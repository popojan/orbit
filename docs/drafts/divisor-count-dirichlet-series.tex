\documentclass[11pt]{article}
\usepackage[utf8]{inputenc}
\usepackage{amsmath,amsthm,amssymb}
\usepackage{geometry}
\geometry{margin=2.5cm}
\usepackage[hidelinks]{hyperref}

\newtheorem{theorem}{Theorem}
\newtheorem{lemma}[theorem]{Lemma}
\newtheorem{corollary}[theorem]{Corollary}
\newtheorem{proposition}[theorem]{Proposition}

\theoremstyle{definition}
\newtheorem{definition}[theorem]{Definition}

\theoremstyle{remark}
\newtheorem{remark}[theorem]{Remark}

\title{Dirichlet Series for Ordered Divisor Pairs}
\author{Jan Popelka\thanks{Email: popojan@protonmail.com}}
\date{}

\begin{document}

\maketitle

\begin{abstract}
We derive a closed form for the Dirichlet series of the function counting
ordered divisor pairs. For $P(n) = \lfloor\tau(n)/2\rfloor$, which counts
divisor pairs $(d, n/d)$ with $d < n/d$, the Dirichlet series admits the
elegant closed form $L_P(s) = \frac{\zeta(s)^2 - \zeta(2s)}{2}$ for $s > 1$.
\end{abstract}

\section{Ordered Divisor Pairs}

\begin{definition}
For $n \geq 1$, define
\[
P(n) = \#\{d \in \mathbb{Z} : d \mid n, \, 1 \leq d < \sqrt{n}\}
\]
the count of divisors strictly below $\sqrt{n}$.
\end{definition}

This counts \emph{ordered divisor pairs}: factorizations $n = d \cdot e$ with $d < e$.

\begin{proposition}[Closed Form]
For $n \geq 1$, $P(n) = \lfloor \tau(n)/2 \rfloor$ where $\tau(n)$ is the divisor function.
\end{proposition}

\begin{proof}
Divisors pair as $(d, n/d)$ with $d \cdot (n/d) = n$. Exactly half satisfy $d < n/d$,
i.e., $d < \sqrt{n}$. For perfect squares, $d = \sqrt{n}$ pairs with itself and is
excluded by the strict inequality. The floor handles both cases.
\end{proof}

\begin{remark}[Non-multiplicativity]
$P(n)$ is not multiplicative: $P(6) = 2$ but $P(2) \cdot P(3) = 1$.
\end{remark}

\section{The Dirichlet Series}

\begin{theorem}[Main Result]
For $s > 1$,
\[
L_P(s) = \sum_{n=1}^{\infty} \frac{P(n)}{n^s} = \frac{\zeta(s)^2 - \zeta(2s)}{2}.
\]
\end{theorem}

\begin{proof}
Since $P(n) = \lfloor \tau(n)/2 \rfloor$ and $\tau(n)$ is odd iff $n$ is a perfect square:
\[
P(n) = \frac{\tau(n)}{2} - \frac{1}{2}\mathbf{1}_{n \text{ is square}}.
\]
Therefore
\[
L_P(s) = \frac{1}{2}\sum_{n=1}^{\infty} \frac{\tau(n)}{n^s}
       - \frac{1}{2}\sum_{k=1}^{\infty} \frac{1}{k^{2s}}
       = \frac{\zeta(s)^2 - \zeta(2s)}{2}. \qedhere
\]
\end{proof}

\begin{corollary}[Special Values]
$L_P(2) = \dfrac{\pi^4 - 90}{180} = \dfrac{\pi^2(\pi^2-9)}{180}$, \quad
$L_P(3) = \dfrac{\zeta(3)^2 - \zeta(6)}{2}$.
\end{corollary}

\section{Variants}

\begin{remark}[Non-trivial Divisor Pairs]\label{rem:M}
If we exclude the trivial pair $(1, n)$ and count $M(n) = \#\{d : 2 \leq d \leq \sqrt{n}, d \mid n\}$,
then $M(n) = \lfloor(\tau(n)-1)/2\rfloor$ and
\[
L_M(s) = \frac{\zeta(s)^2 + \zeta(2s)}{2} - \zeta(s).
\]
The relationship is $L_P(s) - L_M(s) = \zeta(s) - \zeta(2s)$, accounting for:
\begin{itemize}
\item Adding the pair $(1, n)$ contributes $+\zeta(s)$
\item Changing $d \leq \sqrt{n}$ to $d < \sqrt{n}$ for squares contributes $-\zeta(2s)$
\end{itemize}
\end{remark}

\begin{remark}[Hadamard Product Representation]
The closed form $L_P(s) = \frac{\zeta(s)^2 - \zeta(2s)}{2}$ admits interpretation
via Hadamard products over the nontrivial zeros $\rho$ of $\zeta$.
Using the identity $(1-s/\rho)^2 = (1-2s/\rho) + s^2/\rho^2$, one obtains
\[
\prod_\rho (1-s/\rho)^2 = \prod_\rho (1-2s/\rho) \cdot C(s),
\]
where $C(s) = \prod_\rho \left(1 + \frac{s^2}{\rho(\rho-2s)}\right)$
is an absolutely convergent product (terms decay as $O(|\rho|^{-2})$).
This reveals how the structure of $L_P(s)$ relates to products over zeta zeros,
though it does not simplify computation.
\end{remark}

\begin{remark}[Epsilon-Pole Formulation]
The variant $M(n)$ from Remark~\ref{rem:M} also appears as a residue. Define for $\alpha > 1/2$:
\[
F_n(\alpha, \varepsilon) = \sum_{d=2}^{\infty} \sum_{k=0}^{\infty}
\left[(n - kd - d^2)^2 + \varepsilon\right]^{-\alpha}.
\]
The sum has poles at $n = d(k+d)$ with $d \geq 2$ and $k \geq 0$, i.e., at divisor pairs
$(d, n/d)$ with $2 \leq d \leq n/d$. Thus $\lim_{\varepsilon \to 0^+} \varepsilon^\alpha F_n(\alpha, \varepsilon) = M(n)$.
This provides an ``analytic'' primality criterion ($n$ prime iff $M(n) = 0$),
though not a practical one.
\end{remark}

\begin{thebibliography}{9}
\bibitem{apostol}
T.\,M.~Apostol,
\emph{Introduction to Analytic Number Theory},
Springer, 1976.
\end{thebibliography}

\end{document}
