\documentclass[11pt]{article}
\usepackage[utf8]{inputenc}
\usepackage{amsmath,amsthm,amssymb}
\usepackage{geometry}
\geometry{margin=2.5cm}
\usepackage[hidelinks]{hyperref}

\newtheorem{theorem}{Theorem}
\newtheorem{lemma}[theorem]{Lemma}
\newtheorem{corollary}[theorem]{Corollary}
\newtheorem{proposition}[theorem]{Proposition}

\theoremstyle{definition}
\newtheorem{definition}[theorem]{Definition}

\theoremstyle{remark}
\newtheorem{remark}[theorem]{Remark}

\title{Dirichlet Series for Divisor Counts Below the Square Root}
\author{Jan Popelka\thanks{Email: popojan@protonmail.com}}
\date{}

\begin{document}

\maketitle

\begin{abstract}
We derive a closed form for the Dirichlet series $L_M(s) = \sum_{n=2}^{\infty} M(n)/n^s$,
where $M(n) = \lfloor(\tau(n)-1)/2\rfloor$ counts divisors of $n$ in $[2, \sqrt{n}]$.
Despite the non-multiplicative nature of $M(n)$, the series admits the closed form
$L_M(s) = \frac{1}{2}[\zeta(s)^2 + \zeta(2s)] - \zeta(s)$ for $s > 1$.
\end{abstract}

\section{The Function $M(n)$}

\begin{definition}
For $n \geq 2$, define
\[
M(n) = \#\{d \in \mathbb{Z} : d \mid n, \, 2 \leq d \leq \sqrt{n}\}.
\]
\end{definition}

\begin{proposition}[Closed Form]
For $n \geq 2$, $M(n) = \lfloor (\tau(n) - 1)/2 \rfloor$ where $\tau(n)$ is the
divisor function.
\end{proposition}

\begin{proof}
Divisors pair as $(d, n/d)$. Those with $d \leq \sqrt{n}$ constitute exactly half
of divisors $\geq 2$, with the floor handling perfect squares.
\end{proof}

\begin{remark}[Non-multiplicativity]
$M(n)$ is not multiplicative: $M(6) = 1$ but $M(2) \cdot M(3) = 0$.
This precludes a standard Euler product for its Dirichlet series.
\end{remark}

\section{The Dirichlet Series}

\begin{theorem}[Closed Form for $L_M(s)$]
For $s > 1$,
\[
L_M(s) = \sum_{n=2}^{\infty} \frac{M(n)}{n^s} = \frac{\zeta(s)^2 + \zeta(2s)}{2} - \zeta(s).
\]
\end{theorem}

\begin{proof}
Rewrite as a double sum over divisor pairs:
\[
L_M(s) = \sum_{n=2}^{\infty} \frac{1}{n^s} \sum_{\substack{d \mid n \\ 2 \leq d \leq \sqrt{n}}} 1
= \sum_{d=2}^{\infty} \sum_{\substack{n : d \mid n \\ d \leq \sqrt{n}}} \frac{1}{n^s}.
\]
For fixed $d$, write $n = kd$ where $k \geq d$ (from $d \leq \sqrt{kd}$):
\[
L_M(s) = \sum_{d=2}^{\infty} \frac{1}{d^s} \sum_{k=d}^{\infty} \frac{1}{k^s}
= \sum_{d=2}^{\infty} \frac{\zeta(s) - H_{d-1}(s)}{d^s}
\]
where $H_j(s) = \sum_{k=1}^{j} k^{-s}$.

Expanding and using $\sum_{d=2}^{\infty} d^{-s} = \zeta(s) - 1$:
\[
L_M(s) = \zeta(s)[\zeta(s) - 1] - \sum_{d=2}^{\infty} \frac{H_{d-1}(s)}{d^s}.
\]

The correction sum equals $\sum_{1 \leq j < k} (jk)^{-s} = (\zeta(s)^2 - \zeta(2s))/2$.
Substituting:
\[
L_M(s) = \zeta(s)^2 - \zeta(s) - \frac{\zeta(s)^2 - \zeta(2s)}{2}
= \frac{\zeta(s)^2 + \zeta(2s)}{2} - \zeta(s). \qedhere
\]
\end{proof}

\begin{corollary}[Special Value]
$L_M(2) = \dfrac{\pi^2(7\pi^2 - 60)}{360} \approx 0.2491$.
\end{corollary}

\begin{remark}[Convergence Rate]
The direct sum $\sum_{n \leq N} M(n)/n^s$ converges to $L_M(s)$ at rate $O(N^{-1/2})$,
which is slow. This follows from the error term in divisor sum asymptotics.
\end{remark}

\section{Connection to Regularized Sums}

\begin{remark}[Epsilon-Pole Formulation]
The function $M(n)$ also appears as a residue. Define for $\alpha > 1/2$:
\[
F_n(\alpha, \varepsilon) = \sum_{d=2}^{\infty} \sum_{k=0}^{\infty}
\left[(n - kd - d^2)^2 + \varepsilon\right]^{-\alpha}.
\]
Then $\lim_{\varepsilon \to 0^+} \varepsilon^\alpha F_n(\alpha, \varepsilon) = M(n)$.

The sum has poles at $\varepsilon = 0$ precisely when $n = kd + d^2 = d(k+d)$ for some
$d \geq 2$, $k \geq 0$---i.e., when $d$ divides $n$ with $d \leq \sqrt{n}$.
The residue counts these poles, giving $M(n)$.

This provides an ``analytic'' primality criterion: $n$ is prime iff $M(n) = 0$
iff $F_n$ is analytic at $\varepsilon = 0$. However, this is not computationally
useful since evaluating $M(n)$ requires knowing divisors.
\end{remark}

\begin{thebibliography}{9}
\bibitem{apostol}
T.\,M.~Apostol,
\emph{Introduction to Analytic Number Theory},
Springer, 1976.
\end{thebibliography}

\end{document}
