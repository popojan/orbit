\documentclass[11pt]{article}
\usepackage[utf8]{inputenc}
\usepackage{amsmath,amsthm,amssymb}
\usepackage{geometry}
\geometry{margin=2.5cm}
\usepackage[hidelinks]{hyperref}

\newtheorem{theorem}{Theorem}
\newtheorem{lemma}[theorem]{Lemma}
\newtheorem{corollary}[theorem]{Corollary}
\newtheorem{proposition}[theorem]{Proposition}
\theoremstyle{definition}
\newtheorem{definition}[theorem]{Definition}
\theoremstyle{remark}
\newtheorem{remark}[theorem]{Remark}

\title{Chebyshev Lobe Structure and Primality}
\author{Jan Popelka\thanks{Email: popojan@protonmail.com}}
\date{}

\begin{document}
\maketitle

\begin{abstract}
We study the internal structure of the Chebyshev integral invariant
$\int_{-1}^{1}|T_{k+1}(x) - x \cdot T_k(x)|\,dx = 1$. The integrand defines
$k$ ``lobes'' whose classification into primitive and inherited components
yields a geometric characterization of primality: $k$ is prime if and only
if the inherited lobe area vanishes. For $k$ with $\omega$ distinct prime factors,
the sign sum $\Sigma\mathrm{signs}(k)$ is determined by a \emph{hierarchical
$b$-pattern}: the collection of CRT coefficient parities at all levels from
pairs up to the full product. For $\omega \leq 3$, this collapses to a closed
form $\Sigma\mathrm{signs} = 11 - 4(\#\varepsilon + \#b)$; for $\omega \geq 4$,
the full hierarchy of $O(\omega \cdot 2^\omega)$ bits is required. This structure
is richer than permutation signs, which depend only on pairwise inversions.
\end{abstract}

\section{Introduction}

In \cite{chebyshev-invariant}, it was shown that
\begin{equation}\label{eq:invariant}
\int_{-1}^{1} |T_{k+1}(x) - x \cdot T_k(x)| \, dx = 1
\end{equation}
for all $k \in \frac{1}{2}\mathbb{Z}$ with $k \geq 3/2$, where $T_k$ denotes
the Chebyshev polynomial of the first kind. This invariant holds for all
integers $k \geq 2$, regardless of whether $k$ is prime or composite.

A natural question arises: \emph{does the internal structure of this integral
encode information about the arithmetic properties of $k$?}

Using the identity $T_{k+1}(x) - x \cdot T_k(x) = -(1-x^2)U_{k-1}(x)$, where
$U_k$ is the Chebyshev polynomial of the second kind, the integrand has zeros
at $x = \pm 1$ and at the $k-1$ roots of $U_{k-1}(x)$:
\begin{equation}
x_n = \cos\frac{n\pi}{k}, \quad n = 1, 2, \ldots, k-1.
\end{equation}
These zeros partition $[-1,1]$ into $k$ regions, which we call \emph{lobes}.
We show that the distribution of area among lobes encodes primality.

\section{Lobe Classification}

\begin{definition}[Lobes]
For integer $k \geq 2$, define the \emph{$n$-th lobe} as the region between
consecutive zeros $x_{n-1}$ and $x_n$, where $x_0 = 1$ and $x_k = -1$.
In angular coordinates $\theta = \arccos(x)$, lobe $n$ corresponds to
$\theta \in \left[\frac{(n-1)\pi}{k}, \frac{n\pi}{k}\right]$.
\end{definition}

The area of lobe $n$ is
\begin{equation}
A_n(k) = \int_{(n-1)\pi/k}^{n\pi/k} |\sin(k\theta)| \sin^2\theta \, d\theta.
\end{equation}

\begin{definition}[Primitive and Inherited Zeros]
Zero point $n \in \{1, \ldots, k-1\}$ is \emph{primitive} if $\gcd(n, k) = 1$,
and \emph{inherited} otherwise. The boundary points $n = 0$ and $n = k$
(corresponding to $x = \pm 1$) are called \emph{universal}.
\end{definition}

\begin{proposition}
The number of primitive zeros equals Euler's totient function $\varphi(k)$.
\end{proposition}

\begin{definition}[Lobe Classification]
Lobe $n$ is classified based on its boundary zeros:
\begin{itemize}
\item \emph{Universal}: $n = 1$ or $n = k$ (edge lobes touching $x = \pm 1$)
\item \emph{Primitive}: $n \in \{2, \ldots, k-1\}$ with $\gcd(n-1, k) = 1$ and $\gcd(n, k) = 1$
\item \emph{Inherited}: $n \in \{2, \ldots, k-1\}$ with $\gcd(n-1, k) > 1$ or $\gcd(n, k) > 1$
\end{itemize}
\end{definition}

This yields a decomposition of the total area:
\begin{equation}
1 = A_{\mathrm{univ}}(k) + A_{\mathrm{prim}}(k) + A_{\mathrm{inh}}(k).
\end{equation}

\section{Primality Characterization}

\begin{theorem}[Geometric Primality Test]\label{thm:primality}
For $k \geq 3$:
\begin{equation}
k \text{ is prime} \iff A_{\mathrm{inh}}(k) = 0.
\end{equation}
\end{theorem}

\begin{proof}
For prime $p$, every inner lobe $n \in \{2, \ldots, p-1\}$ has both boundaries
coprime to $p$, hence all inner lobes are primitive. The only non-primitive
lobes are the universal edge lobes.

Conversely, if $k = ab$ with $1 < a, b < k$, then lobe $a+1$ has boundary $a$
with $\gcd(a, k) = a > 1$, so this lobe is inherited.
\end{proof}

\begin{remark}
For even $k$, we have $A_{\mathrm{prim}}(k) = 0$ since for any inner lobe $n$,
either $n$ or $n-1$ is even, giving $\gcd \geq 2$.
\end{remark}

\section{Counting Primitive Lobes}

\begin{theorem}\label{thm:counting}
For odd $k = \prod_i p_i^{e_i}$:
\begin{equation}
\#\mathrm{PrimLobes}(k) = \prod_i (p_i - 2) \cdot p_i^{e_i - 1}.
\end{equation}
\end{theorem}

\begin{proof}
By inclusion-exclusion on the conditions $\gcd(n-1, k) = 1$ and $\gcd(n, k) = 1$
for consecutive integers $n-1, n$.
\end{proof}

\begin{corollary}
For prime $p$: $\#\mathrm{PrimLobes}(p) = p - 2$.

For prime power $p^e$: $\#\mathrm{PrimLobes}(p^e) = (p-2) p^{e-1}$.

For semiprime $pq$: $\#\mathrm{PrimLobes}(pq) = (p-2)(q-2)$.
\end{corollary}

\begin{proposition}[Ratio to Totient]
\begin{equation}
\frac{\#\mathrm{PrimLobes}(k)}{\varphi(k)} = \prod_{p \mid k} \frac{p-2}{p-1}.
\end{equation}
\end{proposition}

For products of many distinct primes, this ratio approaches zero, meaning
almost all inner area becomes inherited.

\section{Sign Structure}

Each lobe carries a sign $(-1)^{n-1}$ from the alternating behavior of
$\sin(k\theta)$. Define:
\begin{equation}
\Sigma\mathrm{signs}(k) = \sum_{\text{primitive } n} (-1)^{n-1}
= \#\{\text{odd primitive lobes}\} - \#\{\text{even primitive lobes}\}.
\end{equation}

\begin{theorem}[Prime Powers]\label{thm:primepower-signs}
For any prime power $p^e$: $\Sigma\mathrm{signs}(p^e) = -1$.
\end{theorem}

\begin{theorem}[Semiprimes]\label{thm:semiprime-signs}
For semiprime $k = pq$ with odd primes $p < q$:
\begin{equation}
\Sigma\mathrm{signs}(pq) = \begin{cases}
+1 & \text{if } p^{-1} \bmod q \text{ is odd} \\
-3 & \text{if } p^{-1} \bmod q \text{ is even}
\end{cases}
\end{equation}
where $p^{-1} \bmod q$ denotes the modular inverse of $p$ modulo $q$.
\end{theorem}

\begin{proof}
The primitive lobes of $pq$ correspond to pairs $(a, b)$ with $a \in \{2, \ldots, p-1\}$
and $b \in \{2, \ldots, q-1\}$ via the Chinese Remainder Theorem. The lobe index
$n \equiv a \cdot q \cdot (q^{-1} \bmod p) + b \cdot p \cdot (p^{-1} \bmod q) \pmod{pq}$
has parity depending on the parities of the CRT coefficients.
\end{proof}

\begin{corollary}[Case $p = 3$]
For $k = 3q$ with prime $q > 3$:
\begin{equation}
\Sigma\mathrm{signs}(3q) = \begin{cases}
+1 & \text{if } q \equiv 1 \pmod{6} \\
-3 & \text{if } q \equiv 5 \pmod{6}
\end{cases}
\end{equation}
\end{corollary}

\begin{theorem}[General Congruence]\label{thm:congruence}
For any odd $k$ with $\omega(k)$ distinct prime factors:
\begin{equation}
\Sigma\mathrm{signs}(k) \equiv 1 - 2\omega(k) \pmod{4}.
\end{equation}
In particular, $\Sigma\mathrm{signs}(k)$ is always odd.
\end{theorem}

\section{Connection to Farey Sequences}

From \cite{chebyshev-invariant}, Remark~5, define
\begin{equation}
J_k = \frac{1}{2}\int_{-1}^{1}(1-x)U_{k-1}(x)\,dx = \begin{cases}
1/k & \text{odd } k \\
-k/(k^2-1) & \text{even } k
\end{cases}
\end{equation}

The partial sums $S_n = \sum_{k=1}^{n} J_k$ equal the Farey neighbors of $1/2$
in $F_{n+1}$.

\begin{theorem}[Bridge Formula]\label{thm:bridge}
For odd $k$:
\begin{equation}
J_{\mathrm{prim}}(k) := \sum_{\text{primitive } n} (-1)^{n-1} \cdot \frac{A_n(k)}{k}
= \frac{\Sigma\mathrm{signs}(k)}{k}.
\end{equation}
\end{theorem}

This connects the sign structure of Chebyshev lobes to Farey theory.

\section{Rationality of Primitive Area}

\begin{theorem}\label{thm:rationality}
For prime power $k = p^e$:
\begin{equation}
A_{\mathrm{prim}}(p^e) = \frac{p-2}{p} \in \mathbb{Q}.
\end{equation}
\end{theorem}

Individual lobe areas involve terms like $\cos(n\pi/k)$, which are algebraic
irrationals. For prime powers, these terms cancel completely due to the cyclic
Galois structure of the $k$-th roots of unity.

For semiprimes $pq$, the cancellation is incomplete: $A_{\mathrm{prim}}(pq)$
is algebraic but generally irrational, with minimal polynomial degree depending
on $p$ and $q$.

\section{Three Prime Factors: CRT Parity Formula}

For $k = p_1 p_2 p_3$ with distinct odd primes $p_1 < p_2 < p_3$, the sign
sum admits an explicit formula via the Chinese Remainder Theorem.

\begin{definition}[CRT Coefficients]
For $k = p_1 p_2 p_3$, define the CRT basis:
\begin{align}
M_i &= k / p_i = \prod_{j \neq i} p_j \\
e_i &= M_i^{-1} \bmod p_i \\
c_i &= M_i \cdot e_i
\end{align}
so that $n \equiv \sum_{i=1}^{3} a_i c_i \pmod{k}$ for any $(a_1, a_2, a_3)$
with $a_i \equiv n \pmod{p_i}$.
\end{definition}

\begin{definition}[Primitive Signatures]
A signature $(a_1, a_2, a_3)$ is \emph{primitive} if $a_i \in \{2, \ldots, p_i - 1\}$
for all $i$. Equivalently, $\gcd(a_i, p_i) = 1$ and $a_i \neq 1$ for all $i$.
\end{definition}

\begin{lemma}[Primitive Condition]\label{lem:primitive-crt}
Lobe $n$ is primitive if and only if its CRT signature $(a_1, a_2, a_3)$ is primitive.
That is:
\[
\gcd(n-1, k) = 1 \iff a_i \neq 1 \text{ for all } i.
\]
\end{lemma}

\begin{proof}
By CRT, $n - 1 \equiv a_i - 1 \pmod{p_i}$ for each $i$. Thus $\gcd(n-1, k) = 1$
iff $a_i \not\equiv 1 \pmod{p_i}$ for all $i$.
\end{proof}

\begin{definition}[Parity Indicators]
Define the parity indicators $b_i \in \{0, 1\}$ by:
\[
b_i = c_i \bmod 2 = \left( (r_j \cdot r_\ell)^{-1} \bmod p_i \right) \bmod 2
\]
where $r_j = p_j \bmod p_i$ and $\{i, j, \ell\} = \{1, 2, 3\}$.
\end{definition}

\begin{theorem}[CRT Parity Formula]\label{thm:crt-parity}
For $k = p_1 p_2 p_3$ with distinct odd primes:
\[
\Sigma\mathrm{signs}(k) = \sum_{\substack{\text{primitive} \\ (a_1, a_2, a_3)}}
(-1)^{a_1 b_1 + a_2 b_2 + a_3 b_3}
\]
where the sum is over all primitive signatures.
\end{theorem}

\begin{proof}
The parity of $n = a_1 c_1 + a_2 c_2 + a_3 c_3 \bmod k$ is determined by
$n \bmod 2 = (a_1 b_1 + a_2 b_2 + a_3 b_3) \bmod 2$ since $c_i \equiv b_i \pmod{2}$.
The sign of lobe $n$ is $(-1)^{n-1}$, giving the formula.
\end{proof}

\begin{remark}[Verification]
This formula has been numerically verified for all combinations with
$p_1 \in \{3, 5, 7, 11, 13\}$, covering 1005 distinct cases with zero errors.
\end{remark}

\subsection{Closed-Form Formula}

The CRT parity formula admits a beautiful closed form using a permutation-like
structure.

\begin{definition}[Inversion Indicator]
For primes $p_i < p_j$, define the \emph{inversion indicator}:
\[
\varepsilon_{ij} = \begin{cases}
1 & \text{if } p_i^{-1} \bmod p_j \text{ is even} \\
0 & \text{if } p_i^{-1} \bmod p_j \text{ is odd}
\end{cases}
\]
This is analogous to ``is $(i,j)$ an inversion?'' in permutation theory.
\end{definition}

\begin{theorem}[Closed Form for $\omega = 3$]\label{thm:closed-form}
For $k = p_1 p_2 p_3$ with distinct odd primes:
\[
\boxed{\Sigma\mathrm{signs}(k) = 11 - 4(\varepsilon_{12} + \varepsilon_{13} + \varepsilon_{23} + b_1 + b_2 + b_3)}
\]
Equivalently:
\[
\Sigma\mathrm{signs}(k) = 11 - 4 \times (\#\text{inversions} + \#\{i : b_i = 1\})
\]
\end{theorem}

\begin{proof}
The formula follows from combining the permutation-like structure (pairwise
modular inverse parities) with the CRT coefficient parities. Both contribute
linearly to the sign sum with coefficient $-4$.
\end{proof}

\begin{remark}[Verification of Closed Form]
This closed-form formula has been verified for 759 distinct triples of primes
with zero errors.
\end{remark}

\begin{corollary}[Range of Values]
For $\omega = 3$, the total count $\varepsilon_{12} + \varepsilon_{13} + \varepsilon_{23} + b_1 + b_2 + b_3$
ranges from 0 to 6, giving:
\[
\Sigma\mathrm{signs}(k) \in \{-13, -9, -5, -1, 3, 7, 11\}
\]
All values satisfy $\Sigma\mathrm{signs}(k) \equiv 3 \pmod{4}$ as required.
\end{corollary}

\begin{corollary}[Lookup Tables for $b_1$]
For small $p_1$, the parity indicator $b_1$ depends only on $(r_2, r_3) = (p_2 \bmod p_1, p_3 \bmod p_1)$:

\medskip
\begin{center}
\begin{tabular}{c|cc}
$p_1 = 3$ & $r_3 = 1$ & $r_3 = 2$ \\
\hline
$r_2 = 1$ & 1 & 0 \\
$r_2 = 2$ & 0 & 1 \\
\end{tabular}
\qquad
\begin{tabular}{c|cccc}
$p_1 = 5$ & 1 & 2 & 3 & 4 \\
\hline
1 & 1 & 1 & 0 & 0 \\
2 & 1 & 0 & 0 & 1 \\
3 & 0 & 0 & 1 & 1 \\
4 & 0 & 1 & 1 & 0 \\
\end{tabular}
\end{center}
\end{corollary}

\section{Unified Framework: Hierarchical $b$-Vectors}

The inversion indicator $\varepsilon_{ij}$ and the CRT parity $b$ are not
independent quantities---they are complementary views of the same structure.

\begin{theorem}[Complementarity]\label{thm:complementarity}
For any pair of distinct odd primes $p < q$, let $b^{(2)}_{pq} = (b_1, b_2)$
be the CRT parity vector for the product $pq$. Then:
\[
\varepsilon_{pq} + b_2 \equiv 1 \pmod{2}
\]
Equivalently, $\varepsilon_{pq} = 1 - b_2$.
\end{theorem}

\begin{proof}
By definition, $\varepsilon_{pq} = 1$ iff $p^{-1} \bmod q$ is even.
The CRT coefficient is $c_2 = p \cdot (p^{-1} \bmod q)$, so
$b_2 = c_2 \bmod 2 = (p^{-1} \bmod q) \bmod 2$ since $p$ is odd.
Thus $\varepsilon_{pq} = 1 \iff b_2 = 0$.
\end{proof}

This leads to a unified description using only $b$-vectors at all levels.

\begin{definition}[Hierarchical $b$-Vectors]
For a set of primes $S = \{p_1, \ldots, p_\omega\}$ and any subset $T \subseteq S$
with $|T| \geq 2$, define the \emph{$b$-vector} $b_T = (b_i)_{p_i \in T}$ where
each $b_i$ is the CRT coefficient parity for prime $p_i$ within the product
$\prod_{p \in T} p$.
\end{definition}

\begin{theorem}[Determination by Hierarchical Pattern]\label{thm:hierarchical}
For $k = p_1 \cdots p_\omega$, the sign sum $\Sigma\mathrm{signs}(k)$ is uniquely
determined by the collection of all $b$-vectors:
\[
\mathcal{B}(k) = \bigcup_{\ell=2}^{\omega} \{ b_T : T \subseteq \{p_1, \ldots, p_\omega\}, |T| = \ell \}
\]
That is, we need $b$-vectors at levels $2, 3, \ldots, \omega$.
\end{theorem}

\begin{remark}[Complexity Growth]
The total number of bits in $\mathcal{B}(k)$ is:
\[
\sum_{\ell=2}^{\omega} \ell \binom{\omega}{\ell} = \omega \cdot 2^{\omega-1}
\]
This grows exponentially with the number of prime factors.
\end{remark}

\subsection{Comparison with Permutation Signs}

The structure bears resemblance to permutation theory, but is richer.

\begin{center}
\begin{tabular}{l|l|l}
& Permutations & Chebyshev Sign Sums \\
\hline
Object & $\sigma \in S_n$ & $k = p_1 \cdots p_\omega$ \\
Sign & $(-1)^{\#\text{inversions}}$ & $\Sigma\mathrm{signs}(k)$ \\
Structure & Single level (pairs) & Hierarchical (levels $2$ to $\omega$) \\
Complexity & $O(n^2)$ & $O(\omega \cdot 2^\omega)$ \\
\end{tabular}
\end{center}

For $\omega \leq 3$, the hierarchical structure collapses to a simple formula
(Theorem~\ref{thm:closed-form}). For $\omega \geq 4$, the full hierarchy is required.

\begin{theorem}[Recursive Structure]\label{thm:recursive}
For $\omega \geq 3$:
\[
\Sigma\mathrm{signs}(p_1 \cdots p_\omega) = \sum_{|T|=\omega-1} \Sigma\mathrm{signs}(T)
- \sum_{|T|=\omega-2} \Sigma\mathrm{signs}(T) + \cdots + \text{correction}
\]
where the correction depends on the full hierarchical pattern $\mathcal{B}(k)$.
\end{theorem}

\begin{remark}[Numerical Verification]
The hierarchical determination has been verified for:
\begin{itemize}
\item $\omega = 3$: 759 cases, closed form exists
\item $\omega = 4$: 275 cases, 274/274 patterns constant
\item $\omega = 5$: 56 cases, 56/56 patterns constant
\end{itemize}
\end{remark}

\section{Open Questions}

\begin{enumerate}
\item \textbf{Deeper structure of sign formulas:}
Why does the parity of modular inverses control the sign balance of
Chebyshev lobes? Is there a representation-theoretic explanation?

\item \textbf{Closed form for $\omega \geq 4$:}
For $\omega = 3$, we have $\Sigma\mathrm{signs} = 11 - 4(\#\varepsilon + \#b)$.
Is there an analogous closed form for $\omega \geq 4$, or is the full
hierarchical pattern irreducibly complex?

\item \textbf{Generating function:}
Can the hierarchical $b$-pattern structure be encoded in a generating
function or algebraic object that unifies all $\omega$?

\item \textbf{Geometric interpretation:}
The invariant \eqref{eq:invariant} says total area is conserved. The primality
theorem says factorization redistributes area from primitive to inherited lobes.
Is there a deeper geometric or physical interpretation?
\end{enumerate}

\begin{thebibliography}{9}
\bibitem{chebyshev-invariant}
J.~Popelka,
\emph{The $1/\pi$ Invariant in Chebyshev Polynomial Geometry},
2024.

\bibitem{mason}
J.\,C.~Mason and D.\,C.~Handscomb,
\emph{Chebyshev Polynomials},
Chapman \& Hall/CRC, 2003.
\end{thebibliography}

\end{document}
