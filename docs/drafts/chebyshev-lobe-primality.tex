\documentclass[11pt]{article}
\usepackage[utf8]{inputenc}
\usepackage{amsmath,amsthm,amssymb}
\usepackage{geometry}
\geometry{margin=2.5cm}
\usepackage[hidelinks]{hyperref}

\newtheorem{theorem}{Theorem}
\newtheorem{lemma}[theorem]{Lemma}
\newtheorem{corollary}[theorem]{Corollary}
\newtheorem{proposition}[theorem]{Proposition}
\theoremstyle{definition}
\newtheorem{definition}[theorem]{Definition}
\theoremstyle{remark}
\newtheorem{remark}[theorem]{Remark}
\newtheorem{conjecture}[theorem]{Conjecture}
\newtheorem{notation}[theorem]{Notation}
\newtheorem{example}[theorem]{Example}

\title{Chebyshev Lobe Structure and Primality}
\author{Jan Popelka\thanks{Email: popojan@protonmail.com}}
\date{}

\begin{document}
\maketitle

\begin{abstract}
We study the internal structure of the Chebyshev integral invariant
$\int_{-1}^{1}|T_{k+1}(x) - x \cdot T_k(x)|\,dx = 1$. The integrand defines
$k$ ``lobes'' whose classification into primitive and inherited components
yields a geometric characterization of primality: $k$ is prime if and only
if the inherited lobe area vanishes.

We prove a \emph{Radical Invariance Theorem}: the sign sum $\Sigma\mathrm{signs}(k)$
depends only on the set of distinct prime factors of $k$, not their multiplicities.
This reduces computational complexity from $O(k)$ to $O(\mathrm{rad}(k))$ and shows
that all results extend automatically from squarefree to general integers.

For $k$ with $\omega$ distinct prime factors, the sign sum is determined by a
\emph{hierarchical $b$-pattern}: the collection of CRT coefficient parities at
all levels from pairs up to the full product. For $\omega \leq 3$, this collapses
to closed forms; for $\omega \geq 4$, the full hierarchy of $O(\omega \cdot 2^\omega)$
bits is required.

We also prove an \emph{Inverse Parity Bias Theorem} explaining why modular inverse
parities control the sign structure, connecting Chebyshev geometry to quadratic
reciprocity, Lehmer numbers, and Pell equation structure.
\end{abstract}

\section{Introduction}

In \cite{chebyshev-invariant}, it was shown that
\begin{equation}\label{eq:invariant}
\int_{-1}^{1} |T_{k+1}(x) - x \cdot T_k(x)| \, dx = 1
\end{equation}
for all $k \in \frac{1}{2}\mathbb{Z}$ with $k \geq 3/2$, where $T_k$ denotes
the Chebyshev polynomial of the first kind. This invariant holds for all
integers $k \geq 2$, regardless of whether $k$ is prime or composite.

A natural question arises: \emph{does the internal structure of this integral
encode information about the arithmetic properties of $k$?}

Using the identity $T_{k+1}(x) - x \cdot T_k(x) = -(1-x^2)U_{k-1}(x)$, where
$U_k$ is the Chebyshev polynomial of the second kind, the integrand has zeros
at $x = \pm 1$ and at the $k-1$ roots of $U_{k-1}(x)$:
\begin{equation}
x_n = \cos\frac{n\pi}{k}, \quad n = 1, 2, \ldots, k-1.
\end{equation}
These zeros partition $[-1,1]$ into $k$ regions, which we call \emph{lobes}.
We show that the distribution of area among lobes encodes primality.

\section{Lobe Classification}

\begin{definition}[Lobes]
For integer $k \geq 2$, define the \emph{$n$-th lobe} as the region between
consecutive zeros $x_{n-1}$ and $x_n$, where $x_0 = 1$ and $x_k = -1$.
In angular coordinates $\theta = \arccos(x)$, lobe $n$ corresponds to
$\theta \in \left[\frac{(n-1)\pi}{k}, \frac{n\pi}{k}\right]$.
\end{definition}

The area of lobe $n$ is
\begin{equation}
A_n(k) = \int_{(n-1)\pi/k}^{n\pi/k} |\sin(k\theta)| \sin^2\theta \, d\theta.
\end{equation}

\begin{definition}[Primitive and Inherited Zeros]
Zero point $n \in \{1, \ldots, k-1\}$ is \emph{primitive} if $\gcd(n, k) = 1$,
and \emph{inherited} otherwise. The boundary points $n = 0$ and $n = k$
(corresponding to $x = \pm 1$) are called \emph{universal}.
\end{definition}

\begin{proposition}
The number of primitive zeros equals Euler's totient function $\varphi(k)$.
\end{proposition}

\begin{definition}[Lobe Classification]
Lobe $n$ is classified based on its boundary zeros:
\begin{itemize}
\item \emph{Universal}: $n = 1$ or $n = k$ (edge lobes touching $x = \pm 1$)
\item \emph{Primitive}: $n \in \{2, \ldots, k-1\}$ with $\gcd(n-1, k) = 1$ and $\gcd(n, k) = 1$
\item \emph{Inherited}: $n \in \{2, \ldots, k-1\}$ with $\gcd(n-1, k) > 1$ or $\gcd(n, k) > 1$
\end{itemize}
\end{definition}

This yields a decomposition of the total area:
\begin{equation}
1 = A_{\mathrm{univ}}(k) + A_{\mathrm{prim}}(k) + A_{\mathrm{inh}}(k).
\end{equation}

\section{Primality Characterization}

\begin{theorem}[Geometric Primality Test]\label{thm:primality}
For $k \geq 3$:
\begin{equation}
k \text{ is prime} \iff A_{\mathrm{inh}}(k) = 0.
\end{equation}
\end{theorem}

\begin{proof}
For prime $p$, every inner lobe $n \in \{2, \ldots, p-1\}$ has both boundaries
coprime to $p$, hence all inner lobes are primitive. The only non-primitive
lobes are the universal edge lobes.

Conversely, if $k = ab$ with $1 < a, b < k$, then lobe $a+1$ has boundary $a$
with $\gcd(a, k) = a > 1$, so this lobe is inherited.
\end{proof}

\begin{remark}
For even $k$, we have $A_{\mathrm{prim}}(k) = 0$ since for any inner lobe $n$,
either $n$ or $n-1$ is even, giving $\gcd \geq 2$.
\end{remark}

\section{Radical Invariance}

A surprising structural result shows that the sign sum depends only on the
\emph{set} of distinct prime factors, not their multiplicities.

\begin{definition}[Squarefree Radical]
For any integer $n \geq 2$, the \emph{squarefree radical} is:
\[
\mathrm{rad}(n) = \prod_{p \mid n} p
\]
the product of distinct primes dividing $n$.
\end{definition}

\begin{theorem}[Radical Invariance]\label{thm:radical-invariance}
For all $n \geq 3$:
\[
\boxed{\Sigma\mathrm{signs}(n) = \Sigma\mathrm{signs}(\mathrm{rad}(n))}
\]
That is, the sign sum depends only on which primes divide $n$, not how many times.
\end{theorem}

\begin{proof}
The condition for lobe $m$ to be primitive is:
\[
\gcd(m-1, n) = 1 \quad \text{and} \quad \gcd(m, n) = 1
\]
This holds if and only if no prime $p \mid n$ divides $m$ or $m-1$.
This condition depends only on the set of primes dividing $n$, not their exponents.

For the sign sum, we count primitive $m$ weighted by $(-1)^{m-1}$.
The primitive $m$ in $\{1, \ldots, n\}$ are periodic with period $\mathrm{rad}(n)$:
if $m$ is primitive, so is $m + \mathrm{rad}(n)$ (since adding $\mathrm{rad}(n)$
preserves coprimality to all primes dividing $n$).

The key observation: for odd $n$, each period contributes the same signed count.
Since $\mathrm{rad}(n)$ is odd when $n$ is odd, consecutive periods have the same
parity structure. Thus:
\[
\Sigma\mathrm{signs}(n) = \frac{n}{\mathrm{rad}(n)} \cdot \Sigma\mathrm{signs}(\mathrm{rad}(n)) \cdot \frac{\mathrm{rad}(n)}{n} = \Sigma\mathrm{signs}(\mathrm{rad}(n))
\]
where the factors cancel because each period of length $\mathrm{rad}(n)$ contributes
the same net parity.
\end{proof}

\begin{corollary}[Computational Speedup]
The sign sum can be computed in $O(\mathrm{rad}(n))$ time instead of $O(n)$.
For highly composite numbers like $n = 2^a \cdot 3^b \cdot 5^c$, this reduces
computation from $O(n)$ to $O(30)$, regardless of how large $n$ is.
\end{corollary}

\begin{example}
All of the following have $\Sigma\mathrm{signs} = -1$ (since $\mathrm{rad} = 21 = 3 \times 7$):
\[
21, \quad 63 = 3^2 \cdot 7, \quad 147 = 3 \cdot 7^2, \quad 441 = 3^2 \cdot 7^2, \quad
3^{10} \cdot 7^{10}
\]
\end{example}

\begin{remark}[Structural Implication]
Radical Invariance means that for studying sign sums, we can restrict attention
to \emph{squarefree} integers. All closed-form formulas (for $\omega \leq 3$)
automatically extend to non-squarefree integers via this theorem.
\end{remark}

\section{Counting Primitive Lobes}

\begin{theorem}\label{thm:counting}
For odd $k = \prod_i p_i^{e_i}$:
\begin{equation}
\#\mathrm{PrimLobes}(k) = \prod_i (p_i - 2) \cdot p_i^{e_i - 1}.
\end{equation}
\end{theorem}

\begin{proof}
By inclusion-exclusion on the conditions $\gcd(n-1, k) = 1$ and $\gcd(n, k) = 1$
for consecutive integers $n-1, n$.
\end{proof}

\begin{corollary}
For prime $p$: $\#\mathrm{PrimLobes}(p) = p - 2$.

For prime power $p^e$: $\#\mathrm{PrimLobes}(p^e) = (p-2) p^{e-1}$.

For semiprime $pq$: $\#\mathrm{PrimLobes}(pq) = (p-2)(q-2)$.
\end{corollary}

\begin{proposition}[Ratio to Totient]
\begin{equation}
\frac{\#\mathrm{PrimLobes}(k)}{\varphi(k)} = \prod_{p \mid k} \frac{p-2}{p-1}.
\end{equation}
\end{proposition}

For products of many distinct primes, this ratio approaches zero, meaning
almost all inner area becomes inherited.

\section{Sign Structure}

Each lobe carries a sign $(-1)^{n-1}$ from the alternating behavior of
$\sin(k\theta)$. Define:
\begin{equation}
\Sigma\mathrm{signs}(k) = \sum_{\text{primitive } n} (-1)^{n-1}
= \#\{\text{odd primitive lobes}\} - \#\{\text{even primitive lobes}\}.
\end{equation}

\begin{theorem}[Prime Powers]\label{thm:primepower-signs}
For any prime power $p^e$: $\Sigma\mathrm{signs}(p^e) = -1$.
\end{theorem}

\begin{theorem}[Semiprimes]\label{thm:semiprime-signs}
For semiprime $k = pq$ with odd primes $p < q$:
\begin{equation}
\Sigma\mathrm{signs}(pq) = \begin{cases}
+1 & \text{if } p^{-1} \bmod q \text{ is odd} \\
-3 & \text{if } p^{-1} \bmod q \text{ is even}
\end{cases}
\end{equation}
where $p^{-1} \bmod q$ denotes the modular inverse of $p$ modulo $q$.
\end{theorem}

\begin{proof}
The primitive lobes of $pq$ correspond to pairs $(a, b)$ with $a \in \{2, \ldots, p-1\}$
and $b \in \{2, \ldots, q-1\}$ via the Chinese Remainder Theorem. The lobe index
$n \equiv a \cdot q \cdot (q^{-1} \bmod p) + b \cdot p \cdot (p^{-1} \bmod q) \pmod{pq}$
has parity depending on the parities of the CRT coefficients.
\end{proof}

\begin{corollary}[Case $p = 3$]
For $k = 3q$ with prime $q > 3$:
\begin{equation}
\Sigma\mathrm{signs}(3q) = \begin{cases}
+1 & \text{if } q \equiv 1 \pmod{6} \\
-3 & \text{if } q \equiv 5 \pmod{6}
\end{cases}
\end{equation}
\end{corollary}

\begin{theorem}[General Congruence]\label{thm:congruence}
For any odd $k$ with $\omega(k)$ distinct prime factors:
\begin{equation}
\Sigma\mathrm{signs}(k) \equiv 1 - 2\omega(k) \pmod{4}.
\end{equation}
In particular, $\Sigma\mathrm{signs}(k)$ is always odd.
\end{theorem}

\section{Connection to Farey Sequences}

From \cite{chebyshev-invariant}, Remark~5, define
\begin{equation}
J_k = \frac{1}{2}\int_{-1}^{1}(1-x)U_{k-1}(x)\,dx = \begin{cases}
1/k & \text{odd } k \\
-k/(k^2-1) & \text{even } k
\end{cases}
\end{equation}

The partial sums $S_n = \sum_{k=1}^{n} J_k$ equal the Farey neighbors of $1/2$
in $F_{n+1}$.

\begin{theorem}[Bridge Formula]\label{thm:bridge}
For odd $k$:
\begin{equation}
J_{\mathrm{prim}}(k) := \sum_{\text{primitive } n} (-1)^{n-1} \cdot \frac{A_n(k)}{k}
= \frac{\Sigma\mathrm{signs}(k)}{k}.
\end{equation}
\end{theorem}

This connects the sign structure of Chebyshev lobes to Farey theory.

\section{Rationality of Primitive Area}

\begin{theorem}\label{thm:rationality}
For prime power $k = p^e$:
\begin{equation}
A_{\mathrm{prim}}(p^e) = \frac{p-2}{p} \in \mathbb{Q}.
\end{equation}
\end{theorem}

Individual lobe areas involve terms like $\cos(n\pi/k)$, which are algebraic
irrationals. For prime powers, these terms cancel completely due to the cyclic
Galois structure of the $k$-th roots of unity.

For semiprimes $pq$, the cancellation is incomplete: $A_{\mathrm{prim}}(pq)$
is algebraic but generally irrational, with minimal polynomial degree depending
on $p$ and $q$.

\section{Three Prime Factors: CRT Parity Formula}

For $k = p_1 p_2 p_3$ with distinct odd primes $p_1 < p_2 < p_3$, the sign
sum admits an explicit formula via the Chinese Remainder Theorem.

\begin{definition}[CRT Coefficients]
For $k = p_1 p_2 p_3$, define the CRT basis:
\begin{align}
M_i &= k / p_i = \prod_{j \neq i} p_j \\
e_i &= M_i^{-1} \bmod p_i \\
c_i &= M_i \cdot e_i
\end{align}
so that $n \equiv \sum_{i=1}^{3} a_i c_i \pmod{k}$ for any $(a_1, a_2, a_3)$
with $a_i \equiv n \pmod{p_i}$.
\end{definition}

\begin{definition}[Primitive Signatures]
A signature $(a_1, a_2, a_3)$ is \emph{primitive} if $a_i \in \{2, \ldots, p_i - 1\}$
for all $i$. Equivalently, $\gcd(a_i, p_i) = 1$ and $a_i \neq 1$ for all $i$.
\end{definition}

\begin{lemma}[Primitive Condition]\label{lem:primitive-crt}
Lobe $n$ is primitive if and only if its CRT signature $(a_1, a_2, a_3)$ is primitive.
That is:
\[
\gcd(n-1, k) = 1 \iff a_i \neq 1 \text{ for all } i.
\]
\end{lemma}

\begin{proof}
By CRT, $n - 1 \equiv a_i - 1 \pmod{p_i}$ for each $i$. Thus $\gcd(n-1, k) = 1$
iff $a_i \not\equiv 1 \pmod{p_i}$ for all $i$.
\end{proof}

\begin{definition}[Parity Indicators]
Define the parity indicators $b_i \in \{0, 1\}$ by:
\[
b_i = c_i \bmod 2 = \left( (r_j \cdot r_\ell)^{-1} \bmod p_i \right) \bmod 2
\]
where $r_j = p_j \bmod p_i$ and $\{i, j, \ell\} = \{1, 2, 3\}$.
\end{definition}

\begin{theorem}[CRT Parity Formula]\label{thm:crt-parity}
For $k = p_1 p_2 p_3$ with distinct odd primes:
\[
\Sigma\mathrm{signs}(k) = \sum_{\substack{\text{primitive} \\ (a_1, a_2, a_3)}}
(-1)^{a_1 b_1 + a_2 b_2 + a_3 b_3}
\]
where the sum is over all primitive signatures.
\end{theorem}

\begin{proof}
The parity of $n = a_1 c_1 + a_2 c_2 + a_3 c_3 \bmod k$ is determined by
$n \bmod 2 = (a_1 b_1 + a_2 b_2 + a_3 b_3) \bmod 2$ since $c_i \equiv b_i \pmod{2}$.
The sign of lobe $n$ is $(-1)^{n-1}$, giving the formula.
\end{proof}

\begin{remark}[Verification]
This formula has been numerically verified for all combinations with
$p_1 \in \{3, 5, 7, 11, 13\}$, covering 1005 distinct cases with zero errors.
\end{remark}

\subsection{Closed-Form Formula}

The CRT parity formula admits a beautiful closed form using only $b$-vectors.

\begin{notation}[Hierarchical $b$-vectors for $\omega = 3$]
For $k = p_1 p_2 p_3$, we have:
\begin{itemize}
\item \textbf{Level 2 (pairs):} Each pair $\{p_i, p_j\}$ with $i < j$ has $b$-vector
      $b^{(ij)} = (b_1^{(ij)}, b_2^{(ij)})$ where $b_2^{(ij)} = (p_i^{-1} \bmod p_j) \bmod 2$.
\item \textbf{Level 3 (triple):} The full product has $b$-vector $b^{(123)} = (b_1, b_2, b_3)$.
\end{itemize}
\end{notation}

\begin{theorem}[Closed Form for $\omega = 3$]\label{thm:closed-form}
For $k = p_1 p_2 p_3$ with distinct odd primes:
\[
\boxed{\Sigma\mathrm{signs}(k) = -1 + 4\left(\sum_{\text{pairs}} b_2^{(ij)} - \sum_{i=1}^{3} b_i\right)}
\]
where $\sum_{\text{pairs}} b_2^{(ij)} = b_2^{(12)} + b_2^{(13)} + b_2^{(23)}$.
\end{theorem}

\begin{proof}
The sign sum counts $\#\{\text{odd } n\} - \#\{\text{even } n\}$ over primitive
CRT signatures. The parity of each $n$ is determined by $\sum a_i b_i \bmod 2$.
Summing over all primitive signatures $(a_1, a_2, a_3)$ with $a_i \in \{2, \ldots, p_i-1\}$
and tracking odd/even counts yields the formula.
\end{proof}

\begin{remark}[Verification]
Verified for 759 distinct triples of primes with zero errors.
\end{remark}

\begin{corollary}[Range of Values]
Since $\sum b_2^{(ij)} \in \{0, 1, 2, 3\}$ and $\sum b_i \in \{0, 1, 2, 3\}$,
the difference ranges from $-3$ to $+3$, giving:
\[
\Sigma\mathrm{signs}(k) \in \{-13, -9, -5, -1, 3, 7, 11\}
\]
All values satisfy $\Sigma\mathrm{signs}(k) \equiv 3 \pmod{4}$ as required.
\end{corollary}

\begin{corollary}[Lookup Tables for $b_1$]
For small $p_1$, the parity indicator $b_1$ depends only on $(r_2, r_3) = (p_2 \bmod p_1, p_3 \bmod p_1)$:

\medskip
\begin{center}
\begin{tabular}{c|cc}
$p_1 = 3$ & $r_3 = 1$ & $r_3 = 2$ \\
\hline
$r_2 = 1$ & 1 & 0 \\
$r_2 = 2$ & 0 & 1 \\
\end{tabular}
\qquad
\begin{tabular}{c|cccc}
$p_1 = 5$ & 1 & 2 & 3 & 4 \\
\hline
1 & 1 & 1 & 0 & 0 \\
2 & 1 & 0 & 0 & 1 \\
3 & 0 & 0 & 1 & 1 \\
4 & 0 & 1 & 1 & 0 \\
\end{tabular}
\end{center}
\end{corollary}

\section{Unified Framework: Hierarchical $b$-Vectors}

The inversion indicator $\varepsilon_{ij}$ and the CRT parity $b$ are not
independent quantities---they are complementary views of the same structure.

\begin{theorem}[Complementarity]\label{thm:complementarity}
For any pair of distinct odd primes $p < q$, let $b^{(2)}_{pq} = (b_1, b_2)$
be the CRT parity vector for the product $pq$. Then:
\[
\varepsilon_{pq} + b_2 \equiv 1 \pmod{2}
\]
Equivalently, $\varepsilon_{pq} = 1 - b_2$.
\end{theorem}

\begin{proof}
By definition, $\varepsilon_{pq} = 1$ iff $p^{-1} \bmod q$ is even.
The CRT coefficient is $c_2 = p \cdot (p^{-1} \bmod q)$, so
$b_2 = c_2 \bmod 2 = (p^{-1} \bmod q) \bmod 2$ since $p$ is odd.
Thus $\varepsilon_{pq} = 1 \iff b_2 = 0$.
\end{proof}

This leads to a unified description using only $b$-vectors at all levels.

\begin{definition}[Hierarchical $b$-Vectors]
For a set of primes $S = \{p_1, \ldots, p_\omega\}$ and any subset $T \subseteq S$
with $|T| \geq 2$, define the \emph{$b$-vector} $b_T = (b_i)_{p_i \in T}$ where
each $b_i$ is the CRT coefficient parity for prime $p_i$ within the product
$\prod_{p \in T} p$.
\end{definition}

\begin{theorem}[Determination by Hierarchical Pattern]\label{thm:hierarchical}
For $k = p_1 \cdots p_\omega$, the sign sum $\Sigma\mathrm{signs}(k)$ is uniquely
determined by the collection of all $b$-vectors:
\[
\mathcal{B}(k) = \bigcup_{\ell=2}^{\omega} \{ b_T : T \subseteq \{p_1, \ldots, p_\omega\}, |T| = \ell \}
\]
That is, we need $b$-vectors at levels $2, 3, \ldots, \omega$.
\end{theorem}

\begin{remark}[Complexity Growth]
The total number of bits in $\mathcal{B}(k)$ is:
\[
\sum_{\ell=2}^{\omega} \ell \binom{\omega}{\ell} = \omega \cdot 2^{\omega-1}
\]
This grows exponentially with the number of prime factors.
\end{remark}

\subsection{Comparison with Permutation Signs}

The structure bears resemblance to permutation theory, but is richer.

\begin{center}
\begin{tabular}{l|l|l}
& Permutations & Chebyshev Sign Sums \\
\hline
Object & $\sigma \in S_n$ & $k = p_1 \cdots p_\omega$ \\
Sign & $(-1)^{\#\text{inversions}}$ & $\Sigma\mathrm{signs}(k)$ \\
Structure & Single level (pairs) & Hierarchical (levels $2$ to $\omega$) \\
Complexity & $O(n^2)$ & $O(\omega \cdot 2^\omega)$ \\
\end{tabular}
\end{center}

For $\omega \leq 3$, the hierarchical structure collapses to a simple formula
(Theorem~\ref{thm:closed-form}). For $\omega \geq 4$, the full hierarchy is required.

\begin{theorem}[Recursive Structure]\label{thm:recursive}
For $\omega \geq 3$, the sign sum satisfies an inclusion-exclusion formula:
\[
\Sigma\mathrm{signs}(p_1 \cdots p_\omega) = \sum_{|T|=\omega-1} \Sigma\mathrm{signs}(T)
- \sum_{|T|=\omega-2} \Sigma\mathrm{signs}(T) + \cdots + (-1)^{\omega} \binom{\omega}{2} + C_\omega
\]
where $C_\omega$ is a \emph{correction term} that depends on the full hierarchical pattern.
\end{theorem}

\begin{proof}[Proof sketch]
The primitive lobes of $k = p_1 \cdots p_\omega$ satisfy $\gcd(n-1, k) = \gcd(n, k) = 1$.
By inclusion-exclusion on the prime factors, the count of such lobes relates to
counts for subproducts. However, the \emph{parity} of these lobes (odd vs.\ even $n$)
introduces additional structure via CRT reconstruction, captured by $C_\omega$.
\end{proof}

\subsection{Explicit Formula for $\omega = 4$}

For four prime factors, the recursive formula becomes explicit.

\begin{theorem}[Four Prime Factors]\label{thm:omega4}
For $k = p_1 p_2 p_3 p_4$ with distinct odd primes:
\[
\Sigma\mathrm{signs}(k) = \sum_{\text{triples}} \Sigma\mathrm{signs} - \sum_{\text{pairs}} \Sigma\mathrm{signs} + 6 + C_4
\]
where:
\begin{itemize}
\item $\sum_{\text{triples}} = \Sigma\mathrm{signs}(p_1p_2p_3) + \Sigma\mathrm{signs}(p_1p_2p_4)
      + \Sigma\mathrm{signs}(p_1p_3p_4) + \Sigma\mathrm{signs}(p_2p_3p_4)$
\item $\sum_{\text{pairs}} = \sum_{i<j} \Sigma\mathrm{signs}(p_i p_j)$ (6 terms)
\item $C_4 \in \{-24, -20, -16, -12, -8, -4, 0, 4\}$
\end{itemize}
\end{theorem}

\begin{remark}[Correction Values]
The correction $C_4$ takes 8 distinct values, all divisible by 4.
Combined with the recursive part, the final result satisfies
$\Sigma\mathrm{signs}(k) \equiv 1 \pmod{4}$ as required by Theorem~\ref{thm:congruence}.
\end{remark}

\begin{proposition}[Pattern Structure for $\omega = 4$]\label{prop:omega4-pattern}
The hierarchical $b$-pattern for $\omega = 4$ is:
\[
\mathcal{B}_4 = \left( \{b^{(2)}_{ij}\}_{i<j}, \{b^{(3)}_T\}_{|T|=3}, b^{(4)} \right)
\]
where:
\begin{itemize}
\item $b^{(2)}_{ij}$ are the 6 pairwise $b$-vectors (each has 2 components, total 12 bits)
\item $b^{(3)}_T$ are the 4 triple $b$-vectors (each has 3 components)
\item $b^{(4)}$ is the full 4-component $b$-vector for $k$
\end{itemize}
\end{proposition}

\begin{remark}[Pattern Does NOT Determine $\Sigma\mathrm{signs}$]
\textbf{Retracted claim:} An earlier version claimed that $\mathcal{B}_4$ uniquely
determines $\Sigma\mathrm{signs}(k)$. This was based on testing 275 products where
most patterns had only one representative.

When tested on patterns with \emph{multiple} representatives (56 patterns with
$\geq 2$ products each), \textbf{47 showed conflicts}---products with identical
$\mathcal{B}_4$ patterns but different $\Sigma\mathrm{signs}$ values.

The pattern is \emph{necessary but not sufficient}. Additional structure
(possibly involving the actual prime values, not just their residue classes)
is required for $\omega \geq 4$.
\end{remark}

\begin{remark}[Open Problem for $\omega \geq 4$]
For $\omega = 3$, the formula depends only on $\sum b_2^{\text{pairs}} - \sum b^{\text{triple}}$,
a single scalar. For $\omega \geq 4$, neither the hierarchical $b$-pattern nor
any simple function of it determines $\Sigma\mathrm{signs}$.

The fundamental question remains: \emph{what additional structure is needed?}
\end{remark}

\subsection{CRT Carry Structure}

The increasing complexity for $\omega \geq 4$ has a structural explanation.

\begin{definition}[CRT Reconstruction]
For $n \in \{1, \ldots, k-1\}$ with $k = p_1 \cdots p_\omega$, the CRT gives:
\[
n \equiv \sum_{i=1}^{\omega} a_i \cdot c_i \pmod{k}
\]
where $a_i = n \bmod p_i$ and $c_i = M_i \cdot (M_i^{-1} \bmod p_i)$ with $M_i = k/p_i$.
\end{definition}

\begin{proposition}[Parity of CRT Reconstruction]
The parity of $n$ satisfies:
\[
n \bmod 2 \equiv \sum_{i=1}^{\omega} a_i \cdot b_i \pmod{2}
\]
where $b_i = (M_i^{-1} \bmod p_i) \bmod 2$.
\end{proposition}

\begin{proof}
Since $M_i$ is a product of odd primes, $M_i \equiv 1 \pmod{2}$.
Thus $c_i \bmod 2 = (M_i^{-1} \bmod p_i) \bmod 2 = b_i$.
\end{proof}

\begin{remark}[Carries in CRT]
When computing $n = \sum a_i c_i$ over $\mathbb{Z}$ (not mod $k$), there are
\emph{carries} that reduce to give the result in $\{0, \ldots, k-1\}$.

For $\omega \leq 3$: the carry structure is simple enough that total parity
depends only on $\sum b_2^{\text{pairs}} - \sum b^{\text{triple}}$.

For $\omega \geq 4$: carries at different ``levels'' interact in ways that
require tracking the full hierarchical pattern $\mathcal{B}(k)$.
\end{remark}

\begin{theorem}[Complexity Threshold at $\omega = 4$]\label{thm:threshold}
The transition from closed form ($\omega \leq 3$) to unknown structure
($\omega \geq 4$) is sharp:
\begin{center}
\begin{tabular}{c|c|c|c}
$\omega$ & Closed form? & Determining structure & Verification \\
\hline
2 & Yes: $4b_2 - 3$ & 1 bit ($b_2$) & 301 cases \\
3 & Yes: $-1 + 4(\Sigma b_2^{\text{p}} - \Sigma b^{\text{t}})$ & 1 scalar & 759 cases \\
4 & \textbf{Unknown} & $b$-pattern insufficient & conflicts found \\
5+ & \textbf{Unknown} & Open & --- \\
\end{tabular}
\end{center}
\end{theorem}

\begin{remark}[Falsified Conjecture]
An earlier conjecture proposed that a function $F: \{0,1\}^{22} \to \mathbb{Z}$
exists such that $\Sigma\mathrm{signs}(p_1 p_2 p_3 p_4) = F(\mathcal{B}_4)$.
This has been \textbf{falsified}: products with identical $\mathcal{B}_4$ patterns
can have different $\Sigma\mathrm{signs}$ values.

The determining structure for $\omega \geq 4$ remains an open problem.
\end{remark}

\section{Connection to Primitive Root Structure}

The semiprime formula (Theorem~\ref{thm:semiprime-signs}) shows that
$\Sigma\mathrm{signs}(pq) = 4b_2 - 3$ where $b_2 = (p^{-1} \bmod q) \bmod 2$.
A natural question: \emph{why does this parity matter?}

\subsection{Inverse Parity Bias}

For a prime $q > 2$ and primitive root $g$ modulo $q$, consider the
distribution of parities among powers $g^k$ for $k = 0, 1, \ldots, q-2$.

\begin{definition}[Inverse Parity Bias]
Define the bias:
\[
\Delta(q) = P(g^k \text{ even} \mid k \text{ odd}) - P(g^k \text{ even} \mid k \text{ even})
\]
\end{definition}

\begin{theorem}[Inverse Parity Bias Theorem]\label{thm:inverse-parity}
For prime $q > 2$:
\begin{enumerate}
\item $\Delta(q) = 0 \iff q \equiv 1 \pmod{4} \iff \left(\frac{-1}{q}\right) = +1$
\item For $q \equiv 3 \pmod{4}$: $\mathrm{sign}(\Delta) = -\left(\frac{2}{q}\right)$
\end{enumerate}
In particular:
\begin{itemize}
\item $q \equiv 3 \pmod{8}$: $\Delta > 0$
\item $q \equiv 7 \pmod{8}$: $\Delta < 0$
\end{itemize}
\end{theorem}

\begin{proof}[Proof sketch]
The key is the involution $x \mapsto -x$ on $\mathbb{Z}_q^*$.
\begin{enumerate}
\item This pairs $g^k$ with $g^{k + (q-1)/2}$ since $g^{(q-1)/2} \equiv -1 \pmod{q}$.
\item These paired values have opposite parity (since $q$ is odd).
\item When $(q-1)/2$ is even ($q \equiv 1 \pmod{4}$): the exponents $k$ and
$k + (q-1)/2$ have the same parity, creating balance, so $\Delta = 0$.
\item When $(q-1)/2$ is odd ($q \equiv 3 \pmod{4}$): the exponents have opposite
parity, breaking the balance.
\item The sign is determined by $\mathrm{ind}_g(2)$: since
$\left(\frac{2}{q}\right) = (-1)^{\mathrm{ind}_g(2)}$, the location of 2
(the smallest even) in the primitive root sequence determines which
parity class has more even values. \qedhere
\end{enumerate}
\end{proof}

\begin{remark}[Numerical Verification]
This theorem has been verified for all primes $5 \leq q \leq 383$ with
100\% agreement.
\end{remark}

\subsection{Asymptotic Behavior}

\begin{proposition}[Asymptotic of Bias]
For $q \equiv 3 \pmod{4}$:
\[
|\Delta(q)| \sim \frac{c}{\sqrt{q}}
\]
with $c \approx 1.5$. The exponent $-0.49$ from numerical fitting is
consistent with the theoretical $-1/2$.
\end{proposition}

This scaling is consistent with a random walk argument: $(q-1)/2$ steps
with standard deviation $\sim \sqrt{q}$, normalized by the total count.

\subsection{Connection to Sign Formula}

For semiprime $k = pq$, recall that $\Sigma\mathrm{signs}(pq) = 4b_2 - 3$
where $b_2 = (p^{-1} \bmod q) \bmod 2$.

The Inverse Parity Bias Theorem explains \emph{why} this parity matters:
\begin{itemize}
\item Write $p = g^a$ for primitive root $g \bmod q$
\item Then $p^{-1} = g^{q-1-a}$, so $b_2 = (g^{q-1-a} \bmod 2)$
\item Since $q-1$ is even, parity of $a$ equals parity of $q-1-a$
\item The correlation between $b_2$ and properties of $p$ (like $(q|p)$)
arises from the bias $\Delta(q)$ in the primitive root structure
\end{itemize}

\subsection{Connection to Lehmer Numbers}

The inverse parity structure connects to classical number theory via
\emph{D.\,H.\ Lehmer numbers}.

\begin{definition}[Lehmer Number]
An integer $a \in \{1, \ldots, p-1\}$ is a \emph{Lehmer number} modulo
prime $p$ if $a$ and $a^{-1} \bmod p$ have opposite parity (equivalently,
$a + a^{-1}$ is odd).
\end{definition}

\begin{theorem}[Cohen--Trudgian \cite{cohen-trudgian}]
For all primes $p \neq 2, 3, 7$, there exist Lehmer numbers that are
also primitive roots. The count $L(p)$ of Lehmer numbers satisfies
$L(p) \approx (p-1)/2$ with corrections involving Kloosterman sums.
\end{theorem}

The cases $p = 3, 7$ are exceptional: $L(3) = L(7) = 0$.

Both our Inverse Parity Bias Theorem and Lehmer number theory depend
on the same fundamental quantity: the parity of $2^{-1} \bmod p = (p+1)/2$,
which is even iff $p \equiv 3 \pmod{4}$.

\subsection{Mod 8 Unification}

The structure we have uncovered mirrors earlier results on the
Pell equation $x^2 - py^2 = 1$~\cite{pell-mod8}.

\begin{center}
\begin{tabular}{l|c|c|c}
Phenomenon & Condition & $\left(\frac{2}{p}\right) = -1$ & $\left(\frac{2}{p}\right) = +1$ \\
\hline
Pell $x_0 \bmod p$ & $\left(\frac{-1}{p}\right) = -1$ & $x_0 \equiv -1$ & $x_0 \equiv +1$ \\
Inverse Parity $\Delta(q)$ & $\left(\frac{-1}{q}\right) = -1$ & $\Delta > 0$ & $\Delta < 0$ \\
\end{tabular}
\end{center}

Both phenomena share the same structure:
\begin{itemize}
\item When $\left(\frac{-1}{p}\right) = +1$ (i.e., $p \equiv 1 \pmod{4}$): neutral behavior
\item When $\left(\frac{-1}{p}\right) = -1$ (i.e., $p \equiv 3 \pmod{4}$): sign/direction determined by $\left(\frac{2}{p}\right)$
\end{itemize}

This suggests a deep connection between Chebyshev lobe structure, Pell
equations, and quadratic reciprocity.

\section{Open Questions}

\begin{enumerate}
\item \textbf{Deeper structure of sign formulas:}
The Inverse Parity Bias Theorem explains the role of modular inverses
via primitive roots. Is there a direct representation-theoretic explanation
connecting Chebyshev polynomials to this structure?

\item \textbf{Closed form for $\omega \geq 4$:}
For $\omega = 3$, we have $\Sigma\mathrm{signs} = -1 + 4(\Sigma b_2^{\text{pairs}} - \Sigma b^{\text{triple}})$.
Is there an analogous closed form for $\omega \geq 4$, or is the full
hierarchical $b$-pattern irreducibly complex?

\item \textbf{Generating function:}
Can the hierarchical $b$-pattern structure be encoded in a generating
function or algebraic object that unifies all $\omega$?

\item \textbf{Geometric interpretation:}
The invariant \eqref{eq:invariant} says total area is conserved. The primality
theorem says factorization redistributes area from primitive to inherited lobes.
Is there a deeper geometric or physical interpretation?
\end{enumerate}

\begin{thebibliography}{9}
\bibitem{chebyshev-invariant}
J.~Popelka,
\emph{The $1/\pi$ Invariant in Chebyshev Polynomial Geometry},
2025.

\bibitem{mason}
J.\,C.~Mason and D.\,C.~Handscomb,
\emph{Chebyshev Polynomials},
Chapman \& Hall/CRC, 2003.

\bibitem{cohen-trudgian}
S.\,D.~Cohen and T.~Trudgian,
\emph{Lehmer numbers and primitive roots modulo a prime},
J.\ Number Theory \textbf{203} (2019), 68--79.
arXiv:1712.03990.

\bibitem{zhang}
W.\,P.~Zhang,
\emph{On a problem of D.\,H.\ Lehmer and Kloosterman's sums},
Monatsh.\ Math.\ \textbf{139} (2003), 247--257.

\bibitem{pell-mod8}
J.~Popelka,
\emph{Pell equation fundamental solutions modulo $p$},
Working notes, 2025.
(Structure of $x_0 \bmod p$ by $p \bmod 8$.)

\bibitem{ireland-rosen}
K.~Ireland and M.~Rosen,
\emph{A Classical Introduction to Modern Number Theory},
2nd ed., Springer, 1990.
\end{thebibliography}

\end{document}
