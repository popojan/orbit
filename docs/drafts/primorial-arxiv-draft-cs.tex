\documentclass[11pt]{article}
\usepackage[margin=1in]{geometry}
\usepackage[czech]{babel}
\usepackage[T1]{fontenc}
\usepackage[utf8]{inputenc}
\usepackage{amsmath,amssymb,amsthm}
\usepackage[colorlinks=true,linkcolor=blue,citecolor=blue,urlcolor=blue]{hyperref}

\newtheorem{theorem}{Věta}
\newtheorem{conjecture}[theorem]{Hypotéza}
\newtheorem{proposition}[theorem]{Tvrzení}
\theoremstyle{definition}
\newtheorem{definition}[theorem]{Definice}
\newtheorem{example}[theorem]{Příklad}
\theoremstyle{remark}
\newtheorem{remark}[theorem]{Poznámka}

\title{Vzorec pro primoriál pomocí střídavých faktoriálových součtů}
\author{Jan Geuss Popelka\thanks{Nezávislý výzkumník. Email: popojan@protonmail.com. Kód: \url{https://github.com/popojan/orbit}}}
\date{\today}

\begin{document}

\maketitle

\begin{abstract}
Představujeme výpočetní objev: primoriál $p\#$ (součin všech prvočísel až do $p$) lze získat jako jmenovatel určitého střídavého faktoriálového součtu. Pro $m \geq 3$ platí
\[
\text{den}\left(\frac{1}{2} \sum_{k=1}^{\lfloor(m-1)/2\rfloor} \frac{(-1)^k \cdot k!}{2k+1}\right) = \prod_{\substack{p \text{ prvočíslo} \\ p \leq m}} p
\]
kde $\text{den}(q)$ označuje jmenovatel racionálního čísla $q$ v základním tvaru. Vzorec byl výpočetně ověřen pro všechna $m$ až do 100\,000, přesto však jeho rigorózní důkaz zatím chybí. Ústředním tajemstvím je systematické krácení: ačkoliv jednotlivé jmenovatele $\{2k+1\}$ obsahují mocniny prvočísel ($9 = 3^2$, $25 = 5^2$, $27 = 3^3$, \ldots), výsledný jmenovatel po zkrácení obsahuje pouze první mocniny prvočísel. Tento jev formulujeme jako otevřený problém týkající se $p$-adických valuací a diskutujeme možné teoretické souvislosti.
\end{abstract}

\section{Úvod}

Funkce primoriál, definovaná jako součin všech prvočísel do dané hranice, se v teorii čísel objevuje zcela přirozeně – od odhadů v analytické teorii čísel po konstrukci vysoce složených čísel. Primoriály se obvykle počítají přímým násobením, my jsme však objevili překvapivou alternativu: vystupují jako jmenovatele střídavých faktoriálových součtů.

Tento objev vzešel z výpočetních experimentů s egyptskými zlomky a faktoriálovými řadami. Při zkoumání různých vzorců pro racionální součty jsme si všimli, že některé střídavé součty mají jmenovatele s pozoruhodnou vlastností – po zkrácení na základní tvar obsahují výhradně součiny různých prvočísel. Systematické testování pak vedlo k vzorci, který zde prezentujeme.

Samotný vzorec je jednoduchý a snadno ověřitelný výpočtem, ale pochopit \emph{proč} funguje, vyžaduje objasnění subtilního mechanismu krácení. Čitatelé, vytvořené z faktoriálů, totiž systematicky odstraňují všechny vyšší mocniny prvočísel z naivně očekávaného nejmenšího společného násobku jmenovatelů. V tomto článku vzorec prezentujeme, uvádíme výpočetní důkazy a formulujeme problém krácení jako otevřenou matematickou otázku.

\section{Vzorec}

\begin{theorem}[Výpočetní]
Pro libovolné celé číslo $m \geq 3$ definujeme racionální číslo
\begin{equation}\label{eq:main}
S_m = \frac{1}{2} \sum_{k=1}^{h} \frac{(-1)^k \cdot k!}{2k+1}
\end{equation}
kde $h = \lfloor(m-1)/2\rfloor$. Pak jmenovatel $S_m$ v základním tvaru je roven $m$-primoriálu:
\[
\text{den}(S_m) = \prod_{\substack{p \text{ prvočíslo} \\ p \leq m}} p
\]
\end{theorem}

\begin{remark}
Případ $m = 2$ vyžaduje speciální přístup a dává přímo hodnotu $2$. Pro $m \geq 3$ platí vzorec univerzálně.
\end{remark}

\begin{example}
Pro $m = 13$ dostáváme $h = 6$ a vypočítáme:
\begin{align*}
S_{13} &= \frac{1}{2}\left(-\frac{1!}{3} + \frac{2!}{5} - \frac{3!}{7} + \frac{4!}{9} - \frac{5!}{11} + \frac{6!}{13}\right)\\
&= \frac{695971}{30030}
\end{align*}
Jmenovatel $30030 = 2 \times 3 \times 5 \times 7 \times 11 \times 13$ je přesně roven součinu všech prvočísel až do 13.
\end{example}

\subsection{Struktura objeveného vzoru}

Klíčové pozorování spočívá v tom, že jmenovatel roste postupným přidáváním nových prvočíselných faktorů:

\begin{proposition}
Označme $S_k = \frac{1}{2} \sum_{i=1}^{k} \frac{(-1)^i \cdot i!}{2i+1}$ částečný součet. Potom
\[
\text{den}(S_k) = 2 \times \prod_{\substack{p \text{ prvočíslo} \\ 3 \leq p \leq 2k+1}} p
\]
\end{proposition}

Jmenovatelé zůstávají konstantní, pokud je $2k+1$ složené číslo. Nová prvočísla přibývají pouze tehdy, je-li $2k+1$ samo prvočíslem. Tabulka~\ref{tab:example} ukazuje tento jev pro $m = 13$:

\begin{table}[h]
\centering
\begin{tabular}{c|c|c|c}
$k$ & $2k+1$ & Prvočíslo? & $\text{den}(S_k)$ \\ \hline
1 & 3 & \checkmark & $2 \times 3$ \\
2 & 5 & \checkmark & $2 \times 3 \times 5$ \\
3 & 7 & \checkmark & $2 \times 3 \times 5 \times 7$ \\
4 & 9 & & $2 \times 3 \times 5 \times 7$ \\
5 & 11 & \checkmark & $2 \times 3 \times 5 \times 7 \times 11$ \\
6 & 13 & \checkmark & $2 \times 3 \times 5 \times 7 \times 11 \times 13$
\end{tabular}
\caption{Růst jmenovatele pro $m = 13$. Povšimněte si, že při $k = 4$ zůstává jmenovatel nezměněn, přestože $9 = 3^2$.}
\label{tab:example}
\end{table}

\section{Problém krácení}

Centrální záhada spočívá v následujícím: posloupnost $\{2k+1\}_{k=1}^{h}$ obsahuje mocniny prvočísel i složená čísla:
\[
3, 5, 7, 9=3^2, 11, 13, 15=3 \times 5, 17, 19, 21=3 \times 7, 23, 25=5^2, 27=3^3, \ldots
\]

Při naivním výpočtu $\text{NSN}(3, 5, 7, 9, 11, \ldots)$ bychom očekávali výskyt $3^2$ kvůli 9, $3^3$ kvůli 27, $5^2$ kvůli 25, a tak dále. Přesto po sečtení výrazu (\ref{eq:main}) a zkrácení na základní tvar obsahuje jmenovatel \emph{pouze první mocniny} prvočísel.

\subsection{Formulace pomocí p-adických valuací}

Označme $\nu_p(n)$ $p$-adickou valuaci čísla $n$ (exponent prvočísla $p$ v prvočíselném rozkladu $n$). Naše výpočetní zkoumání ukazuje:

\begin{conjecture}\label{conj:main}
Pro všechna prvočísla $p$ splňující $3 \leq p \leq 2k+1$ platí
\[
\nu_p\left(\text{den}(S_k)\right) = 1
\]
a zároveň
\[
\nu_p\left(\text{čit}(S_k)\right) = 0
\]
kde $\text{čit}(q)$ označuje čitatel čísla $q$ v základním tvaru.
\end{conjecture}

V tom spočívá jádro problému: dokázat, že čitatelé obsahují právě takové prvočíselné faktory, které při redukci největším společným dělitelem odstraní všechny mocniny $p^j$ pro $j > 1$.

\subsection{Dva mechanismy krácení}

Výpočetní analýza odhaluje dva různé režimy chování:

\textbf{Malá $k$ (krácení pomocí NSD):} Pokud platí $\nu_p(k!) < \nu_p(2k+1)$ pro nějaké prvočíslo $p$ dělící $2k+1$, pak kombinovaný čitatel po sečtení obsahuje faktory $p$, přičemž redukce pomocí NSD odstraní právě nadbytečné mocniny.

\textbf{Příklad:} Pro $k = 4$ máme $2k+1 = 9 = 3^2$, ale $\nu_3(4!) = 1 < 2$. Čitatel po zkombinování s předchozími členy obsahuje právě jeden faktor 3, což vede k redukci $3^2 \to 3^1$ ve jmenovateli.

\textbf{Velká $k$ (celočíselné členy):} Pro dostatečně velká $k$ zaručuje Legendreův vzorec
\[
\nu_p(k!) = \sum_{i=1}^{\infty} \left\lfloor \frac{k}{p^i} \right\rfloor
\]
že platí $\nu_p(k!) \geq \nu_p(2k+1)$ pro všechna $p | (2k+1)$. V těchto případech je zlomek $\frac{k!}{2k+1}$ již \emph{celé číslo} a nepřidává žádné nové faktory do jmenovatele.

\textbf{Příklad:} Pro $k = 12$ máme $2k+1 = 25 = 5^2$ a $\nu_5(12!) = 2 \geq 2$, takže příslušný člen je celé číslo.

\subsection{Role střídavého znaménka}

Střídavý faktor $(-1)^k$ je naprosto zásadní. Bez něj vzorec při $k = 4$ ztratí faktor 3 a již ho nikdy nezíská zpět, takže pro všechna $m \geq 9$ dává výsledek $\text{Primoriál}/3$. Střídání znamének ovlivňuje strukturu čitatele tak, že zabraňuje nadměrnému krácení v kritických bodech.

\section{Výpočetní ověření}

Vzorec jsme vyčerpávajícím způsobem ověřili pro všechna $m$ od 3 do 100\,000, přičemž jsme testovali každou celočíselnou hodnotu (složenou i prvočíselnou). Pro $m = 100{,}000$ má primoriál 43\,293 cifer; všechny jmenovatele se přesně shodují.

\subsection{Iterativní formulace}

Pro efektivní ověření ve velkém měřítku jsme odvodili iterativní formulaci. Z přímého součtu (\ref{eq:main}) lze částečné součty počítat pomocí tříčlenné rekurence:

\begin{align}
S_0 &= \{0, 0, 1\}\\
S_{n+1} &= \{n+1, b_n, b_n + (a_n - b_n)\cdot\left(n + \frac{1}{3+2n}\right)\}
\end{align}

kde stav je reprezentován jako $S_n = \{n, a_n, b_n\}$. Po $h = \lfloor(m-1)/2\rfloor$ iteracích získáme primoriál jako:
\[
\text{Primoriál}(m) = 2 \cdot \text{den}(b_h - 1)
\]

Ekvivalenci mezi přímým součtem (\ref{eq:main}) a touto rekurencí lze dokázat indukcí; rekurence sleduje veličinu $b_n = 1 + 2S_n$ a faktor 2 se extrahuje ve finálním vzorci. To umožňuje provést výpočet v $O(m)$ aritmetických operacích a značně usnadňuje ověření pro velké hodnoty. Tabulka~\ref{tab:verification} shrnuje klíčové kontrolní body:

\begin{table}[h]
\centering
\begin{tabular}{r|r|r}
$m$ & $\pi(m)$ & Počet cifer primoriálu \\ \hline
100 & 25 & 37 \\
1{,}000 & 168 & 416 \\
10{,}000 & 1{,}229 & 3{,}393 \\
100{,}000 & 9{,}592 & 43{,}293
\end{tabular}
\caption{Kontrolní body ověření. Všechny mezilehlé hodnoty byly rovněž testovány (každé $m$ od 3 výše).}
\label{tab:verification}
\end{table}

\section{Otevřené otázky}

Přestože je vzorec výpočetně ověřen pro obrovský rozsah hodnot, zůstává několik teoretických otázek nezodpovězených:

\begin{enumerate}
\item \textbf{Rigorózní důkaz:} Dokázat Hypotézu~\ref{conj:main} pomocí $p$-adické analýzy nebo jiných matematických metod.

\item \textbf{Souvislost s vytvořující funkcí:} Má tento součet interpretaci jako vytvořující funkce pro primoriály? Struktura naznačuje existenci hlubšího kombinatorického či analytického rámce.

\item \textbf{Zobecnění:} Co se stane při modifikaci vzorce? Například:
\begin{itemize}
\item Změna koeficientu: $\frac{1}{2} \to \frac{1}{c}$ pro jiné konstanty $c$
\item Úprava jmenovatelů: $2k+1 \to ak+b$ pro jiné aritmetické posloupnosti
\item Změna čitatelů: $k! \to (2k)!$ nebo jiné faktoriálové výrazy
\end{itemize}

\item \textbf{Souvislost s klasickými větami:} Existuje spojení s Wilsonovou větou, Wolstenholmeovou větou nebo s jmenovateli harmonických čísel (které rovněž obsahují primoriály)?

\item \textbf{Výpočetní složitost:} Jaká je složitost tohoto vzorce ve srovnání s přímým výpočtem primoriálu? Iterativní formulace by mohla sloužit jako paměťově efektivní metoda pro generování prvočísel bez explicitního testování primality.
\end{enumerate}

\section{Závěr}

Představili jsme výpočetně ověřený vzorec, který vyjadřuje primoriály jako jmenovatele střídavých faktoriálových součtů. Systematické krácení vyšších mocnin prvočísel zůstává nevysvětlené a naznačuje existenci hlubokých číselně-teoretických struktur. Problém se zdá být vhodný pro formální důkazové asistenty a mohl by těžit z metod $p$-adické analýzy, teorie vytvořujících funkcí nebo modulární aritmetiky.

Věříme, že tento objev podnítí další teoretický výzkum. Budeme rádi za jakoukoli spolupráci směřující k důkazu Hypotézy~\ref{conj:main} nebo k prozkoumání širších souvislostí zde naznačených.

\section*{Poděkování}

Tato práce vznikla při výpočetních experimentech v prostředí Wolfram Language. Děkuji online matematické komunitě za cenné diskuse a připomínky k raným verzím tohoto výsledku.

\begin{thebibliography}{9}

\bibitem{legendre}
A.-M. Legendre,
\emph{Th\'eorie des nombres},
Firmin Didot Fr\`eres, Paris, 1830.

\bibitem{oeis}
OEIS Foundation Inc.,
The On-Line Encyclopedia of Integer Sequences,
\url{https://oeis.org}.
Relevantní posloupnosti: A002110 (primoriály), A034386 (primoriály prvočísel).

\end{thebibliography}

\end{document}