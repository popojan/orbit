\documentclass[11pt]{article}
\usepackage{amsmath,amssymb,amsthm}
\usepackage[margin=1in]{geometry}
\usepackage{hyperref}

\newtheorem{theorem}{Theorem}
\newtheorem{proposition}{Proposition}
\newtheorem{definition}{Definition}
\newtheorem{remark}{Remark}

\title{Chebyshev Polynomials as Rational Twist Algebra:\\
A Bridge to Wildberger's Program}

\author{Jan Popelka \and Claude (Anthropic)}

\date{December 1, 2025\\
\small Draft -- Not peer-reviewed}

\begin{document}

\maketitle

\begin{abstract}
We observe that Chebyshev polynomials provide a natural algebraic framework
for encoding rotational structure without explicit reference to $\pi$.
The ``twist count'' $n$ (an integer) replaces the continuous angle $\theta$
(which requires $\pi$), while the composition property $T_m \circ T_n = T_{mn}$
captures the multiplicative structure of combined rotations.
We show that the total normalized area of Chebyshev polygon lobes equals
exactly 1 for all $n \geq 3$, giving a rational area ratio of $1/4$
relative to the bounding square. This suggests Chebyshev polynomials
may serve as a bridge between classical trigonometry and Wildberger's
rational geometry program, recovering orientation/twist within a
purely algebraic framework.
\end{abstract}

\section{Motivation: Local vs Global Orientation}

Norman Wildberger's \emph{Rational Trigonometry} \cite{wildberger2005}
replaces the classical notions of distance and angle with:
\begin{itemize}
    \item \textbf{Quadrance}: $Q = d^2$ (square of distance, avoiding $\sqrt{\cdot}$)
    \item \textbf{Spread}: $s = \sin^2\theta$ (avoiding $\pi$ in angle measurement)
\end{itemize}

This framework successfully avoids transcendental numbers in many geometric
computations. Wildberger also introduces an oriented version: the \textbf{turn}
$u(l_1, l_2) = \tan\theta$ between lines, with the property that
$\text{twist} = \text{turn}^2$ and $u(l_1, l_2) = -u(l_2, l_1)$.
The Triple Turn Formula $u_1 + u_2 + u_3 = u_1 u_2 u_3$ elegantly captures
local angular relationships.

However, Wildberger's turn measures \emph{local} orientation---the angular
displacement between two specific lines. It does not capture \emph{global}
orientation: how many times one has ``gone around.'' Going around the circle
twice gives the same turn as going around once.

\textbf{Question:} Can we encode the \emph{winding number}---how many complete
cycles---within a rational/algebraic framework?

\section{Chebyshev Polynomials as Twist Encoding}

\subsection{The Classical Definition}

The Chebyshev polynomials of the first kind are defined by:
\begin{equation}
    T_n(\cos\theta) = \cos(n\theta)
\end{equation}

This definition explicitly uses $\pi$ (via $\theta$). However, the
polynomials themselves are purely algebraic:
\begin{align}
    T_0(x) &= 1 \\
    T_1(x) &= x \\
    T_{n+1}(x) &= 2x \cdot T_n(x) - T_{n-1}(x)
\end{align}

\subsection{The Key Insight: Integer Twist Count}

Instead of measuring \emph{how much} rotation (angle $\theta \in [0, 2\pi)$),
we measure \emph{how many} twists (integer $n \in \mathbb{Z}^+$):

\begin{center}
\begin{tabular}{lll}
\textbf{Framework} & \textbf{Rotation measure} & \textbf{Needs $\pi$?} \\
\hline
Classical & angle $\theta \in [0, 2\pi)$ & Yes \\
Wildberger & spread $s \in [0, 1]$ & No, but loses orientation \\
Chebyshev & twist count $n \in \mathbb{Z}^+$ & No, preserves ``how many times''
\end{tabular}
\end{center}

\subsection{Composition Property}

The Chebyshev composition identity:
\begin{equation}
    T_m(T_n(x)) = T_{mn}(x)
\end{equation}
encodes the multiplicative structure of rotations:
\begin{quote}
``$m$ groups of $n$ twists equals $mn$ twists total.''
\end{quote}

This is the algebraic shadow of the angle addition formula,
expressed without any reference to $\pi$.

\section{The Chebyshev Polygon and Lobe Areas}

\subsection{Definition}

\begin{definition}[Chebyshev Polygon \cite{chebyshev-integral}]
For $n \geq 3$, the \emph{Chebyshev polygon} is the curve defined by
\begin{equation}
    L_n(x) = T_{n+1}(x) - x \cdot T_n(x), \quad x \in [-1, 1]
\end{equation}
This function creates $n$ lobes between consecutive roots of $T_n$,
with vertices on the unit circle.
\end{definition}

\subsection{Lobe Area Identity}

\begin{theorem}[Lobe Sum Identity]
For $n \geq 3$, let $A(n,k)$ denote the normalized area of the $k$-th
lobe of the Chebyshev polygon. Then:
\begin{equation}
    \sum_{k=1}^{n} A(n,k) = 1
\end{equation}
\end{theorem}

\begin{proof}[Proof sketch]
The lobe area formula involves terms $\cos(2\pi k/n)$, which are
algebraic numbers (real parts of $n$-th roots of unity). The sum
over all $k$ invokes the root sum identity:
\[
    \sum_{k=0}^{n-1} e^{2\pi i k/n} = 0
\]
The oscillatory terms cancel, leaving only the constant contribution,
which sums to exactly 1.
\end{proof}

\subsection{Rational Area Ratio}

\begin{proposition}[Rational Ratio]
The Chebyshev polygon area, relative to natural reference regions,
gives rational ratios:
\begin{align}
    \frac{\text{Polygon area}}{\text{Circumscribed square area}}
        &= \frac{1}{4} \\
    \frac{\text{Polygon area}}{\text{Inscribed square area}}
        &= \frac{1}{2}
\end{align}
\end{proposition}

\begin{remark}
The circumscribed square (bounding box) has vertices at $(\pm 1, \pm 1)$
and area 4. No $\sqrt{2}$ appears in either the coordinates or the area.
The ratio $1/4$ is purely rational.
\end{remark}

\section{The Bridge to Rational Geometry}

\subsection{What Chebyshev Provides}

\begin{enumerate}
    \item \textbf{Discrete twist count} $n$ replaces continuous angle $\theta$
    \item \textbf{Algebraic polynomials} $T_n(x)$ encode rotational structure
    \item \textbf{Composition law} $T_m \circ T_n = T_{mn}$ preserves multiplicativity
    \item \textbf{Rational invariant} (area ratio $1/4$) independent of $n$
    \item \textbf{Individual areas} are algebraic (roots of unity), sum is integer
\end{enumerate}

\subsection{Where $\pi$ Hides}

We are honest about where $\pi$ remains:

\begin{itemize}
    \item \textbf{Motivation}: The connection $T_n(\cos\theta) = \cos(n\theta)$
          explains \emph{why} these polynomials encode rotation.
    \item \textbf{Notation}: Writing $\cos(2\pi k/n)$ is shorthand for
          $\text{Re}(\zeta_n^k)$ where $\zeta_n$ is a primitive $n$-th
          root of unity.
    \item \textbf{Circle reference}: Comparing to disk area gives ratio
          $1/\pi$ (transcendental). We avoid this by using the square.
\end{itemize}

The key observation: $\pi$ is needed to \emph{ask} the question
(why Chebyshev?), not to \emph{answer} it (what is the area ratio?).

\section{Comparison: Turn vs Twist Count}

Wildberger's \textbf{turn} and our Chebyshev \textbf{twist count} address
different aspects of orientation:

\begin{center}
\begin{tabular}{lll}
\textbf{Aspect} & \textbf{Wildberger's Turn} & \textbf{Chebyshev $n$} \\
\hline
Type & Rational number $u \in \mathbb{Q}$ & Integer $n \in \mathbb{Z}^+$ \\
Measures & Local angle between lines & Global winding count \\
``Go around once'' & $u = \tan\theta$ & $n = 1$ \\
``Go around twice'' & $u = \tan\theta$ (same!) & $n = 2$ \\
Key formula & $u_1 + u_2 + u_3 = u_1 u_2 u_3$ & $T_m \circ T_n = T_{mn}$ \\
Structure & Additive (linear in each $u_i$) & Multiplicative \\
\end{tabular}
\end{center}

The Chebyshev composition property $T_m(T_n(x)) = T_{mn}(x)$ encodes that
``$m$ groups of $n$ twists equals $mn$ twists''---a statement about
\emph{counting} rotations that Wildberger's local turn cannot express.

Thus, Chebyshev polynomials complement rational trigonometry by providing
\textbf{discrete global orientation} (winding number) alongside
Wildberger's \textbf{continuous local orientation} (turn).

\section{Proposal for Wildberger's Program}

We propose that Chebyshev polynomials could complement Wildberger's
rational trigonometry by providing:

\begin{enumerate}
    \item A \textbf{rational theory of twist/orientation} via integer $n$
    \item \textbf{Algebraic encoding} of ``going around'' without angles
    \item \textbf{Multiplicative structure} via composition $T_m \circ T_n = T_{mn}$
    \item \textbf{Rational invariants} that are independent of twist count
\end{enumerate}

\subsection{The Multiplicative Decomposition}

The recent result on lobe area decomposition supports this view:

\begin{theorem}[Multiplicative Decomposition, Dec 2025]
For $n = m \cdot d$ with $m, d \geq 2$:
\begin{equation}
    \sum_{k \equiv r \pmod{m}} A(n, k) = \frac{1}{m}
    \quad \text{for all } r \in \{1, \ldots, m\}
\end{equation}
\end{theorem}

This shows that lobe areas ``factor'' according to the factorization
of $n$, providing a geometric analogue of divisor structure---all
within a rational framework.

\section{Conclusion}

Chebyshev polynomials may offer a path to recovering orientation
within Wildberger's program. The key insight is replacing continuous
angle measurement (requiring $\pi$) with discrete twist counting
(using integers). The algebraic structure of Chebyshev polynomials
encodes rotational information, while the area identities yield
purely rational invariants.

This is not a claim to have ``solved'' the reformulation of
transcendental constants. Rather, it is an observation that
Chebyshev geometry might be a productive direction for investigation.

\section*{Acknowledgments}

This exploration arose from a conversation about the Chebyshev
Integral Identity and its connection to lobe areas. The ``Wildberger
framing'' emerged from asking adversarial questions about whether
$\pi$ could be eliminated from the results.

\begin{thebibliography}{9}

\bibitem{wildberger2005}
N.~J. Wildberger,
\emph{Divine Proportions: Rational Trigonometry to Universal Geometry},
Wild Egg Books, 2005.

\bibitem{chebyshev}
P.~L. Chebyshev,
``Th\'eorie des m\'ecanismes connus sous le nom de parall\'elogrammes,''
\emph{M\'emoires pr\'esent\'es \`a l'Acad\'emie Imp\'eriale des Sciences
de St-P\'etersbourg}, 1854.

\bibitem{chebyshev-integral}
J.~Popelka and Claude (Anthropic),
``Chebyshev Integral Identity and Lobe Areas,''
Orbit repository, v0.1.0-chebyshev-integral,
\url{https://github.com/popojan/orbit}, November 2025.

\bibitem{orbit}
J.~Popelka,
``Orbit: Mathematical Explorations,''
\url{https://github.com/popojan/orbit}, 2025.

\end{thebibliography}

\end{document}
