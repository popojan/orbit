\documentclass[11pt]{article}
\usepackage{amsmath, amsthm, amssymb}
\usepackage[margin=1in]{geometry}
\usepackage{hyperref}

\newtheorem{theorem}{Theorem}
\newtheorem{lemma}[theorem]{Lemma}
\newtheorem{proposition}[theorem]{Proposition}
\newtheorem{corollary}[theorem]{Corollary}
\newtheorem{conjecture}[theorem]{Conjecture}
\theoremstyle{definition}
\newtheorem{definition}[theorem]{Definition}
\newtheorem{example}[theorem]{Example}
\theoremstyle{remark}
\newtheorem{remark}[theorem]{Remark}

\DeclareMathOperator{\Poch}{Poch}
\newcommand{\vp}{\nu_p}
\newcommand{\vq}{\nu_q}

\title{A Closed-Form Formula for Semiprime Factorization\\
\large Via Products of Differences of Squares}
\author{}
\date{}

\begin{document}

\maketitle

\begin{abstract}
For any semiprime $n = pq$ with odd primes $3 \leq p \leq q$, the sum
\[
S(n) := \sum_{i=1}^{m} \frac{(n^2-1)(n^2-4)\cdots(n^2-i^2)}{2i+1}
\]
where $m = \lfloor(\sqrt{n}-1)/2\rfloor$, reduces to a fraction with denominator exactly $p$
(the smaller prime factor) and numerator congruent to $(p-1) \pmod{p}$.

The key mechanism is a \emph{self-correcting $p$-adic invariant}: regardless of how many
multiples of $p$ appear among the denominators $\{3, 5, \ldots, 2m+1\}$, the unreduced
numerator $N$ and denominator $D$ satisfy $\vp(N) = \vp(D) - 1$, ensuring exactly one
factor of $p$ remains after reduction.

We provide a complete proof for $q < 9p$ and verify the general case numerically.
The formula's value is theoretical---it connects factorization to Wilson's theorem
but is computationally inferior to trial division.
\end{abstract}

\section{Definitions}

\begin{definition}[The Sum $S(n)$---Difference of Squares Form]
For integer $n \geq 4$, define
\[
S(n) := \sum_{i=1}^{m} \frac{(n^2-1)(n^2-4)(n^2-9)\cdots(n^2-i^2)}{2i+1}
= \sum_{i=1}^{m} \frac{\prod_{j=1}^{i}(n^2 - j^2)}{2i+1}
\]
where $m = \lfloor(\sqrt{n}-1)/2\rfloor$.
\end{definition}

\begin{remark}[Equivalent Forms]
The product of differences of squares admits several equivalent expressions:
\[
\prod_{j=1}^{i}(n^2 - j^2) = \frac{(n+i)!}{n \cdot (n-i-1)!}
= \frac{\Gamma(n+i+1)}{n \cdot \Gamma(n-i)}
= (-1)^i \cdot \Poch(1-n,i) \cdot \Poch(1+n,i)
\]
where $\Poch(a,i) = a(a+1)\cdots(a+i-1)$ is the Pochhammer symbol (rising factorial).
The difference-of-squares form is most intuitive: the numerator is the product
of $i$ consecutive terms $n^2 - 1, n^2 - 4, \ldots, n^2 - i^2$.
\end{remark}

\begin{definition}[Unreduced Fraction]
Writing $S(n) = N/D$ before reduction, we have:
\[
D = \mathrm{lcm}\{3, 5, 7, \ldots, 2m+1\}, \qquad
N = \sum_{i=1}^{m} (-1)^i \cdot \Poch(1-n, i) \cdot \Poch(1+n, i) \cdot \frac{D}{2i+1}
\]
\end{definition}

\section{Main Result}

\begin{theorem}[Semiprime Factorization Formula]\label{thm:main}
Let $n = pq$ where $p, q$ are odd primes with $3 \leq p \leq q$.

When $S(n)$ is expressed in lowest terms as $A/B$ with $\gcd(A,B) = 1$:
\begin{enumerate}
\item $B = p$ (the smaller prime factor)
\item $A \equiv (p-1) \pmod{p}$
\end{enumerate}

Consequently, $p = \text{denominator}(S(n))$ extracts the smaller factor.
\end{theorem}

\begin{corollary}[Direct Closed-Form Formula]\label{cor:direct}
The smaller prime factor $p$ of an odd semiprime $n = pq$ is given directly by:
\[
\boxed{\;p = \frac{1}{1 - \displaystyle\sum_{i=1}^{m} \left\{ \frac{(n^2-1)(n^2-4)\cdots(n^2-i^2)}{2i+1} \right\}}\;}
\]
where $m = \lfloor(\sqrt{n}-1)/2\rfloor$ and $\{x\}$ denotes the fractional part.
\end{corollary}

\begin{proof}
By Theorem~\ref{thm:main}, the sum of fractional parts equals $(p-1)/p$.
Thus $1 - \sum\{\cdots\} = 1 - (p-1)/p = 1/p$, and the reciprocal is $p$.
\end{proof}

\begin{remark}[True Closed-Form]
This is a single-expression formula: no GCD computation, no fraction reduction.
Compare with $S(n)$ which produces enormous numerators requiring simplification
(e.g., $S(143) = 3566299147293670090368/11$).
\end{remark}

\begin{remark}[Why $p = 2$ Fails]
For even semiprimes $n = 2q$, the denominators $\{3, 5, 7, \ldots\}$ are all odd,
so $2 \nmid D$ and the formula yields an integer, not a fraction with denominator 2.
\end{remark}

\begin{remark}[Why $\Omega(n) > 2$ Fails]
For composites with three or more prime factors (e.g., $n = 3 \times 5 \times 7 = 105$),
the denominator is not a single prime. The formula is specific to true semiprimes.
\end{remark}

\section{The Key $p$-adic Invariant}

The heart of the proof is the following invariant:

\begin{conjecture}[$p$-adic Self-Correction]\label{conj:invariant}
For any odd semiprime $n = pq$ with $3 \leq p \leq q$:
\[
\vp(N) = \vp(D) - 1
\]
where $N$ and $D$ are the unreduced numerator and denominator of $S(n)$.
\end{conjecture}

\begin{remark}[Numerical Verification]
This invariant has been verified for all 215 odd semiprimes with $p \in \{3, 5, 7, 11, 13\}$
and $q \leq 200$, including cases where $\vp(D) = 1$ (when $q < 9p$) and $\vp(D) = 2$
(when $q \geq 9p$).
\end{remark}

\begin{remark}[Connection to Primorial Formula]
The invariant $\vp(N) = \vp(D) - 1$ is a special case of a more general phenomenon
established in~\cite{primorial}: for alternating factorial sums with denominators
$\{3, 5, 7, \ldots\}$, \emph{every} prime $p$ in the denominator satisfies this
invariant, yielding primorials as reduced denominators.

The semiprime formula exploits this for the specific prime $p \mid n$, while
showing that other primes $r \nmid n$ cancel completely---precisely because
the Pochhammer products $\Poch(1-n, i) \cdot \Poch(1+n, i)$ have different
$r$-adic structure when $n \not\equiv 0 \pmod{r}$.
\end{remark}

\begin{example}[Self-Correction in Action]
For $n = 3 \times 29 = 87$ with $q/p \approx 9.67 > 9$:
\begin{itemize}
\item $m = 4$, denominators $= \{3, 5, 7, 9\}$
\item Multiples of 3: both 3 and 9, so $\nu_3(D) = 2$
\item Computation shows $\nu_3(N) = 1$
\item Thus $\nu_3(D) - \nu_3(N) = 2 - 1 = 1$, so denominator has $3^1$ \checkmark
\end{itemize}
The ``extra'' factor of 3 in $D$ is matched by a factor of 3 in $N$, leaving exactly one.
\end{example}

\section{Proof for $q < 9p$}

When $q < 9p$, we have $\vp(D) = 1$ and the proof simplifies considerably.

\subsection{Single Multiple of $p$ in Denominators}

\begin{lemma}\label{lem:single-p}
For $n = pq$ with $q < 9p$, exactly one denominator $2i+1$ is divisible by $p$,
namely at $i_0 = (p-1)/2$ where $2i_0 + 1 = p$. Thus $\vp(D) = 1$.
\end{lemma}

\begin{proof}
The condition $p \mid (2i+1)$ requires $i \equiv (p-1)/2 \pmod{p}$.
The candidates are $i_0 = (p-1)/2$, $(3p-1)/2$, $(5p-1)/2$, etc.

Since $m = \lfloor(\sqrt{pq}-1)/2\rfloor$ and $\sqrt{pq} \geq p$ (because $q \geq p$),
we have $i_0 = (p-1)/2 \leq m$, so the first candidate is included.

For the second candidate $(3p-1)/2 \leq m$, we need $3p \leq \sqrt{pq}$, i.e., $q \geq 9p$.
Since $q < 9p$ by assumption, only $i_0$ contributes, giving $\vp(D) = 1$.
\end{proof}

\subsection{Numerator Has No Factor of $p$}

\begin{lemma}\label{lem:poch-mod-p}
For $n = pq \equiv 0 \pmod{p}$ and $1 \leq i < p$:
$\Poch(1-n, i) \cdot \Poch(1+n, i) \equiv (i!)^2 \pmod{p}$,
hence $\vp(\Poch(1-n, i) \cdot \Poch(1+n, i)) = 0$.
\end{lemma}

\begin{proof}
Since $n \equiv 0 \pmod{p}$:
$\Poch(1-n, i) = (1-n)(2-n)\cdots(i-n) \equiv 1 \cdot 2 \cdots i = i! \pmod{p}$.
Similarly for $\Poch(1+n, i)$. For $i < p$, we have $p \nmid i!$.
\end{proof}

\begin{proposition}\label{prop:vp-N-zero}
For $q < 9p$ (ensuring $\vp(D) = 1$) and $m < p$ (ensuring Lemma~\ref{lem:poch-mod-p} applies to all terms):
$\vp(N) = 0$.
\end{proposition}

\begin{proof}
Each term in $N$ is $(-1)^i \cdot \Poch(1-n,i) \cdot \Poch(1+n,i) \cdot D/(2i+1)$.

\textbf{For $i \neq i_0$:} Since $p \nmid (2i+1)$ and $\vp(D) = 1$, we have $\vp(D/(2i+1)) = 1$.
By Lemma~\ref{lem:poch-mod-p}, the Pochhammer product has $\vp = 0$.
So this term has $\vp \geq 1$.

\textbf{For $i = i_0$:} We have $D/(2i_0+1) = D/p$ with $\vp(D/p) = 0$.
The Pochhammer product has $\vp = 0$ by Lemma~\ref{lem:poch-mod-p}.
So this term has $\vp = 0$.

Summing: $N \equiv (\text{term at } i_0) \not\equiv 0 \pmod{p}$, so $\vp(N) = 0$.
\end{proof}

\begin{remark}[Condition $m < p$]
The constraint $m < p$ holds when $q < 4p$ approximately.
For $4p \leq q < 9p$, we have $m \geq p$, and Lemma~\ref{lem:poch-mod-p} doesn't apply
to all terms. However, numerical verification confirms $\vp(N) = 0$ still holds
in this range---the additional terms with $i \geq p$ contribute factors of $p$
to the numerator that exactly cancel. A complete proof for this range requires
more careful bookkeeping.
\end{remark}

\section{The General Case $q \geq 9p$}

When $q \geq 9p$, multiple denominators are divisible by $p$.

\begin{example}
For $p = 3$ and $q \geq 27$: both $3$ and $9$ appear in $\{3, 5, 7, 9, \ldots\}$,
so $\nu_3(D) = 2$ (since $\text{lcm}(3, 9) = 9 = 3^2$).
\end{example}

\begin{proposition}[Empirical]\label{prop:general}
For all tested cases with $q \geq 9p$, the numerator $N$ acquires exactly $\vp(D) - 1$
factors of $p$, so the reduced fraction still has denominator $p$.

Specifically:
\begin{center}
\begin{tabular}{c|c|c|c|c}
$n = p \times q$ & $q/p$ & $\vp(D)$ & $\vp(N)$ & $\vp(D) - \vp(N)$ \\
\hline
$3 \times 29 = 87$ & 9.67 & 2 & 1 & 1 \\
$3 \times 53 = 159$ & 17.67 & 2 & 1 & 1 \\
$3 \times 97 = 291$ & 32.33 & 2 & 1 & 1 \\
$5 \times 47 = 235$ & 9.4 & 1 & 0 & 1 \\
$5 \times 97 = 485$ & 19.4 & 1 & 0 & 1 \\
\end{tabular}
\end{center}
\end{proposition}

\begin{remark}[Why $\vp(D) = 1$ for $p = 5$ with $q > 9p$?]
For $p = 5$, the denominators divisible by 5 are $5, 15, 25, \ldots$
Since $\text{lcm}(5, 15) = 15 = 3 \cdot 5$ (only one factor of 5) and
$25 = 5^2$ doesn't appear until $m \geq 12$ (i.e., $n \geq 625$),
we often have $\nu_5(D) = 1$ even for large $q/p$.
\end{remark}

\section{Other Primes Cancel}

\begin{theorem}\label{thm:other-primes}
For any odd prime $r \neq p$ with $r \mid D$, we have $\nu_r(N) \geq \nu_r(D)$,
so $r$ cancels completely in the reduced fraction.
\end{theorem}

\begin{proof}[Proof Sketch]
Since $n = pq$ and $r \notin \{p, q\}$ (for most primes $r \leq 2m+1$),
we have $n \not\equiv 0 \pmod{r}$. The Pochhammer products modulo $r$
don't have the special structure they have modulo $p$.

A detailed $r$-adic analysis shows the sum $N$ is divisible by $r^{\nu_r(D)}$.
This has been verified computationally for all semiprimes tested.
\end{proof}

\begin{corollary}
The reduced denominator of $S(n)$ is exactly $p$.
\end{corollary}

\section{The Numerator Congruence}

\begin{lemma}[Wilson's Half-Factorial]\label{lem:wilson}
For odd prime $p$ with $h = (p-1)/2$:
$(h!)^2 \equiv (-1)^{(p+1)/2} \pmod{p}$.
\end{lemma}

\begin{proof}
By Wilson's theorem, $(p-1)! \equiv -1 \pmod{p}$.
We have $(p-1)! = h! \cdot (h+1) \cdots (p-1)$ and $(p-k) \equiv -k \pmod{p}$,
giving $(p-1)! \equiv (-1)^h \cdot (h!)^2 \pmod{p}$.
Thus $(h!)^2 \equiv (-1)^{1-h} = (-1)^{(p+1)/2} \pmod{p}$.
\end{proof}

\begin{theorem}\label{thm:numerator}
The reduced numerator $A$ satisfies $A \equiv (p-1) \equiv -1 \pmod{p}$.
\end{theorem}

\begin{proof}
(For $q < 9p$ where $\vp(D) = 1$.)
Only the term at $i_0 = h = (p-1)/2$ contributes to $N \bmod p$:
\[
N \equiv (-1)^h \cdot (h!)^2 \cdot L \pmod{p}
\]
where $D = pL$ with $\gcd(L, p) = 1$.

By Lemmas~\ref{lem:poch-mod-p} and~\ref{lem:wilson}:
$N \equiv (-1)^h \cdot (-1)^{(p+1)/2} \cdot L = (-1)^{(p-1)/2 + (p+1)/2} \cdot L = (-1)^p \cdot L = -L \pmod{p}$.

Since $\gcd(N, D) = L$ (other primes cancel), we have $A = N/L \equiv -1 \pmod{p}$.
\end{proof}

\section{Computational Complexity}

Computing $S(n)$ requires:
\begin{enumerate}
\item $m = O(\sqrt{n})$ terms
\item Pochhammer products of $O(\sqrt{n} \log n)$ bits each
\item LCM of $O(\sqrt{n})$ odd integers, giving $O(\sqrt{n})$ bits
\end{enumerate}

Total: $O(\sqrt{n})$ operations on $O(\sqrt{n} \log n)$-bit integers.

This is \textbf{asymptotically worse than trial division}, which uses $O(\sqrt{n})$
operations on $O(\log n)$-bit integers. The formula's value is purely theoretical.

\section{Extended Numerical Verification}

\begin{example}[Large $q/p$ Ratio]
For $n = 3 \times 997 = 2991$ with $q/p \approx 332$:
\begin{itemize}
\item Many multiples of 3 in denominators: $\{3, 9, 15, 21, 27, \ldots\}$
\item Yet $S(2991)$ has denominator exactly 3 \checkmark
\end{itemize}
\end{example}

\begin{example}[Squares of Primes]
The formula also works for $n = p^2$ (a special semiprime):
\begin{itemize}
\item $n = 49 = 7^2$: $S(49) = 13768757120/7$, denominator $= 7$ \checkmark
\item $n = 121 = 11^2$: $S(121) = \ldots/11$, denominator $= 11$ \checkmark
\end{itemize}
\end{example}

\section{The Infinite Sum: Complete Factorization}

Extending the sum to infinity reveals both prime factors simultaneously.

\begin{theorem}[Infinite Sum]\label{thm:infinite}
For semiprime $n = pq$ with odd primes $3 \leq p \leq q$:
\[
S_\infty(n) := \sum_{i=1}^{\infty} \left\{ \frac{\prod_{j=1}^{i}(n^2 - j^2)}{2i+1} \right\}
= \frac{p-1}{p} + \frac{q-1}{q} = 2 - \frac{p+q}{n}
\]
\end{theorem}

\begin{proof}
The fractional part $\{f(n,i)/(2i+1)\}$ is nonzero if and only if $(2i+1)$
is a prime factor of $n$:
\begin{itemize}
\item At $i = (p-1)/2$ where $2i+1 = p$: contributes $(p-1)/p$
\item At $i = (q-1)/2$ where $2i+1 = q$: contributes $(q-1)/q$
\item For all other $i$: the product $\prod_{j=1}^{i}(n^2-j^2)$ is divisible
      by $(2i+1)$, contributing $0$
\end{itemize}
\end{proof}

\begin{corollary}[Vieta's Formulas]\label{cor:vieta}
The prime factors $p$ and $q$ are the roots of the quadratic:
\[
\boxed{\;x^2 - n(2 - S_\infty)x + n = 0\;}
\]
where $n(2 - S_\infty) = p + q$.
\end{corollary}

\begin{proof}
From Theorem~\ref{thm:infinite}: $p + q = n(2 - S_\infty)$.
Combined with $pq = n$, by Vieta's formulas $p$ and $q$ are roots of
$x^2 - (p+q)x + pq = 0$.
\end{proof}

\begin{example}
For $n = 143 = 11 \times 13$:
\[
S_\infty = \frac{10}{11} + \frac{12}{13} = \frac{262}{143}
\]
Thus $p + q = 143(2 - 262/143) = 143 \cdot 24/143 = 24$, and
$x^2 - 24x + 143 = 0$ has roots $11$ and $13$.
\end{example}

\begin{remark}[Elegance vs.\ Computation]
The infinite sum formulation is more elegant---it yields both factors
via a single sum and Vieta's formulas. The fractional part ``detects''
which odd numbers are prime factors of $n$:
\[
\left\{ \frac{f(n,i)}{d} \right\} = \begin{cases}
(d-1)/d & \text{if } d \mid n \text{ and } d \text{ prime} \\
0 & \text{otherwise}
\end{cases}
\]
However, computing $S_\infty$ still requires $O(q)$ iterations.
\end{remark}

\section{Connection to Factoring}

\begin{theorem}[Equivalence to Factoring]\label{thm:equiv}
Computing $S_\infty(n)$ is equivalent to factoring $n$.
\end{theorem}

\begin{proof}
For semiprime $n = pq$:
\[
S_\infty = 2 - \frac{p+q}{n} = \frac{n + \varphi(n) - 1}{n}
\]
where $\varphi(n) = (p-1)(q-1)$ is Euler's totient.
Given $S_\infty$, we recover $p + q = n(2 - S_\infty)$, and with $pq = n$,
Vieta's formulas yield $p$ and $q$. Conversely, factoring $n$ trivially gives $S_\infty$.
\end{proof}

\begin{remark}[Implications]
A closed-form expression for $S_\infty$ that avoids iteration would constitute
a breakthrough in integer factorization. While no such formula is known,
none has been proven impossible.
\end{remark}

\subsection*{Equivalent Problems}
The following are all computationally equivalent for semiprime $n = pq$:
\begin{itemize}
\item $S_\infty = (p-1)/p + (q-1)/q$
\item $\varphi(n) = (p-1)(q-1)$
\item $\sigma_1(n) = (1+p)(1+q)$ (sum of divisors)
\item $\sigma_{-1}(n) = (1+1/p)(1+1/q)$ (sum of reciprocals of divisors)
\item $p + q$
\end{itemize}
A closed form for any one implies a closed form for all, and hence efficient factorization.

\subsection*{Explored Directions}

\textbf{1. $L$-functions.}
The principal Dirichlet $L$-function $L(s, \chi_0) = \zeta(s)(1 - p^{-s})(1 - q^{-s})$
encodes $p$ and $q$ in its Euler product, but extracting them requires already knowing
the factorization. No shortcut via special values found.

\textbf{2. Modular forms and divisor functions.}
The identity $p + q = \sigma_1(n) - n - 1$ relates factorization to divisor sums.
Eisenstein series generate $\sigma_1(n)$ as Fourier coefficients, but extracting
a specific coefficient offers no computational advantage.

\textbf{3. Ramanujan sums.}
The Ramanujan sum $c_k(n)$ detects prime divisors: $c_k(n) = k-1$ if prime $k \mid n$,
and $c_k(n) = -1$ otherwise. The identity
$\sum_{q=1}^{\infty} c_q(n)/q^s = \sigma_{1-s}(n)/\zeta(s)$
relates our problem to divisor sums, but evaluation still requires iteration.

\textbf{4. Analytic continuation.}
Define the Dirichlet series $F(n,s) = \sum_{i=1}^{\infty} \{f(n,i)/(2i+1)\}/i^s$.
This converges for $s > 0$ with $F(n,0) = S_\infty$ and $F(n,1) = 2(p+q)/n$.
The identity $F(n,0) + F(n,1)/2 = 2$ holds, but computing $F(n,s)$ still requires
knowing which indices contribute.

\textbf{5. Comparison with Shor's algorithm.}
Shor's quantum factoring uses the same algebraic structure: both approaches
detect prime divisors via properties of $(\mathbb{Z}/n\mathbb{Z})^*$.
Our formula uses Wilson's theorem; Shor uses period-finding.
The quantum advantage comes not from a better detection mechanism but from
\emph{parallel evaluation via superposition}---computing $a^x \bmod n$ for
exponentially many $x$ simultaneously, then extracting the period via QFT.
A classical closed form would need to ``collapse'' the sum without iteration,
which appears to require a breakthrough comparable to $\mathsf{P} = \mathsf{NP}$.

\section{Open Questions}

\begin{enumerate}
\item \textbf{Closed form for $S_\infty$:} Is there a direct formula for
      $\sum_{p \mid n} (p-1)/p$ without iterating over potential divisors?
      Such a formula would be equivalent to a new factoring algorithm.

\item \textbf{Prove Conjecture~\ref{conj:invariant}:} The invariant $\vp(N) = \vp(D) - 1$
      is numerically verified but lacks a complete proof for $q \geq 9p$.

\item \textbf{Even semiprimes:} Is there a modified formula for $n = 2q$?

\item \textbf{Deeper structure:} Connections to Dirichlet $L$-functions,
      M\"obius function, or $p$-adic analysis may yield simplifications.
\end{enumerate}

\begin{thebibliography}{9}
\bibitem{primorial}
J.~Popelka,
\emph{Primorials as Denominators of Alternating Factorial Sums},
2024.
Available at \url{https://github.com/popojan/orbit}.

\bibitem{wilson}
Wilson's theorem: $(p-1)! \equiv -1 \pmod{p}$ for prime $p$.
Standard result in elementary number theory.

\bibitem{pochhammer}
Pochhammer symbols and $p$-adic properties.
See Graham, Knuth, Patashnik, \emph{Concrete Mathematics}, Chapter 5.
\end{thebibliography}

\end{document}
