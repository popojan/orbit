\documentclass[11pt,a4paper]{article}
\usepackage[utf8]{inputenc}
\usepackage[T1]{fontenc}
\usepackage{amsmath,amssymb,amsthm}
\usepackage{geometry}
\usepackage{booktabs}
\usepackage{hyperref}

\geometry{margin=2.5cm}

\newtheorem{theorem}{Theorem}
\newtheorem{proposition}{Proposition}
\newtheorem{definition}{Definition}

\title{Chebyshev Lobe Area Kernel}
\author{Jan Popelka}
\date{November 29, 2025 (updated December 1, 2025)}

\begin{document}
\maketitle

\section{Definition}

\begin{definition}[Chebyshev Lobe Area Function]
For $n \in \mathbb{C} \setminus \{0, \pm 2\}$ and $k \in \mathbb{C}$, define:
\begin{equation}
A(n,k) = \frac{8 - 2n^2 + n^2\left(\cos\frac{2(k-1)\pi}{n} + \cos\frac{2k\pi}{n}\right)}{8n - 2n^3}
\end{equation}
\end{definition}

The denominator simplifies to $8n - 2n^3 = 2n(4-n^2)$.

\textbf{Notation clarification.} The parameter $k$ is the \emph{lobe index}:
\begin{itemize}
\item \textbf{Discrete interpretation:} For integer $n \geq 3$, lobes are numbered $k = 1, 2, \ldots, n$.
\item \textbf{Continuous extension:} The formula accepts any $k \in \mathbb{C}$, with period $n$.
\item \textbf{Not to be confused with:} The Chebyshev polynomial argument $t \in [-1, 1]$ (integration domain).
\end{itemize}

The reflection symmetry $A(n, 1-k) = A(n, k)$ holds for the \emph{continuous} parameter $k$.
For discrete lobes, the equivalent statement is $A(n, k) = A(n, n+1-k)$, since $1-k \equiv n+1-k \pmod{n}$.

\section{Fundamental Decomposition}

\begin{theorem}[Decomposition]
The function $A(n,k)$ decomposes as:
\begin{equation}
A(n,k) = \underbrace{\frac{1}{n}}_{\text{constant}} + \underbrace{\frac{n}{2(4-n^2)}\left(\cos\frac{2(k-1)\pi}{n} + \cos\frac{2k\pi}{n}\right)}_{\text{oscillating, mean zero}}
\end{equation}
\end{theorem}

\begin{center}
\begin{tabular}{lcc}
\toprule
Component & Formula & $\int_0^n (\cdot)\, dk$ \\
\midrule
Constant & $\dfrac{1}{n}$ & $n \cdot \dfrac{1}{n} = 1$ \\[1em]
Oscillating & $\dfrac{n}{2(4-n^2)}(\cos\ldots)$ & $0$ \\[1em]
\textbf{Total} & $A(n,k)$ & $\mathbf{1}$ \\
\bottomrule
\end{tabular}
\end{center}

\section{Key Properties}

\subsection{Periodicity}

\begin{proposition}
$A(n,k)$ has period $n$ in $k$:
\begin{equation}
A(n, k+n) = A(n, k) \quad \forall k
\end{equation}
\end{proposition}

\subsection{Integral Identities}

\begin{theorem}[Discrete Sum --- Chebyshev Integral Theorem]
\begin{equation}
\sum_{k=1}^{n} A(n,k) = 1 \quad \forall n \geq 3
\end{equation}
\end{theorem}

\begin{theorem}[Continuous Integral Identity]
\begin{equation}
\int_{0}^{n} A(n,k) \, dk = 1 \quad \forall n > 2
\end{equation}
\end{theorem}

\begin{theorem}[Multiple Periods]
\begin{equation}
\int_{0}^{mn} A(n,k) \, dk = m \quad \forall m \in \mathbb{Z}^+
\end{equation}
\end{theorem}

\subsection{Normalization}

$A(n,k)$ acts as a \textbf{probability density} on $[0,n]$:
\begin{itemize}
\item Integrates to 1 over one period
\item Discrete samples at $k=1,\ldots,n$ give lobe areas summing to 1
\end{itemize}

\subsection{Asymptotic Behavior}

As $n \to \infty$:
\begin{align}
\text{Constant part:} \quad & \frac{1}{n} \to 0 \\
\text{Oscillation amplitude:} \quad & \frac{n}{2(4-n^2)} \sim -\frac{1}{2n} \to 0
\end{align}
The function becomes increasingly flat, but $\int_0^n A\,dk = 1$ (conservation).

\section{Comparison with Classical Kernels}

\begin{center}
\begin{tabular}{lcccc}
\toprule
Kernel & Period & $\int$ (1 period) & Sign changes & Structure \\
\midrule
Dirichlet $D_n(\theta)$ & $2\pi$ & $2\pi$ & Yes & $\dfrac{\sin((n+\frac{1}{2})\theta)}{2\sin(\theta/2)}$ \\[1em]
Fejér $F_n(\theta)$ & $2\pi$ & $2\pi$ & No ($\geq 0$) & $\dfrac{1}{n}\left(\dfrac{\sin(n\theta/2)}{\sin(\theta/2)}\right)^2$ \\[1em]
\textbf{Lobe Area} $A(n,k)$ & $n$ & $\mathbf{1}$ & Yes & $\dfrac{1}{n} + \text{osc}$ \\
\bottomrule
\end{tabular}
\end{center}

\textbf{Key distinction:}
\begin{itemize}
\item Fejér kernel achieves constant integral trivially (always non-negative)
\item Lobe Area achieves constant integral \textbf{non-trivially} (has sign changes)
\end{itemize}

\section{Fourier Structure}

Using the product-to-sum identity:
\begin{equation}
\cos\frac{2(k-1)\pi}{n} + \cos\frac{2k\pi}{n} = 2\cos\frac{\pi}{n}\cos\frac{(2k-1)\pi}{n}
\end{equation}

Therefore:
\begin{equation}
\boxed{A(n,k) = \frac{1}{n} + \frac{n\cos(\pi/n)}{4-n^2}\cos\frac{(2k-1)\pi}{n}}
\end{equation}

This is a \textbf{DC component plus single harmonic} at frequency $1/n$.

\section{Potential Applications}

\subsection{Low-Pass Filter}

Define the operator:
\begin{equation}
\mathcal{L}_n[f](x) = \int_0^n f(k) A(n,k) \, dk
\end{equation}

Properties:
\begin{itemize}
\item $\mathcal{L}_n[1] = 1$ (preserves constants)
\item $\mathcal{L}_n[\cos(2\pi mk/n)] = 0$ for $m \neq 0$ (filters oscillations)
\item Bandwidth decreases with $n$
\end{itemize}

\subsection{Multi-Scale Averaging}

Different $n$ provides different ``resolution'':
\begin{itemize}
\item Small $n$: coarse averaging (fewer lobes)
\item Large $n$: fine averaging (many lobes, small amplitude)
\end{itemize}

\section{Connection to Polygon Function}

The lobe areas originate from the polygon function:
\begin{equation}
f_k(x) = T_{k+1}(x) - x \cdot T_k(x) = -(1-x^2)U_{k-1}(x)
\end{equation}
where $T_k$, $U_k$ are Chebyshev polynomials of first and second kind.

For $x = \cos\theta$:
\begin{equation}
f_k(\cos\theta) = -\sin\theta \sin(k\theta)
\end{equation}

The $n$ zeros of $f_{n-1}(x)$ divide $[-1,1]$ into $n$ lobes, and $A(n,k)$ is the area of the $k$-th lobe.

\section{Complex Analysis}

The function $A(n,k)$ extends naturally to complex variables, revealing rich analytic structure.

\subsection{Domain and Analyticity}

\begin{theorem}[Two-Variable Extension]
$A(n,k)$ extends to a function on $\mathbb{C}^2$ with:
\begin{itemize}
\item \textbf{In $k$}: Entire function for each fixed $n \neq 0$
\item \textbf{In $n$}: Meromorphic with singularities at $n \in \{0, \pm 2\}$
\end{itemize}
\end{theorem}

\subsection{Singularity Classification}

\begin{center}
\begin{tabular}{lccc}
\toprule
Point & Type & Residue & Limit \\
\midrule
$n = 2$ & Removable & $0$ & $\dfrac{4 - \pi\sin(k\pi)}{8}$ \\[1em]
$n = -2$ & Removable & $0$ & $\dfrac{-4 + \pi\sin(k\pi)}{8}$ \\[1em]
$n = 0$ & Essential & --- & (oscillates wildly) \\
\bottomrule
\end{tabular}
\end{center}

\textbf{Digon limit:} At $n = 2$ with $k \in \mathbb{Z}$, we get $A(2,k) = \frac{1}{2}$ (two equal lobes).

\subsection{Complex Periodicity}

\begin{theorem}[Complex Period]
For all $n, k \in \mathbb{C}$ with $n \neq 0$:
\begin{equation}
A(n, k+n) = A(n, k)
\end{equation}
The period $n$ itself can be complex.
\end{theorem}

\subsection{Path-Independent Integral}

\begin{theorem}[Complex Integral Identity]
For any $n \in \mathbb{C} \setminus \{0, \pm 2\}$ and any contour $\gamma$ from $0$ to $n$:
\begin{equation}
\int_\gamma A(n,k)\, dk = 1
\end{equation}
The integral is path-independent because $A(n,k)$ is entire in $k$.
\end{theorem}

\begin{proof}[Proof sketch]
Since $A(n,k)$ is entire in $k$, Cauchy's theorem implies path independence.
The primitive function evaluated at endpoints gives:
\begin{itemize}
\item Oscillating terms: $0$ at both $k=0$ and $k=n$ (periodicity)
\item Constant term $\frac{1}{n}$: integrates to $\frac{k}{n}\big|_0^n = 1$
\end{itemize}
\end{proof}

\subsection{Behavior Near Essential Singularity}

Near $n = 0$, the function $A(n,k)$ exhibits essential singularity behavior:
\begin{itemize}
\item Values on circles $|n| = \varepsilon$ vary wildly (from $O(1)$ to $O(1/\varepsilon^2)$)
\item By Picard's theorem, $A(n,k)$ takes almost every complex value infinitely often in any neighborhood of $n = 0$
\end{itemize}

\section{Unit Period Normalization}

The variable period $n$ can be normalized to a fixed unit period via substitution.

\subsection{Definition}

\begin{definition}[Normalized Lobe Area Function]
For $n > 2$ and $t \in [0,1]$, define:
\begin{equation}
\tilde{A}(n,t) = n \cdot A(n, nt)
\end{equation}
\end{definition}

Explicitly:
\begin{equation}
\tilde{A}(n,t) = 1 - \frac{n^2\cos(\pi/n)}{n^2-4}\cos\left(\pi\left(\frac{1}{n} - 2t\right)\right)
\end{equation}

\subsection{Properties}

\begin{theorem}[Unit Period Density]
For all $n > 2$:
\begin{enumerate}
\item \textbf{Fixed period:} $\tilde{A}(n,t)$ has period $1$ in $t$
\item \textbf{Normalization:} $\displaystyle\int_0^1 \tilde{A}(n,t)\, dt = 1$
\item \textbf{Non-negativity:} $\tilde{A}(n,t) \geq 0$ for all $t \in [0,1]$
\end{enumerate}
\end{theorem}

The non-negativity follows from the amplitude bound:
\begin{equation}
\alpha(n) = \frac{n^2\cos(\pi/n)}{n^2-4} < 1 \quad \text{for all } n > 2
\end{equation}

with $\lim_{n \to 2^+} \alpha(n) = \pi/4 \approx 0.785$ and $\lim_{n \to \infty} \alpha(n) = 1$.

\subsection{Limiting Behavior}

\begin{theorem}[Hann Window Limit]
\begin{equation}
\lim_{n \to \infty} \tilde{A}(n,t) = 2\sin^2(\pi t) = 1 - \cos(2\pi t)
\end{equation}
\end{theorem}

This is the \textbf{Hann window} (raised cosine), a standard window function in signal processing.

\textbf{Remark (Parameter roles in limits).} In the two-parameter extension $\tilde{A}_{[a,b]}(n,x)$ with $m$ periods:
\begin{itemize}
\item Parameter $n$ controls \textbf{shape} (lobe structure). Only $n \to \infty$ produces Hann limit.
\item Parameter $m$ (number of periods) controls \textbf{repetition count} only---does not affect shape.
\item For fixed $n$, $m \to \infty$: infinitely many copies of the $n$-lobed pattern (no convergence to Hann).
\item For $n \to \infty$, any $m$: $m$ copies of the Hann window.
\end{itemize}
Thus the Hann window is an attractor only in the $n$-direction of parameter space.

\subsection{Interpretation}

\begin{center}
\begin{tabular}{lcc}
\toprule
Property & Original $A(n,k)$ & Normalized $\tilde{A}(n,t)$ \\
\midrule
Period & $n$ (variable) & $1$ (fixed) \\
Integral & $\int_0^n = 1$ & $\int_0^1 = 1$ \\
Non-negative? & No (sign changes) & \textbf{Yes} for $n > 2$ \\
DC component & $1/n$ & $1$ \\
Limit $n \to \infty$ & flat & Hann window \\
\bottomrule
\end{tabular}
\end{center}

The parameter $n$ controls the \textbf{granularity}: smaller $n$ gives fewer, more pronounced lobes; larger $n$ approaches the smooth Hann window. This defines a one-parameter family of probability densities on $[0,1]$.

\subsection{Extension to Arbitrary Intervals}

For any interval $[a,b]$ with width $w = b-a$, define:
\begin{equation}
\tilde{A}_{[a,b]}(n,x) = \frac{1}{w} \cdot \tilde{A}\left(n, \frac{x-a}{w}\right)
\end{equation}

\begin{theorem}[Two-Parameter Family]
For $n > 2$ and $w > 0$:
\begin{enumerate}
\item \textbf{Period:} $\tilde{A}_{[a,b]}(n,x)$ has period $w$ in $x$
\item \textbf{Normalization:} $\displaystyle\int_a^b \tilde{A}_{[a,b]}(n,x)\, dx = 1$
\item \textbf{Multiple periods:} $\displaystyle\int_a^{a+mw} \tilde{A}_{[a,b]}(n,x)\, dx = m$
\item \textbf{Non-negativity:} $\tilde{A}_{[a,b]}(n,x) \geq 0$ for all $x$
\end{enumerate}
\end{theorem}

This gives a \textbf{two-parameter family} of probability densities:
\begin{itemize}
\item Parameter $n > 2$: granularity (number of lobes per period)
\item Parameter $w > 0$: period width
\end{itemize}

The structure is analogous to Fourier series, which can be defined on any interval with periodic extension to $\mathbb{R}$.

\section{Root Sum Identity (General Form)}

The lobe area function exhibits a fundamental reflection symmetry that implies a universal root sum identity.

\subsection{Reflection Symmetry}

\begin{theorem}[Reflection Symmetry]
For all $n \in \mathbb{C} \setminus \{0, \pm 2\}$ and all $k \in \mathbb{C}$:
\begin{equation}
\boxed{A(n, 1-k) = A(n, k)}
\end{equation}
\end{theorem}

\begin{proof}
Using the Fourier form $A(n,k) = \frac{1}{n} + \frac{n\cos(\pi/n)}{4-n^2}\cos\frac{(2k-1)\pi}{n}$:
\begin{align}
A(n, 1-k) &= \frac{1}{n} + \frac{n\cos(\pi/n)}{4-n^2}\cos\frac{(2(1-k)-1)\pi}{n} \\
&= \frac{1}{n} + \frac{n\cos(\pi/n)}{4-n^2}\cos\frac{(1-2k)\pi}{n} \\
&= \frac{1}{n} + \frac{n\cos(\pi/n)}{4-n^2}\cos\frac{-(2k-1)\pi}{n} \\
&= A(n, k) \qquad \text{(since $\cos(-\theta) = \cos\theta$)}
\end{align}
\end{proof}

\textbf{Discrete vs.\ continuous.} The symmetry $k \leftrightarrow 1-k$ is around the axis $k = \frac{1}{2}$.
For discrete lobes ($k \in \{1, \ldots, n\}$), the equivalent symmetry is $k \leftrightarrow n+1-k$ (around $k = \frac{n+1}{2}$),
since $1-k \equiv n+1-k \pmod{n}$.

\subsection{General Root Sum Theorem}

\begin{theorem}[Universal Root Sum Identity]
For any $n > 2$ and any $c \in \mathbb{R}$ such that $A(n, k) = c$ has real solutions:
\begin{equation}
\boxed{k_1 + k_2 \equiv 1 \pmod{n}}
\end{equation}
where $k_1, k_2$ are any two roots in the same period.

In particular, for roots in the \textbf{principal period} $[\frac{1}{2} - \frac{n}{2}, \frac{1}{2} + \frac{n}{2}]$:
\begin{equation}
k_1 + k_2 = 1 \quad \text{(exactly)}
\end{equation}
\end{theorem}

\begin{proof}
The reflection symmetry $A(n, 1-k) = A(n, k)$ implies that if $k_1$ is a root of $A(n, k) = c$, then so is $1-k_1$. These roots are symmetric around $k = \frac{1}{2}$ and sum to $1$.

By periodicity $A(n, k+n) = A(n, k)$, roots in other periods differ by multiples of $n$, giving $k_1 + k_2 \equiv 1 \pmod{n}$.
\end{proof}

\subsection{Special Cases}

\subsubsection{Integer $n$-gons: $A(n, k) = 1/n$}

For integer $n \geq 3$, the mean value $c = 1/n$ gives roots in $[\frac{1-n}{2}, \frac{1+n}{2}]$:
\begin{center}
\begin{tabular}{cccc}
\toprule
$n$ & $k_1$ & $k_2$ & $k_1 + k_2$ \\
\midrule
3 & $-0.25$ & $1.25$ & $1$ \\
4 & $-0.50$ & $1.50$ & $1$ \\
5 & $-0.75$ & $1.75$ & $1$ \\
6 & $-1.00$ & $2.00$ & $1$ \\
\bottomrule
\end{tabular}
\end{center}

\subsubsection{Fractional $\sqrt{m}$-gons: $A(\sqrt{m}, k) = 1/m$}

For integer $m \geq 3$, $m \neq 4$, the equation $A(\sqrt{m}, k) = 1/m$ (where $1/m = 1/n^2$) gives roots summing to $1$ in the principal period. In the standard period $[0, \sqrt{m}]$:
\begin{equation}
k_1 + k_2 = \sqrt{m} \quad \text{(shifted by $\frac{\sqrt{m}-1}{2}$ from principal period)}
\end{equation}

\textbf{Exceptional cases:}
\begin{itemize}
\item $m = 2$: Period $\sqrt{2}$ is irrational; symbolic solvers may mix periods.
\item $m = 4$: Singularity at $n = 2$; digon limit applies.
\end{itemize}

\subsubsection{Other level sets: $A(n, k) = c$}

For $c \neq 1/n$, roots still satisfy $k_1 + k_2 = 1$ (in principal period), but may be complex if $|c - 1/n|$ exceeds the oscillation amplitude $\left|\frac{n\cos(\pi/n)}{4-n^2}\right|$.

\textbf{Example:} $A(3, k) = 1/2$ gives roots $\{-1.469, -0.531\}$ summing to $-2$.
These are in period $[-2.5, 0.5]$; shifting by $+3$ gives $\{1.531, 2.469\}$ in $[0.5, 3.5]$, summing to $4 = 1 + 3$.

\subsection{Geometric Interpretation}

The reflection symmetry $A(n, 1-k) = A(n, k)$ means:
\begin{itemize}
\item The lobe area function is symmetric around $k = \frac{1}{2}$ (continuous) or around $k = \frac{n+1}{2}$ (discrete)
\item Level curves of $A(n, k)$ come in symmetric pairs
\item The ``center of mass'' of any level set is at $k = \frac{1}{2}$ (mod $n$)
\end{itemize}

This extends naturally to fractional ``polygon numbers'' $n = \sqrt{m}$, $n = \sqrt[3]{m}$, etc.

\subsection{Functional Equation (Riemann $\xi$ Analogy)}

The symmetry structure of $A(n,k)$ is reminiscent of the Riemann zeta functional equation.

\subsubsection{Symmetry Properties of $A(n,k)$}

\begin{theorem}[Triple Symmetry]
The function $A(n,k)$ satisfies three fundamental symmetries:
\begin{align}
A(n, 1-k) &= A(n, k) \qquad \text{(reflection in $k$ around $1/2$)} \\
A(-n, k) &= -A(n, k) \qquad \text{(antisymmetry in $n$)} \\
A(\bar{n}, \bar{k}) &= \overline{A(n, k)} \qquad \text{(Schwarz reflection)}
\end{align}
\end{theorem}

The Schwarz reflection principle ensures that $A$ maps real arguments to real values and preserves the conjugation structure in the complex plane.

\subsubsection{Completed Form}

Define the \textbf{completed lobe area function}:
\begin{equation}
B(n, k) := n \cdot A(n, k)
\end{equation}

\begin{theorem}[Double Even Symmetry]
$B(n,k)$ is even in both variables:
\begin{equation}
\boxed{B(n, 1-k) = B(n, k), \qquad B(-n, k) = B(n, k)}
\end{equation}
\end{theorem}

\begin{proof}
\begin{align}
B(-n, k) &= (-n) \cdot A(-n, k) = (-n) \cdot (-A(n, k)) = n \cdot A(n, k) = B(n, k) \\
B(n, 1-k) &= n \cdot A(n, 1-k) = n \cdot A(n, k) = B(n, k)
\end{align}
\end{proof}

\textbf{Geometric interpretation.} The completed form $B(n,k) = n \cdot A(n,k)$ has a natural geometric meaning:
\begin{center}
\begin{tabular}{lccc}
\toprule
Form & Disk radius & Disk area & Total lobe area \\
\midrule
$A(n,k)$ & $1$ & $\pi$ & $1$ \\
$B(n,k)$ & $\sqrt{n}$ & $\pi n$ & $n$ \\
\bottomrule
\end{tabular}
\end{center}
Since area scales as (radius)$^2$, scaling the inscribed curve to a disk of radius $\sqrt{n}$ multiplies all areas by $n$. The completed form normalizes each lobe area to be $O(1)$ regardless of $n$, paralleling how $\xi(s)$ incorporates natural scaling factors to achieve its symmetry.

\subsubsection{Analogy Tables}

\textbf{Raw functions} (original geometric definitions):
\begin{center}
\begin{tabular}{lcc}
\toprule
Property & Riemann $\zeta(s)$ & Lobe Area $A(n, k)$ \\
\midrule
Schwarz reflection & $\zeta(\bar{s}) = \overline{\zeta(s)}$ & $A(\bar{n}, \bar{k}) = \overline{A(n, k)}$ \\
Sign at negative & $\zeta(-m) = -\frac{B_{m+1}}{m+1}$ ($m \in \mathbb{Z}^+$, $B_j$ = Bernoulli) & $A(-n, k) = -A(n, k)$ (exact) \\
Singularities & $s = 1$ (simple pole) & $n = 0, \pm 2$ (poles) \\
Periodicity & --- & $A(n, k+n) = A(n, k)$ \\
\bottomrule
\end{tabular}
\end{center}

\textbf{Completed functions} (symmetric forms):
\begin{center}
\begin{tabular}{lcc}
\toprule
Property & Completed $\xi(s)$ & Completed $B(n, k) = nA(n,k)$ \\
\midrule
Reflection symmetry & $\xi(1-s) = \xi(s)$ & $B(n, 1-k) = B(n, k)$ \\
Sign symmetry & $\xi(-s) = \xi(s)$ (even) & $B(-n, k) = B(n, k)$ (even) \\
Schwarz reflection & $\xi(\bar{s}) = \overline{\xi(s)}$ & $B(\bar{n}, \bar{k}) = \overline{B(n, k)}$ \\
Symmetry center & $s = 1/2$ & $k = 1/2$ \\
Critical region & $0 < \mathrm{Re}(s) < 1$ & $|n| > 2$ \\
Root pairing & $\rho \leftrightarrow 1-\rho$ & $k \leftrightarrow 1-k$ \\
\bottomrule
\end{tabular}
\end{center}

The parallel suggests that $B(n,k)$ may have deeper analytic structure worth investigating.

\section{Multiplicative Decomposition}

The lobe area function exhibits a remarkable multiplicative structure that mirrors the Chebyshev polynomial composition property.

\subsection{Chebyshev Composition Property}

Chebyshev polynomials satisfy the composition identity:
\begin{equation}
T_m(T_d(x)) = T_{md}(x)
\end{equation}

This means an $md$-gon can be viewed as a ``composition'' of an $m$-gon with a $d$-gon.

\subsection{Multiplicative Decomposition Theorem}

\begin{theorem}[Multiplicative Decomposition of Lobe Areas]
For composite $n = md$ with $m \geq 2$, $d \geq 2$, and $n > 2$:
\begin{equation}
\boxed{\sum_{k \equiv r \pmod{m}} A(n, k) = \frac{1}{m} \quad \text{for all } r \in \{1, \ldots, m\}}
\end{equation}
Equivalently, using the completed form:
\begin{equation}
\sum_{k \equiv r \pmod{m}} B(n, k) = d \quad \text{for all } r
\end{equation}
\end{theorem}

\begin{proof}
Using the decomposition $A(n,k) = \frac{1}{n} + \text{oscillating}(\cos\frac{2\pi k}{n})$:

\textbf{1. Constant part:} Summing over $k = r, r+m, r+2m, \ldots, r+(d-1)m$:
\begin{equation}
\sum_{j=0}^{d-1} \frac{1}{n} = \frac{d}{n} = \frac{d}{md} = \frac{1}{m}
\end{equation}

\textbf{2. Oscillating part:} The sum over an arithmetic progression with common difference $m$:
\begin{equation}
\sum_{j=0}^{d-1} e^{2\pi i(r+jm)/(md)} = e^{2\pi ir/(md)} \cdot \sum_{j=0}^{d-1} e^{2\pi ij/d} = e^{2\pi ir/(md)} \cdot 0 = 0
\end{equation}
The geometric sum vanishes because the terms are $d$-th roots of unity.

Therefore only the constant part survives, giving $\frac{1}{m}$.
\end{proof}

\subsection{Geometric Interpretation}

The $md$-gon can be ``viewed'' in two equivalent ways:
\begin{itemize}
\item As $m$ groups of $d$ lobes each, with each group having total area $\frac{1}{m}$
\item As $d$ groups of $m$ lobes each, with each group having total area $\frac{1}{d}$
\end{itemize}

\begin{center}
\begin{tabular}{lccc}
\toprule
Grouping & Number of groups & Lobes per group & Area per group \\
\midrule
By $k \bmod m$ & $m$ & $d$ & $1/m$ \\
By $k \bmod d$ & $d$ & $m$ & $1/d$ \\
\bottomrule
\end{tabular}
\end{center}

\subsection{Divisor Decomposition Structure}

This multiplicative structure is reminiscent of (but distinct from) how $L$-functions factor over primes:

\begin{center}
\begin{tabular}{lcc}
\toprule
Property & $L$-functions & Lobe Areas \\
\midrule
Factorization base & Primes $p$ & Divisors of $n$ \\
Product structure & $L(s) = \prod_p F_p(s)$ & $\sum_{k \equiv r} A(md,k) = \frac{1}{m}$ \\
Composition & --- & $T_m(T_d) = T_{md}$ \\
Local-global & Each prime contributes & Each divisor contributes \\
\bottomrule
\end{tabular}
\end{center}

The lobe areas ``factor'' according to the divisor structure of $n$, providing a geometric analogue of arithmetic multiplicativity.

\subsection{Example: 12-gon}

For $n = 12 = 3 \times 4$:

\textbf{Grouped by $k \bmod 3$:}
\begin{center}
\begin{tabular}{ccc}
\toprule
$k \equiv 1 \pmod{3}$ & $k \equiv 2 \pmod{3}$ & $k \equiv 0 \pmod{3}$ \\
\midrule
$k = 1, 4, 7, 10$ & $k = 2, 5, 8, 11$ & $k = 3, 6, 9, 12$ \\
Sum $= 4 = d$ & Sum $= 4$ & Sum $= 4$ \\
\bottomrule
\end{tabular}
\end{center}

\textbf{Grouped by $k \bmod 4$:}
\begin{center}
\begin{tabular}{cccc}
\toprule
$k \equiv 1$ & $k \equiv 2$ & $k \equiv 3$ & $k \equiv 0$ \\
\midrule
$k = 1, 5, 9$ & $k = 2, 6, 10$ & $k = 3, 7, 11$ & $k = 4, 8, 12$ \\
Sum $= 3 = m$ & Sum $= 3$ & Sum $= 3$ & Sum $= 3$ \\
\bottomrule
\end{tabular}
\end{center}

\subsection{Open Questions}

\begin{enumerate}
\item \textbf{Prime powers:} Does a similar decomposition hold for $n = p^k$?
\item \textbf{Three or more factors:} For $n = abc$, how do the groupings interact?
\item \textbf{Continuous extension:} What happens for non-integer divisors?
\item \textbf{Connection to characters:} Is there a Dirichlet character interpretation?
\end{enumerate}

\section{Summary}

\begin{center}
\begin{tabular}{ll}
\toprule
Property & Value/Formula \\
\midrule
Definition & $\dfrac{8 - 2n^2 + n^2(\cos\frac{2(k-1)\pi}{n} + \cos\frac{2k\pi}{n})}{2n(4-n^2)}$ \\[1em]
Period & $n$ \\
Discrete sum & $\sum_{k=1}^n A(n,k) = 1$ \\
Continuous integral & $\int_0^n A(n,k)\,dk = 1$ \\
Decomposition & $\dfrac{1}{n} + O(1/n)$ oscillation \\
Domain & $n \in \mathbb{C} \setminus \{0, \pm 2\}$, $k \in \mathbb{C}$ \\
Normalized form & $\tilde{A}(n,t) = n \cdot A(n, nt)$ on $[0,1]$ \\
Limit $n \to \infty$ & $\tilde{A} \to 2\sin^2(\pi t)$ (Hann window) \\
Reflection Symmetry & $A(n, 1-k) = A(n, k)$ \\
Root Sum Identity & $A(n, k_1) = A(n, k_2) = c \Rightarrow k_1 + k_2 \equiv 1 \pmod{n}$ \\
Completed form $B$ & $B(n,k) := n \cdot A(n,k)$; even in both $n$ and around $k=\frac{1}{2}$ \\
$n$-antisymmetry & $A(-n, k) = -A(n, k)$ \\
Schwarz reflection & $A(\bar{n}, \bar{k}) = \overline{A(n, k)}$ \\
Multiplicative decomp. & $\sum_{k \equiv r \pmod{m}} A(md, k) = \frac{1}{m}$ \\
\bottomrule
\end{tabular}
\end{center}

\textbf{Notation:} $k$ is the lobe index (discrete: $k \in \{1, \ldots, n\}$; continuous extension: $k \in \mathbb{C}$).
The symmetry $k \leftrightarrow 1-k$ is equivalent to $k \leftrightarrow n+1-k$ for discrete lobes via periodicity.

\section{Implementation}

Available in the Orbit Mathematica paclet:
\begin{verbatim}
<< Orbit`
ChebyshevLobeAreaSymbolic[n, k]  (* symbolic version *)
ChebyshevLobeArea[n, k]          (* integer n >= 3, 1 <= k <= n *)

(* Normalized version on [0,1] *)
NormalizedLobeArea[n_, t_] := n * ChebyshevLobeAreaSymbolic[n, n*t]
\end{verbatim}

\end{document}
