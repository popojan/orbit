\documentclass[11pt]{article}
\usepackage[margin=1in]{geometry}
\usepackage{amsmath,amssymb,amsthm}

\newtheorem{theorem}{Theorem}
\newtheorem{lemma}[theorem]{Lemma}
\theoremstyle{definition}
\newtheorem{definition}[theorem]{Definition}

\title{Proof of Primorial Recurrence Equivalence}
\author{Supplementary Material}
\date{\today}

\begin{document}

\maketitle

\section{Statement of Result}

This document provides a rigorous proof that the iterative recurrence formulation is equivalent to the direct alternating factorial sum for computing primorials.

\begin{theorem}[Recurrence Equivalence]
Define the recurrence relation starting from $S_0 = \{0, 0, 1\}$:
\begin{equation}\label{eq:recurrence}
S_{n+1} = \left\{n+1,\, b_n,\, b_n + (a_n - b_n)\left(n + \frac{1}{3+2n}\right)\right\}
\end{equation}
where $S_n = \{n, a_n, b_n\}$ denotes the state after $n$ iterations.

Then for all $n \geq 0$:
\begin{equation}\label{eq:main-relation}
b_n = 1 + 2\sum_{k=1}^{n} \frac{(-1)^k \cdot k!}{2k+1}
\end{equation}
\end{theorem}

\section{Proof}

\subsection{Base Case}

For $n = 0$:
\begin{itemize}
\item From initialization: $b_0 = 1$
\item From equation (\ref{eq:main-relation}): $1 + 2 \cdot \text{(empty sum)} = 1 + 0 = 1$
\end{itemize}

Thus the base case holds.

\subsection{Recurrence Structure}

\begin{lemma}\label{lem:a-equals-prev-b}
For all $n \geq 1$, we have $a_n = b_{n-1}$.
\end{lemma}

\begin{proof}
This follows directly from the recurrence update: $S_{n+1} = \{n+1, b_n, \ldots\}$ sets $a_{n+1}$ to the previous value of $b_n$.
\end{proof}

\subsection{Inductive Step}

Assume equation (\ref{eq:main-relation}) holds for $n$, i.e., $b_n = 1 + 2S_n$ where
\[
S_n = \sum_{k=1}^{n} \frac{(-1)^k \cdot k!}{2k+1}
\]

By Lemma \ref{lem:a-equals-prev-b}, we also have:
\[
a_n = b_{n-1} = 1 + 2S_{n-1}
\]

From the recurrence (\ref{eq:recurrence}):
\begin{align}
b_{n+1} &= b_n + (a_n - b_n)\left(n + \frac{1}{3+2n}\right) \nonumber\\
&= (1 + 2S_n) + (1 + 2S_{n-1} - 1 - 2S_n)\left(n + \frac{1}{3+2n}\right) \nonumber\\
&= 1 + 2S_n + 2(S_{n-1} - S_n)\left(n + \frac{1}{3+2n}\right) \label{eq:step1}
\end{align}

\begin{lemma}\label{lem:sum-diff}
The difference of consecutive partial sums is:
\[
S_n - S_{n-1} = \frac{(-1)^n \cdot n!}{2n+1}
\]
\end{lemma}

\begin{proof}
By definition, $S_n$ includes all terms from $k=1$ to $k=n$, while $S_{n-1}$ includes only $k=1$ to $k=n-1$. The difference is precisely the $n$-th term.
\end{proof}

Substituting Lemma \ref{lem:sum-diff} into equation (\ref{eq:step1}):
\begin{align}
b_{n+1} &= 1 + 2S_n - 2 \cdot \frac{(-1)^n \cdot n!}{2n+1} \cdot \left(n + \frac{1}{3+2n}\right) \label{eq:step2}
\end{align}

The key is to show that the update term equals the next term in the doubled sum:
\[
2(S_{n+1} - S_n) = 2 \cdot \frac{(-1)^{n+1} \cdot (n+1)!}{2(n+1)+1} = 2 \cdot \frac{(-1)^{n+1} \cdot (n+1)!}{2n+3}
\]

\subsection{Core Algebraic Identity}

\begin{lemma}\label{lem:algebraic-identity}
For all $n \geq 0$:
\[
-2 \cdot \frac{(-1)^n \cdot n!}{2n+1} \cdot \left(n + \frac{1}{3+2n}\right) = 2 \cdot \frac{(-1)^{n+1} \cdot (n+1)!}{2n+3}
\]
\end{lemma}

\begin{proof}
First, note that $(-1)^{n+1} = -(-1)^n$, so we can factor out $-2(-1)^n$ from both sides:
\begin{equation}\label{eq:factored}
\frac{n!}{2n+1} \cdot \left(n + \frac{1}{3+2n}\right) = \frac{(n+1)!}{2n+3}
\end{equation}

Simplify the left-hand side:
\begin{align*}
n + \frac{1}{3+2n} &= \frac{n(3+2n) + 1}{3+2n} = \frac{3n + 2n^2 + 1}{3+2n} = \frac{1 + 3n + 2n^2}{3+2n}
\end{align*}

Therefore:
\begin{align*}
\text{LHS} &= \frac{n!}{2n+1} \cdot \frac{1 + 3n + 2n^2}{3+2n}\\
&= \frac{n! \cdot (1 + 3n + 2n^2)}{(2n+1)(3+2n)}
\end{align*}

Note that $(2n+1)(3+2n) = 6n + 4n^2 + 3 + 2n = 3 + 8n + 4n^2$ and $1 + 3n + 2n^2 = (n+1)(2n+1)$:
\begin{align*}
\text{LHS} &= \frac{n! \cdot (n+1)(2n+1)}{(2n+1)(2n+3)}\\
&= \frac{n! \cdot (n+1)}{2n+3}\\
&= \frac{(n+1)!}{2n+3} = \text{RHS}
\end{align*}

This completes the proof.
\end{proof}

\subsection{Conclusion of Inductive Step}

By Lemma \ref{lem:algebraic-identity}, equation (\ref{eq:step2}) becomes:
\begin{align*}
b_{n+1} &= 1 + 2S_n + 2 \cdot \frac{(-1)^{n+1} \cdot (n+1)!}{2n+3}\\
&= 1 + 2S_n + 2(S_{n+1} - S_n)\\
&= 1 + 2S_{n+1}
\end{align*}

This completes the induction.

\section{Extraction Formula}

\begin{theorem}[Primorial Extraction]
After $h = \lfloor(m-1)/2\rfloor$ iterations, the primorial of $m$ is given by:
\[
\text{Primorial}(m) = 2 \cdot \text{den}(b_h - 1)
\]
where $\text{den}(q)$ denotes the denominator of rational $q$ in lowest terms.
\end{theorem}

\begin{proof}
From Theorem 1, $b_h = 1 + 2S_h$, so:
\[
b_h - 1 = 2S_h = 2 \cdot \frac{1}{2}\sum_{k=1}^{h} \frac{(-1)^k \cdot k!}{2k+1} = \sum_{k=1}^{h} \frac{(-1)^k \cdot k!}{2k+1}
\]

Since primorials always contain the factor 2, and $\text{den}(S_h) = \text{Primorial}(m)$ (the main result to be proved), we have:
\[
\text{den}(2S_h) = \frac{\text{Primorial}(m)}{2}
\]

Therefore:
\[
2 \cdot \text{den}(b_h - 1) = 2 \cdot \frac{\text{Primorial}(m)}{2} = \text{Primorial}(m)
\]
\end{proof}

\end{document}
