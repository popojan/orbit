\documentclass[11pt]{article}
\usepackage{amsmath,amssymb,amsthm}
\usepackage[margin=1in]{geometry}
\usepackage{booktabs}

\newtheorem{theorem}{Theorem}
\newtheorem{proposition}{Proposition}
\newtheorem{definition}{Definition}

\title{The Diagonal Parametrization:\\De Moivre Without the Imaginary Unit}
\author{Jan Popelka}
\date{\today}

\begin{document}
\maketitle

\begin{abstract}
We present a parametrization of the unit circle that expresses complex multiplication
as pure real arithmetic. Building on Hartley's cas function (1942), we define
$\mathrm{Circ}(t) = \cos(3\pi/4 + \pi t)$ and show that the point
$P[t] = (\mathrm{Circ}[-t], \mathrm{Circ}[t])$ satisfies a multiplication law
$P[t_1] \otimes P[t_2] = P[t_1 + t_2 + 5/4]$ involving only addition of real parameters.
The offset $5/4$ is not arbitrary but one of exactly two values forced by the
requirement that $|\cos\phi| = |\sin\phi|$ (the ``diagonal'' condition).
The two choices form a duality related by the reflection $t \leftrightarrow 1-t$,
and together with the shift operator generate a $D_4$ symmetry group.
\end{abstract}

\section{Introduction}

The unit circle admits many parametrizations. The standard choice
$(\cos\theta, \sin\theta)$ requires two functions, while the complex exponential
$e^{i\theta}$ requires the imaginary unit. Hartley \cite{hartley1942} introduced
the function $\mathrm{cas}(x) = \cos x + \sin x$, motivated by signal processing applications.

We revisit this idea with a specific phase choice that yields a clean algebraic structure.
The main observation is that complex multiplication on the unit circle becomes
\emph{addition of real parameters} plus a fixed offset---and this offset is constrained
to one of exactly two values by geometric requirements.

\section{The Circ Function}

\begin{definition}
The \emph{Circ function} is defined by
\[
\mathrm{Circ}(t) = \cos\left(\frac{3\pi}{4} + \pi t\right).
\]
\end{definition}

This is equivalent to Hartley's cas function up to scaling and sign:
\[
\mathrm{Circ}(t) = -\frac{1}{\sqrt{2}} \, \mathrm{cas}(\pi t).
\]

The phase $3\pi/4$ places us on the \emph{diagonal} of the unit circle,
where $|\cos\phi| = |\sin\phi| = 1/\sqrt{2}$.

Key properties:
\begin{itemize}
\item \textbf{Antiperiod:} $\mathrm{Circ}(t+1) = -\mathrm{Circ}(t)$ (period 2)
\item \textbf{Derivative:} $\mathrm{Circ}'(t) = -\pi\,\mathrm{Circ}(t + 1/2)$ (shift, not function change)
\item \textbf{Symmetry:} $\cos(\pi t) = -\frac{\mathrm{Circ}(t) + \mathrm{Circ}(-t)}{\sqrt{2}}$, \quad
      $\sin(\pi t) = -\frac{\mathrm{Circ}(t) - \mathrm{Circ}(-t)}{\sqrt{2}}$
\end{itemize}

\section{Circle Points and Multiplication}

\begin{definition}
For $t \in \mathbb{R}$, define the \emph{circle point}
\[
P[t] = (\mathrm{Circ}[-t], \mathrm{Circ}[t]).
\]
\end{definition}

This is a point on the unit circle: $\mathrm{Circ}[-t]^2 + \mathrm{Circ}[t]^2 = 1$.

\begin{theorem}[Real Multiplication Law]
Complex multiplication of circle points satisfies
\[
P[t_1] \otimes P[t_2] = P[t_1 + t_2 + 5/4]
\]
where $\otimes$ denotes component-wise complex multiplication:
$(a,b) \otimes (c,d) = (ac - bd, ad + bc)$.
\end{theorem}

\begin{proof}
Direct computation using the angle-addition property of complex multiplication.
The offset $5/4$ arises from the phase $3\pi/4$ in the Circ definition.
\end{proof}

\begin{theorem}[De Moivre, Real Form]
\[
P[t]^n = P\left[nt + \frac{5(n-1)}{4}\right]
\]
\end{theorem}

\begin{proof}
By induction on $n$, using the multiplication law.
\end{proof}

The formulas involve \emph{only real arithmetic}---no $i$, no explicit $\pi$.

\section{The Diagonal Constraint}

Why $5/4$? The phase $3\pi/4$ was chosen so that
$|\cos(3\pi/4)| = |\sin(3\pi/4)| = 1/\sqrt{2}$---the \emph{diagonal} condition
ensuring symmetric treatment of sine and cosine.

There are exactly four such phases: $\pi/4$, $3\pi/4$, $5\pi/4$, $7\pi/4$.
Phases differing by $\pi$ yield the same Circ up to sign, leaving two equivalence classes:

\begin{center}
\begin{tabular}{@{}lcc@{}}
\toprule
Phase pair & Start point $P[0]$ & Multiplication offset \\
\midrule
$\{\pi/4, 7\pi/4\}$ & $(+1/\sqrt{2}, +1/\sqrt{2})$ & $7/4$ \\
$\{3\pi/4, 5\pi/4\}$ & $(-1/\sqrt{2}, -1/\sqrt{2})$ & $5/4$ \\
\bottomrule
\end{tabular}
\end{center}

The offset is \emph{not arbitrary}---it is one of exactly two values,
determined by the diagonal geometry.

\section{The $D_4$ Symmetry}

Let $P_A$ denote the framework with offset $5/4$, and $P_B$ the framework with offset $7/4$.

\begin{proposition}
$P_B[t] = P_A[1-t]$.
\end{proposition}

The two frameworks are related by reflection around $t = 1/2$.

Define operators:
\begin{itemize}
\item $S$: shift $t \mapsto t + 1/2$ (corresponds to multiplication by $i$)
\item $T$: reflection $t \mapsto 1 - t$ (swaps the two frameworks)
\end{itemize}

These satisfy $S^4 = 1$, $T^2 = 1$, and generate the dihedral group $D_4$:

\begin{center}
\begin{tabular}{@{}lll@{}}
\toprule
Element & Action on $t$ & Meaning \\
\midrule
$1$ & $t$ & identity \\
$S$ & $t + 1/2$ & $90^\circ$ rotation \\
$S^2$ & $t + 1$ & $180^\circ$ rotation \\
$S^3$ & $t + 3/2$ & $270^\circ$ rotation \\
$T$ & $1 - t$ & framework reflection \\
$S^2 T$ & $-t$ & coordinate swap \\
\bottomrule
\end{tabular}
\end{center}

A \emph{unified framework} invariant under full $D_4$ symmetry is given by the pair:
\[
Q[t] = (P_A[t], P_A[1-t]).
\]

\section{Taylor Series}

The Taylor expansion of Circ around $t=0$ has a closed form:
\[
\mathrm{Circ}(t) = \sum_{k=0}^{\infty} \frac{\varepsilon_k \cdot \pi^k}{\sqrt{2} \cdot k!} \, t^k
\]
where the sign pattern $\varepsilon_k = -\mathrm{cas}(\pi k/2)$ cycles as
$\{-1, -1, +1, +1, -1, -1, \ldots\}$.

Unlike $\sin$ (odd powers) or $\cos$ (even powers), Circ contains \emph{all} powers---the
price of unification.

\section{Discussion}

What does this parametrization achieve?

\textbf{Operationally:} The multiplication formula $P[t_1 + t_2 + 5/4]$ involves only
real addition. The imaginary unit $i$ becomes the shift operator $S$, satisfying
$S^2 = -1$ because Circ has antiperiod 1.

\textbf{Geometrically:} The two-ness of trigonometry (sin vs cos) relocates to a
discrete gauge choice ($5/4$ vs $7/4$), unified by $D_4$ symmetry.

\textbf{Historically:} This is Hartley (1942) with a specific phase choice that
reveals the diagonal duality structure.

The framework does not eliminate trigonometry---it is \emph{defined} via cosine.
But it demonstrates that the operational complexity of complex arithmetic
can be absorbed into a single real parameter, with the algebraic structure
encoded in the offset $5/4$.

\begin{thebibliography}{9}
\bibitem{hartley1942}
R.~V.~L. Hartley,
``A More Symmetrical Fourier Analysis Applied to Transmission Problems,''
\emph{Proceedings of the IRE}, vol.~30, no.~3, pp.~144--150, 1942.

\bibitem{bracewell1986}
R.~N. Bracewell,
\emph{The Hartley Transform},
Oxford University Press, 1986.
\end{thebibliography}

\end{document}
