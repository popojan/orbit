\documentclass[11pt,a4paper]{article}
\usepackage{amsmath,amssymb,amsthm}
\usepackage{hyperref}
\usepackage{booktabs}

\newtheorem{theorem}{Theorem}[section]
\newtheorem{corollary}[theorem]{Corollary}
\newtheorem{lemma}[theorem]{Lemma}
\newtheorem{definition}[theorem]{Definition}
\newtheorem{remark}[theorem]{Remark}

\title{Complex Analysis of the Completed Lobe Area Function $B(n,k)$}
\author{Jan Popelka}
\date{December 2025}

\begin{document}
\maketitle

\begin{abstract}
We study the analytic properties of the Completed Lobe Area function $B(n,k)$ as a function
of complex $k$. This Chebyshev-derived function exhibits remarkable parallels to Riemann's
xi function: it is entire, satisfies a functional equation, and has all zeros on a critical line.
Unlike the Riemann Hypothesis, these properties are \emph{proven} from the explicit Fourier structure.
The universal constant $\pi^2 - 8$ emerges in three independent contexts.
\end{abstract}

\section{Definitions}

\begin{definition}[Lobe Area Function]
The lobe area function is defined as
\begin{equation}
A(n,k) = \frac{1}{n} + \alpha(n) \cos\frac{(2k-1)\pi}{n}
\end{equation}
where $\alpha(n) = \frac{n\cos(\pi/n)}{4-n^2} < 0$ for $n > 2$.
\end{definition}

\begin{definition}[Completed Lobe Area]
The completed lobe area function is
\begin{equation}
B(n,k) = n \cdot A(n,k) = 1 + \beta(n) \cos\frac{(2k-1)\pi}{n}
\end{equation}
where $\beta(n) = \frac{n^2\cos(\pi/n)}{4-n^2}$.
\end{definition}

\section{Key Results}

\subsection{Holomorphicity}

\begin{theorem}
$B(n,k)$ is an entire function in $k$ (holomorphic on all of $\mathbb{C}$).
\end{theorem}

\begin{proof}
$B(n,k)$ is a composition of polynomial and cosine functions, both entire.
\end{proof}

\subsection{Symmetries}

\begin{theorem}[Symmetry Properties]
The function $B(n,k)$ satisfies:
\begin{enumerate}
\item Periodicity: $B(n, k+n) = B(n, k)$
\item Reflection: $B(n, 1-k) = B(n, k)$
\item Complementary: $B(n, n+1-k) = B(n, k)$
\item Even in $n$: $B(-n, k) = B(n, k)$
\end{enumerate}
\end{theorem}

\subsection{Critical Points}

\begin{theorem}
The critical points of $B(n,k)$ for real $k$ are:
\begin{itemize}
\item Maximum at $k = \frac{n+1}{2}$, with value $\to 2$ as $n \to \infty$
\item Minimum at $k = \frac{1}{2}$, with value $\to 0$ as $n \to \infty$
\end{itemize}
\end{theorem}

\subsection{Zeros}

\begin{theorem}[Closed-Form Zeros]
The zeros of $B(n,k)$ are located at
\begin{equation}
k = \frac{1}{2} + mn \pm i\delta(n), \quad m \in \mathbb{Z}
\end{equation}
where
\begin{equation}
\delta(n) = \frac{n}{2\pi} \operatorname{arccosh}\left(-\frac{1}{\beta(n)}\right)
= \frac{n}{2\pi} \operatorname{arccosh}\frac{n^2 - 4}{n^2 \cos(\pi/n)}
\end{equation}
\end{theorem}

\begin{remark}[Removable Singularity at $n=2$]
The formula for $\beta(n)$ has a $\frac{0}{0}$ form at $n = 2$:
\[
\beta(2) = \frac{4 \cdot \cos(\pi/2)}{4 - 4} = \frac{0}{0}
\]
By L'H\^{o}pital's rule:
\[
\lim_{n \to 2} \beta(n) = -\frac{\pi}{4}, \qquad
\lim_{n \to 2} \delta(n) = \frac{\operatorname{arccosh}(4/\pi)}{\pi} \approx 0.23026
\]
\end{remark}

\begin{theorem}[Universal Limit]
As $n \to \infty$,
\begin{equation}
\delta(n) \to \delta_\infty = \frac{\sqrt{\pi^2 - 8}}{2\pi} = 0.21761808912708625\ldots
\end{equation}
\end{theorem}

\begin{proof}
For large $n$:
\begin{align}
\beta(n) &= -1 + \frac{\pi^2/2 - 4}{n^2} + O(1/n^4) \\
-\frac{1}{\beta(n)} &= 1 + \frac{\pi^2/2 - 4}{n^2} + O(1/n^4)
\end{align}
Using $\operatorname{arccosh}(1 + \epsilon) \sim \sqrt{2\epsilon}$ for small $\epsilon$:
\[
\operatorname{arccosh}\left(-\frac{1}{\beta(n)}\right) \sim \frac{\sqrt{\pi^2 - 8}}{n}
\]
Therefore $\delta(n) = \frac{n}{2\pi} \cdot \frac{\sqrt{\pi^2 - 8}}{n} = \frac{\sqrt{\pi^2 - 8}}{2\pi}$.
\end{proof}

\subsection{Critical Line Property}

\begin{theorem}
All zeros of $B(n,k)$ lie on the critical lines $\operatorname{Re}(k) \equiv \frac{1}{2} \pmod{n}$.
\end{theorem}

\begin{proof}
From the closed form, zeros are at $k = \frac{1}{2} + mn \pm i\delta(n)$,
so $\operatorname{Re}(k) = \frac{1}{2} + mn \equiv \frac{1}{2} \pmod{n}$.
\end{proof}

\begin{remark}
Unlike the Riemann Hypothesis, this is not a conjecture but a \emph{proven fact}
following from the explicit Fourier structure.
\end{remark}

\subsection{No Real Zeros}

\begin{theorem}
$B(n,k)$ has no real zeros for any $n \geq 1$.
\end{theorem}

\begin{proof}
Zeros occur where $\cos\frac{(2k-1)\pi}{n} = -\frac{1}{\beta(n)}$.
For all $n \geq 1$, we have $-\frac{1}{\beta(n)} > 1$, so the equation has no real solutions.
\end{proof}

\subsection{Hadamard Factorization}

\begin{theorem}
As an entire function of order 1, $B(n,k)$ admits the product representation:
\begin{equation}
B(n,k) = B\left(n, \frac{n+1}{2}\right) \cdot \prod_{m \in \mathbb{Z}}
\left[1 + \left(\frac{k - \frac{1}{2} - mn}{\delta(n)}\right)^2\right]
\end{equation}
\end{theorem}

\section{Comparison with Riemann Xi}

\begin{center}
\begin{tabular}{lcc}
\toprule
Property & Riemann $\xi(s)$ & $B(n,k)$ \\
\midrule
Type & Entire function & Entire function \\
Order & 1 & 1 \\
Critical line & $\operatorname{Re}(s) = \frac{1}{2}$ & $\operatorname{Re}(k) \equiv \frac{1}{2} \pmod{n}$ \\
Zeros on CL & Conjectured (RH) & \textbf{Proven} \\
Functional equation & $\xi(s) = \xi(1-s)$ & $B(n,1-k) = B(n,k)$ \\
Zero locations & Unknown & Explicit formula \\
\bottomrule
\end{tabular}
\end{center}

\section{The Universal Constant $\pi^2 - 8$}

The constant $\pi^2 - 8 \approx 1.8696$ appears in three independent contexts:

\subsection{Zero Offset}
\begin{equation}
\delta_\infty = \frac{\sqrt{\pi^2 - 8}}{2\pi} \approx 0.2176
\end{equation}

\subsection{Asymptotic Decay}
\begin{equation}
B\left(n, \frac{1}{2}\right) \sim \frac{\pi^2 - 8}{2n^2} \quad \text{as } n \to \infty
\end{equation}

\subsection{Multiplicativity Correction}
\begin{equation}
\frac{B(mn, \frac{1}{2})}{B(m, \frac{1}{2}) \cdot B(n, \frac{1}{2})} \to \frac{2}{\pi^2 - 8} \approx 1.0697
\end{equation}
This shows $B$ is \emph{not} multiplicative, but the deviation is controlled by $\pi^2 - 8$.

\subsection{Algebraic Origin}

The constant $\pi^2 - 8$ is a \emph{hybrid} of geometry and algebra:
\begin{itemize}
\item \textbf{Source of $\pi^2$}: From Taylor expansion $\cos(\pi/n) = 1 - \frac{\pi^2}{2n^2} + O(1/n^4)$
\item \textbf{Source of 8}: From the pole at $n = 2$ (digon): $\beta(n) = \frac{n^2\cos(\pi/n)}{4-n^2}$
\end{itemize}

\section{Connection to Riemann Zeta}

\subsection{Sum Formula}

\begin{theorem}
The sum of $B(n, \frac{1}{2})$ over all $n \geq 1$ equals a power series in $\pi$:
\begin{equation}
\sum_{n=1}^{\infty} B\left(n, \frac{1}{2}\right) =
\frac{2}{3} + \left(1 - \frac{\pi}{4}\right) +
\sum_{k=1}^{\infty} c_{2k} \cdot \left(\zeta(2k) - 1 - 2^{-2k}\right)
\end{equation}
where $c_{2k}$ are the Taylor coefficients of $B(n, \frac{1}{2})$ at $n = \infty$.
\end{theorem}

\begin{proof}
\textbf{Step 1: Taylor expansion of $B(n, \frac{1}{2})$.}

At $k = \frac{1}{2}$, the argument of cosine is $\frac{(2 \cdot \frac{1}{2} - 1)\pi}{n} = 0$,
so $\cos(0) = 1$ and:
\[
B\left(n, \frac{1}{2}\right) = 1 + \beta(n)
\]

For the Taylor expansion of $\beta(n)$ at $n = \infty$, we expand each factor:
\begin{align}
\cos\frac{\pi}{n} &= 1 - \frac{\pi^2}{2n^2} + \frac{\pi^4}{24n^4} - \cdots \\
\frac{1}{4 - n^2} &= -\frac{1}{n^2} \cdot \frac{1}{1 - 4/n^2}
= -\frac{1}{n^2}\left(1 + \frac{4}{n^2} + \frac{16}{n^4} + \cdots\right)
\end{align}

Combining:
\[
\beta(n) = n^2 \cdot \cos\frac{\pi}{n} \cdot \frac{1}{4 - n^2}
= -1 + \frac{\pi^2/2 - 4}{n^2} + O(1/n^4)
\]

Therefore:
\[
B\left(n, \frac{1}{2}\right) = 1 + \beta(n) = \frac{\pi^2 - 8}{2n^2} + \frac{c_4}{n^4} + \cdots
= \sum_{k=1}^{\infty} \frac{c_{2k}}{n^{2k}}
\]
where $c_2 = \frac{\pi^2 - 8}{2}$ is the leading coefficient.

\textbf{Step 2: Sum over $n \geq 3$.}

For $n \geq 3$, sum term by term:
\[
\sum_{n=3}^{\infty} B\left(n, \frac{1}{2}\right)
= \sum_{n=3}^{\infty} \sum_{k=1}^{\infty} \frac{c_{2k}}{n^{2k}}
= \sum_{k=1}^{\infty} c_{2k} \sum_{n=3}^{\infty} \frac{1}{n^{2k}}
= \sum_{k=1}^{\infty} c_{2k} \left(\zeta(2k) - 1 - \frac{1}{2^{2k}}\right)
\]

\textbf{Step 3: Special values at $n = 1, 2$.}

For $n = 1$: $\beta(1) = \frac{1 \cdot \cos\pi}{4 - 1} = \frac{-1}{3}$, so $B(1, \frac{1}{2}) = 1 - \frac{1}{3} = \frac{2}{3}$.

For $n = 2$: Using L'H\^{o}pital (removable singularity), $\lim_{n \to 2} \beta(n) = -\frac{\pi}{4}$,
so $B(2, \frac{1}{2}) = 1 - \frac{\pi}{4}$.

\textbf{Step 4: Euler's formula converts $\zeta(2k)$ to powers of $\pi$.}

By Euler's formula:
\[
\zeta(2k) = (-1)^{k+1} \frac{(2\pi)^{2k} B_{2k}}{2(2k)!}
\]
where $B_{2k}$ are Bernoulli numbers. Thus each term $c_{2k} \cdot \zeta(2k)$ contributes
a rational multiple of $\pi^{2k}$, and the full sum is a power series in $\pi$.
\end{proof}

\textbf{Truncated approximation} (first 4 terms):
\begin{equation}
\sum_{n=1}^{\infty} B\left(n, \frac{1}{2}\right) \approx
\frac{408960 - 4320\pi - 59040\pi^2 - 867\pi^4 + 384\pi^6 - 8\pi^8}{17280}
\approx 1.244
\end{equation}

\subsection{Mellin-Type Sum}

\begin{theorem}
Define the Mellin-type sum
\begin{equation}
M(s) = \sum_{n \geq 1} \frac{B(n, \frac{1}{2})}{n^{s-1}}
\end{equation}
Then asymptotically:
\begin{equation}
M(s) \sim \frac{\pi^2-8}{2} \cdot \zeta(s+1)
\end{equation}
\end{theorem}

\begin{corollary}[Geometric Representation of Zeta]
The Riemann zeta function admits a representation via the Chebyshev lobe area:
\begin{equation}
\zeta(s+1) \sim \frac{2}{\pi^2-8} \sum_{n=1}^{\infty} \frac{B(n, \frac{1}{2})}{n^{s-1}}
\end{equation}
where $B(n, \frac{1}{2}) = 1 + \frac{n^2 \cos(\pi/n)}{4-n^2}$ is defined purely in terms of
elementary functions (no zeta dependence).
\end{corollary}

\begin{remark}[Non-circularity]
This representation is \emph{not} circular. The function $B(n,k)$ has a closed form
derived from Chebyshev polygon geometry:
\[
B(n, \tfrac{1}{2}) = 1 + \beta(n), \quad \beta(n) = \frac{n^2 \cos(\pi/n)}{4-n^2}
\]
containing only $n$, $\pi$, $\cos$, and rational arithmetic. The appearance of $\zeta$
in the sum is a \emph{discovered consequence}, not a defining dependency.
\end{remark}

\subsection{Mean Value Identity}

\begin{theorem}
For all $n$:
\begin{equation}
\sum_{k=1}^{n} B(n, k) = n
\end{equation}
\end{theorem}

\section{Open Questions}

\begin{enumerate}
\item Why does the digon ($n=2$) singularity contribute exactly 8?
\item Does the $M(s) \sim \zeta(s+1)$ relation have analytic continuation?
\item Can the ``proven RH'' structure inform approaches to the actual RH?
\end{enumerate}

\section{Geometric Interpretation}

The zeros form a \textbf{lattice} in the complex plane:
\begin{itemize}
\item Real coordinates: $\frac{1}{2}, \frac{1}{2} + n, \frac{1}{2} + 2n, \ldots$ (critical lines)
\item Imaginary coordinates: $\pm \delta(n)$ (symmetric about real axis)
\end{itemize}

As $n \to \infty$, the lattice becomes denser along the real axis while maintaining
fixed imaginary offset $\approx 0.2176$.

\end{document}
