\documentclass[11pt,a4paper]{article}
\usepackage{amsmath,amssymb,amsthm}
\usepackage{hyperref}
\usepackage{booktabs}

\newtheorem{theorem}{Theorem}[section]
\newtheorem{corollary}[theorem]{Corollary}
\newtheorem{lemma}[theorem]{Lemma}
\newtheorem{definition}[theorem]{Definition}
\newtheorem{remark}[theorem]{Remark}

\title{Complex Analysis of the Completed Lobe Area Function $B(n,k)$}
\author{Jan Popelka}
\date{December 2025}

\begin{document}
\maketitle

\begin{abstract}
We study the analytic properties of the Completed Lobe Area function $B(n,k)$ as a function
of complex $k$. This Chebyshev-derived function exhibits remarkable parallels to Riemann's
xi function: it is entire, satisfies a functional equation, and has all zeros on a critical line.
Unlike the Riemann Hypothesis, these properties are \emph{proven} from the explicit Fourier structure.
The universal constant $\pi^2 - 8$ emerges in three independent contexts.
\end{abstract}

\section{Definitions}

\begin{definition}[Lobe Area Function]
The lobe area function is defined as
\begin{equation}
A(n,k) = \frac{1}{n} + \alpha(n) \cos\frac{(2k-1)\pi}{n}
\end{equation}
where $\alpha(n) = \frac{n\cos(\pi/n)}{4-n^2} < 0$ for $n > 2$.
\end{definition}

\begin{definition}[Completed Lobe Area]
The completed lobe area function is
\begin{equation}
B(n,k) = n \cdot A(n,k) = 1 + \beta(n) \cos\frac{(2k-1)\pi}{n}
\end{equation}
where $\beta(n) = \frac{n^2\cos(\pi/n)}{4-n^2}$.
\end{definition}

\section{Key Results}

\subsection{Holomorphicity}

\begin{theorem}
$B(n,k)$ is an entire function in $k$ (holomorphic on all of $\mathbb{C}$).
\end{theorem}

\begin{proof}
$B(n,k)$ is a composition of polynomial and cosine functions, both entire.
\end{proof}

\subsection{Symmetries}

\begin{theorem}[Symmetry Properties]
The function $B(n,k)$ satisfies:
\begin{enumerate}
\item Periodicity: $B(n, k+n) = B(n, k)$
\item Reflection: $B(n, 1-k) = B(n, k)$
\item Complementary: $B(n, n+1-k) = B(n, k)$
\item Even in $n$: $B(-n, k) = B(n, k)$
\end{enumerate}
\end{theorem}

\subsection{Critical Points}

\begin{theorem}
The critical points of $B(n,k)$ for real $k$ are:
\begin{itemize}
\item Maximum at $k = \frac{n+1}{2}$, with value $\to 2$ as $n \to \infty$
\item Minimum at $k = \frac{1}{2}$, with value $\to 0$ as $n \to \infty$
\end{itemize}
\end{theorem}

\subsection{Zeros}

\begin{theorem}[Closed-Form Zeros]
The zeros of $B(n,k)$ are located at
\begin{equation}
k = \frac{1}{2} + mn \pm i\delta(n), \quad m \in \mathbb{Z}
\end{equation}
where
\begin{equation}
\delta(n) = \frac{n}{2\pi} \operatorname{arccosh}\left(-\frac{1}{\beta(n)}\right)
= \frac{n}{2\pi} \operatorname{arccosh}\frac{n^2 - 4}{n^2 \cos(\pi/n)}
\end{equation}
\end{theorem}

\begin{remark}[Removable Singularity at $n=2$]
The formula for $\beta(n)$ has a $\frac{0}{0}$ form at $n = 2$:
\[
\beta(2) = \frac{4 \cdot \cos(\pi/2)}{4 - 4} = \frac{0}{0}
\]
By L'H\^{o}pital's rule:
\[
\lim_{n \to 2} \beta(n) = -\frac{\pi}{4}, \qquad
\lim_{n \to 2} \delta(n) = \frac{\operatorname{arccosh}(4/\pi)}{\pi} \approx 0.23026
\]
\end{remark}

\begin{theorem}[Universal Limit]
As $n \to \infty$,
\begin{equation}
\delta(n) \to \delta_\infty = \frac{\sqrt{\pi^2 - 8}}{2\pi} = 0.21761808912708625\ldots
\end{equation}
\end{theorem}

\begin{proof}
For large $n$:
\begin{align}
\beta(n) &= -1 + \frac{\pi^2/2 - 4}{n^2} + O(1/n^4) \\
-\frac{1}{\beta(n)} &= 1 + \frac{\pi^2/2 - 4}{n^2} + O(1/n^4)
\end{align}
Using $\operatorname{arccosh}(1 + \epsilon) \sim \sqrt{2\epsilon}$ for small $\epsilon$:
\[
\operatorname{arccosh}\left(-\frac{1}{\beta(n)}\right) \sim \frac{\sqrt{\pi^2 - 8}}{n}
\]
Therefore $\delta(n) = \frac{n}{2\pi} \cdot \frac{\sqrt{\pi^2 - 8}}{n} = \frac{\sqrt{\pi^2 - 8}}{2\pi}$.
\end{proof}

\subsection{Critical Line Property}

\begin{theorem}
All zeros of $B(n,k)$ lie on the critical lines $\operatorname{Re}(k) \equiv \frac{1}{2} \pmod{n}$.
\end{theorem}

\begin{proof}
From the closed form, zeros are at $k = \frac{1}{2} + mn \pm i\delta(n)$,
so $\operatorname{Re}(k) = \frac{1}{2} + mn \equiv \frac{1}{2} \pmod{n}$.
\end{proof}

\begin{remark}
Unlike the Riemann Hypothesis, this is not a conjecture but a \emph{proven fact}
following from the explicit Fourier structure.
\end{remark}

\subsection{No Real Zeros}

\begin{theorem}
$B(n,k)$ has no real zeros for any $n \geq 1$.
\end{theorem}

\begin{proof}
Zeros occur where $\cos\frac{(2k-1)\pi}{n} = -\frac{1}{\beta(n)}$.
For all $n \geq 1$, we have $-\frac{1}{\beta(n)} > 1$, so the equation has no real solutions.
\end{proof}

\subsection{Hadamard Factorization}

\begin{theorem}
As an entire function of order 1, $B(n,k)$ admits the product representation:
\begin{equation}
B(n,k) = B\left(n, \frac{n+1}{2}\right) \cdot \prod_{m \in \mathbb{Z}}
\left[1 + \left(\frac{k - \frac{1}{2} - mn}{\delta(n)}\right)^2\right]
\end{equation}
\end{theorem}

\section{Comparison with Riemann Xi}

\begin{center}
\begin{tabular}{lcc}
\toprule
Property & Riemann $\xi(s)$ & $B(n,k)$ \\
\midrule
Type & Entire function & Entire function \\
Order & 1 & 1 \\
Critical line & $\operatorname{Re}(s) = \frac{1}{2}$ & $\operatorname{Re}(k) \equiv \frac{1}{2} \pmod{n}$ \\
Zeros on CL & Conjectured (RH) & \textbf{Proven} \\
Functional equation & $\xi(s) = \xi(1-s)$ & $B(n,1-k) = B(n,k)$ \\
Zero locations & Unknown & Explicit formula \\
\bottomrule
\end{tabular}
\end{center}

\section{The Universal Constant $\pi^2 - 8$}

The constant $\pi^2 - 8 \approx 1.8696$ appears in three independent contexts:

\subsection{Zero Offset}
\begin{equation}
\delta_\infty = \frac{\sqrt{\pi^2 - 8}}{2\pi} \approx 0.2176
\end{equation}

\subsection{Asymptotic Decay}
\begin{equation}
B\left(n, \frac{1}{2}\right) \sim \frac{\pi^2 - 8}{2n^2} \quad \text{as } n \to \infty
\end{equation}

\subsection{Multiplicativity Correction}
\begin{equation}
\frac{B(mn, \frac{1}{2})}{B(m, \frac{1}{2}) \cdot B(n, \frac{1}{2})} \to \frac{2}{\pi^2 - 8} \approx 1.0697
\end{equation}
This shows $B$ is \emph{not} multiplicative, but the deviation is controlled by $\pi^2 - 8$.

\subsection{Algebraic Origin}

The constant $\pi^2 - 8$ is a \emph{hybrid} of geometry and algebra:
\begin{itemize}
\item \textbf{Source of $\pi^2$}: From Taylor expansion $\cos(\pi/n) = 1 - \frac{\pi^2}{2n^2} + O(1/n^4)$
\item \textbf{Source of 8}: From the pole at $n = 2$ (digon): $\beta(n) = \frac{n^2\cos(\pi/n)}{4-n^2}$
\end{itemize}

\section{Connection to Riemann Zeta}

\subsection{Sum Formula}

\begin{theorem}
The sum of $B(n, \frac{1}{2})$ over all $n \geq 1$ equals a power series in $\pi$:
\begin{equation}
\sum_{n=1}^{\infty} B\left(n, \frac{1}{2}\right) =
\frac{2}{3} + \left(1 - \frac{\pi}{4}\right) +
\sum_{k=1}^{\infty} c_{2k} \cdot \left(\zeta(2k) - 1 - 2^{-2k}\right)
\end{equation}
where $c_{2k}$ are the Taylor coefficients of $B(n, \frac{1}{2})$ at $n = \infty$.
\end{theorem}

\begin{proof}
\textbf{Step 1: Taylor expansion of $B(n, \frac{1}{2})$.}

At $k = \frac{1}{2}$, the argument of cosine is $\frac{(2 \cdot \frac{1}{2} - 1)\pi}{n} = 0$,
so $\cos(0) = 1$ and:
\[
B\left(n, \frac{1}{2}\right) = 1 + \beta(n)
\]

For the Taylor expansion of $\beta(n)$ at $n = \infty$, we expand each factor:
\begin{align}
\cos\frac{\pi}{n} &= 1 - \frac{\pi^2}{2n^2} + \frac{\pi^4}{24n^4} - \cdots \\
\frac{1}{4 - n^2} &= -\frac{1}{n^2} \cdot \frac{1}{1 - 4/n^2}
= -\frac{1}{n^2}\left(1 + \frac{4}{n^2} + \frac{16}{n^4} + \cdots\right)
\end{align}

Combining:
\[
\beta(n) = n^2 \cdot \cos\frac{\pi}{n} \cdot \frac{1}{4 - n^2}
= -1 + \frac{\pi^2/2 - 4}{n^2} + O(1/n^4)
\]

Therefore:
\[
B\left(n, \frac{1}{2}\right) = 1 + \beta(n) = \frac{\pi^2 - 8}{2n^2} + \frac{c_4}{n^4} + \cdots
= \sum_{k=1}^{\infty} \frac{c_{2k}}{n^{2k}}
\]
where $c_2 = \frac{\pi^2 - 8}{2}$ is the leading coefficient.

\textbf{Step 2: Sum over $n \geq 3$.}

For $n \geq 3$, sum term by term:
\[
\sum_{n=3}^{\infty} B\left(n, \frac{1}{2}\right)
= \sum_{n=3}^{\infty} \sum_{k=1}^{\infty} \frac{c_{2k}}{n^{2k}}
= \sum_{k=1}^{\infty} c_{2k} \sum_{n=3}^{\infty} \frac{1}{n^{2k}}
= \sum_{k=1}^{\infty} c_{2k} \left(\zeta(2k) - 1 - \frac{1}{2^{2k}}\right)
\]

\textbf{Step 3: Special values at $n = 1, 2$.}

For $n = 1$: $\beta(1) = \frac{1 \cdot \cos\pi}{4 - 1} = \frac{-1}{3}$, so $B(1, \frac{1}{2}) = 1 - \frac{1}{3} = \frac{2}{3}$.

For $n = 2$: Using L'H\^{o}pital (removable singularity), $\lim_{n \to 2} \beta(n) = -\frac{\pi}{4}$,
so $B(2, \frac{1}{2}) = 1 - \frac{\pi}{4}$.

\textbf{Step 4: Euler's formula converts $\zeta(2k)$ to powers of $\pi$.}

By Euler's formula:
\[
\zeta(2k) = (-1)^{k+1} \frac{(2\pi)^{2k} B_{2k}}{2(2k)!}
\]
where $B_{2k}$ are Bernoulli numbers. Thus each term $c_{2k} \cdot \zeta(2k)$ contributes
a rational multiple of $\pi^{2k}$, and the full sum is a power series in $\pi$.
\end{proof}

\textbf{Truncated approximation} (first 4 terms):
\begin{equation}
\sum_{n=1}^{\infty} B\left(n, \frac{1}{2}\right) \approx
\frac{408960 - 4320\pi - 59040\pi^2 - 867\pi^4 + 384\pi^6 - 8\pi^8}{17280}
\approx 1.244
\end{equation}

\subsection{Mellin-Type Sum}

\begin{theorem}[Mellin-Zeta Identity]
Define the Mellin-type sum
\begin{equation}
M(s) = \sum_{n \geq 1} \frac{B(n, \frac{1}{2})}{n^{s-1}}
\end{equation}
Then \textbf{exactly}:
\begin{equation}
M(s) = \frac{2}{3} + \frac{1 - \pi/4}{2^{s-1}} +
\sum_{k=1}^{\infty} c_{2k} \cdot \left(\zeta(s + 2k - 1) - 1 - 2^{-(s+2k-1)}\right)
\end{equation}
where $c_{2k}$ are the Taylor coefficients of $B(n, \frac{1}{2})$.

\end{theorem}

\begin{remark}[No Simple Asymptotic]
One might expect $M(s) \sim c_2 \cdot \zeta(s+1)$ for large $s$, but this is \textbf{false}.
Numerical verification shows the ratio $M(s)/(c_2 \cdot \zeta(s+1)) \to 0.71$ as $s \to \infty$,
not 1. The reason: for large $s$, the Dirichlet series is dominated by the $n=1$ term
($B(1, \frac{1}{2}) = \frac{2}{3}$), which does not follow the Taylor expansion valid for large $n$.

Additionally, even for the tail sum $\sum_{n \geq 3}$, the higher zeta terms maintain a
\emph{fixed ratio} to the leading term (approximately 3--4\%), not vanishing.
\end{remark}

\begin{remark}[Structural Connection]
The identity (17) reveals a deep connection between the Chebyshev lobe geometry and
the Riemann zeta function. Each Taylor coefficient $c_{2k}$ of $B(n, \frac{1}{2})$
couples to $\zeta(s + 2k - 1)$. This is \emph{not} a representation of zeta in terms
of $B$ alone, but rather a structural identity relating both functions.

The function $B(n,k)$ has a closed form derived from Chebyshev polygon geometry:
\[
B(n, \tfrac{1}{2}) = 1 + \beta(n), \quad \beta(n) = \frac{n^2 \cos(\pi/n)}{4-n^2}
\]
containing only $n$, $\pi$, $\cos$, and rational arithmetic---no zeta dependence.
\end{remark}

\subsection{Mean Value Identity}

\begin{theorem}
For all $n$:
\begin{equation}
\sum_{k=1}^{n} B(n, k) = n
\end{equation}
\end{theorem}

\subsection{Computational Method for Zeta Values}

The Mellin-Zeta identity (17) leads to a practical computational method for evaluating
$\zeta(s)$ at complex arguments with $\operatorname{Re}(s) > 2$.

\begin{theorem}[Triangular System for Zeta]
Define $M_3(s) = \sum_{n=3}^{\infty} \frac{B(n, \frac{1}{2})}{n^{s-1}}$ (converges for $\operatorname{Re}(s) > 0$).
For a starting point $s_0$ with $\operatorname{Re}(s_0) > 0$, consider the values
\[
M_3(s_0), \; M_3(s_0 + 2), \; M_3(s_0 + 4), \; \ldots
\]
These satisfy an upper triangular linear system with unknowns
\[
Z_k = \zeta(s_0 + 2k - 1) - 1 - 2^{-(s_0 + 2k - 1)}, \quad k = 1, 2, 3, \ldots
\]
The coefficient matrix has entries $C_{jk} = c_{2(k-j+1)}$ for $k \geq j$ (and 0 otherwise),
where $c_{2m}$ are the Taylor coefficients of $B(n, \frac{1}{2})$.
\end{theorem}

\begin{remark}[Numerical Verification]
Truncating to $N$ equations yields approximations with rapidly improving accuracy:
\begin{center}
\begin{tabular}{lcc}
\toprule
Equations & $\zeta(3)$ error & $\zeta(5)$ error \\
\midrule
5 & $10^{-7}$ & $10^{-11}$ \\
10 & $5 \times 10^{-9}$ & $10^{-17}$ \\
\bottomrule
\end{tabular}
\end{center}
The method works for complex arguments: $\zeta(3+i)$, $\zeta(7/2)$, etc.
\end{remark}

\begin{remark}[Computational Assessment---Honest Evaluation]
While the triangular system provides an elegant identity, its \textbf{practical computational value is negligible}:

\begin{center}
\begin{tabular}{ll}
\toprule
Criterion & Result \\
\midrule
Speed & $\sim$3000$\times$ slower than built-in $\zeta$ \\
Region & $\operatorname{Re}(s) > 1$ only (where Dirichlet series converges anyway) \\
Accuracy & $10^{-4}$ to $10^{-5}$ (worse than direct methods) \\
Critical strip & Inaccessible (Gap $\frac{1}{2}$ barrier) \\
\bottomrule
\end{tabular}
\end{center}

The value of this work lies in the \textbf{identity itself}---an unexpected connection
between Chebyshev polygon geometry and the Riemann zeta function---not in computational
utility. The constant $\pi^2 - 8$ appearing in both contexts suggests deeper structure,
but the method provides no new information about $\zeta$ that simpler approaches cannot obtain.

For practical zeta computation, standard methods (Euler-Maclaurin, Riemann-Siegel, eta function)
remain vastly superior.
\end{remark}

\begin{remark}[Limitation: Critical Line]
To compute $\zeta(\tfrac{1}{2} + it)$ would require $s_0 = -\tfrac{1}{2} + it$,
but $M_3(s)$ only converges for $\operatorname{Re}(s) > 0$. Thus the critical line
is not directly accessible via this method. Analytic continuation of $M_3(s)$
remains an open question.
\end{remark}

\subsection{The Gap $\frac{1}{2}$ Barrier}

A natural question: can we choose a different $k$ value to extend the convergence region
and reach the critical line? We systematically investigated this.

\begin{theorem}[Decay Classification]
The asymptotic decay of $B(n,k)$ depends on the choice of $k$:
\begin{center}
\begin{tabular}{lll}
\toprule
Choice of $k$ & Decay & Convergence of $M_k(s)$ \\
\midrule
$k = \frac{1}{2}$ & $O(1/n^2)$ & $\operatorname{Re}(s) > 0$ \\
$k = m$ (integer) & $O(1/n^3)$ & $\operatorname{Re}(s) > -1$ \\
$k = \alpha n$ (linear) & $O(1)$ & Diverges \\
$k = \sqrt{n}$ & $O(1/n^{3/2})$ & $\operatorname{Re}(s) > \frac{1}{2}$ \\
\bottomrule
\end{tabular}
\end{center}
\end{theorem}

\begin{proof}
For integer $k$, we have the lobe area formula $B(n,k) = \pi k/n - \sin(2\pi k/n)/2$.
Taylor expansion gives:
\[
B(n,k) = \frac{2\pi^3 k^3}{3n^3} - \frac{2\pi^5 k^5}{15 n^5} + O(1/n^7)
\]
The leading term is $O(1/n^3)$ for any fixed integer $k$.

For $k = \alpha n$ with $0 < \alpha < 1$:
\[
B(n, \alpha n) = \pi\alpha - \frac{\sin(2\pi\alpha)}{2} = \text{constant}
\]
No decay at all---the series diverges.

For $k = \sqrt{n}$, the decay is $O(1/n^{3/2})$, \emph{slower} than fixed integer $k$.
\end{proof}

\begin{theorem}[Universal Gap $\frac{1}{2}$]
For any choice of $k$, the critical line $\operatorname{Re}(s) = \frac{1}{2}$ lies exactly
$\frac{1}{2}$ outside the convergence region for accessing $\zeta(\frac{1}{2} + it)$.

Specifically, if $M_k(s)$ involves $\zeta(s + d)$ as its leading zeta term (where $d$ is
the offset determined by the Taylor expansion), then:
\begin{itemize}
\item To access $\zeta(\frac{1}{2} + it)$, we need $s = \frac{1}{2} - d + it$
\item The convergence region is $\operatorname{Re}(s) > 1 - d$
\item The gap is: $(1 - d) - (\frac{1}{2} - d) = \frac{1}{2}$
\end{itemize}
\end{theorem}

\begin{center}
\begin{tabular}{lccccc}
\toprule
$k$ & Offset $d$ & Need $s$ for $\zeta(\frac{1}{2})$ & Convergence & Gap \\
\midrule
$\frac{1}{2}$ & 1 & $s = -\frac{1}{2}$ & $\operatorname{Re}(s) > 0$ & $\frac{1}{2}$ \\
1 (integer) & 2 & $s = -\frac{3}{2}$ & $\operatorname{Re}(s) > -1$ & $\frac{1}{2}$ \\
\bottomrule
\end{tabular}
\end{center}

\begin{remark}[The Critical Line as Fixed Point]
This universal gap of $\frac{1}{2}$ is not coincidental---it reflects a deeper truth.
The critical line $\operatorname{Re}(s) = \frac{1}{2}$ is the \textbf{symmetry axis}
of Riemann's functional equation $\zeta(s) = \chi(s)\zeta(1-s)$.

Attempts to overcome the barrier via functional equation fail because:
\begin{itemize}
\item At $s = \frac{1}{2}$, the equation relates $\zeta(\frac{1}{2})$ to itself (tautology)
\item At $s = \frac{1}{2} + it$, it relates conjugate values on the same line
\item No ``external'' information is provided---the critical line maps to itself
\end{itemize}

This explains why the critical line is simultaneously:
\begin{enumerate}
\item Just outside convergence for our $M_k(s)$ series (gap $\frac{1}{2}$)
\item A fixed set of the functional equation (no new constraints)
\item The locus of the Riemann Hypothesis (hardest to analyze)
\end{enumerate}

All three approaches to overcome the barrier were tested and failed:
\begin{center}
\begin{tabular}{ll}
\toprule
Approach & Result \\
\midrule
Ramanujan summation & Requires analytic continuation (circular) \\
$B$-symmetry $\to$ functional eq. & Gives $M_{1-k}(s) = M_k(s)$, not $s$-reflection \\
Integral representation & Regularization needs $\zeta$ values (circular) \\
Self-referential trick & $\zeta(\frac{1}{2})$ appears with coefficient 1 (tautology) \\
\bottomrule
\end{tabular}
\end{center}

The $B(n,k)$ framework elegantly reaches the boundary of what is achievable without
invoking the functional equation of $\zeta$ itself---and then stops.
\end{remark}

\subsection{Breakthrough: Exact Identity for $n^{-s}$ via $B$}

A key observation overcomes the Gap $\frac{1}{2}$ barrier by evaluating $B(n,k)$ at
\emph{complex} values of $k$.

\begin{theorem}[Exact $n^{-s}$ Identity]
For any $n \geq 2$ and $s \in \mathbb{C}$, define
\begin{equation}
k_s(n) = \frac{1}{2} - \frac{i s n \log n}{2\pi}
\end{equation}
Then:
\begin{equation}
n^{-s} = \frac{B(n, k_s) - 1}{\beta(n)} + \frac{in}{2\pi\beta(n)} \cdot \left.\frac{\partial B}{\partial k}\right|_{k=k_s}
\end{equation}
where $\beta(n) = \frac{n^2\cos(\pi/n)}{4-n^2}$ (with $\beta(2) = -\frac{\pi}{4}$ by L'H\^{o}pital).
\end{theorem}

\begin{proof}
The argument of cosine at $k = k_s$ is:
\[
\frac{(2k_s - 1)\pi}{n} = -is\log n
\]
Therefore:
\[
\cos\left(\frac{(2k_s-1)\pi}{n}\right) = \cos(-is\log n) = \cosh(s\log n) = \frac{n^s + n^{-s}}{2}
\]
Similarly:
\[
\sin\left(\frac{(2k_s-1)\pi}{n}\right) = i\sinh(s\log n) = i\frac{n^s - n^{-s}}{2}
\]
From $B(n,k) = 1 + \beta(n)\cos(\cdot)$ and $\frac{\partial B}{\partial k} = -\beta(n)\frac{2\pi}{n}\sin(\cdot)$,
we obtain $n^s + n^{-s}$ and $n^s - n^{-s}$. Solving for $n^{-s}$ yields the identity.
\end{proof}

\begin{corollary}[Dirichlet Eta via $B$]
The Dirichlet eta function can be expressed entirely in terms of $B$:
\begin{equation}
\eta(s) = \sum_{n=1}^{\infty} (-1)^{n-1} n^{-s} = 1 + \sum_{n=2}^{\infty} (-1)^{n-1} \left[\frac{B(n,k_s)-1}{\beta(n)} + \frac{in}{2\pi\beta(n)}\frac{\partial B}{\partial k}\Big|_{k_s}\right]
\end{equation}
This series converges for $\operatorname{Re}(s) > 0$, including the critical line.
\end{corollary}

\begin{corollary}[Zeta on Critical Line via $B$]
Since $\zeta(s) = \frac{\eta(s)}{1 - 2^{1-s}}$, we can express $\zeta(s)$ on the critical line
$\operatorname{Re}(s) = \frac{1}{2}$ entirely in terms of $B(n,k)$ evaluated at complex $k$.
\end{corollary}

\begin{remark}[Numerical Verification]
The identity $n^{-s}$ via $B$ is exact to machine precision ($\sim 10^{-15}$ error).
For $\zeta(3)$ via this method: error $\sim 10^{-8}$ with 200 terms.
For $\zeta(\frac{1}{2} + 14.13i)$: error $\sim 0.01$ with 500 terms (slow convergence of
alternating series on critical line, but \emph{does converge}).
\end{remark}

\begin{remark}[Significance]
This result shows that $B(n,k)$ at complex $k$ contains \emph{all information} about the
Riemann zeta function, including values on the critical line. The Gap $\frac{1}{2}$ barrier
for real $k$ is bypassed by analytic continuation to complex $k$.

However, this does not trivialize the Riemann Hypothesis: the slow convergence on the
critical line means practical computation still requires many terms, and the location of
zeros is encoded non-obviously in the interplay of infinitely many $B(n,k_s)$ values.
\end{remark}

\subsection{Geometric Interpretation: Wick Rotation}

The complex $k$ values used in the exact identity have a natural geometric interpretation
as a \textbf{Wick rotation} from circular to hyperbolic geometry.

\begin{theorem}[Circle-to-Hyperbola Transition]
The fundamental identity
\begin{equation}
\cos(i\varphi) = \cosh(\varphi)
\end{equation}
transforms circular geometry (bounded) into hyperbolic geometry (unbounded):
\begin{center}
\begin{tabular}{lccc}
\toprule
Argument & Function & Curve & Range \\
\midrule
Real $\theta$ & $\cos\theta, \sin\theta$ & Circle $x^2 + y^2 = 1$ & Bounded $[-1,1]$ \\
Imaginary $i\varphi$ & $\cosh\varphi, \sinh\varphi$ & Hyperbola $x^2 - y^2 = 1$ & Unbounded $[1,\infty)$ \\
\bottomrule
\end{tabular}
\end{center}
\end{theorem}

\begin{definition}[Circular vs Hyperbolic Chebyshev Structures]
\textbf{Circular case} (real $k$):
\begin{itemize}
\item Polygon vertices at $e^{2\pi ij/n}$ on unit circle
\item Chebyshev curve: $T_n(\cos\theta) = \cos(n\theta)$, bounded
\item Lobes: bounded regions with area $B(n,k) \in [B_{\min}, B_{\max}]$
\end{itemize}

\textbf{Hyperbolic case} (complex $k$):
\begin{itemize}
\item ``Vertices'' at $(\cosh(t_j), \sinh(t_j))$ on hyperbola
\item Hyperbolic Chebyshev: $T_n(\cosh\varphi) = \cosh(n\varphi)$, unbounded
\item ``Lobes'': unbounded regions extending to infinity
\end{itemize}
\end{definition}

\begin{remark}[Why Hyperbolic Geometry Enables $n^{-s}$]
For the special value $k_s = \frac{1}{2} - \frac{isn\log n}{2\pi}$:
\begin{equation}
\frac{(2k_s - 1)\pi}{n} = -is\log n
\end{equation}
Therefore:
\begin{equation}
\cos\left(-is\log n\right) = \cosh(s\log n) = \frac{n^s + n^{-s}}{2}
\end{equation}
The \textbf{unboundedness} of hyperbolic geometry allows $B(n,k)$ to take values like
$(n^s + n^{-s})/2$ for any $s$, which would be impossible in the bounded circular setting
where $\cos(\cdot) \in [-1,1]$.
\end{remark}

\begin{remark}[Physical Analogy]
In physics, \textbf{Wick rotation} $t \to i\tau$ transforms:
\begin{itemize}
\item Minkowski spacetime $\leftrightarrow$ Euclidean spacetime
\item Oscillating solutions $e^{i\omega t}$ $\leftrightarrow$ Exponential decay $e^{-\omega\tau}$
\item Real frequencies $\leftrightarrow$ Imaginary (evanescent) modes
\end{itemize}
Similarly here, extending $k$ to complex values transforms:
\begin{itemize}
\item Bounded circular lobe areas $\leftrightarrow$ Unbounded hyperbolic ``lobe areas''
\item Oscillating $\cos(\cdot)$ $\leftrightarrow$ Exponential $\cosh(\cdot)$
\item Access to $B \in [B_{\min}, B_{\max}]$ $\leftrightarrow$ Access to $n^{\pm s}$ for any $s$
\end{itemize}
\end{remark}

\begin{theorem}[Hyperbolic Lobe Area and Sign Change]
For $k = \frac{1}{2} + ib$ with $b \in \mathbb{R}$, the function
\begin{equation}
B\left(n, \tfrac{1}{2} + ib\right) = 1 + \beta(n) \cosh\frac{2b\pi}{n}
\end{equation}
represents a \textbf{signed hyperbolic lobe area} that changes sign at $b = \delta(n)$:
\begin{center}
\begin{tabular}{lcl}
\toprule
Region & Value & Interpretation \\
\midrule
$b < \delta(n)$ & $B > 0$ & Positive hyperbolic area \\
$b = \delta(n)$ & $B = 0$ & \textbf{Zero of $B(n,k)$} \\
$b > \delta(n)$ & $B < 0$ & Negative hyperbolic area \\
\bottomrule
\end{tabular}
\end{center}
where $\delta(n) = \frac{n}{2\pi}\operatorname{arccosh}\left(-\frac{1}{\beta(n)}\right)$
is the imaginary offset from the critical line $\operatorname{Re}(k) = \frac{1}{2}$.
\end{theorem}

\begin{proof}
Since $\beta(n) < 0$ for all $n > 2$, and $\cosh(x) \geq 1$ for all real $x$:
\begin{itemize}
\item At $b = 0$: $B = 1 + \beta(n) > 0$ (since $|\beta(n)| < 1$ for large $n$)
\item As $b \to \infty$: $\cosh(2b\pi/n) \to \infty$, so $B \to -\infty$
\item $B = 0$ when $\cosh(2b\pi/n) = -1/\beta(n)$, i.e., $b = \delta(n)$
\end{itemize}
\end{proof}

\begin{remark}[Geometric Meaning of Zeros]
The zeros of $B(n,k)$ on the critical line are \emph{not} arbitrary points---they are
precisely the values where the hyperbolic lobe area transitions from positive to negative.
This provides a natural geometric interpretation: just as a function's roots are where it
crosses the $x$-axis, the zeros of $B(n,k)$ are where the hyperbolic ``area'' changes sign.

The universal limit $\delta_\infty = \frac{\sqrt{\pi^2-8}}{2\pi}$ is therefore the
asymptotic sign-change threshold for hyperbolic lobe areas as $n \to \infty$.
\end{remark}

\begin{remark}[Assessment]
This interpretation elevates the Wick rotation from a mere algebraic trick to a
geometrically meaningful extension. The zeros of $B(n,k)$ now have intrinsic meaning
as sign-change boundaries in hyperbolic geometry, analogous to how roots of polynomials
mark transitions between positive and negative regions.

While this does not provide new computational capabilities for studying zeta zeros,
it demonstrates that the $B(n,k)$ framework has genuine geometric depth beyond its
original circular polygon setting.
\end{remark}

\subsection{Hyperbolic Area Invariance}

The Chebyshev integral theorem (area invariance) extends naturally to the hyperbolic setting.

\begin{theorem}[Hyperbolic Conservation Law]
For any $n \geq 2$ and any $b \in \mathbb{C}$:
\begin{equation}
\sum_{k=1}^{n} B(n, k+ib) = n
\end{equation}
where $\beta(2) = -\frac{\pi}{4}$ is defined by L'H\^{o}pital's rule.
The case $n=1$ is excluded as geometrically undefined (no 1-gon exists).
\end{theorem}

\begin{proof}
Expanding $B(n, k+ib) = 1 + \beta(n)\cos\frac{(2(k+ib)-1)\pi}{n}$:
\begin{align}
\sum_{k=1}^{n} B(n, k+ib) &= n + \beta(n) \sum_{k=1}^{n} \cos\left(\frac{(2k-1)\pi}{n} + \frac{2ib\pi}{n}\right) \\
&= n + \beta(n)\cosh\frac{2b\pi}{n} \sum_{k=1}^{n}\cos\frac{(2k-1)\pi}{n}
   - i\beta(n)\sinh\frac{2b\pi}{n} \sum_{k=1}^{n}\sin\frac{(2k-1)\pi}{n}
\end{align}

\textbf{Key lemma:} Let $\omega = e^{2\pi i/n}$. Then:
\[
\sum_{k=1}^{n} e^{i(2k-1)\pi/n} = e^{i\pi/n} \sum_{j=0}^{n-1} \omega^j
= e^{i\pi/n} \cdot \frac{1 - \omega^n}{1 - \omega} = e^{i\pi/n} \cdot \frac{1-1}{1-\omega} = 0
\]
Therefore $\sum \cos\frac{(2k-1)\pi}{n} = \operatorname{Re}(0) = 0$ and
$\sum \sin\frac{(2k-1)\pi}{n} = \operatorname{Im}(0) = 0$.

Substituting: $\sum_{k=1}^{n} B(n, k+ib) = n + \beta(n)(\cosh(\cdot)\cdot 0 - i\sinh(\cdot)\cdot 0) = n$.
\end{proof}

\begin{remark}[Physical Interpretation]
This is a \textbf{conservation law} for hyperbolic lobe areas:
\begin{center}
\begin{tabular}{lcc}
\toprule
Setting & Individual lobes & Sum \\
\midrule
Circular ($b=0$) & All positive & $n$ \\
Hyperbolic ($b \neq 0$) & Some positive, some negative & $n$ \\
Limit ($b \to \infty$) & Diverge to $\pm\infty$ & $n$ \\
\bottomrule
\end{tabular}
\end{center}
The excess of positive lobes exactly compensates the deficit of negative lobes,
analogous to energy conservation where kinetic and potential energies exchange
but total energy remains constant.
\end{remark}

\begin{remark}[Relation to Zeta]
The area invariance sums over $k$ for fixed $n$, while the Dirichlet eta sums over $n$
with $k = k_s(n)$ depending on $n$. These are different summation structures, so the
conservation law does not directly constrain the eta function. However, it demonstrates
that the $B(n,k)$ framework maintains internal consistency in the hyperbolic extension.
\end{remark}

\begin{remark}[Numerical Verification]
The theorem was subjected to Popper-style falsification testing:
\begin{itemize}
\item $n \in \{3, 5, 7, 10, 20, 50, 100\}$: All pass with error $< 10^{-15}$
\item Real $b \in [-100, 100]$: All pass (high precision needed for $|b| > 50$)
\item Complex $b$: Tested with random $b \in \mathbb{C}$, max error $\sim 10^{-12}$
\item $n = 2$ with $\beta(2) = -\pi/4$: Passes exactly
\item $n = 1$: Fails (but geometrically meaningless---excluded from theorem)
\end{itemize}
Status: \textbf{Numerically verified} for $n \geq 2$.
\end{remark}

\section{Open Questions}

\begin{enumerate}
\item Why does the digon ($n=2$) singularity contribute exactly 8 to $\pi^2 - 8$?
\item Is there a closed-form expression for $\sum_{n=1}^{\infty} B(n, \frac{1}{2})$ not involving infinite zeta sums?
\item Can the ``proven RH'' structure for $B(n,k)$ inform approaches to the actual Riemann Hypothesis?
\item What is the geometric meaning of the Taylor coefficients $c_{2k}$ coupling to $\zeta(s+2k-1)$?
\item Does $M_3(s)$ admit analytic continuation to $\operatorname{Re}(s) < 0$, enabling computation
      of $\zeta$ on the critical line via the triangular system method?
\item \textbf{Is the Gap $\frac{1}{2}$ barrier fundamental?} Can any regularization, functional equation,
      or integral representation overcome this barrier and provide non-circular access to $\zeta(\frac{1}{2}+it)$
      via the $B(n,k)$ framework?
\item Does the symmetry $B(n,1-k) = B(n,k)$ induce a functional equation for $M_k(s)$ analogous
      to the Riemann functional equation $\xi(s) = \xi(1-s)$?
\item[8.] \textbf{[ANSWERED]} What is the precise analog of ``lobe area'' in hyperbolic geometry?
      \emph{Answer:} $B(n,k)$ at complex $k$ is a signed hyperbolic area;
      zeros correspond to sign-change boundaries (Theorem 7.9).
\end{enumerate}

\section{Geometric Interpretation}

The zeros form a \textbf{lattice} in the complex plane:
\begin{itemize}
\item Real coordinates: $\frac{1}{2}, \frac{1}{2} + n, \frac{1}{2} + 2n, \ldots$ (critical lines)
\item Imaginary coordinates: $\pm \delta(n)$ (symmetric about real axis)
\end{itemize}

As $n \to \infty$, the lattice becomes denser along the real axis while maintaining
fixed imaginary offset $\approx 0.2176$.

\end{document}
