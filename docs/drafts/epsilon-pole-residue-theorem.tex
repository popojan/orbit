\documentclass[11pt]{amsart}

\usepackage{amsmath,amssymb,amsthm}
\usepackage{hyperref}

\newtheorem{theorem}{Theorem}[section]
\newtheorem{lemma}[theorem]{Lemma}
\newtheorem{corollary}[theorem]{Corollary}
\newtheorem{proposition}[theorem]{Proposition}
\theoremstyle{definition}
\newtheorem{definition}[theorem]{Definition}
\theoremstyle{remark}
\newtheorem{remark}[theorem]{Remark}

\DeclareMathOperator{\dist}{dist}

\title{Epsilon-Pole Residues and Prime Factorization Structure}

\author{Jan Popelka}
\address{Prague, Czech Republic}
\email{popojan@protonmail.com}

\date{\today}

\begin{document}

\begin{abstract}
We establish a connection between the residue structure of a regularized double sum and the factorization properties of positive integers. For $n \geq 2$ and $\alpha > 1/2$, we define
\[
F_n(\alpha, \varepsilon) = \sum_{d=2}^{\infty} \sum_{k=0}^{\infty} \left[(n - kd - d^2)^2 + \varepsilon\right]^{-\alpha}
\]
and prove that the residue at $\varepsilon = 0$ equals the number of non-trivial divisors of $n$ up to $\sqrt{n}$. This yields a new analytic characterization of primality: an integer $n$ is prime if and only if $F_n(\alpha, \varepsilon)$ is analytic at $\varepsilon = 0$.
\end{abstract}

\maketitle

\section{Introduction}

Classical analytic number theory connects prime numbers to the pole structure of $L$-functions, most notably through the Riemann zeta function $\zeta(s)$, which has a simple pole at $s = 1$. The residue at this pole encodes fundamental arithmetic information about the distribution of primes.

In this paper, we introduce a different type of pole structure arising from geometric factorizations of integers. Rather than poles in a complex variable, we study poles in a regularization parameter $\varepsilon$ that prevents division by zero in sums over integer pairs $(d, k)$ satisfying $n = kd + d^2$.

Our main result shows that the residue of $F_n(\alpha, \varepsilon)$ at $\varepsilon = 0$ equals the count of non-trivial divisors of $n$ up to $\sqrt{n}$. This provides a new characterization: primes are precisely those integers for which $F_n$ has no pole at $\varepsilon = 0$.

\subsection{Main Results}

We introduce the \emph{factorization count function}:

\begin{definition}
For $n \geq 2$, let
\[
M(n) = \#\{d \in \mathbb{Z} : d \mid n, \, 2 \leq d \leq \sqrt{n}\}
\]
denote the number of divisors of $n$ strictly between 2 and $\sqrt{n}$ (inclusive).
\end{definition}

Our central object of study is the following regularized sum:

\begin{definition}
For $n \geq 2$, $\alpha \in \mathbb{R}$ with $\alpha > 0$, and $\varepsilon > 0$, define
\[
F_n(\alpha, \varepsilon) = \sum_{d=2}^{\infty} \sum_{k=0}^{\infty} \left[(n - kd - d^2)^2 + \varepsilon\right]^{-\alpha}.
\]
\end{definition}

\begin{remark}
The regularization parameter $\varepsilon$ ensures that the summand is well-defined even when $n = kd + d^2$, which occurs precisely when $d$ divides $n$ and $2 \leq d \leq \sqrt{n}$.
\end{remark}

Our main theorem establishes the residue formula:

\begin{theorem}[Epsilon-Pole Residue Formula]
\label{thm:main}
For $\alpha > 1/2$ and $n \geq 2$,
\[
\lim_{\varepsilon \to 0^+} \varepsilon^\alpha \cdot F_n(\alpha, \varepsilon) = M(n).
\]
\end{theorem}

We derive an explicit closed form for $M(n)$:

\begin{theorem}[Closed Form of $M(n)$]
\label{thm:closed-form}
For $n \geq 2$, let $\tau(n)$ denote the divisor function (total number of positive divisors of $n$). Then
\[
M(n) = \left\lfloor \frac{\tau(n) - 1}{2} \right\rfloor.
\]
\end{theorem}

Combining these results yields a primality criterion:

\begin{corollary}[Analytic Primality Criterion]
\label{cor:primality}
An integer $n \geq 2$ is prime if and only if
\[
\lim_{\varepsilon \to 0^+} \varepsilon^\alpha \cdot F_n(\alpha, \varepsilon) = 0
\]
for some (equivalently, all) $\alpha > 1/2$.
\end{corollary}

\section{Factorizations and Divisors}

We begin by establishing the bijection between factorizations and divisors.

\begin{definition}
For $n \geq 2$, define the set of \emph{factorizing pairs}:
\[
\mathcal{F}_n = \{(d, k) \in \mathbb{Z}^2 : n = kd + d^2, \, d \geq 2, \, k \geq 0\}.
\]
\end{definition}

\begin{proposition}[Bijection with Divisors]
\label{prop:bijection}
There exists a bijection between $\mathcal{F}_n$ and the set $\{d \in \mathbb{Z} : d \mid n, \, 2 \leq d \leq \sqrt{n}\}$.
\end{proposition}

\begin{proof}
Define the maps:
\begin{align*}
\phi: \mathcal{F}_n &\to \{d \in \mathbb{Z} : d \mid n, \, 2 \leq d \leq \sqrt{n}\} \\
(d, k) &\mapsto d
\end{align*}
and
\begin{align*}
\psi: \{d \in \mathbb{Z} : d \mid n, \, 2 \leq d \leq \sqrt{n}\} &\to \mathcal{F}_n \\
d &\mapsto \left(d, \frac{n}{d} - d\right).
\end{align*}

\textbf{Well-definedness of $\phi$:} If $(d, k) \in \mathcal{F}_n$, then $n = kd + d^2 = d(k + d)$, so $d \mid n$. Since $k \geq 0$, we have $k + d \geq d$, hence $\frac{n}{d} = k + d \geq d$, giving $d^2 \leq n$. Thus $d \in \{d \in \mathbb{Z} : d \mid n, \, 2 \leq d \leq \sqrt{n}\}$.

\textbf{Well-definedness of $\psi$:} If $d \mid n$ and $2 \leq d \leq \sqrt{n}$, then $k := \frac{n}{d} - d$ is an integer. Since $d \leq \sqrt{n}$, we have $\frac{n}{d} \geq \sqrt{n} \geq d$, so $k = \frac{n}{d} - d \geq 0$. Moreover,
\[
kd + d^2 = \left(\frac{n}{d} - d\right)d + d^2 = n - d^2 + d^2 = n,
\]
so $(d, k) \in \mathcal{F}_n$.

\textbf{$\phi \circ \psi = \mathrm{id}$:} For $d \mid n$ with $2 \leq d \leq \sqrt{n}$,
\[
\phi(\psi(d)) = \phi\left(d, \frac{n}{d} - d\right) = d.
\]

\textbf{$\psi \circ \phi = \mathrm{id}$:} For $(d, k) \in \mathcal{F}_n$ with $n = kd + d^2$,
\[
\psi(\phi(d, k)) = \psi(d) = \left(d, \frac{n}{d} - d\right) = \left(d, \frac{kd + d^2}{d} - d\right) = (d, k + d - d) = (d, k).
\]

Thus $\phi$ and $\psi$ are inverses, establishing the bijection.
\end{proof}

\begin{proof}[Proof of Theorem \ref{thm:closed-form}]
By Proposition \ref{prop:bijection},
\[
M(n) = |\mathcal{F}_n| = |\{d \in \mathbb{Z} : d \mid n, \, 2 \leq d \leq \sqrt{n}\}|.
\]

Let $D(n) = \{d \in \mathbb{Z} : d \mid n, \, d > 0\}$ denote the set of all positive divisors of $n$. Then $|D(n)| = \tau(n)$.

For each divisor $d$ of $n$, the complementary divisor is $\frac{n}{d}$. The divisors pair up as $(d, \frac{n}{d})$ unless $d = \frac{n}{d}$, which occurs when $d^2 = n$, i.e., when $n$ is a perfect square and $d = \sqrt{n}$.

Consider the divisors satisfying $d \geq 2$. Excluding $d = 1$, there are $\tau(n) - 1$ such divisors.

Among these, the divisors satisfying $2 \leq d \leq \sqrt{n}$ constitute exactly half of the pairs $(d, \frac{n}{d})$ with $d \geq 2$, except when $n$ is a perfect square, in which case the middle divisor $d = \sqrt{n}$ pairs with itself.

If $n$ is not a perfect square, divisors $d \geq 2$ pair up with distinct complementary divisors, and exactly half satisfy $d \leq \sqrt{n}$:
\[
M(n) = \frac{\tau(n) - 1}{2}.
\]

If $n = m^2$ is a perfect square, then $\sqrt{n} = m$ is a divisor. Among the $\tau(n) - 1$ divisors $d \geq 2$, the divisor $m$ pairs with itself, and the remaining $\tau(n) - 2$ divisors form pairs. Half of these satisfy $d < m$, so
\[
M(n) = \frac{\tau(n) - 2}{2} + 1 = \frac{\tau(n)}{2}.
\]

Both cases are unified by the floor formula:
\[
M(n) = \left\lfloor \frac{\tau(n) - 1}{2} \right\rfloor. \qedhere
\]
\end{proof}

\section{The Residue Theorem}

To prove Theorem \ref{thm:main}, we partition the double sum into contributions from factorizing and non-factorizing pairs.

\begin{definition}
Define the \emph{singular part}:
\[
S_n(\varepsilon) = \sum_{(d,k) \in \mathcal{F}_n} \varepsilon^{-\alpha}
\]
and the \emph{regular part}:
\[
R_n(\alpha, \varepsilon) = \sum_{\substack{d \geq 2, k \geq 0 \\ (d,k) \notin \mathcal{F}_n}} \left[(n - kd - d^2)^2 + \varepsilon\right]^{-\alpha}.
\]
\end{definition}

Then $F_n(\alpha, \varepsilon) = S_n(\varepsilon) + R_n(\alpha, \varepsilon)$.

\subsection{Singular Part Contribution}

\begin{proposition}
\label{prop:singular}
For all $\varepsilon > 0$,
\[
\varepsilon^\alpha \cdot S_n(\varepsilon) = M(n).
\]
\end{proposition}

\begin{proof}
For $(d, k) \in \mathcal{F}_n$, we have $n = kd + d^2$, so $n - kd - d^2 = 0$. Thus each term contributes
\[
\left[(0)^2 + \varepsilon\right]^{-\alpha} = \varepsilon^{-\alpha}.
\]

Summing over all $M(n)$ factorizing pairs,
\[
S_n(\varepsilon) = \sum_{(d,k) \in \mathcal{F}_n} \varepsilon^{-\alpha} = M(n) \cdot \varepsilon^{-\alpha}.
\]

Multiplying by $\varepsilon^\alpha$,
\[
\varepsilon^\alpha \cdot S_n(\varepsilon) = \varepsilon^\alpha \cdot M(n) \cdot \varepsilon^{-\alpha} = M(n). \qedhere
\]
\end{proof}

\subsection{Regular Part Vanishes}

The key step is proving that the non-factorizing contributions vanish.

\begin{lemma}[Regular Part Vanishes]
\label{lem:regular}
For $\alpha > 1/2$,
\[
\lim_{\varepsilon \to 0^+} \varepsilon^\alpha \cdot R_n(\alpha, \varepsilon) = 0.
\]
\end{lemma}

\begin{proof}
For $(d, k) \notin \mathcal{F}_n$, define $\dist(d, k) = n - kd - d^2 \neq 0$.

Consider the sum at $\varepsilon = 0$:
\[
R_n(\alpha, 0) = \sum_{\substack{d \geq 2, k \geq 0 \\ (d,k) \notin \mathcal{F}_n}} |\dist(d, k)|^{-2\alpha}.
\]

We show $R_n(\alpha, 0) < \infty$ for $\alpha > 1/2$ by analyzing the sum over each fixed $d$.

\textbf{Fixed $d$ analysis:} For fixed $d \geq 2$, at most one value $k_0 = \frac{n - d^2}{d}$ (if it is a non-negative integer) satisfies $\dist(d, k_0) = 0$. For all other $k \geq 0$,
\[
\dist(d, k) = n - kd - d^2 = d\left(\frac{n - d^2}{d} - k\right) = d(k_0 - k),
\]
where $k_0$ may be non-integer or negative.

For $k \neq k_0$,
\[
|\dist(d, k)|^{-2\alpha} = |d(k_0 - k)|^{-2\alpha} = d^{-2\alpha} |k_0 - k|^{-2\alpha}.
\]

Summing over $k \geq 0$ with $k \neq k_0$,
\[
\sum_{\substack{k \geq 0 \\ k \neq k_0}} |\dist(d, k)|^{-2\alpha} \leq d^{-2\alpha} \sum_{j=1}^{\infty} j^{-2\alpha} + d^{-2\alpha} \sum_{j=1}^{\infty} j^{-2\alpha} = 2d^{-2\alpha} \zeta(2\alpha),
\]
where $\zeta(s) = \sum_{j=1}^{\infty} j^{-s}$ is the Riemann zeta function. This converges if and only if $2\alpha > 1$, i.e., $\alpha > 1/2$.

\textbf{Sum over $d$:} Summing over $d \geq 2$,
\[
R_n(\alpha, 0) \leq 2\zeta(2\alpha) \sum_{d=2}^{\infty} d^{-2\alpha} = 2\zeta(2\alpha)(\zeta(2\alpha) - 1) < \infty.
\]

\textbf{Monotonicity:} For $\varepsilon_1 < \varepsilon_2$, each summand satisfies
\[
\left[(\dist(d, k))^2 + \varepsilon_1\right]^{-\alpha} \geq \left[(\dist(d, k))^2 + \varepsilon_2\right]^{-\alpha},
\]
so $R_n(\alpha, \varepsilon)$ is monotone decreasing in $\varepsilon$, and
\[
R_n(\alpha, \varepsilon) \downarrow R_n(\alpha, 0) \quad \text{as } \varepsilon \to 0^+.
\]

\textbf{Conclusion:} Since $R_n(\alpha, \varepsilon) \leq R_n(\alpha, 0) < \infty$ for all $\varepsilon > 0$,
\[
\varepsilon^\alpha R_n(\alpha, \varepsilon) \leq \varepsilon^\alpha R_n(\alpha, 0) \to 0 \quad \text{as } \varepsilon \to 0^+. \qedhere
\]
\end{proof}

\subsection{Proof of Main Theorem}

\begin{proof}[Proof of Theorem \ref{thm:main}]
By the partition $F_n(\alpha, \varepsilon) = S_n(\varepsilon) + R_n(\alpha, \varepsilon)$,
\[
\varepsilon^\alpha F_n(\alpha, \varepsilon) = \varepsilon^\alpha S_n(\varepsilon) + \varepsilon^\alpha R_n(\alpha, \varepsilon).
\]

Taking the limit as $\varepsilon \to 0^+$,
\[
\lim_{\varepsilon \to 0^+} \varepsilon^\alpha F_n(\alpha, \varepsilon) = \lim_{\varepsilon \to 0^+} \varepsilon^\alpha S_n(\varepsilon) + \lim_{\varepsilon \to 0^+} \varepsilon^\alpha R_n(\alpha, \varepsilon).
\]

By Proposition \ref{prop:singular}, the first limit equals $M(n)$.

By Lemma \ref{lem:regular}, the second limit equals $0$.

Therefore,
\[
\lim_{\varepsilon \to 0^+} \varepsilon^\alpha F_n(\alpha, \varepsilon) = M(n). \qedhere
\]
\end{proof}

\section{Corollaries and Applications}

\begin{proof}[Proof of Corollary \ref{cor:primality}]
An integer $n \geq 2$ is prime if and only if $n$ has no divisors $d$ with $2 \leq d \leq \sqrt{n}$, i.e., $M(n) = 0$.

By Theorem \ref{thm:main}, this is equivalent to
\[
\lim_{\varepsilon \to 0^+} \varepsilon^\alpha F_n(\alpha, \varepsilon) = 0
\]
for any $\alpha > 1/2$.
\end{proof}

\begin{corollary}[Compositeness Measure]
The quantity $M(n)$ provides a measure of compositeness: primes have $M(n) = 0$, while highly composite numbers have large $M(n)$.
\end{corollary}

\begin{remark}
The residue formula provides an analytic characterization of factorization structure analogous to the role of poles in $L$-functions for prime distribution.
\end{remark}

\section{Laurent Expansion Perspective}

Near $\varepsilon = 0$, the function $F_n(\alpha, \varepsilon)$ admits a Laurent expansion:
\[
F_n(\alpha, \varepsilon) = \frac{M(n)}{\varepsilon^\alpha} + c_0 + c_1 \varepsilon + c_2 \varepsilon^2 + \cdots
\]

The leading coefficient (pole strength) equals $M(n)$.

For primes, $M(p) = 0$, so
\[
F_p(\alpha, \varepsilon) = c_0 + c_1 \varepsilon + c_2 \varepsilon^2 + \cdots
\]
is analytic at $\varepsilon = 0$.

This parallels classical residue theory in complex analysis, where the residue at a pole extracts arithmetic information.

\section{Remarks on the Range of $\alpha$}

The theorem requires $\alpha > 1/2$ for convergence of the sum $\sum_{j=1}^{\infty} j^{-2\alpha}$ appearing in the proof of Lemma \ref{lem:regular}.

For $\alpha \leq 1/2$, the series diverges, and the regular part $R_n(\alpha, 0)$ is infinite. In this regime, the limit behavior may differ, and further analysis is required.

\section{Connection to Classical Number Theory}

The divisor function $\tau(n)$ satisfies well-known asymptotic formulas:
\[
\sum_{n \leq x} \tau(n) = x \log x + (2\gamma - 1)x + O(\sqrt{x}),
\]
where $\gamma$ is the Euler-Mascheroni constant.

Since $M(n) = \lfloor (\tau(n) - 1)/2 \rfloor$, we have
\[
\sum_{n \leq x} M(n) \sim \frac{1}{2} \sum_{n \leq x} \tau(n) \sim \frac{x \log x}{2}.
\]

This suggests that on average, integers have logarithmically many factorizations of the form $n = kd + d^2$.

\section{Open Questions}

Several directions for future research emerge:

\begin{enumerate}
\item \textbf{Higher-order coefficients}: What is the arithmetic significance of the coefficients $c_0, c_1, c_2, \ldots$ in the Laurent expansion?

\item \textbf{Generalizations}: Can the result be extended to decompositions $n = kd + d^r$ for $r \geq 3$?

\item \textbf{Connections to $L$-functions}: Is there a relationship between $F_n(\alpha, \varepsilon)$ and classical Dirichlet $L$-functions?

\item \textbf{Analytic continuation}: Can $F_n(\alpha, \varepsilon)$ be continued to complex values of $\alpha$ or $\varepsilon$?

\item \textbf{Distribution of $M(n)$}: What is the distribution of $M(n)$ among integers? Are there infinitely many $n$ with $M(n) = k$ for each fixed $k$?
\end{enumerate}

\section*{Acknowledgments}

This work was developed through human-AI collaboration, with geometric intuition and problem formulation by the author, and formal proof development assisted by Claude (Anthropic). All results have been independently verified through extensive numerical computation.

Computational verification performed with Wolfram Language.

\begin{thebibliography}{99}

\bibitem{apostol}
T. M. Apostol, \emph{Introduction to Analytic Number Theory}, Springer, 1976.

\bibitem{hardy-wright}
G. H. Hardy and E. M. Wright, \emph{An Introduction to the Theory of Numbers}, 6th ed., Oxford University Press, 2008.

\bibitem{titchmarsh}
E. C. Titchmarsh, \emph{The Theory of the Riemann Zeta-Function}, 2nd ed., Oxford University Press, 1986.

\end{thebibliography}

\end{document}
