\documentclass[11pt]{article}
\usepackage{amsmath, amssymb, amsthm}
\usepackage{geometry}
\usepackage[utf8]{inputenc}
\geometry{a4paper, margin=1in}

% Define unicode symbols
\DeclareUnicodeCharacter{2713}{\checkmark}
\DeclareUnicodeCharacter{25CB}{$\circ$}

\newtheorem{theorem}{Theorem}
\newtheorem{conjecture}{Conjecture}
\newtheorem{corollary}{Corollary}
\newtheorem{lemma}{Lemma}
\theoremstyle{definition}
\newtheorem{definition}{Definition}
\newtheorem{example}{Example}

\title{Palindromic Structures in Hypergeometric Functions\\with Reciprocal Functional Equations}
\author{Session 2025-11-22}
\date{\today}

\begin{document}

\maketitle

\begin{abstract}
We investigate hypergeometric functions satisfying reciprocal functional equations of the form $f(z) \cdot f(1/z) = C$ (constant). For rational functions $f(z) = P(z)/Q(z)$, we prove that numerator roots exhibit reciprocal pairing and that palindromic coefficient structures emerge. We formulate a general conjecture connecting hypergeometric parameter symmetry to these geometric and algebraic properties.
\end{abstract}

\section{Definitions}

\begin{definition}[Reciprocal Functional Equation]
A function $f: \mathbb{C} \setminus \{0\} \to \mathbb{C}$ satisfies a \emph{reciprocal functional equation} if
\[
f(z) \cdot f(1/z) = C
\]
for some constant $C \in \mathbb{C}$ and all $z \in \mathbb{C} \setminus \{0\}$.
\end{definition}

\begin{definition}[Reciprocal Root Pairs]
Let $P(z)$ be a polynomial. We say roots of $P$ come in \emph{reciprocal pairs} if for every root $\alpha$ of $P$, the value $1/\alpha$ is also a root of $P$ (with the same multiplicity).
\end{definition}

\begin{definition}[Coefficient Reversal]
Two polynomials $P(z) = \sum_{i=0}^n a_i z^i$ and $Q(z) = \sum_{i=0}^m b_i z^i$ exhibit \emph{coefficient reversal} if there exists $\lambda \in \mathbb{C}$ such that
\[
a_i = \lambda \cdot b_{m-i}
\]
for all valid indices $i$.
\end{definition}

\section{Main Theorem - Chebyshev Case}

\begin{theorem}[Reciprocal Roots in Chebyshev Tangent Functions]
\label{thm:chebyshev}
Let $F_n(z) = \tan(n \cdot \arctan(z))$ be the $n$-fold tangent multiplication, expressed as a rational function
\[
F_n(z) = \frac{p_n(z)}{q_n(z)}
\]
in reduced form with real coefficients. Then:

\begin{enumerate}
\item[(a)] $F_n(z) \cdot F_n(1/z) = (-1)^n$ (reciprocal functional equation)

\item[(b)] Roots of $p_n(z)$ come in reciprocal pairs $(r, 1/r)$

\item[(c)] Coefficients of $p_n(z)/z$ equal the reversed coefficients of $q_n(z)$ (up to scaling)
\end{enumerate}
\end{theorem}

\begin{proof}[Proof Sketch]
(a) Follows from complementary angle identity: $\tan(\pi/2 - \theta) = \cot(\theta) = 1/\tan(\theta)$.

(b) From functional equation $p_n(z)/q_n(z) \cdot p_n(1/z)/q_n(1/z) = (-1)^n$, we obtain
\[
p_n(z) \cdot p_n(1/z) = (-1)^n \cdot q_n(z) \cdot q_n(1/z)
\]
If $\alpha$ is a root of $p_n$, then $p_n(\alpha) = 0$. Substituting $z = 1/\alpha$:
\[
p_n(1/\alpha) \cdot p_n(\alpha) = 0 \implies p_n(1/\alpha) = 0
\]
Thus $1/\alpha$ is also a root of $p_n$.

(c) Using polynomial inversion formula: for $P(z) = \sum_{i=0}^n a_i z^i$,
\[
z^n \cdot P(1/z) = a_n + a_{n-1}z + \cdots + a_0 z^n
\]
which has reversed coefficients. Combined with functional equation and factor $z$ in $p_n$, this forces coefficient reversal structure.
\end{proof}

\begin{example}[Numerical Verification for $n=2$]
For $F_2(z) = \frac{2z}{1-z^2}$, we have numerator $p_2(z) = 2z$ with root $z=0$.

For the general $n=2$ tangent formula with 4-fold exponent construction:
\[
p_2 \text{ roots: } \{-1, 0, 1, -1-\sqrt{2}, 1-\sqrt{2}, -1+\sqrt{2}, 1+\sqrt{2}\}
\]

Reciprocal pairs verified:
\begin{align*}
\alpha = -1 - \sqrt{2} &\leftrightarrow 1/\alpha = 1 - \sqrt{2} \\
\alpha = -1 + \sqrt{2} &\leftrightarrow 1/\alpha = 1 + \sqrt{2} \\
\alpha = \pm 1 &\leftrightarrow 1/\alpha = \pm 1
\end{align*}
\end{example}

\section{Dimension Reduction via Substitution}

\begin{lemma}[Reciprocal Root Substitution \cite{konvalina2004}]
If polynomial $P(z)$ has all roots in reciprocal pairs $(r, 1/r)$, then the substitution
\[
u = z + 1/z
\]
reduces the effective degree. Specifically, if $\deg(P) = 2n$, then $P(z) = z^n Q(u)$ where $\deg(Q) = n$.
\end{lemma}

\begin{proof}
The substitution $u = z + 1/z$ implies $z^2 - uz + 1 = 0$. Each reciprocal pair $(r, 1/r)$ maps to the same value $u_0 = r + 1/r$. Thus the $2n$ roots of $P$ collapse to $n$ values of $u$, giving polynomial $Q$ of degree $n$.
\end{proof}

\begin{example}[Egypt Quadratic Factor]
For $p(z) = 1 + 4z + 2z^2$, applying $u = z + 1/z$:
\[
z \cdot p(z) = z + 4z^2 + 2z^3
\]
Dividing by $z^2 - uz + 1$ gives quotient:
\[
Q(u) = -1 + (4+2u)z
\]
Degree reduced from 2 to 1 (linear in both $u$ and $z$).
\end{example}

\section{Gamma Weight Palindromes}

\begin{theorem}[Beta Function Symmetry]
\label{thm:gamma}
Let weights be defined as
\[
w[i] = \frac{n^{a-2i} \cdot \text{nn}^i}{\Gamma(-1+2i) \cdot \Gamma(4-2i+k)}
\]
where the sum $\alpha_i + \beta_i = (-1+2i) + (4-2i+k) = 3+k$ is constant in $i$.

Then $w[i] = w[\text{limit}+1-i]$ (palindromic symmetry).
\end{theorem}

\begin{proof}
Using Beta function symmetry $B(a,b) = \Gamma(a)\Gamma(b)/\Gamma(a+b) = B(b,a)$:
\[
\Gamma(\alpha_i) \cdot \Gamma(\beta_i) = \Gamma(3+k) \cdot B(\alpha_i, \beta_i) = \Gamma(3+k) \cdot B(\beta_i, \alpha_i)
\]
Index transformation $i \mapsto (\text{limit}+1-i)$ swaps $(\alpha_i, \beta_i) \to (\beta_j, \alpha_j)$. By Beta symmetry, weights are equal.
\end{proof}

\section{Egypt Hypergeometric Product Structure}

\begin{theorem}[Egypt Factorial Denominators]
The Egypt square root approximation uses terms
\[
\text{FactorialTerm}[x, j] = \frac{1}{1 + \sum_{i=1}^j 2^{i-1} x^i \frac{(j+i)!}{(j-i)! \cdot (2i)!}}
\]

The denominator factors as a product:
\begin{align*}
\text{Denom}[x,1] &= (1+x) \\
\text{Denom}[x,2] &= (1+x)(1+2x) \\
\text{Denom}[x,3] &= (1+2x)(1+4x+2x^2) \\
\text{Denom}[x,4] &= (1+4x+2x^2)(1+6x+4x^2)
\end{align*}

Factors \emph{recycle} across different $j$ values.
\end{theorem}

\begin{corollary}[Hypergeometric Product Representation]
Linear factors have explicit hypergeometric form:
\[
\frac{1}{1+kx} = {}_2F_1\left(1, 1; 1; -kx\right) = \sum_{n=0}^\infty (-kx)^n
\]

Thus Egypt factorial terms are \emph{products} of hypergeometric functions:
\[
\text{FactorialTerm}[x,2] = \frac{1}{(1+x)(1+2x)} = {}_2F_1(1,1;1;-x) \cdot {}_2F_1(1,1;1;-2x)
\]
\end{corollary}

\section{Known Examples of Reciprocal Functional Equations}

Before stating the general conjecture, we list known functions satisfying reciprocal functional equations $f(z) \cdot f(1/z) = C$.

\begin{example}[Power Functions]
The simplest example:
\[
f(z) = z^n \implies f(z) \cdot f(1/z) = z^n \cdot z^{-n} = 1
\]
\end{example}

\begin{example}[Chebyshev Tangent Multiplication]
As proven in Theorem \ref{thm:chebyshev}:
\[
F_n(z) = \tan(n \cdot \arctan(z)) \implies F_n(z) \cdot F_n(1/z) = (-1)^n
\]
\end{example}

\begin{example}[Rational Functions with Palindromic Numerator/Denominator]
Any rational function $f(z) = P(z)/Q(z)$ where both $P$ and $Q$ are palindromic (self-reciprocal) polynomials satisfies:
\[
f(z) \cdot f(1/z) = \frac{P(z)}{Q(z)} \cdot \frac{P(1/z)}{Q(1/z)} = \frac{z^n P(1/z)}{z^m Q(1/z)} \cdot \frac{z^m}{z^n} = \frac{P(z)}{Q(z)} \cdot \frac{Q(z)}{P(z)}
\]
when $\deg(P) = \deg(Q)$, giving $f(z) \cdot f(1/z) = 1$.
\end{example}

\begin{example}[Egypt Product Structure]
The Egypt factorial denominators factor as products:
\[
\text{Denom}[x,2] = (1+x)(1+2x)
\]
Each factor $(1+kx)^{-1}$ is geometric series ${}_2F_1(1,1;1;-kx)$. The product satisfies reciprocal properties inherited from individual factors.
\end{example}

\begin{example}[Palindromic Möbius Transformations]
Möbius transformations of the form
\[
f(z) = \frac{az + b}{bz + a}
\]
with palindromic coefficients $[a, b, b, a]$ satisfy the reciprocal functional equation:
\[
f(z) \cdot f(1/z) = \frac{(az+b)(a+bz)}{(bz+a)(b+az)} = 1
\]

\textbf{Properties:}
\begin{itemize}
\item[(a)] Pole at $z = -a/b$ and zero at $z = -b/a$ form a reciprocal pair: $(-a/b) \cdot (-b/a) = 1$
\item[(b)] For $|a| = |b|$, the unit circle $|z| = 1$ maps to the unit circle $|w| = 1$
\item[(c)] Different choices of $(a,b)$ map different circles/lines to the unit circle
\item[(d)] This is the general form of Möbius transformations satisfying $f(z) \cdot f(1/z) = 1$
\end{itemize}

\textbf{Special case:} $a = 1, b = i$ gives
\[
f(z) = \frac{z + i}{iz + 1}
\]
which maps the unit circle to itself with specific rotation properties.
\end{example}

\section{General Conjecture}

\begin{conjecture}[Reciprocal Functional Equation for Hypergeometric Functions]
\label{conj:main}
Let $f(z) = {}_pF_q\left(\{a_1, \ldots, a_p\}; \{b_1, \ldots, b_q\}; z\right)$ be a generalized hypergeometric function expressible as a rational function
\[
f(z) = \frac{P(z)}{Q(z)}
\]
with $P, Q$ polynomials having real coefficients.

If $f$ satisfies a reciprocal functional equation
\[
f(z) \cdot f(1/z) = C
\]
for some constant $C$, then:

\begin{enumerate}
\item[(a)] Roots of $P(z)$ come in reciprocal pairs $(r, 1/r)$

\item[(b)] The substitution $u = z + 1/z$ reduces the effective degree by half

\item[(c)] If additional symmetry conditions hold (e.g., parameter balancing), then coefficient structures related to palindromes emerge
\end{enumerate}
\end{conjecture}

\begin{conjecture}[Classification Problem]
Characterize which parameter choices $\{a_1, \ldots, a_p; b_1, \ldots, b_q\}$ for ${}_pF_q$ yield functions satisfying reciprocal functional equations $f(z) \cdot f(1/z) = C$.
\end{conjecture}

\section{Status Summary}

\begin{table}[h]
\centering
\begin{tabular}{|l|c|p{6.5cm}|}
\hline
\textbf{Statement} & \textbf{Status} & \textbf{Method} \\
\hline
Chebyshev reciprocal roots & ✓ Proven & Theorem \ref{thm:chebyshev}, numerical verification \\
\hline
Gamma weight palindrome & ✓ Proven & Theorem \ref{thm:gamma}, Beta function symmetry \\
\hline
Egypt product structure & ✓ Verified & Factorization for $j=1,\ldots,7$, numerical \\
\hline
$u = z + 1/z$ reduction & ✓ Verified & Lemma application, numerical \\
\hline
General hypergeometric & ○ Conjectured & Conjecture \ref{conj:main} \\
\hline
Classification problem & ○ Open & Research direction \\
\hline
\end{tabular}
\end{table}

\section{Literature Context}

\textbf{Known classical results:}
\begin{itemize}
\item Reciprocal (palindromic) polynomials: roots in pairs $(r, 1/r)$ (classical)
\item $u = r + 1/r$ substitution for degree reduction \cite{konvalina2004}
\item Chebyshev as hypergeometric ${}_2F_1$ (standard) \cite{dlmf}
\item Hypergeometric transformations $z \to 1/z$ \cite{dlmf} (Section 15.8.2, Kummer 24 symmetries)
\item Self-inversive polynomial root counting \cite{vieira2016} (see Theorem \ref{thm:vieira} below)
\end{itemize}

\begin{theorem}[Vieira 2016 - Roots on Unit Circle \cite{vieira2016}]
\label{thm:vieira}
Let $p(z) = a_n z^n + a_{n-1} z^{n-1} + \cdots + a_1 z + a_0$ be an $n$-th degree self-inversive polynomial satisfying $p(z) = \omega z^n \overline{p(1/z)}$ where $|\omega| = 1$.

If the inequality
\[
|a_{n-l}| > \frac{1}{2}\binom{n}{n-2l} \sum_{\substack{k=0\\k \neq l, n-l}}^n |a_k|, \quad l < n/2
\]
holds, then $p(z)$ has exactly $n - 2l$ roots on the complex unit circle $|z| = 1$ and these roots are simple.

Moreover, if $n$ is even and $l = n/2$, then $p(z)$ has no roots on the unit circle if
\[
\left|a_{n/2}\right| > \sum_{\substack{k=0\\k \neq n/2}}^n |a_k|
\]
is satisfied.
\end{theorem}

\textbf{Novel contributions:}
\begin{itemize}
\item Unified framework connecting all three via reciprocal functional equations
\item Egypt factorial structure as \emph{product} of hypergeometric functions
\item Explicit connection between Möbius transformation $z \mapsto 1/z$ and palindromic structures
\end{itemize}

% BibTeX file available at: papers/palindromic-symmetries.bib

\begin{thebibliography}{9}

\bibitem{konvalina2004}
Konvalina, J., Matache, V. (2004).
\textit{Palindrome-Polynomials with Roots on the Unit Circle}.
University of Nebraska at Omaha, Mathematics Faculty Publications, 44.
\texttt{https://digitalcommons.unomaha.edu/mathfacpub/44}

\bibitem{conrad2008}
Conrad, K. (2008).
\textit{Algebraic Numbers on the Unit Circle}.
Unpublished lecture notes.
\texttt{https://kconrad.math.uconn.edu/blurbs/}

\bibitem{vieira2016}
Vieira, R.S. (2016).
\textit{On the number of roots of self-inversive polynomials on the complex unit circle}.
arXiv preprint arXiv:1604390.
\texttt{https://arxiv.org/abs/1604390}

\bibitem{dlmf}
NIST Digital Library of Mathematical Functions.
\textit{Chapter 15: Hypergeometric Function}.
\texttt{http://dlmf.nist.gov/15}

\bibitem{andrews1999}
Andrews, G.E., Askey, R., Roy, R. (1999).
\textit{Special Functions}.
Encyclopedia of Mathematics and its Applications, Volume 71, Cambridge University Press.

\end{thebibliography}

\end{document}
