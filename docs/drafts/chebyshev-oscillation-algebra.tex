\documentclass[11pt]{article}
\usepackage{amsmath,amssymb,amsthm}
\usepackage[margin=1in]{geometry}
\usepackage{hyperref}

\newtheorem{theorem}{Theorem}
\newtheorem{proposition}{Proposition}
\newtheorem{definition}{Definition}
\newtheorem{remark}{Remark}

\title{Chebyshev Polynomials and Discrete Oscillation Counting:\\
Notes Toward Wildberger's Program}

\author{Jan Popelka\thanks{Email: popojan@protonmail.com}}

\date{December 1, 2025\\
\small Draft -- Not peer-reviewed}

\begin{document}

\maketitle

\begin{abstract}
We observe that Chebyshev polynomials, defined purely via algebraic
recurrence, encode rotational structure through discrete parameters.
The integer $n$ (``oscillation count'') determines the polynomial's
roots, degree, and lobe structure, while the composition property
$T_m(T_n(x)) = T_{mn}(x)$ captures multiplicative combination.
The total area under the Chebyshev lobe curve $L_n(x) = T_{n+1}(x) - xT_n(x)$
equals exactly 1 for all $n \geq 2$, yielding a rational area ratio
of $1/4$ relative to the bounding square. The lobe area function extends
naturally to rational and real parameters, suggesting these structures
are genuinely algebraic rather than merely discrete samples of transcendental
functions. We discuss how these features might complement Wildberger's
rational trigonometry program, noting both potential connections and
significant limitations.
\end{abstract}

\section{Motivation: Local vs Global Orientation}

Norman Wildberger's \emph{Rational Trigonometry} \cite{wildberger2005}
replaces the classical notions of distance and angle with:
\begin{itemize}
    \item \textbf{Quadrance}: $Q = d^2$ (square of distance, avoiding $\sqrt{\cdot}$)
    \item \textbf{Spread}: $s = \sin^2\theta$ (avoiding $\pi$ in angle measurement)
\end{itemize}

This framework successfully avoids transcendental numbers in many geometric
computations. Wildberger also introduces an oriented version: the \textbf{turn}
$u(l_1, l_2) = \tan\theta$ between lines, with the property that
$\text{twist} = \text{turn}^2$ and $u(l_1, l_2) = -u(l_2, l_1)$.
The Triple Turn Formula $u_1 + u_2 + u_3 = u_1 u_2 u_3$ elegantly captures
local angular relationships.

However, Wildberger's turn measures \emph{local} orientation---the angular
displacement between two specific lines. It does not capture \emph{global}
orientation: how many times one has ``gone around.'' Going around the circle
twice gives the same turn as going around once.

\textbf{Question:} Can we encode the \emph{winding number}---how many complete
cycles---within a rational/algebraic framework?

\section{Chebyshev Polynomials as Oscillation Encoding}

\subsection{The Classical Definition}

The Chebyshev polynomials of the first kind are defined by:
\begin{equation}
    T_n(\cos\theta) = \cos(n\theta)
\end{equation}

This definition explicitly uses $\pi$ (via $\theta$). However, the
polynomials themselves are purely algebraic:
\begin{align}
    T_0(x) &= 1 \\
    T_1(x) &= x \\
    T_{n+1}(x) &= 2x \cdot T_n(x) - T_{n-1}(x)
\end{align}

\subsection{The Key Insight: Integer Oscillation Count}

Instead of measuring \emph{how much} rotation (angle $\theta \in [0, 2\pi)$),
we measure \emph{how many} oscillations (integer $n \in \mathbb{Z}^+$):

\begin{center}
\begin{tabular}{lll}
\textbf{Framework} & \textbf{What it measures} & \textbf{Needs $\pi$?} \\
\hline
Classical & angle $\theta \in [0, 2\pi)$ & Yes \\
Wildberger & spread $s \in [0, 1]$ & No, but loses global count \\
Chebyshev & oscillation count $n \in \mathbb{Z}^+$ & No, preserves ``how many times''
\end{tabular}
\end{center}

\subsection{The $n$-gon Correspondence}

The integer $n$ has concrete geometric meaning through three equivalent
characterizations:

\begin{proposition}[Oscillation Count Characterizations]
For integer $n \geq 2$, the following are equivalent ways to define
the ``$n$-fold oscillation'':
\begin{enumerate}
    \item \textbf{Root count:} $T_n(x)$ has exactly $n$ roots in $[-1, 1]$.
          These are the real parts of certain $2n$-th roots of unity
          (classically written $x_k = \cos\frac{(2k-1)\pi}{2n}$,
          but this is just notation for algebraic numbers).
    \item \textbf{Lobe count:} The curve $L_n(x) = T_{n+1}(x) - xT_n(x)$
          has exactly $n$ lobes (sign changes) over $[-1,1]$.
    \item \textbf{Circle touchings:} The parametric curve
          $(x, L_n(x))$ touches the unit circle at exactly $n$ points,
          located at the vertices of a regular $n$-gon (the $x$-coordinates
          being roots of $T_n$).
    \item \textbf{Algebraic degree:} $\deg(T_n) = n$, and this degree
          is preserved under composition: $\deg(T_m \circ T_n) = mn$.
\end{enumerate}
\end{proposition}

\begin{remark}
The key point is that $n$ counts something discrete and algebraically
accessible: roots, lobes, touchings, or degree. We do not need to
invoke continuous angles to define or compute with $n$. The polynomial
$T_n$ is determined by the recurrence $T_{n+1} = 2xT_n - T_{n-1}$
with initial conditions $T_0 = 1$, $T_1 = x$---no trigonometry required.
\end{remark}

\begin{remark}[Why $n \geq 2$]
The restriction $n \geq 2$ in the lobe area identity is essential.
For $n = 1$: $L_1(x) = T_2(x) - xT_1(x) = 2x^2 - 1 - x^2 = x^2 - 1$,
giving $\int_{-1}^{1}|x^2-1|\,dx = 4/3 \neq 1$. The ``monogon'' does
not participate in the constant-area phenomenon. The case $n = 2$
is also special (a ``digon'') and corresponds to a removable singularity
in the general lobe area formula $A(n,k)$.
\end{remark}

\subsection{Composition Property}

The Chebyshev composition identity:
\begin{equation}
    T_m(T_n(x)) = T_{mn}(x)
\end{equation}
encodes the multiplicative structure of rotations:
\begin{quote}
``$m$ groups of $n$ twists equals $mn$ twists total.''
\end{quote}

This is the algebraic shadow of the angle addition formula,
expressed without any reference to $\pi$.

\section{The Chebyshev Lobe Curve and Areas}

\subsection{Definition}

\begin{definition}[Chebyshev Lobe Curve \cite{chebyshev-integral}]
For $n \geq 2$, the \emph{Chebyshev lobe curve} is defined by
\begin{equation}
    L_n(x) = T_{n+1}(x) - x \cdot T_n(x), \quad x \in [-1, 1]
\end{equation}
This function creates $n$ lobes between consecutive roots of $T_n$.
The curve touches the unit circle at exactly $n$ points, namely at
$x_j = \cos\frac{(2j+1)\pi}{2n}$ for $j = 0, \ldots, n-1$.
\end{definition}

\subsection{Lobe Area Identity}

\begin{theorem}[Lobe Sum Identity \cite{chebyshev-integral}]
For $n \geq 2$, the total unsigned area under the Chebyshev lobe curve is:
\begin{equation}
    \int_{-1}^{1} |L_n(x)| \, dx = 1
\end{equation}
\end{theorem}

\begin{proof}
Via the substitution $x = \cos\theta$, one shows that
$L_n(\cos\theta) = -\sin(n\theta)\sin\theta$. The integral becomes
\[
    \int_{0}^{\pi} |\sin(n\theta)| \sin^2\theta \, d\theta
    = \frac{1}{2}\left(\int_{0}^{\pi} |\sin(n\theta)| \, d\theta
      - \int_{0}^{\pi} |\sin(n\theta)| \cos(2\theta) \, d\theta\right).
\]
The first integral equals $2$ (by periodicity of $|\sin(n\theta)|$),
and the second vanishes by symmetry about $\theta = \pi/2$.
See \cite{chebyshev-integral} for details.
\end{proof}

\begin{remark}[Continuous Extension]
The lobe area formula extends naturally beyond integer parameters.
Define \cite{lobe-kernel}:
\[
A(n,k) = \frac{1}{n} + \frac{n\cos(\pi/n)}{4-n^2}\cos\frac{(2k-1)\pi}{n}
\]
This function is well-defined for all $n \in \mathbb{R} \setminus \{0, \pm 2\}$
and $k \in \mathbb{R}$ (or even $\mathbb{C}$). The integral identity
$\int_0^n A(n,k)\,dk = 1$ holds for arbitrary real $n > 2$, not just integers.
Thus the ``polygon number'' $n$ becomes a continuous parameter, and rational
values like $n = 7/2$ (a ``triangle-and-a-half'') are meaningful.
\end{remark}

\subsection{Rational Area Ratio}

\begin{proposition}[Rational Ratio]
The Chebyshev lobe curve area, relative to natural reference regions,
gives rational ratios:
\begin{align}
    \frac{\text{Lobe curve area}}{\text{Circumscribed square area}}
        &= \frac{1}{4} \\
    \frac{\text{Lobe curve area}}{\text{Inscribed square area}}
        &= \frac{1}{2}
\end{align}
\end{proposition}

\begin{remark}
The circumscribed square (bounding box) has vertices at $(\pm 1, \pm 1)$
and area 4. No $\sqrt{2}$ appears in either the coordinates or the area.
The ratio $1/4$ is purely rational.
\end{remark}

\section{The Bridge to Rational Geometry}

\subsection{What Chebyshev Provides}

\begin{enumerate}
    \item \textbf{Discrete twist count} $n$ replaces continuous angle $\theta$
    \item \textbf{Algebraic polynomials} $T_n(x)$ encode rotational structure
    \item \textbf{Composition law} $T_m \circ T_n = T_{mn}$ preserves multiplicativity
    \item \textbf{Rational invariant} (area ratio $1/4$) independent of $n$
    \item \textbf{Individual areas} are algebraic (roots of unity), sum is integer
\end{enumerate}

\subsection{Where $\pi$ Appears}

We distinguish three roles of $\pi$ in this context:

\begin{itemize}
    \item \textbf{Historical motivation}: The identity $T_n(\cos\theta) = \cos(n\theta)$
          explains \emph{why} these polynomials encode rotational structure.
          But $T_n$ can be defined purely algebraically via recurrence.
    \item \textbf{Notational convenience}: Writing $\cos(2\pi k/n)$ is shorthand for
          $\text{Re}(\zeta_n^k)$ where $\zeta_n$ is a primitive $n$-th
          root of unity---an algebraic number satisfying $z^n = 1$.
          Similarly, $e^{2\pi i/n}$ is just ``the'' $n$-th root of unity.
    \item \textbf{Circle comparisons}: Comparing to disk area gives ratio
          $1/\pi$ (transcendental). We deliberately avoid this by using
          the bounding square, yielding the rational ratio $1/4$.
\end{itemize}

The key distinction: $\pi$ appears as a \emph{parameterization} of algebraic
objects (roots of unity), not as an essential transcendental constant.
The lobe area function $A(n,k)$ extends to all real (even complex) arguments,
and the identity $\int_0^n A(n,k)\,dk = 1$ holds without reference to circles.
In this sense, $\pi$ in the formula is like using degrees instead of radians---a
notational choice, not a mathematical necessity.

\section{Comparison: Turn vs Oscillation Count}

Wildberger's \textbf{turn} and Chebyshev \textbf{oscillation count} address
different aspects of orientation. (Note: Wildberger uses ``twist'' to mean
turn$^2$; we use ``oscillation count'' to avoid confusion.)

\begin{center}
\begin{tabular}{lll}
\textbf{Aspect} & \textbf{Wildberger's Turn} & \textbf{Chebyshev $n$} \\
\hline
Type & Rational number $u \in \mathbb{Q}$ & Integer $n \in \mathbb{Z}^+$ \\
Measures & Local angle between lines & Oscillation frequency \\
``One oscillation'' & $u = \tan\theta$ & $n = 1$ \\
``Two oscillations'' & $u = \tan\theta$ (same!) & $n = 2$ \\
Key formula & $u_1 + u_2 + u_3 = u_1 u_2 u_3$ & $T_m(T_n(x)) = T_{mn}(x)$ \\
Structure & Additive (linear in each $u_i$) & Multiplicative \\
\end{tabular}
\end{center}

The Chebyshev composition property $T_m(T_n(x)) = T_{mn}(x)$ encodes that
``$m$ groups of $n$ oscillations equals $mn$ oscillations''---a statement about
\emph{counting} that Wildberger's local turn cannot express.

Thus, Chebyshev polynomials complement rational trigonometry by providing
\textbf{discrete oscillation counting} (the integer $n$) alongside
Wildberger's \textbf{continuous local orientation} (the rational turn $u$).

\section{Observations for Wildberger's Program}

We observe that Chebyshev polynomials exhibit several features that
may be of interest to Wildberger's rational trigonometry program:

\begin{enumerate}
    \item A \textbf{discrete oscillation count} via integer $n$
    \item \textbf{Algebraic encoding} of ``going around'' without angles
    \item \textbf{Multiplicative structure} via composition $T_m \circ T_n = T_{mn}$
    \item \textbf{Rational invariants} that are independent of twist count
\end{enumerate}

\subsection{The Multiplicative Decomposition}

The following result on lobe area decomposition supports this view:

\begin{theorem}[Multiplicative Decomposition \cite{lobe-kernel}]
For composite $n = m \cdot d$ with $m, d \geq 2$, let $A(n, k)$ denote
the area of the $k$-th lobe. Then:
\begin{equation}
    \sum_{k \equiv r \pmod{m}} A(n, k) = \frac{1}{m}
    \quad \text{for all } r \in \{1, \ldots, m\}
\end{equation}
\end{theorem}

The proof follows from roots of unity cancellation in the lobe area
formula. This shows that lobe areas ``factor'' according to the
factorization of $n$, providing a geometric analogue of divisor
structure---all within a rational framework.

\subsection{Spread as Chebyshev Polynomial}

Wildberger's \textbf{spread} between two lines is defined as $s = \sin^2\theta$,
avoiding the angle $\theta$ itself. We observe that spread has a purely
polynomial expression via Chebyshev polynomials of the second kind.

\begin{proposition}[Spread via Chebyshev $U_n$]
Let $x = \cos\theta$ parametrize an ``angular position'' algebraically.
The spread of the $n$-fold angle is:
\begin{equation}
    s_n(x) = (1 - x^2) \cdot U_{n-1}(x)^2
\end{equation}
where $U_{n-1}$ is the Chebyshev polynomial of the second kind.
\end{proposition}

\begin{proof}
The identity $U_{n-1}(\cos\theta) = \frac{\sin(n\theta)}{\sin\theta}$ gives:
\[
    \sin^2(n\theta) = \sin^2\theta \cdot U_{n-1}(\cos\theta)^2
    = (1 - \cos^2\theta) \cdot U_{n-1}(\cos\theta)^2
\]
Substituting $x = \cos\theta$ yields the result.
\end{proof}

\begin{remark}
The polynomials $U_n$ satisfy the same recurrence as $T_n$:
\begin{align}
    U_0(x) &= 1, \quad U_1(x) = 2x \\
    U_{n+1}(x) &= 2x \cdot U_n(x) - U_{n-1}(x)
\end{align}
Thus $s_n(x)$ is defined purely algebraically---no trigonometry required.
\end{remark}

\textbf{Examples:}
\begin{center}
\begin{tabular}{ccc}
$n$ & $U_{n-1}(x)$ & $s_n(x) = (1-x^2)U_{n-1}(x)^2$ \\
\hline
$1$ & $1$ & $1 - x^2$ \quad (Wildberger's spread!) \\
$2$ & $2x$ & $4x^2(1-x^2)$ \\
$3$ & $4x^2-1$ & $(1-x^2)(4x^2-1)^2$ \\
$4$ & $8x^3-4x$ & $(1-x^2)(8x^3-4x)^2$ \\
\end{tabular}
\end{center}

\noindent The case $n=1$ recovers Wildberger's original spread $s = \sin^2\theta = 1 - \cos^2\theta = 1 - x^2$ exactly.
Higher $n$ give the ``spread of the $n$-fold angle.''

This suggests a potential bridge: Wildberger's spread algebra could be
reformulated entirely in terms of Chebyshev polynomials $U_n$, with
the parameter $x \in [-1,1]$ replacing the transcendental angle $\theta$.
For integer $n$, the ``$n$-fold spread'' $s_n(x)$ is a polynomial of degree $2n$
in $x$, with integer coefficients.

\begin{remark}[Limitation: Integer $n$ Only]
Unlike the lobe area function $A(n,k)$ which extends smoothly to non-integer $n$
while remaining algebraic (involving only $\cos(2\pi k/n)$), the spread formula
$s_n(x) = (1-x^2)U_{n-1}(x)^2$ loses its polynomial character for non-integer $n$.
For $n \notin \mathbb{Z}$, the Chebyshev function $U_{n-1}(x)$ is defined as
$\sin(n\theta)/\sin\theta$ where $x = \cos\theta$---a transcendental function.
The identity still holds numerically, but the purely algebraic nature is lost.
\end{remark}

\section{Conclusion}

Chebyshev polynomials may offer a path to recovering orientation
within Wildberger's program. The key insight is replacing continuous
angle measurement (requiring $\pi$) with discrete twist counting
(using integers). The algebraic structure of Chebyshev polynomials
encodes rotational information, while the area identities yield
purely rational invariants.

This is not a claim to have ``solved'' the reformulation of
transcendental constants. Rather, it is an observation that
Chebyshev geometry might be a productive direction for investigation.

\section*{Acknowledgments}

This exploration arose from a conversation with Claude (Anthropic)
about the Chebyshev Integral Identity and its connection to lobe areas.
The ``Wildberger framing'' emerged from asking adversarial questions
about whether $\pi$ could be eliminated from the results.
Claude provided substantial assistance in formulating the arguments
and identifying logical gaps.

\begin{thebibliography}{9}

\bibitem{wildberger2005}
N.~J. Wildberger,
\emph{Divine Proportions: Rational Trigonometry to Universal Geometry},
Wild Egg Books, 2005.

\bibitem{chebyshev}
P.~L. Chebyshev,
``Th\'eorie des m\'ecanismes connus sous le nom de parall\'elogrammes,''
\emph{M\'emoires pr\'esent\'es \`a l'Acad\'emie Imp\'eriale des Sciences
de St-P\'etersbourg}, 1854.

\bibitem{chebyshev-integral}
J.~Popelka,
``The $1/\pi$ Invariant in Chebyshev Polynomial Geometry,''
Orbit repository, v0.1.0-chebyshev-integral,
\url{https://github.com/popojan/orbit}, November 2025.

\bibitem{lobe-kernel}
J.~Popelka,
``Chebyshev Lobe Area Kernel and Modular Structure,''
Orbit repository,
\url{https://github.com/popojan/orbit}, December 2025.

\bibitem{orbit}
J.~Popelka,
``Orbit: Mathematical Explorations,''
\url{https://github.com/popojan/orbit}, 2025.

\end{thebibliography}

\end{document}
