\documentclass[11pt]{article}
\usepackage[margin=1in]{geometry}
\usepackage{amsmath,amssymb,amsthm}
\usepackage[colorlinks=true,linkcolor=blue,citecolor=blue,urlcolor=blue]{hyperref}

\newtheorem{theorem}{Theorem}
\newtheorem{conjecture}[theorem]{Conjecture}
\newtheorem{proposition}[theorem]{Proposition}
\theoremstyle{definition}
\newtheorem{definition}[theorem]{Definition}
\newtheorem{example}[theorem]{Example}
\theoremstyle{remark}
\newtheorem{remark}[theorem]{Remark}

\title{A Primorial Formula via Alternating Factorial Sums}
\author{Jan Geuss Popelka\thanks{Independent researcher. Email: popojan@protonmail.com. Code: \url{https://github.com/popojan/orbit}}}
\date{\today}

\begin{document}

\maketitle

\begin{abstract}
We present a computational discovery: the primorial $p\#$ (product of all primes up to $p$) can be extracted as the denominator of a specific alternating factorial sum. For $m \geq 3$, we have
\[
\text{den}\left(\frac{1}{2} \sum_{k=1}^{\lfloor(m-1)/2\rfloor} \frac{(-1)^k \cdot k!}{2k+1}\right) = \prod_{\substack{p \text{ prime} \\ p \leq m}} p
\]
where $\text{den}(q)$ denotes the denominator of rational $q$ in lowest terms. The formula has been verified computationally for $m$ up to 100,000. A rigorous proof has not yet been established. The primary theoretical challenge concerns a systematic cancellation: although the individual denominators $\{2k+1\}$ contain prime powers ($9 = 3^2$, $25 = 5^2$, $27 = 3^3$, \ldots), the final reduced denominator contains only first powers of primes. We formulate this as an open problem about $p$-adic valuations and discuss possible theoretical connections.
\end{abstract}

\section{Introduction}

The primorial function, defined as the product of all primes up to a given bound, appears naturally in number theory, from bounds in analytic number theory to the construction of highly composite numbers. While primorials are typically computed by direct multiplication, we have discovered an unexpected alternative: they emerge as denominators of alternating factorial sums.

This discovery arose through computational experimentation with Egyptian fraction decompositions and factorial series. While investigating various rational sum patterns, we observed that certain alternating sums exhibited denominators containing only products of distinct primes after reduction to lowest terms. Systematic testing revealed the formula presented here.

While the formula admits a concise statement and computational verification is straightforward, the underlying cancellation mechanism requires careful analysis. The numerators, built from factorials, systematically eliminate all higher prime powers from the naive least common multiple of the denominators. This paper presents the formula, provides computational evidence, and formulates the cancellation as a formal open problem.

\section{The Formula}

\begin{theorem}[Computational]
For any integer $m \geq 3$, define the rational number
\begin{equation}\label{eq:main}
S_m = \frac{1}{2} \sum_{k=1}^{h} \frac{(-1)^k \cdot k!}{2k+1}
\end{equation}
where $h = \lfloor(m-1)/2\rfloor$. Then the denominator of $S_m$ in lowest terms equals the $m$-primorial:
\[
\text{den}(S_m) = \prod_{\substack{p \text{ prime} \\ p \leq m}} p
\]
\end{theorem}

\begin{remark}
The case $m = 2$ requires special handling and returns $2$ directly. For $m \geq 3$, the formula applies universally.
\end{remark}

\begin{example}
For $m = 13$, we have $h = 6$ and compute:
\begin{align*}
S_{13} &= \frac{1}{2}\left(-\frac{1!}{3} + \frac{2!}{5} - \frac{3!}{7} + \frac{4!}{9} - \frac{5!}{11} + \frac{6!}{13}\right)\\
&= \frac{695971}{30030}
\end{align*}
The denominator $30030 = 2 \times 3 \times 5 \times 7 \times 11 \times 13$ is precisely the product of all primes up to 13.
\end{example}

\subsection{Pattern Discovery}

The key observation is that the denominator grows by accumulating new prime factors:

\begin{proposition}
Let $S_k = \frac{1}{2} \sum_{i=1}^{k} \frac{(-1)^i \cdot i!}{2i+1}$ denote the partial sum. Then
\[
\text{den}(S_k) = 2 \times \prod_{\substack{p \text{ prime} \\ 3 \leq p \leq 2k+1}} p
\]
\end{proposition}

The denominators stabilize when $2k+1$ is composite. New primes enter only when $2k+1$ itself is prime. Table~\ref{tab:example} illustrates this for $m = 13$:

\begin{table}[h]
\centering
\begin{tabular}{c|c|c|c}
$k$ & $2k+1$ & Prime? & $\text{den}(S_k)$ \\ \hline
1 & 3 & \checkmark & $2 \times 3$ \\
2 & 5 & \checkmark & $2 \times 3 \times 5$ \\
3 & 7 & \checkmark & $2 \times 3 \times 5 \times 7$ \\
4 & 9 & & $2 \times 3 \times 5 \times 7$ \\
5 & 11 & \checkmark & $2 \times 3 \times 5 \times 7 \times 11$ \\
6 & 13 & \checkmark & $2 \times 3 \times 5 \times 7 \times 11 \times 13$
\end{tabular}
\caption{Denominator growth for $m = 13$. Note that the denominator remains unchanged at $k = 4$ despite $9 = 3^2$.}
\label{tab:example}
\end{table}

\section{The Cancellation Problem}

The primary theoretical challenge is as follows: the sequence $\{2k+1\}_{k=1}^{h}$ contains prime powers and composites:
\[
3, 5, 7, 9=3^2, 11, 13, 15=3 \times 5, 17, 19, 21=3 \times 7, 23, 25=5^2, 27=3^3, \ldots
\]

A direct computation of $\text{LCM}(3, 5, 7, 9, 11, \ldots)$ would retain $3^2$ from 9, $3^3$ from 27, $5^2$ from 25, and so on. Yet after summing and reducing (\ref{eq:main}) to lowest terms, the denominator contains \emph{only first powers} of primes.

\subsection{P-adic Valuation Formulation}

Let $\nu_p(n)$ denote the $p$-adic valuation (exponent of $p$ in the prime factorization of $n$). Our computational investigations reveal:

\begin{conjecture}\label{conj:main}
For all primes $p$ with $3 \leq p \leq 2k+1$, we have
\[
\nu_p\left(\text{den}(S_k)\right) = 1
\]
and
\[
\nu_p\left(\text{num}(S_k)\right) = 0
\]
where $\text{num}(q)$ denotes the numerator of $q$ in lowest terms.
\end{conjecture}

This represents the fundamental theoretical question: proving that the numerators systematically contain exactly the right prime factors to cancel all $p^j$ with $j > 1$ through GCD reduction.

\subsection{Two Cancellation Mechanisms}

Computational investigation reveals two distinct regimes:

\textbf{Small $k$ (GCD cancellation):} When $\nu_p(k!) < \nu_p(2k+1)$ for some prime $p$ dividing $2k+1$, the combined numerator after addition contains factors of $p$, and GCD reduction eliminates exactly the excess powers.

\textbf{Example:} At $k = 4$, we have $2k+1 = 9 = 3^2$, but $\nu_3(4!) = 1 < 2$. The numerator after combining with previous terms contains exactly one factor of 3, reducing $3^2 \to 3^1$ in the denominator.

\textbf{Large $k$ (integer terms):} For sufficiently large $k$, Legendre's formula
\[
\nu_p(k!) = \sum_{i=1}^{\infty} \left\lfloor \frac{k}{p^i} \right\rfloor
\]
ensures that $\nu_p(k!) \geq \nu_p(2k+1)$ for all $p | (2k+1)$. In these cases, the term $\frac{k!}{2k+1}$ reduces to an \emph{integer}, and no new denominator factors enter.

\textbf{Example:} At $k = 12$, we have $2k+1 = 25 = 5^2$ and $\nu_5(12!) = 2 \geq 2$, so the term is already an integer.

\subsection{The Role of the Alternating Sign}

The alternating factor $(-1)^k$ is essential. Without it, the formula loses the factor 3 at $k = 4$ and never recovers, yielding $\text{Primorial}/3$ for all $m \geq 9$. The alternating sign controls the numerator structure to prevent over-cancellation at critical steps.

\section{Computational Verification}

The formula has been verified exhaustively for all $m$ from 3 to 100,000, testing every integer value (both composite and prime). At $m = 100{,}000$, the primorial has 43,293 digits; all denominators match exactly.

\subsection{Iterative Formulation}

To enable large-scale verification, we derived an efficient iterative formulation. Starting from the direct sum (\ref{eq:main}), the partial sums can be computed via a three-term recurrence:

\begin{align}
S_0 &= \{0, 0, 1\}\\
S_{n+1} &= \{n+1, b_n, b_n + (a_n - b_n)\cdot\left(n + \frac{1}{3+2n}\right)\}
\end{align}

where the state is $S_n = \{n, a_n, b_n\}$. After $h = \lfloor(m-1)/2\rfloor$ iterations, the primorial is extracted as:
\[
\text{Primorial}(m) = 2 \cdot \text{den}(b_h - 1)
\]

The recurrence tracks $b_n = 1 + 2S_n$, and the factor of 2 is unwrapped by the extraction formula. This enables $O(m)$ arithmetic operations and facilitates verification at scale. Table~\ref{tab:verification} shows key verification checkpoints:

\begin{table}[h]
\centering
\begin{tabular}{r|r|r}
$m$ & $\pi(m)$ & Primorial digits \\ \hline
100 & 25 & 37 \\
1{,}000 & 168 & 416 \\
10{,}000 & 1{,}229 & 3{,}393 \\
100{,}000 & 9{,}592 & 43{,}293
\end{tabular}
\caption{Verification checkpoints. All values tested exhaustively (every $m$ from 3 onwards).}
\label{tab:verification}
\end{table}

\section{Open Questions}

Several theoretical questions remain:

\begin{enumerate}
\item \textbf{Rigorous proof:} Prove Conjecture~\ref{conj:main} using $p$-adic analysis or other techniques.

\item \textbf{Generating function connection:} Does this sum have an interpretation as a generating function for primorials?

\item \textbf{Generalization:} What happens if we modify the formula? For instance:
\begin{itemize}
\item Changing the coefficient: $\frac{1}{2} \to \frac{1}{c}$ for other constants $c$
\item Modifying denominators: $2k+1 \to ak+b$ for other progressions
\item Altering numerators: $k! \to (2k)!$ or other factorial-like sequences
\end{itemize}

\item \textbf{Connection to other identities:} Are there links to Wilson's theorem, Wolstenholme's theorem, or harmonic number denominators (which also involve primorials)?

\item \textbf{Complexity:} What is the computational complexity of this formula compared to direct primorial computation?
\end{enumerate}

\section{Conclusion}

We have presented a computationally verified formula expressing primorials as denominators of alternating factorial sums. The systematic cancellation of higher prime powers remains unexplained. Techniques from $p$-adic analysis, generating functions, or modular arithmetic may provide insight into the underlying mechanism.

\section*{Acknowledgments}

This work emerged from computational explorations using the Wolfram Language. I thank the online mathematical community for preliminary discussions and feedback on early versions of this result.

\begin{thebibliography}{9}

\bibitem{legendre}
A.-M. Legendre,
\emph{Th\'eorie des nombres},
Firmin Didot Fr\`eres, Paris, 1830.

\bibitem{oeis}
OEIS Foundation Inc.,
The On-Line Encyclopedia of Integer Sequences,
\url{https://oeis.org}.
Relevant sequences: A002110 (primorials), A034386 (primorials of primes).

\end{thebibliography}

\end{document}
