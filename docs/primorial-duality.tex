\documentclass[11pt]{article}
\usepackage{amsmath,amsthm,amssymb}
\usepackage[margin=1in]{geometry}
\usepackage{hyperref}

\newtheorem{theorem}{Theorem}[section]
\newtheorem{lemma}[theorem]{Lemma}
\newtheorem{corollary}[theorem]{Corollary}
\newtheorem{proposition}[theorem]{Proposition}
\theoremstyle{definition}
\newtheorem{definition}[theorem]{Definition}
\theoremstyle{remark}
\newtheorem{remark}[theorem]{Remark}

\newcommand{\N}{\mathbb{N}}
\newcommand{\Z}{\mathbb{Z}}
\newcommand{\Q}{\mathbb{Q}}
\newcommand{\Primorial}{\mathrm{Primorial}}
\newcommand{\nup}{\nu_p}

\title{Primorials Through Factorial Sums:\\
A Dual Characterization via Primes and Composites}

\author{Anonymous}
\date{\today}

\begin{document}

\maketitle

\begin{abstract}
We establish alternating factorial sum formulas that characterize primorials through rational arithmetic, revealing an unexpected structural duality: both prime and composite numbers appear in explicit closed forms obtained through reduction to lowest terms, while all remaining complexity concentrates in the numerators. The formulas exhibit perfect mathematical circularity---composite numbers are needed to compute the greatest common divisor, but identifying composites requires knowing primes. This self-referential structure suggests fundamental computational limits on closed-form prime characterizations. We prove our results using p-adic valuation analysis and demonstrate a precise three-way decomposition of complexity.
\end{abstract}

\section{Introduction}

The primorial function, defined as the product of all primes up to a given bound, is fundamental in number theory:
\[
\Primorial(m) = \prod_{\substack{p \le m \\ p \text{ prime}}} p
\]

We present a surprising alternative characterization: primorials emerge as denominators of certain factorial sums when reduced to lowest terms. Moreover, the reduction process reveals a perfect duality between primes and composite numbers, with each appearing in explicit closed forms.

\begin{theorem}[Alternating Formula]
\label{thm:alternating}
For odd integers $m \geq 3$, let
\[
S_m = \frac{1}{2} \sum_{k=1}^{\lfloor (m-1)/2 \rfloor} \frac{(-1)^k \cdot k!}{2k+1}
\]
Then $\mathrm{Denominator}[S_m] = \Primorial(m)$.
\end{theorem}

\begin{theorem}[Non-Alternating Formula]
\label{thm:nonalternating}
For odd integers $m \geq 9$, let
\[
T_m = \frac{1}{6} \sum_{k=1}^{\lfloor (m-1)/2 \rfloor} \frac{k!}{2k+1}
\]
Then $\mathrm{Denominator}[T_m] = \Primorial(m)$.
\end{theorem}

These theorems establish two parallel characterizations of primorials. Remarkably, the reduction to lowest terms producing these denominators follows a predictable pattern involving composite numbers.

\section{The Three-Way Decomposition}

Before proving the main theorems, we establish the structural framework underlying both formulas. To facilitate the analysis, we work with the \emph{bare sums} (without the prefactors $1/2$ or $1/6$ from Theorems~\ref{thm:alternating} and~\ref{thm:nonalternating}).

\subsection{Unreduced vs Reduced Forms}

\begin{definition}
For odd $m \geq 3$, let $k = \lfloor(m-1)/2\rfloor$ and define the bare sums:
\begin{align*}
\Sigma_m^{\text{alt}} &= \sum_{i=1}^{k} \frac{(-1)^i \cdot i!}{2i+1} \quad \text{(alternating, no prefactor)} \\
\Sigma_m^{\text{nonalt}} &= \sum_{i=1}^{k} \frac{i!}{2i+1} \quad \text{(non-alternating, no prefactor)}
\end{align*}

For each bare sum, write the unreduced representation as $N_{\text{unred}} / D_{\text{unred}}$ where:
\begin{itemize}
\item $N_{\text{unred}}$ = unreduced numerator
\item $D_{\text{unred}} = 2 \cdot (2k+1)!! = 2 \cdot 3 \cdot 5 \cdot 7 \cdots (2k+1)$ = unreduced denominator
\end{itemize}

After reducing to lowest terms, let $N_{\text{red}} / D_{\text{red}}$ denote the reduced form, and define:
\[
G = \gcd(N_{\text{unred}}, D_{\text{unred}}) = \frac{D_{\text{unred}}}{D_{\text{red}}}
\]
\end{definition}

\begin{remark}
The final formulas $S_m$ and $T_m$ in Theorems~\ref{thm:alternating} and~\ref{thm:nonalternating} are obtained by multiplying these bare sums by $1/2$ and $1/6$ respectively. The denominators transform as:
\begin{align*}
\text{Denominator}[S_m] &= \text{Denominator}\left[\frac{1}{2} \cdot \Sigma_m^{\text{alt}}\right] = 2 \cdot D_{\text{red}}^{\text{alt}} = \Primorial(m) \\
\text{Denominator}[T_m] &= \text{Denominator}\left[\frac{1}{6} \cdot \Sigma_m^{\text{nonalt}}\right] = 6 \cdot D_{\text{red}}^{\text{nonalt}} = \Primorial(m)
\end{align*}
\end{remark}

\begin{theorem}[GCD Closed Form]
\label{thm:gcd}
Let $\mathcal{C}_m = \{9, 15, 21, 25, 27, \ldots\}$ denote the odd composite numbers not exceeding $m$.

For the bare alternating sum $\Sigma_m^{\text{alt}}$:
\[
G_{\text{alt}} = \begin{cases}
2 & \text{if } m \in \{3,5,7\} \\
2 \cdot \prod_{c \in \mathcal{C}_m} c & \text{if } m \geq 9
\end{cases}
\]

For the bare non-alternating sum $\Sigma_m^{\text{nonalt}}$:
\[
G_{\text{nonalt}} = \begin{cases}
2 & \text{if } m \in \{3,5,7\} \\
6 \cdot \prod_{c \in \mathcal{C}_m} c & \text{if } m \geq 9
\end{cases}
\]
\end{theorem}

\begin{proof}
We prove this using p-adic valuations. The proof proceeds in several steps.

\textbf{Step 1: Reduced denominator of bare alternating sum.}

By Theorem~\ref{thm:invariant} (the p-adic invariant), each odd prime $p \leq m$ satisfies $\nup(D_{\text{red}}) = 1$ in the reduced representation. Additionally, the factor of 2 from $D_{\text{unred}} = 2 \cdot (2k+1)!!$ survives reduction because numerators are odd.

The reduced denominator for the bare sum $\Sigma_m^{\text{alt}}$ contains one factor of 2 and each odd prime to the first power. Since $\Primorial(m) = 2 \times (\text{odd primes } \leq m)$, we have:
\[
D_{\text{red}}^{\text{alt}} = 2 \times \prod_{\substack{3 \leq p \leq m \\ p \text{ prime}}} p = \frac{\Primorial(m)}{2} \times 2
\]

Note that $\Primorial(m)/2$ denotes the product of odd primes only, so the extra factor of 2 gives us half of $\Primorial(m)$.

\textbf{Step 2: 2-adic valuation of GCD.}

Since $(2k+1)!!$ is a product of odd numbers only:
\begin{align*}
\nu_2(G_{\text{alt}}) &= \nu_2(D_{\text{unred}}) - \nu_2(D_{\text{red}}^{\text{alt}}) \\
&= \nu_2(2 \cdot (2k+1)!!) - \nu_2(\Primorial(m)/2) \\
&= 1 - 0 = 1
\end{align*}

Therefore $G_{\text{alt}}$ has exactly one factor of 2.

\textbf{Step 3: p-adic valuation for odd primes.}

For odd prime $p$:
\begin{align*}
\nup(G_{\text{alt}}) &= \nup(D_{\text{unred}}) - \nup(D_{\text{red}}^{\text{alt}}) \\
&= \nup((2k+1)!!) - \nup(\Primorial(m)/2) \\
&= \nup((2k+1)!!) - 1
\end{align*}

\textbf{Step 4: Structure of odd double factorial.}

The odd double factorial $(2k+1)!! = 1 \cdot 3 \cdot 5 \cdots (2k+1)$ contains each odd prime $q \leq 2k+1$ at least once, each odd composite $c \leq 2k+1$ as a factor, and higher odd multiples contributing additional prime factors.

\textbf{Step 5: Excess valuation from composites.}

The key observation is that the excess valuation $\nup((2k+1)!!) - 1$ comes precisely from odd composites $c \leq m$: each odd prime $q \leq m$ contributes valuation 1 (matching $\Primorial(m)/2$, no excess), while each odd composite $c \leq m$ contributes its full prime factorization to $(2k+1)!!$, creating excess valuation.

Summing over all primes:
\[
G_{\text{alt}} = 2 \cdot \prod_{c \in \mathcal{C}_m} c
\]

for $m \geq 9$. For $m \in \{3,5,7\}$, there are no odd composites, so $G_{\text{alt}} = 2$.

\textbf{Step 6: Non-alternating sum and the factor of 3.}

For the bare non-alternating sum $\Sigma_m^{\text{nonalt}}$, removing the alternating sign $(-1)^i$ affects the cancellation at $i = 1$ (where the denominator is 3). Direct computation shows:
\begin{align*}
\Sigma_3^{\text{alt}} &= -\frac{1}{3}, \quad D_{\text{red}}^{\text{alt}} = 3 \\
\Sigma_3^{\text{nonalt}} &= +\frac{1}{3}, \quad D_{\text{red}}^{\text{nonalt}} = 3
\end{align*}

For $m = 5$:
\begin{align*}
\Sigma_5^{\text{alt}} &= -\frac{1}{3} + \frac{2}{5} = \frac{1}{15}, \quad D_{\text{red}}^{\text{alt}} = 15 \\
\Sigma_5^{\text{nonalt}} &= +\frac{1}{3} + \frac{2}{5} = \frac{11}{15}, \quad D_{\text{red}}^{\text{nonalt}} = 15
\end{align*}

For $m = 9$ (first composite in denominator sequence):
\begin{align*}
D_{\text{red}}^{\text{alt}} &= 105 = 3 \times 5 \times 7 \\
D_{\text{red}}^{\text{nonalt}} &= 35 = 5 \times 7 = \frac{105}{3}
\end{align*}

The pattern emerges: for $m \geq 9$, the non-alternating sum loses a factor of 3 in the reduced denominator compared to the alternating sum:
\[
D_{\text{red}}^{\text{nonalt}} = \frac{D_{\text{red}}^{\text{alt}}}{3} = \frac{\Primorial(m)}{6}
\]

Since $G = D_{\text{unred}} / D_{\text{red}}$ and $D_{\text{unred}}$ is identical for both sums, we have:
\[
G_{\text{nonalt}} = \frac{D_{\text{unred}}}{D_{\text{red}}^{\text{nonalt}}} = \frac{D_{\text{unred}}}{D_{\text{red}}^{\text{alt}} / 3} = 3 \cdot G_{\text{alt}} = 6 \cdot \prod_{c \in \mathcal{C}_m} c
\]

for $m \geq 9$. For $m \in \{3, 5, 7\}$, the reduced denominators are identical, so $G_{\text{nonalt}} = G_{\text{alt}} = 2$.

This completes the proof.
\end{proof}

\begin{corollary}[Structural Decomposition]
\label{cor:decomposition}
The bare factorial sums admit a canonical three-way decomposition:
\[
\Sigma_m = \frac{N_{\text{red}}}{G \cdot D_{\text{red}}}
\]
where:
\begin{enumerate}
\item $D_{\text{red}}$ encodes \textbf{prime structure}: $\Primorial(m)/2$ for alternating, $\Primorial(m)/6$ for non-alternating
\item $G$ encodes \textbf{composite structure}: $2 \cdot \prod\{\text{odd composites } \leq m\}$ for alternating, $6 \times \prod\{\text{odd composites } \leq m\}$ for non-alternating (for $m \geq 9$)
\item $N_{\text{red}}$ absorbs \textbf{residual complexity}: no closed form known
\end{enumerate}

The final formulas $S_m = \frac{1}{2} \Sigma_m^{\text{alt}}$ and $T_m = \frac{1}{6} \Sigma_m^{\text{nonalt}}$ have denominators equal to $\Primorial(m)$ exactly, as the prefactors $1/2$ and $1/6$ cancel with the $2$ and $6$ in the respective $D_{\text{red}}$ values.
\end{corollary}

\section{Proof of Main Theorems}

We prove Theorem~\ref{thm:alternating} using p-adic valuation analysis. The non-alternating case follows similarly.

\subsection{Recurrence Relations}

Define the cumulative sums via recurrence. Let $N_0 = 0, D_0 = 2$, and for $k \geq 1$:
\begin{align*}
N_k &= N_{k-1} \cdot (2k+1) + (-1)^k \cdot k! \cdot D_{k-1} \\
D_k &= D_{k-1} \cdot (2k+1)
\end{align*}

The recurrence computes the unreduced representation of the bare alternating sum $\Sigma_m^{\text{alt}}$ from Section~3. The final formula in Theorem~\ref{thm:alternating} is obtained by multiplying by the prefactor:
\[
S_m = \frac{1}{2} \cdot \frac{N_k}{D_k}
\]
where $k = \lfloor(m-1)/2\rfloor$.

\begin{remark}
For the non-alternating formula (Theorem~\ref{thm:nonalternating}), the recurrence is identical except the numerator update omits the alternating sign:
\[
N_k^{\text{nonalt}} = N_{k-1}^{\text{nonalt}} \cdot (2k+1) + k! \cdot D_{k-1}
\]
with the same denominator recurrence. The final formula is $T_m = \frac{1}{6} \cdot \frac{N_k^{\text{nonalt}}}{D_k}$.
\end{remark}

\subsection{The p-adic Invariant}

For a prime $p$ and integer $n$, let $\nup(n)$ denote the p-adic valuation (the exponent of $p$ in the prime factorization of $n$).

\begin{lemma}[Factorial Inequality]
\label{lem:factorial-ineq}
For any prime $p$ dividing $2k+1$ with $\nup(2k+1) = \alpha \geq 2$, we have:
\[
\nup(k!) \geq \alpha - 1
\]
\end{lemma}

\begin{proof}
By Legendre's formula, $\nup(k!) = \sum_{i \geq 1} \lfloor k/p^i \rfloor$. If $\nup(2k+1) = \alpha$, then $2k+1 \equiv 0 \pmod{p^\alpha}$, so $k \geq (p^\alpha - 1)/2$. For $\alpha \geq 2$:
\[
\nup(k!) \geq \left\lfloor \frac{k}{p} \right\rfloor \geq \left\lfloor \frac{p^\alpha - 1}{2p} \right\rfloor = \left\lfloor \frac{p^{\alpha-1}}{2} - \frac{1}{2p} \right\rfloor \geq \alpha - 1
\]
for all primes $p \geq 3$.
\end{proof}

\begin{theorem}[p-adic Invariant]
\label{thm:invariant}
For all primes $p \leq 2k+1$ with $p \geq 3$, we have $\nup(D_k) - \nup(N_k) = 1$.
\end{theorem}

\begin{proof}
By induction on $k$. Base case $k=1$ is verified directly. For the inductive step, assume the invariant holds at $k-1$.

Let $p$ be a prime with $p \leq 2k+1$ and $\nup(2k+1) = \alpha$. From the recurrence:
\[
N_k = N_{k-1} \cdot (2k+1) + (-1)^k \cdot k! \cdot D_{k-1}
\]

\textbf{Case A: $\alpha = 0$ or $\alpha = 1$.}
Then $\nup(D_k) = \nup(D_{k-1}) + \alpha$. The term $N_{k-1} \cdot (2k+1)$ has $\nup = \nup(N_{k-1}) + \alpha = (\nup(D_{k-1}) - 1) + \alpha = \nup(D_k) - 1$. The term $k! \cdot D_{k-1}$ has $\nup \geq \nup(D_{k-1}) = \nup(D_k) - \alpha \geq \nup(D_k) - 1$. By valuation of sums, $\nup(N_k) = \nup(D_k) - 1$.

\textbf{Case B: $\alpha \geq 2$.}

\textbf{Subcase B1:} $\nup(k!) > \alpha - 1$. Then the term $k! \cdot D_{k-1}$ has strictly higher valuation than $N_{k-1} \cdot (2k+1)$, so $\nup(N_k) = \nup(D_k) - 1$.

\textbf{Subcase B2:} $\nup(k!) = \alpha - 1$ (boundary case). This occurs when $2k+1 = p^2$. We show this cannot happen for $p \geq 5$.

If $2k+1 = p^2$, then $k+1 = (p^2-1)/2$. By Legendre's formula:
\[
\nup((k+1)!) \geq \left\lfloor \frac{p^2-1}{2p} \right\rfloor = \left\lfloor \frac{p}{2} - \frac{1}{2p} \right\rfloor \geq 2 \text{ for } p \geq 5
\]
Thus the boundary condition cannot occur for $p \geq 5$. For $p=3$, direct computation at $k=4$ (where $2k+1 = 9$) verifies the invariant holds.
\end{proof}

\begin{corollary}
The bare alternating sum $\Sigma_m^{\text{alt}} = \frac{N_k}{D_k}$ has reduced denominator $\Primorial(m)/2$ for odd $m \geq 3$. Multiplying by the prefactor $1/2$ yields the final formula $S_m$ with $\mathrm{Denominator}[S_m] = \Primorial(m)$.
\end{corollary}

\begin{proof}
Theorem~\ref{thm:invariant} shows that each odd prime $p \leq m$ appears to exactly the first power in $D_k/\gcd(N_k, D_k)$. The factor of 2 in the unreduced denominator remains after reduction (since numerators are odd for $k \geq 1$). Thus the bare sum has reduced denominator $\Primorial(m)/2$. The prefactor $1/2$ introduces an additional factor of 2 in the final denominator, giving $\mathrm{Denominator}[S_m] = 2 \times (\Primorial(m)/2) = \Primorial(m)$.
\end{proof}

\section{The Circular Structure}

The closed forms in Theorem~\ref{thm:gcd} reveal a profound self-referential property.

\begin{theorem}[Computational Circularity]
\label{thm:circularity}
The formulas exhibit the following logical dependency cycle:
\begin{enumerate}
\item To compute $\Primorial(m)$ from the formula → need $D_{\text{red}}$
\item To compute $D_{\text{red}}$ → need to reduce fraction via $\gcd$
\item To compute $\gcd$ → need product of odd composites $\leq m$ (Theorem~\ref{thm:gcd})
\item To identify odd composites → need to test which odd numbers are prime
\item To test primality → need knowledge of primes $\leq m$
\item Primes $\leq m$ are exactly the factors of $\Primorial(m)$
\end{enumerate}
\end{theorem}

\begin{remark}[Gödelian Aspect]
This circularity resembles Gödel's incompleteness theorem: any formula that completely characterizes primes must contain a computational step at least as hard as identifying primes. The formulas are \emph{mathematically complete} (all components have closed forms) yet \emph{computationally circular} (cannot be evaluated without solving the problem they define).
\end{remark}

\begin{corollary}[Complexity Conservation]
The predictability of the prime structure (Item 1 in Corollary~\ref{cor:decomposition}) and composite structure (Item 2) necessitates that all irreducible computational complexity concentrates in the numerators (Item 3) or in the primality testing required to apply Theorem~\ref{thm:gcd}.
\end{corollary}

The numerators $N_{\text{red}}$ exhibit no discernible pattern across extensive computational tests. Many are prime, but not all. They follow no polynomial recurrence, no simple generating function, and show chaotic growth rates. This appears to be a fundamental consequence of the circular structure.

\section{Prime-Composite Duality}

The formulas reveal a beautiful symmetry between primes and composites.

\begin{theorem}[Symmetric Decomposition of Odd Double Factorial]
\label{thm:duality}
The odd double factorial decomposes as:
\[
(2k+1)!! = \left(\prod_{\substack{p \leq 2k+1 \\ p \text{ odd prime}}} p^{\alpha_p}\right) \cdot \left(\prod_{c \in \mathcal{C}_{2k+1}} c\right)
\]
where the exponents $\alpha_p$ account for repeated odd multiples of each prime $p$.
\end{theorem}

This decomposition manifests in our formulas:
\begin{itemize}
\item \textbf{Primes} appear explicitly in $D_{\text{red}}$ (the primorial factors)
\item \textbf{Composites} appear explicitly in $G$ (Theorem~\ref{thm:gcd})
\item Their \textbf{interaction} produces $N_{\text{red}}$ (unpredictable)
\end{itemize}

\begin{proposition}[Relation Between Formulas]
The two formulas are related by:
\[
G_{\text{nonalt}} = 3 \cdot G_{\text{alt}}
\]
\end{proposition}

\begin{proof}
Both share the same unreduced denominator $D_{\text{unred}} = 2 \cdot (2k+1)!!$. The reduced denominators differ by a factor of 3:
\[
\frac{D_{\text{red}}^{\text{alt}}}{D_{\text{red}}^{\text{nonalt}}} = \frac{\Primorial/2}{\Primorial/6} = 3
\]
Since $G = D_{\text{unred}}/D_{\text{red}}$, the GCDs differ by the reciprocal factor.
\end{proof}

This relation shows that the alternating sign changes the final primorial factor (from $/2$ to $/6$) and the GCD coefficient (from $2$ to $6$), but preserves the underlying composite product structure.

\section{Open Problems and Future Directions}

\begin{enumerate}
\item \textbf{Numerator Structure:} Do the reduced numerators $N_{\text{red}}$ have any predictable number-theoretic properties? Many appear to be prime.

\item \textbf{Breaking Circularity:} Can asymptotic analysis or probabilistic methods extract information about primorials without explicit primality testing?

\item \textbf{Generalizations:} Are there factorial sum formulas producing $\Primorial(m)/c$ for other constants $c$?

\item \textbf{p-adic Lifting:} Can these formulas be lifted to p-adic L-functions or connected to Iwasawa theory?

\item \textbf{Algorithmic Complexity:} Can we prove rigorously that computing via these formulas requires work equivalent to classical primorial computation?

\item \textbf{Connection to PNT:} The asymptotic $\Primorial(m) \sim e^m$ translates to denominator growth. Can this lead to alternative approaches to the Prime Number Theorem?

\item \textbf{Chaos Transfer Principle:} Can the concentration of complexity in numerators be formalized using Kolmogorov complexity or algorithmic information theory?
\end{enumerate}

\section{Conclusion}

We have established factorial sum characterizations of primorials that reveal a perfect duality: primes appear predictably in reduced denominators, composites appear predictably in the greatest common divisors, and all remaining complexity concentrates in numerators. The formulas are mathematically complete (closed forms exist for all components) yet computationally circular (composites and primes mutually reference each other).

This structure suggests that any formula characterizing primes must contain computational steps at least as hard as identifying primes---a kind of complexity conservation law. The beauty lies not in computational advantage, but in the elegant three-way partition of number-theoretic structure it reveals.

\end{document}
