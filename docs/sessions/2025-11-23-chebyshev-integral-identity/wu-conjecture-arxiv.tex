\documentclass[11pt]{article}
\usepackage[utf8]{inputenc}
\usepackage{amsmath,amsthm,amssymb}
\usepackage{geometry}
\geometry{margin=2.5cm}
\usepackage[hidelinks]{hyperref}

\newtheorem{theorem}{Theorem}
\newtheorem{lemma}[theorem]{Lemma}
\newtheorem{conjecture}[theorem]{Conjecture}

\title{Proof of Wu's Conjecture for OEIS A028355}
\author{Jan Popelka\thanks{Email: popojan@protonmail.com}}
\date{}

\begin{document}
\maketitle

\begin{abstract}
We prove a conjecture by Chai Wah Wu (2024) concerning the OEIS sequence A028355,
which arises from the periodic digit sequence $[1,2,3,4,3,2]$ associated with the
striking pattern of the Prague Astronomical Clock. The recurrence
$a(n) = 1000001 \cdot a(n-15) - 1000000 \cdot a(n-30)$ for $n > 30$
follows directly from the periodicity of the underlying digit sequence.
\end{abstract}

\section{The Sequence A028355}

The OEIS sequence A028355 \cite{oeis} is constructed from the infinite periodic
digit sequence
\begin{equation}
d(k) = [1,2,3,4,3,2,1,2,3,4,3,2,\ldots]
\end{equation}
with period $p = [1,2,3,4,3,2]$ of length 6.

For each $n \geq 1$, the term $a(n)$ is defined as the integer formed by
concatenating certain digits from this sequence. The construction satisfies:
\begin{itemize}
\item $\mathrm{start}(n)$ is periodic with period 15
\item $\mathrm{len}(n+15) = \mathrm{len}(n) + 6$
\end{itemize}

The first terms are: $1, 12, 123, 1234, 12343, 123432, 2, 23, 234, 2343, 23432, 234321, 3, 34, 343, \ldots$

\section{Wu's Conjecture}

\begin{conjecture}[Wu, 2024]
For $n > 30$:
\begin{equation}
a(n) = 1000001 \cdot a(n-15) - 1000000 \cdot a(n-30)
\end{equation}
\end{conjecture}

\section{Proof}

\begin{lemma}\label{lem:shift}
For all $n \geq 1$:
\begin{equation}
a(n+15) = a(n) \cdot 10^6 + c(n)
\end{equation}
where $c(n)$ is the 6-digit number formed by the appended digits.
\end{lemma}

\begin{proof}
Since $\mathrm{start}(n+15) = \mathrm{start}(n)$ and $\mathrm{len}(n+15) = \mathrm{len}(n) + 6$,
the term $a(n+15)$ consists of the same digits as $a(n)$ followed by 6 additional digits.
Shifting $a(n)$ by 6 decimal places and adding the new digits gives the result.
\end{proof}

\begin{lemma}\label{lem:periodic}
The sequence $c(n)$ is periodic with period 15, i.e., $c(n+15) = c(n)$.
\end{lemma}

\begin{proof}
The value $c(n)$ is determined by digits $d[\mathrm{start}(n) + \mathrm{len}(n)]$ through
$d[\mathrm{start}(n) + \mathrm{len}(n) + 5]$.

For $c(n+15)$: the starting position shifts by $\mathrm{len}(n+15) - \mathrm{len}(n) = 6$.
Since $d$ has period 6, we have $d[k+6] = d[k]$ for all $k$.
Therefore $c(n+15) = c(n)$.
\end{proof}

\begin{theorem}
For $n > 30$:
\begin{equation}
a(n) = 1000001 \cdot a(n-15) - 1000000 \cdot a(n-30)
\end{equation}
\end{theorem}

\begin{proof}
From Lemma~\ref{lem:shift}:
\begin{equation}
a(n) = a(n-15) \cdot 10^6 + c(n-15)
\end{equation}

From Lemma~\ref{lem:periodic}, for $n > 30$:
\begin{equation}
c(n-15) = c(n-30)
\end{equation}

Applying Lemma~\ref{lem:shift} to $a(n-15)$:
\begin{equation}
a(n-15) = a(n-30) \cdot 10^6 + c(n-30)
\end{equation}

Solving for $c(n-30)$:
\begin{equation}
c(n-30) = a(n-15) - a(n-30) \cdot 10^6
\end{equation}

Substituting:
\begin{align}
a(n) &= a(n-15) \cdot 10^6 + a(n-15) - a(n-30) \cdot 10^6 \\
     &= a(n-15) \cdot (10^6 + 1) - a(n-30) \cdot 10^6 \\
     &= 1000001 \cdot a(n-15) - 1000000 \cdot a(n-30)
\end{align}
\end{proof}

\section{Remarks}

The sequence $[1,2,3,4,3,2]$ represents the striking pattern of the Prague
Astronomical Clock (Pražský orloj), where the clock strikes $1, 1+2, 1+2+3, \ldots$
times at hours $1, 2, 3, \ldots, 12$, with the pattern repeating \cite{horsky}.
Křížek, Šolcová, and Somer \cite{krizek} introduced the mathematical theory
of Šindel sequences in this context. This connection to medieval horology
gives the sequence its historical significance.

The proof demonstrates that Wu's conjecture is a direct consequence of the
6-periodicity of the digit sequence combined with the 15-periodicity of the
construction parameters.

\begin{thebibliography}{9}
\bibitem{oeis}
OEIS Foundation Inc., \textit{The On-Line Encyclopedia of Integer Sequences},
A028355, \url{https://oeis.org/A028355}.

\bibitem{krizek}
M.~Křížek, A.~Šolcová, and L.~Somer,
``Šindel sequences and the Prague horologe,''
in \textit{Programs and Algorithms of Numerical Mathematics 13},
Institute of Mathematics AS CR, Prague, 2006, pp.~156--164.

\bibitem{horsky}
Z.~Horský,
\textit{Pražský orloj},
Panorama, Prague, 1988.
\end{thebibliography}

\end{document}
