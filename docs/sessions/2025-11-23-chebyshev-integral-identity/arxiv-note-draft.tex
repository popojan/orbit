\documentclass[11pt]{article}
\usepackage[utf8]{inputenc}
\usepackage{amsmath,amsthm,amssymb}
\usepackage{geometry}
\geometry{margin=2.5cm}
\usepackage[hidelinks,pdfborder={0 0 0}]{hyperref}
\usepackage{graphicx}

\newtheorem{theorem}{Theorem}
\newtheorem{lemma}[theorem]{Lemma}
\newtheorem{corollary}[theorem]{Corollary}

\title{The $1/\pi$ Invariant in Chebyshev Polynomial Geometry}
\author{Jan Popelka\thanks{Email: popojan@protonmail.com}}
\date{}

\begin{document}
\maketitle

\begin{abstract}
We prove that the integral $\int_{-1}^{1} |T_{k+1}(x) - x \cdot T_k(x)| \, dx = 1$
for all $k \in \tfrac{1}{2}\mathbb{Z}$ with $k \geq 3/2$, where $T_k$ denotes the
Chebyshev polynomial of the first kind. Geometrically, this means that the area
between the curve $y = T_{k+1}(x) - x \cdot T_k(x)$ and the $x$-axis equals exactly
$1/\pi$ of the unit circle's area, regardless of the polynomial degree. For general
real $k$, we analyze the complex-valued extension and show that the sum of areas
under real and imaginary parts lies in $[1, \sqrt{2}]$. This invariant produces
visually striking interference patterns as $k$ varies continuously.
\end{abstract}

\section{Introduction}

The Chebyshev polynomials of the first kind $T_k(x) = \cos(k \arccos x)$ are
fundamental objects in approximation theory. The combination
$f_k(x) := T_{k+1}(x) - x \cdot T_k(x)$ arises naturally from the product identity
\cite{mason}
\begin{equation}
x T_k(x) = \frac{1}{2}\bigl[T_{k+1}(x) + T_{k-1}(x)\bigr],
\end{equation}
which gives $f_k(x) = \frac{1}{2}[T_{k+1}(x) - T_{k-1}(x)]$.

The curve $(x, f_k(x))$ oscillates within the unit disk, with increasing frequency
as $k$ grows. Remarkably, the total unsigned area under this curve is independent
of~$k$.

A key geometric property: for integer $k \geq 2$, the equation $x^2 + f_k(x)^2 = 1$
has exactly $k$ solutions in $(-1, 1)$. To see this, substitute $x = \cos\theta$
and use $f_k(\cos\theta) = -\sin(k\theta)\sin\theta$ to obtain
$\cos^2\theta + \sin^2(k\theta)\sin^2\theta = 1$, which simplifies to
$\sin^2(k\theta) = 1$ (for $\sin\theta \neq 0$). Thus $k\theta = \frac{\pi}{2} + n\pi$,
giving
\begin{equation}
x_n = \cos\frac{(2n+1)\pi}{2k}, \quad n = 0, 1, \ldots, k-1.
\end{equation}
These are the $x$-coordinates of vertices of a regular $k$-gon inscribed in the
unit circle, rotated by angle $\pi/(2k)$ from the standard position. This connects
the Chebyshev oscillations to discrete rotational symmetry.

\begin{figure}[ht]
\centering
\includegraphics[width=0.24\textwidth]{chebyshev-k2.pdf}%
\includegraphics[width=0.24\textwidth]{chebyshev-k3.pdf}%
\includegraphics[width=0.24\textwidth]{chebyshev-k17.pdf}%
\includegraphics[width=0.24\textwidth]{chebyshev-k31.pdf}
\caption{The curve $y = f_k(x) = T_{k+1}(x) - x \cdot T_k(x)$ for $k = 2, 3, 17, 31$.
The shaded area equals $1$ in each case (i.e., $1/\pi$ of the unit disk area).
Dashed lines show the regular $k$-gon whose vertices satisfy $x^2 + f_k(x)^2 = 1$.}
\label{fig:lobes}
\end{figure}

\section{Main Result}

\begin{theorem}[The $1/\pi$ Invariant]\label{thm:main}
For all $k \in \tfrac{1}{2}\mathbb{Z}$ with $k \geq 3/2$,
\begin{equation}
\int_{-1}^{1} |T_{k+1}(x) - x \cdot T_k(x)| \, dx = 1.
\end{equation}
The identity fails for $k \notin \tfrac{1}{2}\mathbb{Z}$.
\end{theorem}

\begin{proof}
Using the substitution $x = \cos\theta$, $dx = -\sin\theta \, d\theta$, and the
trigonometric representation $T_n(\cos\theta) = \cos(n\theta)$, we obtain
\begin{equation}
f_k(\cos\theta) = \cos((k+1)\theta) - \cos\theta \cdot \cos(k\theta).
\end{equation}

Applying the product-to-sum formula
$\cos\theta \cdot \cos(k\theta) = \frac{1}{2}[\cos((k+1)\theta) + \cos((k-1)\theta)]$,
we get
\begin{equation}
f_k(\cos\theta) = \frac{1}{2}[\cos((k+1)\theta) - \cos((k-1)\theta)] = -\sin(k\theta)\sin\theta,
\end{equation}
using the difference formula $\cos A - \cos B = -2\sin\frac{A+B}{2}\sin\frac{A-B}{2}$.

The integral transforms to
\begin{equation}
I_k := \int_{-1}^{1} |f_k(x)| \, dx = \int_{0}^{\pi} |\sin(k\theta)| \sin^2\theta \, d\theta.
\end{equation}

Expanding $\sin^2\theta = \frac{1}{2}(1 - \cos 2\theta)$, we have
\begin{equation}
I_k = \frac{1}{2}(A_k - B_k),
\end{equation}
where $A_k = \int_{0}^{\pi} |\sin(k\theta)| \, d\theta$ and
$B_k = \int_{0}^{\pi} |\sin(k\theta)| \cos(2\theta) \, d\theta$.

\begin{lemma}\label{lem:A}
$A_k = 2$ for all $k \in \tfrac{1}{2}\mathbb{Z}$ with $k \geq 1$.
\end{lemma}
\begin{proof}
For integer $k$, the function $|\sin(k\theta)|$ completes exactly $k$ half-periods
over $[0,\pi]$, each contributing $2/k$, so $A_k = 2$.

For half-integer $k = n + \tfrac{1}{2}$, observe that $\sin(k\pi) = \pm 1$, so the
endpoint $\theta = \pi$ lands at a peak rather than a zero. Direct calculation shows
$A_k = 2$ in this case as well. For example, $A_{5/2} = \tfrac{4}{5} + \tfrac{4}{5} + \tfrac{2}{5} = 2$.

For $k \notin \tfrac{1}{2}\mathbb{Z}$, the integral $A_k \neq 2$ in general.
\end{proof}

\begin{lemma}\label{lem:B}
$B_k = 0$ for all $k \in \tfrac{1}{2}\mathbb{Z}$ with $k \geq 3/2$.
\end{lemma}
\begin{proof}
Let $\varphi = \theta - \pi/2$. The function $|\sin(k(\varphi + \pi/2))|$ is even
in $\varphi$ for $k \in \tfrac{1}{2}\mathbb{Z}$, while $\cos(2(\varphi + \pi/2)) = -\cos(2\varphi)$
is odd. The integral of an even function times an odd function over the symmetric
interval $[-\pi/2, \pi/2]$ vanishes.

For $k \notin \tfrac{1}{2}\mathbb{Z}$, the symmetry breaks and $B_k \neq 0$.
\end{proof}

Combining the lemmas: $I_k = \frac{1}{2}(2 - 0) = 1$ for $k \in \tfrac{1}{2}\mathbb{Z}$, $k \geq 3/2$.
\end{proof}

\begin{corollary}[Geometric Interpretation]
Let $D$ denote the unit disk with area $\pi$. The region bounded by the curve
$y = T_{k+1}(x) - x \cdot T_k(x)$ and the $x$-axis has area exactly $1/\pi$ of
the disk area, for all $k \in \tfrac{1}{2}\mathbb{Z}$ with $k \geq 3/2$.
\end{corollary}

\section{Complex Extension for Real $k$}

For non-half-integer $k$, we consider the complex-valued extension arising from
the factor $(-1)^k = e^{ik\pi}$ in the shader implementation. Writing
\begin{equation}
g_k(\theta) := e^{ik\pi} \sin(k\theta) \sin\theta,
\end{equation}
we obtain real and imaginary parts:
\begin{align}
\operatorname{Re}[g_k(\theta)] &= \cos(k\pi) \sin(k\theta) \sin\theta, \\
\operatorname{Im}[g_k(\theta)] &= \sin(k\pi) \sin(k\theta) \sin\theta.
\end{align}

\begin{theorem}[Complex Area Bounds]\label{thm:complex}
For any real $k \geq 3/2$, define
\begin{align}
I_{\mathrm{Re}}(k) &:= \int_0^\pi |\operatorname{Re}[g_k(\theta)]| \sin\theta \, d\theta, \\
I_{\mathrm{Im}}(k) &:= \int_0^\pi |\operatorname{Im}[g_k(\theta)]| \sin\theta \, d\theta.
\end{align}
Then
\begin{equation}
I_{\mathrm{Re}}(k) + I_{\mathrm{Im}}(k) = |\cos(k\pi)| + |\sin(k\pi)| \in [1, \sqrt{2}].
\end{equation}
The minimum value $1$ is attained precisely when $k \in \tfrac{1}{2}\mathbb{Z}$,
and the maximum $\sqrt{2}$ is attained when $k \in \mathbb{Z} + \tfrac{1}{4}$.
\end{theorem}

\begin{proof}
The base integral $\int_0^\pi |\sin(k\theta)| \sin^2\theta \, d\theta = 1$ for
$k \in \tfrac{1}{2}\mathbb{Z}$. The $\cos(k\pi)$ and $\sin(k\pi)$ factors scale
the real and imaginary contributions respectively. Since
$|\cos\alpha| + |\sin\alpha|$ achieves its minimum $1$ at $\alpha \in \tfrac{\pi}{2}\mathbb{Z}$
(i.e., $k \in \tfrac{1}{2}\mathbb{Z}$) and maximum $\sqrt{2}$ at $\alpha = \tfrac{\pi}{4} + \tfrac{\pi}{2}\mathbb{Z}$
(i.e., $k \in \mathbb{Z} + \tfrac{1}{4}$), the result follows.
\end{proof}

\section{Visual Demonstration}

As $k$ varies continuously, the curve $f_k(x)$ splits into real and imaginary
components that oscillate with varying amplitudes. The total shaded area ranges
from $1$ (at half-integers) to $\sqrt{2}$ (at quarter-integers), creating
interference-like patterns that can be visualized interactively at:

\begin{center}
\url{https://www.shadertoy.com/view/MXc3Rj}
\end{center}

\section{Remarks}

\begin{enumerate}
\item The case $k=1$ is exceptional: $\int_{-1}^{1}|x^2 - 1|\,dx = 4/3$.

\item Using the mutual recurrence relation from Mason \& Handscomb \cite{mason},
we have $T_{k+1}(x) - x \cdot T_k(x) = -(1-x^2) U_{k-1}(x)$, where $U_k$ is the
Chebyshev polynomial of the second kind. Thus the identity can be written as
\[
\int_{-1}^{1} (1-x^2)|U_{k-1}(x)| \, dx = 1 \quad \text{for } k \geq 2.
\]

\item The generating function for the sequence $(I_1, I_2, I_3, \ldots) = (4/3, 1, 1, \ldots)$
is the rational function $G(z) = z(4-z)/[3(1-z)]$.
\end{enumerate}

\begin{thebibliography}{9}
\bibitem{mason}
J.\,C.~Mason and D.\,C.~Handscomb,
\textit{Chebyshev Polynomials},
Chapman \& Hall/CRC, 2003.

\bibitem{dlmf}
NIST Digital Library of Mathematical Functions,
\url{https://dlmf.nist.gov/18}, Release 1.2.3.
\end{thebibliography}

\end{document}
