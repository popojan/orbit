\documentclass[11pt]{article}
\usepackage{amsmath,amssymb,amsthm}
\usepackage{graphicx}
\usepackage{xcolor}
\usepackage{tikz}
\usepackage[margin=1in]{geometry}
\usepackage{listings}

\title{The Primal Forest:\\
A Bird's Eye View of the Sieve of Eratosthenes}

\author{Jan Pospíšil}

\date{\today}

\begin{document}

\maketitle

\begin{abstract}
We present an educational visualization that transforms the classical Sieve of Eratosthenes from a one-dimensional list into a two-dimensional ``primal forest.'' By mapping composite numbers $n = x(x+k)$ to coordinates $\{kx + x^2, kx + 1\}$, we create a top-down view where composites form a regular pattern like trees in a forest. Prime numbers become visible as gaps---positions on the number line that no composite point lies above. Special ``see-through'' positions occur when $x=1$, creating windows that reveal primes directly. This geometric perspective makes the structure of the sieve visually intuitive and provides an engaging introduction to multiplicative structure in number theory.
\end{abstract}

\section{Introduction: Why Are Primes So Hard to Find?}

Prime numbers are the atoms of arithmetic---every whole number breaks down into a unique product of primes. Yet despite their fundamental importance, primes seem to appear unpredictably along the number line:

\[
2, 3, 5, 7, 11, 13, 17, 19, 23, 29, 31, 37, 41, 43, 47, \ldots
\]

The gaps between consecutive primes vary: sometimes just 2 (twin primes like 11 and 13), sometimes much larger (there's a gap of 14 between 113 and 127). Is there a pattern hidden in this chaos?

The ancient Greeks developed a clever algorithm---the \emph{Sieve of Eratosthenes}---that systematically finds all primes by eliminating composites. But working through the sieve on a linear list can feel mechanical. \textbf{What if we could \emph{see} the structure?}

Enter the \textbf{Primal Forest}: a geometric transformation that turns the sieve into a visual landscape where composites form regular patterns and primes appear as gaps. Imagine you're a bird flying over a forest, looking down at the trees below. The trees are composites, blocking your view. But in the gaps between trees, you can see down to the primes on the ground.

\section{The Classical Sieve (Linear View)}

Let's briefly recall how the Sieve of Eratosthenes works in its traditional form. We list all integers starting from 2:

\[
2, 3, 4, 5, 6, 7, 8, 9, 10, 11, 12, 13, 14, 15, 16, 17, 18, 19, 20, 21, 22, 23, 24, \ldots
\]

Then we repeatedly:
\begin{enumerate}
\item Mark the first unmarked number as prime
\item Cross out all its multiples (composites)
\item Repeat
\end{enumerate}

For example:
\begin{itemize}
\item Mark 2 as prime, cross out 4, 6, 8, 10, 12, 14, 16, 18, 20, 22, 24, \ldots
\item Mark 3 as prime, cross out 6, 9, 12, 15, 18, 21, 24, \ldots
\item Mark 5 as prime, cross out 10, 15, 20, 25, \ldots
\item And so on\ldots
\end{itemize}

The numbers that remain unmarked are the primes. This works beautifully, but it's a one-dimensional view. Every composite disappears into a crossed-out symbol. Can we visualize the \emph{structure} of why composites get eliminated?

\section{The Geometric Transformation}

Here's the key insight: every composite number can be written as $n = x(x+k)$ for some positive integers $x$ and $k \geq 0$. For example:
\begin{align*}
4 &= 2 \times 2 = 2(2+0) \quad \text{(so } x=2, k=0\text{)} \\
6 &= 2 \times 3 = 2(2+1) \quad \text{(so } x=2, k=1\text{)} \\
8 &= 2 \times 4 = 2(2+2) \quad \text{(so } x=2, k=2\text{)} \\
9 &= 3 \times 3 = 3(3+0) \quad \text{(so } x=3, k=0\text{)}
\end{align*}

Now, instead of placing these composites on a line, we'll place them in a 2D plane using this mapping:

\[
\boxed{n = x(x+k) \quad \mapsto \quad \text{point at } (kx + x^2,\, kx + 1)}
\]

Let's unpack this:
\begin{itemize}
\item \textbf{x-coordinate}: $kx + x^2 = x(k+x)$ is exactly the composite number $n$ itself
\item \textbf{y-coordinate}: $kx + 1$ is chosen to spread the points vertically in a regular pattern
\end{itemize}

\subsection{Why This Coordinate Choice?}

The y-coordinate $kx + 1$ isn't arbitrary---it creates a \textbf{regular, square-like forest}. As we plot points for different values of $k$ and $x$, they form a symmetric grid pattern. In fact, if we reflect the entire forest across the x-axis (flipping $y \to -y$), the positions where primes are visible remain unchanged! This symmetry reveals the deep regularity in how composites are distributed.

\section{The Forest Emerges}

Figure~\ref{fig:forest} shows the primal forest for composites up to $n=31$.

\begin{figure}[h]
\centering
\includegraphics[width=0.8\textwidth]{../visualizations/prime-grid-demo.png}
\caption{The Primal Forest (top-down view). Each blue dot represents a factorization: a composite at x-coordinate $n = x(x+k)$ with smaller factor $x \geq 2$. Orange dots mark the main diagonal where $x=1$, revealing primes directly. Vertical lines connect orange dots to the primes on the x-axis. Prime numbers are the x-coordinates with no dots above them---gaps in the factor forest.}
\label{fig:forest}
\end{figure}

\subsection{What Are We Seeing?}

\begin{itemize}
\item \textbf{Blue dots}: Each dot represents a \emph{factorization}. Specifically, a dot at $(kx+x^2, kx+1)$ shows that the number $n = x(x+k)$ has $x$ as a factor. The more dots above a given x-coordinate, the more factorizations that number has. These dots form the dense ``forest'' blocking your view down to the primes.

\item \textbf{Orange dots}: Special positions where $x=1$. When $x=1$, the coordinates become:
\[
(k \cdot 1 + 1^2,\, k \cdot 1 + 1) = (k+1, k+1)
\]
So we get points at $(2,2), (3,3), (4,4), (5,5), \ldots$ along the \textbf{main diagonal} $y=x$ of the square forest. The orange dots highlight when $k+1$ is \textbf{prime}---these are ``see-through'' positions where you can look down directly to the prime on the x-axis.

\item \textbf{Vertical lines}: Drop from orange points to the x-axis, showing which prime they sit above.

\item \textbf{Gaps on the x-axis}: Look at the x-coordinates: most values have blue or orange dots above them (composites). But some x-coordinates have \emph{no points at all}---these are the \textbf{primes}! They're gaps in the forest.
\end{itemize}

\subsection{The Core Insight}

\begin{quote}
\textbf{Each blue dot represents a factorization: ``the number $x(x+k)$ has $x$ as a factor.'' Primes have no such factorizations, so they have no dots above them---they're gaps in the factor forest.}
\end{quote}

In this view, a composite number may have \emph{multiple} dots above it (one for each distinct factorization), while a prime has \emph{zero} dots. The forest isn't just showing which numbers are composite---it's showing \emph{how they factor}.

\section{Exploring the Forest}

\subsection{Why Do Composites Form This Pattern?}

Each row $k = \text{const}$ represents all composites of the form $x(x+k)$:
\begin{itemize}
\item $k=0$: Perfect squares $x^2$ at positions $(x^2, 1)$
\item $k=1$: Products $x(x+1)$ at positions $(x^2+x, x+1)$
\item $k=2$: Products $x(x+2)$ at positions $(x^2+2x, 2x+1)$
\item And so on\ldots
\end{itemize}

Each column $x = \text{const}$ represents all composites with $x$ as the smaller factor.

\subsection{What About Prime Powers?}

Prime powers like $4 = 2^2$, $8 = 2^3$, $9 = 3^2$ appear multiple times in the forest! For example:
\begin{itemize}
\item $9 = 3 \times 3 = 3(3+0)$ appears at $(9, 1)$ (the $k=0$ row)
\item $9 = 9 \times 1$ would be $(9, \ldots)$ if we extended to $x < k$ cases
\end{itemize}

This redundancy shows that some numbers have multiple factorizations, while primes have none.

\subsection{The Parabolic Curve and Growing Gaps}

Notice how the forest curves upward as numbers grow? The x-coordinate $kx + x^2$ has a quadratic term $x^2$, creating a parabolic arrangement. This reveals a profound truth about prime distribution:

\begin{quote}
\textbf{The higher the x-coordinate (the larger the number), the higher the probability that some factor (tree) blocks your sight down to that position.}
\end{quote}

As we move right along the number line, more and more factor dots accumulate above each position. The chance of finding a gap---a prime---decreases. This geometric view makes the Prime Number Theorem intuitive: \emph{primes thin out} because the ``forest of factors'' grows denser.

\section{Try It Yourself}

Here's the Wolfram Language code that generates the forest:

\begin{lstlisting}[language=Mathematica, basicstyle=\small\ttfamily]
(* Generate all composite points up to n *)
comp1[m_, n_] := Join @@ Table[
  {k*x + x^2, k*x + 1},
  {k, 0, n},
  {x, Max[1, Ceiling[1/2 (-k + Sqrt[k^2 + 4*m])]],
      Floor[1/2 (-k + Sqrt[k^2 + 4*n])]}
]

(* Visualize the forest *)
lpl[hi_] := ListPlot[
  GatherBy[#, PrimeQ@*First],
  PlotRange -> {Automatic, {-1/2, Automatic}},
  AxesOrigin -> {0, 0},
  AspectRatio -> 1,
  PlotMarkers -> {Automatic, Large},
  GridLines -> {Range@hi - 1/2, Range[0, hi] - 1/2},
  Epilog -> {Thick, ColorData[97, "ColorList"][[2]],
    Line[{#, {First@#, 0}}] & /@
      Select[#, PrimeQ@*First]}
] &@ comp1[1, hi]

(* Generate the plot *)
lpl@31
\end{lstlisting}

\subsection{Exercises}

\begin{enumerate}
\item \textbf{Predict the pattern}: Before running the code, can you predict which x-coordinates will have no points above them for $n \leq 20$?

\item \textbf{Modify the transformation}: What happens if you change the y-coordinate to $kx + 2$ or $kx + k$? Does the forest remain regular?

\item \textbf{Count the trees}: For a given range, how many composite points appear? How does this compare to the number of primes?

\item \textbf{Twin primes}: Can you identify twin primes (primes differing by 2) in the forest? What pattern do they form?

\item \textbf{Reflection symmetry}: Implement the reflection across the x-axis (plot points at negative y-coordinates too). Verify that the see-through positions remain symmetric.
\end{enumerate}

\section{Connections to Other Visualizations}

The Primal Forest joins a family of geometric prime visualizations:

\begin{itemize}
\item \textbf{Ulam Spiral} (1963): Arrange integers in a spiral; primes cluster along diagonal lines
\item \textbf{Sacks Spiral}: Arrange integers on an Archimedean spiral; primes form curved patterns
\item \textbf{Prime Grids}: Various 2D arrangements revealing structure
\item \textbf{Primal Forest}: Composites as a regular lattice, primes as gaps
\end{itemize}

Each visualization highlights different aspects of prime distribution. The forest metaphor is particularly intuitive: \textbf{primes aren't randomly scattered---they're the gaps left behind when composites are systematically placed.}

\section{Educational Value}

This visualization helps answer common student questions:

\paragraph{Q: Why are primes ``special''?}
A: Look at the forest---primes are the only numbers with \emph{no tree above them}. They're fundamentally different from composites.

\paragraph{Q: Why do prime gaps grow?}
A: See how the forest gets denser as numbers increase (the parabolic curve)? More composites crowd the larger regions, leaving bigger gaps between primes.

\paragraph{Q: What does factorization mean geometrically?}
A: Each blue dot is a composite, positioned according to its factors $x$ and $k+x$. Primes can't be positioned this way---they have no factorization.

\paragraph{Q: Is there a pattern to primes?}
A: Yes! Primes are precisely the gaps in the regular factor lattice. The pattern is visible once we map factorizations to 2D.

\paragraph{Q: Why does the probability of finding a prime decrease?}
A: As numbers grow larger, each position on the x-axis has more potential factors. The forest grows denser because there are simply more ways to factor larger numbers. Geometrically, more trees block your sight, so gaps (primes) become rarer.

\section{Conclusion}

The Primal Forest transforms the abstract notion of ``sieving out composites'' into a concrete visual experience. By mapping factorizations to coordinates, we see composites form a regular geometric pattern, while primes emerge as the gaps---numbers that don't fit the pattern.

This approach has pedagogical value at multiple levels:
\begin{itemize}
\item \textbf{Elementary}: Visual intuition before algorithms
\item \textbf{Secondary}: Connection between factorization and geometry
\item \textbf{University}: Coordinate transformations reveal structure
\item \textbf{Research}: Alternative perspectives may inspire new approaches
\end{itemize}

The next time you think about primes, imagine flying over a forest. The trees are composites, neatly arranged by their factors. And in the clearings between the trees? That's where the primes are hiding.

\section*{Acknowledgments}

This visualization emerged from recreational explorations in computational number theory. Thanks to the Wolfram Language community for tools that make such experiments delightful.

\end{document}
