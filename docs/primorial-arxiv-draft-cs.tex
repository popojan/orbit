\documentclass[11pt]{article}
\usepackage[margin=1in]{geometry}
\usepackage[czech]{babel}
\usepackage[T1]{fontenc}
\usepackage[utf8]{inputenc}
\usepackage{amsmath,amssymb,amsthm}
\usepackage[colorlinks=true,linkcolor=blue,citecolor=blue,urlcolor=blue]{hyperref}

\newtheorem{theorem}{Věta}
\newtheorem{conjecture}[theorem]{Hypotéza}
\newtheorem{proposition}[theorem]{Tvrzení}
\theoremstyle{definition}
\newtheorem{definition}[theorem]{Definice}
\newtheorem{example}[theorem]{Příklad}
\theoremstyle{remark}
\newtheorem{remark}[theorem]{Poznámka}

\title{Vzorec pro primoriál pomocí střídavých faktoriálových součtů}
\author{Jan Geuss Popelka\thanks{Nezávislý výzkumník. Email: popojan@protonmail.com. Kód: \url{https://github.com/popojan/orbit}}}
\date{\today}

\begin{document}

\maketitle

\begin{abstract}
Představujeme výpočetní objev: primoriál $p\#$ (součin všech prvočísel až do $p$) lze extrahovat jako jmenovatele specifického střídavého faktoriálového součtu. Pro $m \geq 3$ máme
\[
\text{den}\left(\frac{1}{2} \sum_{k=1}^{\lfloor(m-1)/2\rfloor} \frac{(-1)^k \cdot k!}{2k+1}\right) = \prod_{\substack{p \text{ prvočíslo} \\ p \leq m}} p
\]
kde $\text{den}(q)$ značí jmenovatele racionálního čísla $q$ v nejnižších číslech. Vzorec byl ověřen výpočetně pro $m$ až do 100\,000, avšak rigorózní důkaz zůstává nepolapitelný. Ústřední záhadou je systematické krácení: ačkoli jednotlivé jmenovatele $\{2k+1\}$ obsahují mocniny prvočísel ($9 = 3^2$, $25 = 5^2$, $27 = 3^3$, \ldots), konečný zkrácený jmenovatel obsahuje pouze první mocniny prvočísel. Formulujeme toto jako otevřený problém o $p$-adických valuacích a diskutujeme možné teoretické souvislosti.
\end{abstract}

\section{Úvod}

Funkce primoriál, definovaná jako součin všech prvočísel až do dané hranice, se přirozeně vyskytuje v teorii čísel, od odhadů v analytické teorii čísel až po konstrukci vysoce složených čísel. Zatímco primoriály jsou typicky počítány přímým násobením, objevili jsme nečekanou alternativu: vznikají jako jmenovatelé střídavých faktoriálových součtů.

Tento objev vznikl prostřednictvím výpočetního experimentování s rozklady na egyptské zlomky a faktoriálovými řadami. Při zkoumání různých vzorců racionálních součtů jsme pozorovali, že určité střídavé součty vykazují jmenovatele s pozoruhodnou strukturou---po zkrácení na nejnižší čísla se objevují pouze součiny různých prvočísel. Systematické testování odhalilo vzorec prezentovaný zde.

Vzorec je snadno formulovatelný a výpočetně ověřitelný, avšak pochopení \emph{proč} funguje vyžaduje vysvětlení jemného mechanismu krácení. Čitatele, konstruované z faktoriálů, systematicky eliminují všechny vyšší mocniny prvočísel z naivního nejmenšího společného násobku jmenovatelů. Tento článek prezentuje vzorec, poskytuje výpočetní důkazy a formuluje krácení jako formální otevřený problém.

\section{Vzorec}

\begin{theorem}[Výpočetní]
Pro libovolné celé číslo $m \geq 3$ definujme racionální číslo
\begin{equation}\label{eq:main}
S_m = \frac{1}{2} \sum_{k=1}^{h} \frac{(-1)^k \cdot k!}{2k+1}
\end{equation}
kde $h = \lfloor(m-1)/2\rfloor$. Pak jmenovatel $S_m$ v nejnižších číslech se rovná $m$-primoriálu:
\[
\text{den}(S_m) = \prod_{\substack{p \text{ prvočíslo} \\ p \leq m}} p
\]
\end{theorem}

\begin{remark}
Případ $m = 2$ vyžaduje speciální zacházení a vrací přímo $2$. Pro $m \geq 3$ se vzorec aplikuje univerzálně.
\end{remark}

\begin{example}
Pro $m = 13$ máme $h = 6$ a počítáme:
\begin{align*}
S_{13} &= \frac{1}{2}\left(-\frac{1!}{3} + \frac{2!}{5} - \frac{3!}{7} + \frac{4!}{9} - \frac{5!}{11} + \frac{6!}{13}\right)\\
&= \frac{695971}{30030}
\end{align*}
Jmenovatel $30030 = 2 \times 3 \times 5 \times 7 \times 11 \times 13$ je přesně součin všech prvočísel až do 13.
\end{example}

\subsection{Objevení vzoru}

Klíčovým pozorováním je, že jmenovatel roste akumulací nových prvočíselných faktorů:

\begin{proposition}
Nechť $S_k = \frac{1}{2} \sum_{i=1}^{k} \frac{(-1)^i \cdot i!}{2i+1}$ značí dílčí součet. Pak
\[
\text{den}(S_k) = 2 \times \prod_{\substack{p \text{ prvočíslo} \\ 3 \leq p \leq 2k+1}} p
\]
\end{proposition}

Jmenovatele se stabilizují, když $2k+1$ je složené. Nová prvočísla vstupují pouze tehdy, když $2k+1$ samo je prvočíslem. Tabulka~\ref{tab:example} ilustruje toto pro $m = 13$:

\begin{table}[h]
\centering
\begin{tabular}{c|c|c|c}
$k$ & $2k+1$ & Prvočíslo? & $\text{den}(S_k)$ \\ \hline
1 & 3 & \checkmark & $2 \times 3$ \\
2 & 5 & \checkmark & $2 \times 3 \times 5$ \\
3 & 7 & \checkmark & $2 \times 3 \times 5 \times 7$ \\
4 & 9 & & $2 \times 3 \times 5 \times 7$ \\
5 & 11 & \checkmark & $2 \times 3 \times 5 \times 7 \times 11$ \\
6 & 13 & \checkmark & $2 \times 3 \times 5 \times 7 \times 11 \times 13$
\end{tabular}
\caption{Růst jmenovatele pro $m = 13$. Všimněte si, že jmenovatel zůstává nezměněn při $k = 4$ navzdory $9 = 3^2$.}
\label{tab:example}
\end{table}

\section{Problém krácení}

Ústřední záhadou je toto: sekvence $\{2k+1\}_{k=1}^{h}$ obsahuje mocniny prvočísel a složená čísla:
\[
3, 5, 7, 9=3^2, 11, 13, 15=3 \times 5, 17, 19, 21=3 \times 7, 23, 25=5^2, 27=3^3, \ldots
\]

Naivně, při výpočtu $\text{LCM}(3, 5, 7, 9, 11, \ldots)$ bychom si ponechali $3^2$ z 9, $3^3$ z 27, $5^2$ z 25, a tak dále. Avšak po sečtení a zkrácení (\ref{eq:main}) na nejnižší čísla obsahuje jmenovatel \emph{pouze první mocniny} prvočísel.

\subsection{Formulace p-adických valuací}

Nechť $\nu_p(n)$ značí $p$-adickou valuaci (exponent $p$ v prvočíselném rozkladu $n$). Naše výpočetní vyšetřování odhaluje:

\begin{conjecture}\label{conj:main}
Pro všechna prvočísla $p$ s $3 \leq p \leq 2k+1$ máme
\[
\nu_p\left(\text{den}(S_k)\right) = 1
\]
a
\[
\nu_p\left(\text{čit}(S_k)\right) = 0
\]
kde $\text{čit}(q)$ značí čitatele $q$ v nejnižších číslech.
\end{conjecture}

Toto je jádrem problému: dokázat, že čitatele systematicky obsahují přesně správné prvočíselné faktory pro zkrácení všech $p^j$ s $j > 1$ prostřednictvím redukce největším společným dělitelem.

\subsection{Dva mechanismy krácení}

Výpočetní vyšetřování odhaluje dva odlišné režimy:

\textbf{Malé $k$ (krácení pomocí NSD):} Když $\nu_p(k!) < \nu_p(2k+1)$ pro nějaké prvočíslo $p$ dělící $2k+1$, kombinovaný čitatel po sčítání obsahuje faktory $p$ a redukce pomocí NSD eliminuje přesně nadbytečné mocniny.

\textbf{Příklad:} Při $k = 4$ máme $2k+1 = 9 = 3^2$, ale $\nu_3(4!) = 1 < 2$. Čitatel po kombinaci s předchozími členy obsahuje přesně jeden faktor 3, redukující $3^2 \to 3^1$ ve jmenovateli.

\textbf{Velké $k$ (celočíselné členy):} Pro dostatečně velké $k$ Legendrův vzorec
\[
\nu_p(k!) = \sum_{i=1}^{\infty} \left\lfloor \frac{k}{p^i} \right\rfloor
\]
zajišťuje, že $\nu_p(k!) \geq \nu_p(2k+1)$ pro všechna $p | (2k+1)$. V těchto případech se člen $\frac{k!}{2k+1}$ redukuje na \emph{celé číslo} a žádné nové jmenovatelové faktory nevstupují.

\textbf{Příklad:} Při $k = 12$ máme $2k+1 = 25 = 5^2$ a $\nu_5(12!) = 2 \geq 2$, takže člen je již celé číslo.

\subsection{Úloha střídavého znaménka}

Střídavý faktor $(-1)^k$ je podstatný. Bez něj vzorec ztrácí faktor 3 při $k = 4$ a nikdy se nezotaví, vydávajíc $\text{Primoriál}/3$ pro všechna $m \geq 9$. Střídavé znaménko řídí strukturu čitatele, aby zabránilo nadměrnému krácení v kritických krocích.

\section{Výpočetní ověření}

Vzorec byl ověřen vyčerpávajícím způsobem pro všechna $m$ od 3 do 100\,000, testujíc každou celočíselnou hodnotu (jak složenou, tak prvočíselnou). Při $m = 100{,}000$ má primoriál 43\,293 číslic; všechny jmenovatele se přesně shodují.

\subsection{Iterativní formulace}

Pro umožnění rozsáhlého ověření jsme odvodili efektivní iterativní formulaci. Vycházejíc z přímého součtu (\ref{eq:main}) mohou být dílčí součty počítány prostřednictvím tříprvkové rekurence:

\begin{align}
S_0 &= \{0, 0, 1\}\\
S_{n+1} &= \{n+1, b_n, b_n + (a_n - b_n)\cdot\left(n + \frac{1}{3+2n}\right)\}
\end{align}

kde stav je $S_n = \{n, a_n, b_n\}$. Po $h = \lfloor(m-1)/2\rfloor$ iteracích je primoriál extrahován jako:
\[
\text{Primoriál}(m) = 2 \cdot \text{den}(b_h - 1)
\]

Ekvivalence mezi přímým součtem (\ref{eq:main}) a touto rekurencí může být ustanovena indukcí; rekurence sleduje $b_n = 1 + 2S_n$ a faktor 2 je odstaven extrakčním vzorcem. Rigorózní důkaz ekvivalence je poskytnut v doprovodném dokumentu. Toto umožňuje $O(m)$ aritmetických operací a usnadňuje ověření ve velkém měřítku. Tabulka~\ref{tab:verification} ukazuje klíčové kontrolní body ověření:

\begin{table}[h]
\centering
\begin{tabular}{r|r|r}
$m$ & $\pi(m)$ & Počet číslic primoriálu \\ \hline
100 & 25 & 37 \\
1{,}000 & 168 & 416 \\
10{,}000 & 1{,}229 & 3{,}393 \\
100{,}000 & 9{,}592 & 43{,}293
\end{tabular}
\caption{Kontrolní body ověření. Všechny hodnoty testovány vyčerpávajícím způsobem (každé $m$ od 3 dále).}
\label{tab:verification}
\end{table}

\section{Otevřené otázky}

Zatímco vzor je výpočetně robustní, několik teoretických otázek zůstává:

\begin{enumerate}
\item \textbf{Rigorózní důkaz:} Dokázat Hypotézu~\ref{conj:main} pomocí $p$-adické analýzy nebo jiných technik.

\item \textbf{Souvislost s vytvořující funkcí:} Má tento součet interpretaci jako vytvořující funkce pro primoriály? Struktura naznačuje hlubší kombinatorický nebo analytický rámec.

\item \textbf{Zobecnění:} Co se stane, když modifikujeme vzorec? Například:
\begin{itemize}
\item Změna koeficientu: $\frac{1}{2} \to \frac{1}{c}$ pro jiné konstanty $c$
\item Modifikace jmenovatelů: $2k+1 \to ak+b$ pro jiné posloupnosti
\item Změna čitatelů: $k! \to (2k)!$ nebo jiné faktoriálové posloupnosti
\end{itemize}

\item \textbf{Souvislost s jinými identitami:} Existují spojení s Wilsonovou větou, Wolstenholmeovou větou nebo jmenovateli harmonických čísel (které také zahrnují primoriály)?

\item \textbf{Složitost:} Jaká je výpočetní složitost tohoto vzorce ve srovnání s přímým výpočtem primoriálu? Iterativní formulace může sloužit jako paměťově efektivní prvočíselné síto bez testování primalit.
\end{enumerate}

\section{Závěr}

Představili jsme výpočetně ověřený vzorec vyjadřující primoriály jako jmenovatele střídavých faktoriálových součtů. Systematické krácení vyšších mocnin prvočísel zůstává nevysvětleno, naznačujíc hlubokou číselně-teoretickou strukturu. Tento problém je vhodný pro formální důkazové asistenty a mohl by profitovat z technik v $p$-adické analýze, vytvořujících funkcích nebo modulární aritmetice.

Doufáme, že tento objev inspiruje teoretické vyšetřování a vítáme spolupráci na důkazu Hypotézy~\ref{conj:main} nebo zkoumání širších otázek zde nastolených.

\section*{Poděkování}

Tato práce vzešla z výpočetních průzkumů používajících Wolfram Language. Děkuji online matematické komunitě za předběžné diskuse a zpětnou vazbu k raným verzím tohoto výsledku.

\begin{thebibliography}{9}

\bibitem{legendre}
A.-M. Legendre,
\emph{Th\'eorie des nombres},
Firmin Didot Fr\`eres, Paris, 1830.

\bibitem{oeis}
OEIS Foundation Inc.,
The On-Line Encyclopedia of Integer Sequences,
\url{https://oeis.org}.
Relevantní posloupnosti: A002110 (primoriály), A034386 (primoriály prvočísel).

\end{thebibliography}

\end{document}
