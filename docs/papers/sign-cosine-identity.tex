\documentclass[11pt,a4paper]{article}

\usepackage{amsmath,amsthm,amssymb}
\usepackage{hyperref}
\usepackage{geometry}
\geometry{margin=2.5cm}

\newtheorem{theorem}{Theorem}
\newtheorem{lemma}[theorem]{Lemma}
\newtheorem{corollary}[theorem]{Corollary}
\newtheorem{remark}{Remark}

\title{A Sign-Cosine Identity, Quadratic Residues, and Class Numbers}
\author{Jan Popelka}
\date{December 2025 (v2)}

\begin{document}

\maketitle

\begin{abstract}
We prove an elementary identity for the sum of signs of cosines at odd multiples of $\pi/p$ for prime $p$. The result exhibits a dichotomy depending on $p \bmod 4$, mirroring classical results on quadratic residue patterns dating to Aladov (1896). We establish a deeper connection: the \emph{character-weighted} version of this sum equals $2h(-p) - 2$ for $p \equiv 1 \pmod 4$, where $h(-p)$ is the class number of the imaginary quadratic field $\mathbb{Q}(\sqrt{-p})$. While the algebraic content reduces to Dirichlet's classical quarter-interval character sum, our sign-cosine formulation provides two novel features: (1)~a geometric interpretation where $h(-p)$ measures quadratic residue distribution between small and large lobes of the Chebyshev $p$-gon, and (2)~\emph{complementarity} with classical half-interval counting---our approach extracts class number information for $p \equiv 1 \pmod 4$ precisely where the classical formula $h(-p) = V - N$ (counting residues vs.\ non-residues in $[1, (p-1)/2]$) gives $V = N$.
\end{abstract}

\section{Introduction}

For an odd prime $p$, consider the sum
\begin{equation}\label{eq:Ap}
A(p) = \sum_{k=1}^{p-1} \operatorname{sign}\left(\cos\frac{(2k-1)\pi}{p}\right).
\end{equation}

This sum arises naturally in the study of Chebyshev polynomial lobe areas, where one considers functions of the form $B(n,k) = 1 + \beta\cos\frac{(2k-1)\pi}{n}$ for geometric constants $\beta$.

The main result of this note is:

\begin{theorem}\label{thm:main}
For any odd prime $p$,
\[
A(p) = \begin{cases}
-2 & \text{if } p \equiv 1 \pmod{4}, \\
0 & \text{if } p \equiv 3 \pmod{4}.
\end{cases}
\]
Equivalently, $A(p) = -(1 + (-1)^{(p-1)/2})$.
\end{theorem}

The mod~4 dichotomy in this result is not coincidental. The condition $p \equiv 1 \pmod 4$ is equivalent to $-1$ being a quadratic residue modulo $p$, a fact that underlies both our theorem and classical results on consecutive quadratic residue patterns.

\section{Historical Context}

The study of consecutive quadratic residue patterns modulo a prime $p$ dates back to \textbf{N.~S.~Aladov} (1896), who proved exact formulas for the counts $N_p(\varepsilon_1, \varepsilon_2)$ of consecutive pairs $(k, k+1)$ with prescribed Legendre symbols $(\varepsilon_1, \varepsilon_2)$ \cite{Aladov1896}.

Aladov's results show that these counts depend on $p \bmod 4$:
\begin{center}
\begin{tabular}{c|cccc}
$p \bmod 4$ & $N_p(+,+)$ & $N_p(+,-)$ & $N_p(-,+)$ & $N_p(-,-)$ \\
\hline
1 & $(p-5)/4$ & $(p-1)/4$ & $(p-1)/4$ & $(p-1)/4$ \\
3 & $(p-3)/4$ & $(p+1)/4$ & $(p-3)/4$ & $(p-3)/4$
\end{tabular}
\end{center}

The work was extended by von~Sterneck (1898) and Jacobsthal \cite{Jacobsthal1906}, with the definitive asymptotic result established by Weil \cite{Weil1948} as a consequence of the Riemann hypothesis for curves over finite fields.

Our Theorem~\ref{thm:main} concerns a different quantity---a sum of signs of cosines rather than counts of Legendre symbol patterns---but exhibits the same mod~4 structure. In Section~\ref{sec:connection}, we explain this connection.

\begin{remark}
Aladov's priority was preserved in Russian mathematical literature (see Kiritchenko et al.\ \cite{KTVZ2024}) but remained largely unknown in anglophone sources until Keith Conrad's expository notes \cite{Conrad}.
\end{remark}

\section{Proof of the Main Theorem}

\begin{proof}[Proof of Theorem~\ref{thm:main}]
Let $N^+$ denote the number of $k \in \{1, \ldots, p-1\}$ with $\cos\frac{(2k-1)\pi}{p} > 0$, and let $N^- = (p-1) - N^+$. Then
\[
A(p) = N^+ - N^- = 2N^+ - (p-1).
\]

The cosine is positive when its argument lies in $(0, \pi/2) \cup (3\pi/2, 2\pi)$. For $\theta_k = \frac{(2k-1)\pi}{p}$, this gives two conditions:

\begin{enumerate}
\item[(i)] $\frac{(2k-1)\pi}{p} < \frac{\pi}{2}$, i.e., $k < \frac{p+2}{4}$;
\item[(ii)] $\frac{(2k-1)\pi}{p} > \frac{3\pi}{2}$, i.e., $k > \frac{3p+2}{4}$.
\end{enumerate}

\textbf{Case 1:} $p \equiv 1 \pmod 4$. Write $p = 4m + 1$.

From (i): $k < m + \frac{3}{4}$, so $k \in \{1, \ldots, m\}$. Count: $m$.

From (ii): $k > 3m + \frac{5}{4}$, so $k \in \{3m+2, \ldots, 4m\}$. Count: $m - 1$.

Thus $N^+ = m + (m-1) = 2m - 1$, and
\[
A(p) = 2(2m-1) - 4m = -2.
\]

\textbf{Case 2:} $p \equiv 3 \pmod 4$. Write $p = 4m + 3$.

From (i): $k < m + \frac{5}{4}$, so $k \in \{1, \ldots, m+1\}$. Count: $m + 1$.

From (ii): $k > 3m + \frac{11}{4}$, so $k \in \{3m+3, \ldots, 4m+2\}$. Count: $m$.

Thus $N^+ = (m+1) + m = 2m + 1$, and
\[
A(p) = 2(2m+1) - (4m+2) = 0.
\]

Finally, we verify the closed form: $(-1)^{(p-1)/2} = 1$ when $p \equiv 1 \pmod 4$ (since $(p-1)/2$ is even), and $(-1)^{(p-1)/2} = -1$ when $p \equiv 3 \pmod 4$. Hence $-(1 + (-1)^{(p-1)/2})$ equals $-2$ and $0$ respectively.
\end{proof}

\section{Connection to Quadratic Residues and Class Numbers}\label{sec:connection}

The appearance of $(-1)^{(p-1)/2}$ in our theorem is significant: this expression equals the Legendre symbol $\left(\frac{-1}{p}\right)$, which determines whether $-1$ is a quadratic residue modulo $p$.

To understand the connection more deeply, define the \emph{character-weighted sum}
\[
W(p) = \sum_{k=1}^{p-1} \chi(k) \cdot \operatorname{sign}\left(\cos\frac{(2k-1)\pi}{p}\right),
\]
where $\chi(k) = \left(\frac{k}{p}\right)$ is the Legendre symbol.

\begin{theorem}[Class Number Connection]\label{thm:class}
For any odd prime $p$,
\[
W(p) = \begin{cases}
2h(-p) - 2 & \text{if } p \equiv 1 \pmod{4}, \\
2 & \text{if } p \equiv 3 \pmod{4},
\end{cases}
\]
where $h(-p)$ is the class number of the imaginary quadratic field $\mathbb{Q}(\sqrt{-p})$.
\end{theorem}

\begin{proof}
\textbf{Case 1: $p \equiv 1 \pmod 4$.} Let $a = (p-1)/4$, which is an integer. The sign of $\cos\frac{(2k-1)\pi}{p}$ partitions $\{1, \ldots, p-1\}$ into three regions:
\begin{align*}
A &= \{1, \ldots, a\} \quad (\text{sign} = +1), \\
B &= \{a+1, \ldots, 3a+1\} \quad (\text{sign} = -1), \\
C &= \{3a+2, \ldots, p-1\} \quad (\text{sign} = +1).
\end{align*}
with sizes $|A| = a$, $|B| = 2a+1$, $|C| = a-1$.

Since $p \equiv 1 \pmod 4$, we have $\chi(-1) = 1$, so $\chi(p-k) = \chi(k)$. The image of $A$ under $k \mapsto p-k$ is $A' = \{3a+1, \ldots, 4a\}$, but region $C = A' \setminus \{3a+1\}$.

The missing element is $3a+1 = (3p+1)/4 \equiv 4^{-1} \pmod p$. Since $\chi(4^{-1}) = \chi(2)^{-2} = 1$, we have $\chi((3p+1)/4) = 1$ for all $p \equiv 1 \pmod 4$.

Let $\Sigma_A = \sum_{k \in A} \chi(k)$, similarly for $\Sigma_B$, $\Sigma_C$. By symmetry, $\Sigma_{A'} = \Sigma_A$, so $\Sigma_C = \Sigma_A - 1$. By character orthogonality, $\Sigma_B = -2\Sigma_A + 1$. Therefore:
\[
W(p) = \Sigma_A + \Sigma_C - \Sigma_B = \Sigma_A + (\Sigma_A - 1) - (-2\Sigma_A + 1) = 4\Sigma_A - 2.
\]

Since $\Sigma_A = S(1, p/4) = \sum_{k=1}^{\lfloor p/4 \rfloor} \chi(k)$ is Dirichlet's quarter-interval character sum, we have:
\begin{equation}\label{eq:reduction}
W(p) = 4 \cdot S(1, p/4) - 2.
\end{equation}

From Chattopadhyay et al.\ \cite{CRST2020}, Lemma~2(2): For $p \equiv 1 \pmod 4$, $S(1, p/4) = \frac{\sqrt{p}}{\pi} L(1, \chi_4 \chi_p)$. Combined with Dirichlet's class number formula \cite{IrelandRosen} $h(-4p) = \frac{\sqrt{4p}}{2\pi} L(1, \chi_4 \chi_p)$, we get $S(1, p/4) = h(-4p)/2 = h(-p)/2$ (since $\mathbb{Q}(\sqrt{-p}) = \mathbb{Q}(\sqrt{-4p})$ have the same class number for $p \equiv 1 \pmod 4$). Thus $W(p) = 4 \cdot \frac{h(-p)}{2} - 2 = 2h(-p) - 2$.

\textbf{Case 2: $p \equiv 3 \pmod 4$.} Now $\chi(-1) = -1$, creating a perfect pairing symmetry. For $k \leftrightarrow p-k$:
\begin{itemize}
\item $\chi(p-k) = -\chi(k)$,
\item $\operatorname{sign}\left(\cos\frac{(2(p-k)-1)\pi}{p}\right) = -\operatorname{sign}\left(\cos\frac{(2k-1)\pi}{p}\right)$.
\end{itemize}
The product $\chi(p-k) \cdot \operatorname{sign}_{p-k} = (-\chi(k))(-\operatorname{sign}_k) = \chi(k) \cdot \operatorname{sign}_k$. Since all $(p-1)$ terms pair up and contribute equally, only boundary effects at $k = (p \pm 1)/4$ determine the sum, yielding $W(p) = 2$.
\end{proof}

\begin{remark}[Novelty Assessment]
Equation~\eqref{eq:reduction} shows that $W(p)$ reduces to the classical quarter-interval character sum studied by Dirichlet, Berndt \cite{Berndt1976}, and others. To our knowledge, the novelty lies in the sign-cosine formulation itself, which provides a geometric interpretation via Chebyshev polygon lobes.
\end{remark}

\begin{corollary}[Explicit Class Number Formula]\label{cor:explicit}
For $p \equiv 1 \pmod 4$, the class number can be computed directly from our sums:
\[
h(-p) = \frac{W(p) + 2}{2}.
\]
\end{corollary}

\begin{remark}[Complementarity with Half-Interval Counting]
The classical approach of counting quadratic residues $V$ and non-residues $N$ in the half-interval $[1, (p-1)/2]$ yields $h(-p) = V - N$ (or $(V-N)/3$) for $p \equiv 3 \pmod 4$. However, for $p \equiv 1 \pmod 4$, the symmetry $\chi(p-k) = \chi(k)$ forces $V = N$, so this counting formula gives no information. The classical remedy is the quarter-interval sum $S(1, p/4) = h(-p)/2$, which is precisely what our sign-cosine formulation reduces to (equation~\eqref{eq:reduction}). The novelty is the geometric interpretation: the sign-cosine weighting corresponds to Chebyshev lobe sizes.
\end{remark}

\section{Geometric Origin and Arithmetic-Geometric Duality}

This investigation arose from studying Chebyshev polynomial lobe areas \cite{Popelka2025}. For the $n$-gon Chebyshev polygon, the normalized lobe area at position $k$ is
\[
B(n,k) = 1 + \beta_{\text{geom}}(n) \cos\frac{(2k-1)\pi}{n},
\]
where $\beta_{\text{geom}}(n) = \frac{n^2\cos(\pi/n)}{4-n^2}$ is a geometric constant derived from the Chebyshev polynomial $T_n(x)$. Since $\beta_{\text{geom}} < 0$ for $n \geq 3$:
\begin{itemize}
\item $\operatorname{sign}(\cos) = +1 \Rightarrow B(n,k) < 1$ (\textbf{small lobe}),
\item $\operatorname{sign}(\cos) = -1 \Rightarrow B(n,k) > 1$ (\textbf{large lobe}).
\end{itemize}

This provides a geometric interpretation of the class number formula: Theorem~\ref{thm:class} states that
\[
h(-p) = \frac{W(p) + 2}{2} = \frac{(\Sigma\chi_{\text{small}}) - (\Sigma\chi_{\text{large}}) + 2}{2}
\]
for $p \equiv 1 \pmod 4$.

\begin{corollary}[Arithmetic-Geometric Duality]
The class number $h(-p)$ measures how quadratic residues distribute between small and large lobes of the Chebyshev $p$-gon:
\begin{itemize}
\item $\chi(k) = \left(\frac{k}{p}\right)$ is \textbf{purely arithmetic} (modular, multiplicative),
\item $\operatorname{sign}(\cos\frac{(2k-1)\pi}{p})$ is \textbf{purely geometric} (Chebyshev lobe indicator).
\end{itemize}
Their product encodes the class number of $\mathbb{Q}(\sqrt{-p})$.
\end{corollary}

This duality suggests that the classical connection between class numbers and character sums has a previously unrecognized geometric avatar in Chebyshev polynomial geometry.

\begin{thebibliography}{99}

\bibitem{Aladov1896}
N.~S.~Aladov,
\emph{Sur la distribution des r\'esidus quadratiques et non-quadratiques d'un nombre premier $P$ dans la suite $1, 2, \ldots, P-1$},
Mat.\ Sb.\ \textbf{18} (1896), 61--75.
\url{http://mi.mathnet.ru/eng/msb/v18/i1/p61}

\bibitem{Berndt1976}
B.~C.~Berndt,
\emph{Classical theorems on quadratic residues},
Enseign.\ Math.\ (2) \textbf{22} (1976), 261--304.

\bibitem{Conrad}
K.~Conrad,
\emph{Quadratic Residue Patterns Modulo a Prime},
expository notes, University of Connecticut.
\url{https://kconrad.math.uconn.edu/blurbs/ugradnumthy/QuadraticResiduePatterns.pdf}

\bibitem{CRST2020}
D.~Chattopadhyay, B.~Roy, R.~Sarkar, T.~Thangadurai,
\emph{Distribution of Residues Modulo $p$ Using the Dirichlet's Class Number Formula},
in: Chakraborty, K.\ et al.\ (eds) Class Groups of Number Fields and Related Topics, Springer, 2020.
\href{https://doi.org/10.1007/978-981-15-1514-9_9}{doi:10.1007/978-981-15-1514-9\_9}.
Open access preprint: \href{https://arxiv.org/abs/1810.00227}{arXiv:1810.00227}.

\bibitem{IrelandRosen}
K.~Ireland, M.~Rosen,
\emph{A Classical Introduction to Modern Number Theory}, 2nd ed.,
Springer GTM~84, 1990.

\bibitem{Jacobsthal1906}
E.~Jacobsthal,
\emph{Anwendung einer Formel aus der Theorie der quadratischen Reste},
Dissertation, University of Berlin, 1906.

\bibitem{KTVZ2024}
V.~Kiritchenko, M.~Tsfasman, S.~Vl\u{a}du\c{t}, I.~Zakharevich,
\emph{Quadratic residue patterns, algebraic curves and a K3 surface},
arXiv:2403.16326 (2024).

\bibitem{Popelka2025}
J.~Popelka,
\emph{The $1/\pi$ Invariant in Chebyshev Polynomial Geometry},
preprint, 2025.
Repository: \url{https://doi.org/10.5281/zenodo.17802021}.
Paper: \url{https://github.com/popojan/orbit/releases/tag/v0.1.0-chebyshev-integral}

\bibitem{Weil1948}
A.~Weil,
\emph{On some exponential sums},
Proc.\ Nat.\ Acad.\ Sci.\ U.S.A.\ \textbf{34} (1948), 204--207.

\end{thebibliography}

\end{document}
