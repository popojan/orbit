\documentclass[11pt]{article}
\usepackage[utf8]{inputenc}
\usepackage{amsmath,amsthm,amssymb}
\usepackage{geometry}
\geometry{margin=2.5cm}
\usepackage[hidelinks]{hyperref}

\newtheorem{theorem}{Theorem}
\newtheorem{lemma}[theorem]{Lemma}
\newtheorem{corollary}[theorem]{Corollary}
\newtheorem{proposition}[theorem]{Proposition}

\theoremstyle{definition}
\newtheorem{definition}[theorem]{Definition}

\theoremstyle{remark}
\newtheorem{remark}[theorem]{Remark}
\newtheorem{example}[theorem]{Example}

\title{Primorials as Denominators of Alternating Factorial Sums}
\author{Jan Popelka\thanks{Email: popojan@protonmail.com. Code: \url{https://github.com/popojan/orbit}}}
\date{}

\begin{document}

\maketitle

\begin{abstract}
We prove that primorials emerge as denominators of alternating factorial sums.
For odd $m \geq 3$, the rational number
$S_m = \frac{1}{2} \sum_{k=1}^{(m-1)/2} \frac{(-1)^k \cdot k!}{2k+1}$
has denominator exactly equal to the primorial $\prod_{p \leq m} p$ when reduced
to lowest terms. The proof reveals a $p$-adic structure: each prime $p \leq m$
appears with valuation exactly~1 in the reduced denominator, despite higher
powers appearing in individual terms. We establish this via induction using
Legendre's formula for factorial valuations.
\end{abstract}

\section{Introduction}

The primorial function $m\# = \prod_{p \leq m} p$ appears throughout number theory,
from bounds in prime number theory to highly composite numbers. Primorials are
typically computed by direct multiplication. We present an alternative: they
emerge as denominators of alternating factorial sums.

This formula was discovered through computational experimentation. While
investigating rational sum patterns, we observed that certain alternating sums
exhibited denominators containing only products of distinct primes after
reduction. The underlying mechanism is a systematic cancellation: although
individual terms have denominators with prime powers ($9 = 3^2$, $25 = 5^2$,
$27 = 3^3$, \ldots), the factorial numerators eliminate all higher powers,
leaving exactly first powers of each prime.

\section{Statement of Results}

\begin{definition}
For integer $k \geq 1$, define the partial sum:
\[
S_k = \frac{1}{2} \sum_{j=1}^{k} \frac{(-1)^j \cdot j!}{2j+1}
\]
Write $S_k = N_k / D_k$ in \emph{unreduced form}, where:
\begin{itemize}
\item $D_k = 2 \cdot \prod_{j=1}^{k} (2j+1) = 2 \cdot 3 \cdot 5 \cdot 7 \cdots (2k+1)$
\item $N_k$ is the corresponding integer numerator
\end{itemize}
\end{definition}

\begin{theorem}[Main Result]\label{thm:main}
For all $k \geq 1$ and all primes $p \leq 2k+1$:
\[
\nu_p(D_k) - \nu_p(N_k) = 1
\]
where $\nu_p(n)$ denotes the $p$-adic valuation of $n$.
\end{theorem}

\begin{corollary}[Primorial Formula]
For odd $m \geq 3$, let $k = (m-1)/2$. Then:
\[
\mathrm{den}(S_k) = \prod_{\substack{p \text{ prime} \\ p \leq m}} p = m\#
\]
where $\mathrm{den}(q)$ denotes the denominator of $q$ in lowest terms.
\end{corollary}

\begin{proof}
By Theorem~\ref{thm:main}, each prime $p \leq 2k+1 = m$ satisfies
$\nu_p(D_k) - \nu_p(N_k) = 1$. After reduction to lowest terms, each such
prime appears exactly once in the denominator.
\end{proof}

\begin{example}
For $m = 13$ (so $k = 6$):
\[
S_6 = \frac{1}{2}\left(-\frac{1}{3} + \frac{2}{5} - \frac{6}{7} + \frac{24}{9} - \frac{120}{11} + \frac{720}{13}\right) = \frac{695971}{30030}
\]
The denominator $30030 = 2 \times 3 \times 5 \times 7 \times 11 \times 13$ is the primorial of 13.
\end{example}

\section{Recurrence Relations}

\begin{lemma}[Recurrence]\label{lem:recurrence}
The unreduced numerators and denominators satisfy:
\begin{align}
D_0 &= 2, \quad N_0 = 0 \\
D_{k+1} &= D_k \cdot (2k+3) \\
N_{k+1} &= N_k \cdot (2k+3) + (-1)^{k+1} \cdot (k+1)! \cdot \frac{D_k}{2}
\end{align}
\end{lemma}

\begin{proof}
From $S_{k+1} = S_k + \frac{(-1)^{k+1}(k+1)!}{2(2k+3)}$, expressing with common
denominator $D_{k+1} = D_k(2k+3)$ yields the stated recurrence.
\end{proof}

\section{The Factorial Inequality}

\begin{lemma}[Key Inequality]\label{lem:factorial}
For odd prime $p \geq 3$ and integer $k \geq 1$ with $p \mid (2k+1)$ and $p < 2k+1$:
\[
\nu_p(k!) \geq \nu_p(2k+1) - 1
\]
\end{lemma}

\begin{proof}
Let $\alpha = \nu_p(2k+1) \geq 1$. Since $p < 2k+1$, we have $p^\alpha \leq 2k+1$,
giving $k \geq (p^\alpha - 1)/2$.

By Legendre's formula, $\nu_p(k!) = \sum_{i \geq 1} \lfloor k/p^i \rfloor \geq \lfloor k/p \rfloor$.

\textbf{Case $\alpha = 1$:} Need $\lfloor k/p \rfloor \geq 0$. True for all $k \geq 1$.

\textbf{Case $\alpha = 2$:} From $k \geq (p^2-1)/2$:
\[
\frac{k}{p} \geq \frac{p^2-1}{2p} = \frac{p}{2} - \frac{1}{2p} \geq \frac{3}{2} - \frac{1}{6} > 1
\]
So $\lfloor k/p \rfloor \geq 1 = \alpha - 1$.

\textbf{Case $\alpha \geq 3$:} From $k \geq (p^\alpha - 1)/2$:
\[
\frac{k}{p} \geq \frac{p^\alpha - 1}{2p} \geq \frac{27 - 1}{6} > 4 \geq \alpha - 1
\]
So $\lfloor k/p \rfloor \geq \alpha - 1$.
\end{proof}

\section{Proof of Main Theorem}

\begin{proof}[Proof of Theorem~\ref{thm:main}]
We prove the result separately for $p = 2$ and odd primes $p \geq 3$.

\subsection*{The prime $p = 2$}

The unreduced denominator $D_k = 2 \cdot 3 \cdot 5 \cdots (2k+1)$ has $\nu_2(D_k) = 1$.
We show $\nu_2(N_k) = 0$ for all $k \geq 1$.

From the recurrence, $N_{k+1} = N_k(2k+3) + (-1)^{k+1}(k+1)! \cdot D_k/2$.
Since $2k+3$ is odd, the first term has $\nu_2(N_k(2k+3)) = \nu_2(N_k)$.
The second term has $\nu_2((k+1)! \cdot D_k/2) = \nu_2((k+1)!) + \nu_2(D_k) - 1 = \nu_2((k+1)!)$.

For $k = 1$: $N_1 = 0 \cdot 3 + (-1) \cdot 1! \cdot 1 = -1$, so $\nu_2(N_1) = 0$.

For $k \geq 1$: By induction, if $\nu_2(N_k) = 0$, then both terms have finite
2-adic valuation: Term~1 has $\nu_2 = 0$, Term~2 has $\nu_2 = \nu_2((k+1)!) \geq 0$.
By the ultrametric property, $\nu_2(N_{k+1}) = \min(0, \nu_2((k+1)!)) = 0$.

Thus $\nu_2(D_k) - \nu_2(N_k) = 1 - 0 = 1$ for all $k \geq 1$.

\subsection*{Odd primes $p \geq 3$}

By strong induction on $k$.

\textbf{Base case $k = 1$:} $D_1 = 6$, $N_1 = -1$.
For $p = 3$: $\nu_3(D_1) - \nu_3(N_1) = 1 - 0 = 1$. \checkmark

\textbf{Inductive step:} Fix $k \geq 1$, assume the result for all $k' \leq k$.
Let $p$ be an odd prime with $p \leq 2k+3$.

From the recurrence: $\nu_p(D_{k+1}) = \nu_p(D_k) + \nu_p(2k+3)$.

Let $\alpha = \nu_p(2k+3)$.

\textbf{Case A ($p = 2k+3$, prime entry):}
Prime $p$ appears for the first time. Then $\nu_p(D_k) = 0$ and $\nu_p(N_k) = 0$.
Also $\nu_p((k+1)!) = 0$ since $k+1 < p$.

From the recurrence, $N_{k+1} = N_k(2k+3) + (-1)^{k+1}(k+1)! \cdot D_k/2$.
The first term has $\nu_p = 1$, the second has $\nu_p = 0$.
By ultrametric property: $\nu_p(N_{k+1}) = 0$.

Thus $\nu_p(D_{k+1}) - \nu_p(N_{k+1}) = 1 - 0 = 1$. \checkmark

\textbf{Case B ($\alpha \geq 1$, $p < 2k+3$, synchronized jump):}
By induction, $\nu_p(N_k) = \nu_p(D_k) - 1$.

The recurrence gives:
\begin{align*}
\text{Term 1: } & \nu_p(N_k \cdot (2k+3)) = (\nu_p(D_k) - 1) + \alpha \\
\text{Term 2: } & \nu_p((k+1)! \cdot D_k/2) = \nu_p((k+1)!) + \nu_p(D_k)
\end{align*}

By Lemma~\ref{lem:factorial}: $\nu_p((k+1)!) \geq \alpha - 1$.

\textbf{Subcase B1} ($\nu_p((k+1)!) > \alpha - 1$): Term~1 has smaller valuation.
\[
\nu_p(N_{k+1}) = \nu_p(D_k) + \alpha - 1
\]
Thus $\nu_p(D_{k+1}) - \nu_p(N_{k+1}) = (\nu_p(D_k) + \alpha) - (\nu_p(D_k) + \alpha - 1) = 1$. \checkmark

\textbf{Subcase B2} ($\nu_p((k+1)!) = \alpha - 1$, boundary):
Both terms have valuation $\nu_p(D_k) + \alpha - 1$.

This can only occur when $p = 3$ and $\alpha = 2$ (i.e., $2k+3 = 9$, so $k = 3$).
For primes $p \geq 5$, Lemma~\ref{lem:factorial} gives strict inequality.

Direct verification for $k = 3$: Computing $N_4$ from the recurrence yields
$\nu_3(N_4) = 2$ and $\nu_3(D_4) = 3$.
Thus $\nu_3(D_4) - \nu_3(N_4) = 1$. \checkmark

\textbf{Case C ($\alpha = 0$, no new factor):}
Prime $p$ does not divide $2k+3$, so $\nu_p(D_{k+1}) = \nu_p(D_k)$.

By induction, $\nu_p(N_k) = \nu_p(D_k) - 1$. From the recurrence:
\begin{align*}
\text{Term 1: } & \nu_p(N_k \cdot (2k+3)) = \nu_p(D_k) - 1 \\
\text{Term 2: } & \nu_p((k+1)! \cdot D_k/2) \geq \nu_p(D_k)
\end{align*}
since $\nu_p((k+1)!) \geq 0$.

Term~1 has strictly smaller $p$-adic valuation, so by the ultrametric property:
\[
\nu_p(N_{k+1}) = \nu_p(D_k) - 1 = \nu_p(D_{k+1}) - 1
\]
Thus $\nu_p(D_{k+1}) - \nu_p(N_{k+1}) = 1$. \checkmark

This completes the induction.
\end{proof}

\section{Remarks}

\begin{enumerate}
\item The alternating sign is essential. Without it, the formula loses the
factor 3 at $k = 4$ (where $2k+1 = 9 = 3^2$) and yields $\text{Primorial}/3$
for all $m \geq 9$.

\item The numerator sequence $(-1, 23, 719, 5039, 40319, \ldots)$ for
$m = 3, 5, 7, 9, 11, \ldots$ contains many primes (23, 719 are prime;
$5039 = 7! - 1$, $40319 = 8! - 1$). This structure remains unexplained.

\item For comparison, the denominators of harmonic sums $H_n = \sum_{k=1}^n 1/k$
satisfy $\mathrm{den}(H_n) \mid \mathrm{lcm}(1,\ldots,n)$, which contains prime
powers. Our formula yields squarefree denominators (primorials) via factorials.

\item The GCD used in reducing $N_k/D_k$ to lowest terms has an explicit form:
$\gcd(N_k, D_k) = 2 \cdot \prod_{c} c$ where the product runs over odd
composites $c \leq m$. (For $m < 9$, the product is empty, giving $\gcd = 2$.)
Thus primes appear in the reduced denominator while composites appear in the
GCD---a complementary structure.

\item When $m$ is prime, $S_k \cdot (m-1)!$ has fractional part $n/m$ where
$n \equiv (-1)^{(m+1)/2} \cdot ((m-1)/2)! \pmod{m}$. This follows from Wilson's
theorem: only the last term (with denominator $2k+1 = m$) contributes to the
fractional part, and $(m-1)! \equiv -1 \pmod{m}$. The sign connects to the
Stickelberger relation and quadratic residue theory.
\end{enumerate}

\begin{thebibliography}{9}
\bibitem{legendre}
A.-M. Legendre,
\emph{Th\'eorie des nombres}, Firmin Didot Fr\`eres, Paris, 1830.

\bibitem{oeis}
OEIS Foundation Inc., The On-Line Encyclopedia of Integer Sequences,
\url{https://oeis.org}. Sequences A002110 (primorials).
\end{thebibliography}

\end{document}
