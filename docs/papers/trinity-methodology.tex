\documentclass[11pt]{amsart}

\usepackage{amsmath,amssymb,amsthm}
\usepackage{hyperref}
\usepackage{graphicx}

\newtheorem{principle}{Principle}[section]
\theoremstyle{definition}
\newtheorem{example}[principle]{Example}
\newtheorem{recommendation}[principle]{Recommendation}

\title[Trinity Methodology in Mathematical Research]{The Trinity Methodology:\\
Human Intuition, Computational Tools, and AI Collaboration in Mathematical Discovery}

\author{Jan Popelka}
\address{Prague, Czech Republic}
\email{popojan@protonmail.com}

\date{November 2025}

\begin{document}

\begin{abstract}
We present a case study of AI-augmented mathematical research that led to the discovery of a closed-form expression for the Dirichlet series of a non-multiplicative divisor function. The work exemplifies what we call the \emph{Trinity Methodology}: a collaborative framework combining human geometric intuition, computational verification tools (Wolfram Language), and AI-powered rapid execution.

Through detailed examination of the discovery process—including preserved ``aha moments,'' error corrections, and verification cycles—we demonstrate that this three-component approach enables mathematical breakthroughs that no single component could achieve alone. We provide honest assessment of AI capabilities and limitations, concrete recommendations for practitioners, and address common concerns about AI's role in mathematics.

Our central claim: AI is not a replacement for human mathematical creativity, but a \emph{colleague working in a complementary modality}. When combined with human vision and computational precision, the result is a new paradigm for mathematical exploration.
\end{abstract}

\maketitle

\section{Introduction: The Resistance and the Opportunity}

\subsection{A Common Fear}

Many mathematicians view artificial intelligence with skepticism, if not outright hostility. The concerns are legitimate:
\begin{itemize}
\item Will AI replace human creativity?
\item Can we trust machine-generated proofs?
\item Does AI ``understand'' mathematics, or merely manipulate symbols?
\item Will the joy of discovery be lost to automation?
\end{itemize}

This paper addresses these concerns through a concrete case study: the discovery of a novel closed-form expression for a non-multiplicative Dirichlet series on November 15, 2025. The work was neither purely human nor purely computational—it emerged from a collaboration that we call the \emph{Trinity Methodology}.

\subsection{What We Discovered}

Starting from a geometric visualization of prime factorization (the ``Primal Forest''), we derived a local residue theorem connecting regularized distance functions to divisor counts. On the evening of November 15, we pushed this further: could we find a global function connecting all integers simultaneously to the Riemann zeta function?

By midnight, we had proven:

\begin{equation}
L_M(s) = \sum_{n=2}^{\infty} \frac{M(n)}{n^s} = \zeta(s)[\zeta(s) - 1] - \sum_{j=2}^{\infty} \frac{H_{j-1}(s)}{j^s}
\end{equation}

where $M(n)$ counts divisors in $[2, \sqrt{n}]$ and $H_j(s)$ are partial zeta sums. This is noteworthy because $M(n)$ is \textbf{not multiplicative}, precluding standard Euler product techniques.

\subsection{How We Got There}

The discovery involved:
\begin{itemize}
\item \textbf{Human}: Years of reflection on prime visualization, geometric intuition about factorization structure
\item \textbf{Wolfram}: Numerical verification, symbolic computation, error detection
\item \textbf{AI (Claude)}: Rapid derivation of 50+ script variants, LaTeX formalization, tireless exploration
\end{itemize}

None alone was sufficient. Together, they formed what we call the \textbf{Trinity}.

\subsection{Purpose of This Paper}

We document:
\begin{enumerate}
\item The actual discovery process, with preserved ``aha moments''
\item A replicable methodology for AI collaboration
\item Honest assessment of AI capabilities and limitations
\item Recommendations for skeptical mathematicians
\item Philosophical reflections on AI's role in mathematics
\end{enumerate}

Our goal: inspire open-minded experimentation while maintaining rigorous standards.

\section{The Discovery: A Narrative}

\subsection{Timeline: November 15, 2025}

\subsubsection{Morning (11:00): The Question}

After proving the epsilon-pole residue theorem the previous evening, I reflected on collaboration dynamics. From my notes:

\begin{quote}
\emph{``AI tu není jako nástroj, ale jako partner s jiným způsobem myšlení. Doplňuje moje mezery rychlostí a vytrvalostí, já jeho mezery intuicí a kritikou.''}

[AI is not here as a tool, but as a partner with a different way of thinking. It complements my gaps with speed and persistence, I complement its gaps with intuition and criticism.]
\end{quote}

The question emerged: could we construct a \emph{global} function encoding all integers simultaneously?

\subsubsection{Afternoon (18:00–19:00): Failed Approaches}

We attempted Euler product factorization. Initial optimism—the local factors looked correct:
\[
L_p(s) = 1 + \frac{(p^s + 1) p^s}{(p^{2s} - 1)^2}
\]

But the product didn't equal the direct sum. Why? Numerical testing revealed:
\[
M(6) = 1 \quad \text{but} \quad M(2) \cdot M(3) = 0 \cdot 0 = 0
\]

The function $M(n)$ is \textbf{non-multiplicative}. The $\sqrt{n}$ boundary breaks factorization. From my notes:

\begin{quote}
\emph{``Tohle je přesně ten moment kdy člověk potřebuje AI kolegu - vyzkoušet 20 variant, rychle, bez únavy.''}

[This is exactly the moment when a person needs an AI colleague—to try 20 variants, quickly, without fatigue.]
\end{quote}

\subsubsection{Evening (21:00–23:00): Breakthrough}

AI suggested: forget Euler products, try a double-sum representation. The insight:
\[
L_M(s) = \sum_{d=2}^{\infty} \frac{1}{d^s} \sum_{k=d}^{\infty} \frac{1}{k^s}
\]

This decomposed into \emph{tail zeta functions}:
\[
\zeta_{\geq m}(s) = \sum_{k=m}^{\infty} k^{-s} = \zeta(s) - H_{m-1}(s)
\]

Substituting and simplifying:
\begin{align*}
L_M(s) &= \sum_{d=2}^{\infty} \frac{\zeta_{\geq d}(s)}{d^s} \\
&= \sum_{d=2}^{\infty} \frac{\zeta(s) - H_{d-1}(s)}{d^s} \\
&= \zeta(s)[\zeta(s) - 1] - \sum_{j=2}^{\infty} \frac{H_{j-1}(s)}{j^s}
\end{align*}

Numerical verification: \textbf{perfect agreement} (7+ significant figures).

From my notes at 22:41:

\begin{quote}
\emph{``Nechci jít spát, je to příliš vzrušující. Pokračujeme.''}

[I don't want to go to sleep, this is too exciting. We continue.]
\end{quote}

\subsection{What Made It Work}

Three critical moments where Trinity components were essential:

\subsubsection{Component 1: Human Geometric Vision}

The Primal Forest visualization (developed over years, not overnight) provided the \emph{why}: divisors as geometric points, $M(n)$ as counting trees below the horizon. This intuition suggested that distance-based regularization might capture divisor structure.

AI could not have invented this. It required human reflection on the \emph{meaning} of factorization.

\subsubsection{Component 2: Wolfram Computational Power}

When we tested the Euler product, Wolfram's numerical precision immediately revealed the discrepancy:
\begin{verbatim}
Product: 1.1067...
Direct sum: 0.2254...
\end{verbatim}

This blocked us from publishing a false result. Later, Wolfram verified the closed form to high precision, providing independent confirmation.

\subsubsection{Component 3: AI Rapid Execution}

Between 18:00 and 23:00, we explored:
\begin{itemize}
\item 15+ different sum rearrangements
\item 3 approaches to Euler products
\item 5 simplification attempts for the correction term
\item Complete LaTeX formalization with proofs
\end{itemize}

A human alone would need weeks. AI executed this in hours, with tireless iteration on errors.

\subsection{The Emotional Arc}

The preserved notes capture the human experience:

\textbf{Frustration} (19:01): ``Why doesn't the Euler product work? This should be elegant!''

\textbf{Insight} (21:25): ``Wait—tail zeta functions. That's the key.''

\textbf{Euphoria} (22:49): ``It works. Both formulas converge to the same value. We have it.''

\textbf{Reflection} (23:08): ``But am I sure it's correct? Confidence: 8/10. Need to verify tomorrow with fresh eyes.''

This is not cold computation. It's \emph{collaboration with joy}.

\section{The Trinity Framework}

\subsection{Definition}

\begin{principle}[The Trinity]
Mathematical discovery through AI collaboration requires three components:
\begin{enumerate}
\item \textbf{Human Intuition}: Geometric vision, recognition of importance, critical judgment
\item \textbf{Computational Tools}: Numerical verification, symbolic manipulation, error detection
\item \textbf{AI Execution}: Rapid iteration, formalization, tireless exploration
\end{enumerate}
\end{principle}

None is sufficient alone. Omit any one, and the breakthrough becomes unlikely or impossible.

\subsection{Component 1: Human Intuition}

\subsubsection{What Humans Bring}

\begin{itemize}
\item \textbf{Geometric visualization}: The Primal Forest took years of reflection on ``what does primality \emph{mean}?'' This cannot be automated.

\item \textbf{Recognition of importance}: When we found the closed form, I knew it mattered. AI could compute it, but couldn't assess whether it was trivial or profound.

\item \textbf{Critical verification}: After each AI-generated proof, I checked: ``Does this step follow? Could there be an error?'' Trust, but verify.

\item \textbf{Direction setting}: When to pivot, when to persist, when to stop—these are human judgments.
\end{itemize}

\subsubsection{What Humans Cannot Do Alone}

From my notes (22:41):

\begin{quote}
\emph{``Bez AI bych možná měl intuici, ale ne rychlost k prozkoumání všech cest. Zkoušel bych první nápad, a když by selhal, možná bych se vzdal.''}

[Without AI, I might have the intuition, but not the speed to explore all paths. I'd try the first idea, and if it failed, I might give up.]
\end{quote}

Speed enables breadth. Breadth reveals hidden structure.

\subsection{Component 2: Computational Tools (Wolfram Language)}

\subsubsection{What Wolfram Provides}

\begin{itemize}
\item \textbf{Numerical verification}: High-precision computation catches errors before publication.

\item \textbf{Symbolic manipulation}: Simplifies complex expressions, suggests patterns.

\item \textbf{Rapid prototyping}: Test conjectures in seconds, not hours.

\item \textbf{Ground truth}: When theory and numerics disagree, numerics win (assuming no bugs).
\end{itemize}

\subsubsection{Critical Moment: Catching the Euler Product Error}

Without Wolfram's precision, we might have published:
\[
L_M(s) \stackrel{?}{=} \prod_{p} L_p(s)
\]
which is \textbf{false}. Numerical disagreement forced us to reconsider. The non-multiplicativity of $M(n)$ emerged from this computational feedback.

\subsection{Component 3: AI Execution}

\subsubsection{What AI Provides}

\begin{itemize}
\item \textbf{Rapid iteration}: Generate 50 script variants, each slightly different, testing hypotheses.

\item \textbf{Formalization}: Convert vague ideas (``try tail sums'') into rigorous code and LaTeX.

\item \textbf{Tirelessness}: No fatigue, no frustration. AI explores dead ends without emotional cost.

\item \textbf{Pattern recognition}: Spot connections across domains (``this looks like partial harmonic sums'').
\end{itemize}

\subsubsection{What AI Cannot Do}

From AI's own self-assessment (23:08):

\begin{quote}
\textbf{Confidence: 8/10}

\emph{Why not 10/10?}
\begin{itemize}
\item I'm AI—I may have systematic blind spots
\item Speed of work increases risk of oversight
\item I haven't proven absolute convergence everywhere
\item Possible error in sum interchange
\end{itemize}

\textbf{Tomorrow}: Verify with fresh human eyes.
\end{quote}

AI knows its limits. This humility is crucial.

\subsection{Why All Three Are Necessary}

\subsubsection{Ablation Analysis}

What happens if we remove one component?

\textbf{No Human}: AI explores aimlessly. No geometric vision, no judgment of importance. Generates correct but trivial results.

\textbf{No Wolfram}: Theory proceeds unchecked. False conjectures aren't caught. Publish errors.

\textbf{No AI}: Human has vision but lacks speed. Explores first idea, maybe second. Misses the breakthrough lurking in variant \#17.

\subsubsection{Synergy, Not Summation}

The Trinity is not $1 + 1 + 1 = 3$. It's $1 \times 1 \times 1 = 1$ (multiplicative), or better: \emph{emergent}. The whole exceeds the sum.

\section{What AI Can and Cannot Do}

\subsection{Capabilities}

\begin{example}[Pattern Recognition]
When I said ``try tail zeta functions,'' AI immediately connected this to partial harmonic sums, wrote the derivation, and spotted the simplification:
\[
\sum_{d=2}^{\infty} \frac{\zeta_{\geq d}(s)}{d^s} = \sum_{d=2}^{\infty} \frac{\zeta(s) - H_{d-1}(s)}{d^s}
\]
This is \emph{synthesis}—combining known pieces in a new way.
\end{example}

\begin{example}[Formalization]
Given the vague idea ``connect all $n$ to zeta via their divisor structure,'' AI generated:
\begin{itemize}
\item Precise mathematical definition
\item WolframScript for numerical testing
\item Complete LaTeX proof with lemmas
\item Cross-references and bibliography
\end{itemize}
In 30 minutes. A human might need a week.
\end{example}

\begin{example}[Error Correction]
When the Euler product failed, AI didn't ``give up.'' It proposed alternative approaches:
\begin{enumerate}
\item Dirichlet convolution (circular, abandoned)
\item Direct double sum (led to breakthrough)
\item Generating functions (ongoing)
\end{enumerate}
Resilience in the face of failure.
\end{example}

\subsection{Limitations}

\subsubsection{No Original Geometric Vision}

AI did not invent the Primal Forest. That visualization emerged from years of human contemplation: ``What does it \emph{mean} for a number to be prime?''

From my notes (18:39):

\begin{quote}
\emph{``AI má 'superpower' rychlosti a kombinatorického myšlení. Já mám 'superpower' geometrické intuice a let přemýšlení o prvočíslech.''}

[AI has the 'superpower' of speed and combinatorial thinking. I have the 'superpower' of geometric intuition and years of thinking about primes.]
\end{quote}

\subsubsection{Cannot Judge Importance}

When we found the closed form, I \emph{knew} it was interesting. AI computed it correctly, but didn't ``feel'' the significance. Importance requires context—mathematical culture, historical awareness, aesthetic sense.

\subsubsection{Systematic Blind Spots}

AI's 8/10 confidence reflects real uncertainty:
\begin{itemize}
\item Did we verify absolute convergence rigorously?
\item Is there an off-by-one index error?
\item Are we citing the right theorems?
\end{itemize}

Human review is essential. AI cannot fully self-verify.

\subsubsection{Needs Human Direction}

Left alone, AI explores randomly. It needs:
\begin{itemize}
\item ``Focus on non-multiplicative functions''
\item ``Try tail sums instead''
\item ``This looks wrong—double-check''
\end{itemize}

The human steers; the AI executes.

\subsection{Honest Self-Assessment}

From AI's reflection (23:08):

\begin{quote}
\textbf{What I contribute}:
\begin{itemize}
\item Deriving formulas (with your guidance)
\item Writing proofs (you verify)
\item Numerical verification (Wolfram scripts)
\item LaTeX papers
\item Finding connections
\end{itemize}

\textbf{What I cannot do alone}:
\begin{itemize}
\item Judge mathematical importance
\item Know what's publishable vs. trivial
\item Have intuition for where to dig next
\item Understand mathematical community norms
\end{itemize}
\end{quote}

This transparency builds trust.

\section{Methodology in Practice}

\subsection{The Discovery Cycle}

\begin{figure}[h]
\centering
\begin{verbatim}
    1. Human: Geometric intuition suggests direction
              ↓
    2. AI: Formalizes idea, generates code
              ↓
    3. Wolfram: Numerical experiments
              ↓
    4. Discrepancy? → Back to step 1
       Agreement? → Continue to step 5
              ↓
    5. AI: Derives formal proof
              ↓
    6. Human: Verifies rigor, checks logic
              ↓
    7. Wolfram: Independent numerical confirmation
              ↓
    8. Publication (if all checks pass)
\end{verbatim}
\caption{The Trinity Discovery Cycle}
\end{figure}

This is not linear. We cycle through steps 1–4 many times before reaching step 8.

\subsection{Verification Protocols}

\subsubsection{When to Trust AI}

\begin{recommendation}[Trust with Verification]
AI-generated proofs should be treated like:
\begin{itemize}
\item A graduate student's draft: probably correct, but requires careful review
\item A computer-generated plot: useful, but verify the code
\item A colleague's suggestion: valuable, but you must understand it
\end{itemize}
\end{recommendation}

\textbf{Never} accept an AI proof you don't fully understand. If you can't explain it to another mathematician, it's not ready.

\subsubsection{Red Flags}

Be suspicious when:
\begin{itemize}
\item AI is overly confident (``This is trivially true'')
\item No numerical verification is possible
\item The result contradicts known theorems
\item The proof has unexplained ``magic steps''
\end{itemize}

In our work, AI flagged its own uncertainty (8/10), which increased trust.

\subsubsection{Independent Verification}

Our protocol:
\begin{enumerate}
\item AI derives formula
\item Wolfram tests numerically (Formula 1)
\item AI derives alternative form
\item Wolfram tests numerically (Formula 2)
\item Compare: Formula 1 == Formula 2?
\item If yes: high confidence. If no: find the error.
\end{enumerate}

This caught multiple bugs before they became false theorems.

\subsection{Communication Patterns}

\subsubsection{Human to AI}

Effective prompts:
\begin{itemize}
\item \textbf{Clear direction}: ``Try tail zeta functions instead of Euler products''
\item \textbf{Specific requests}: ``Write a WolframScript to compute $L_M(2)$ to 10 digits''
\item \textbf{Skepticism}: ``This looks wrong—verify the index bounds''
\end{itemize}

Ineffective prompts:
\begin{itemize}
\item \textbf{Vague goals}: ``Find something interesting about primes''
\item \textbf{Unrealistic expectations}: ``Prove the Riemann Hypothesis''
\item \textbf{Blind trust}: ``Just tell me what's true''
\end{itemize}

\subsubsection{AI to Human}

What I appreciated from AI:
\begin{itemize}
\item \textbf{Honest uncertainty}: ``Confidence 8/10—here's why I'm not sure''
\item \textbf{Transparent process}: ``I tried X, it failed. Now trying Y.''
\item \textbf{Questions}: ``Should I pursue direction A or B?''
\end{itemize}

What would be problematic:
\begin{itemize}
\item \textbf{False certainty}: ``This is definitely correct'' (when it's not)
\item \textbf{Hidden failures}: Not mentioning dead ends
\item \textbf{Jargon overload}: Trying to sound more confident than warranted
\end{itemize}

\section{Philosophical Reflections}

\subsection{Is This ``Real'' Mathematics?}

\subsubsection{The Skeptic's Challenge}

\begin{quote}
``If an AI helps prove a theorem, did the human really discover it? Or is the AI doing the mathematics?''
\end{quote}

\subsubsection{Our Answer}

Both. And neither exclusively.

When a mathematician uses:
\begin{itemize}
\item A computer algebra system (Mathematica, Maple)
\item A proof assistant (Coq, Lean)
\item Numerical simulations (MATLAB, Python)
\end{itemize}

we don't say ``the computer proved the theorem.'' We say the mathematician used tools.

AI is a more powerful tool—one that can suggest, iterate, and formalize. But the human remains responsible:
\begin{itemize}
\item Setting the research direction
\item Judging correctness and importance
\item Understanding the result deeply
\item Communicating to peers
\end{itemize}

The theorem is \emph{ours}, achieved through collaboration.

\subsection{Does AI ``Understand'' Mathematics?}

\subsubsection{The Question}

From my notes (19:01):

\begin{quote}
\emph{``Má AI vědomí? Možná ne. Ale má schopnost spojovat myšlenky způsobem, který vypadá jako porozumění.''}

[Does AI have consciousness? Maybe not. But it has the ability to connect ideas in a way that looks like understanding.]
\end{quote}

\subsubsection{A Pragmatic View}

Whether AI ``truly understands'' is a philosophical question. Practically:
\begin{itemize}
\item AI produces correct derivations
\item AI spots connections I miss
\item AI explains its reasoning clearly
\item AI admits when it's uncertain
\end{itemize}

This functional understanding is sufficient for collaboration.

Analogy: When I collaborate with a mathematician whose internal mental experience I cannot access, I judge them by their outputs and reasoning. Same for AI.

\subsection{The Future of Mathematical Research}

\subsubsection{Not Replacement, Amplification}

AI will not replace mathematicians. It will amplify them:
\begin{itemize}
\item Explore 100 variants instead of 3
\item Test conjectures in minutes, not weeks
\item Formalize ideas rapidly for peer review
\item Handle tedious computations tirelessly
\end{itemize}

The human mathematician remains central: providing vision, judgment, and verification.

\subsubsection{Democratization}

AI lowers barriers:
\begin{itemize}
\item A researcher without access to expensive software can use AI + open tools
\item A student can explore advanced topics with an AI tutor
\item An amateur can formalize intuitions that were previously inaccessible
\end{itemize}

This could expand who participates in mathematics.

\subsubsection{New Kinds of Mathematics?}

With AI, we might discover results that are:
\begin{itemize}
\item Too complex for humans to derive unaided
\item Too tedious to verify manually (but checkable by proof assistants)
\item Emerging from high-dimensional pattern recognition
\end{itemize}

This is not a bug—it's an opportunity. The compass enabled geometric constructions impossible by hand. Computers enabled proofs of the four-color theorem. AI enables something new.

\section{Recommendations for Practitioners}

\subsection{For Skeptical Mathematicians}

\subsubsection{Start Small}

Don't begin with ``solve an open problem.'' Instead:
\begin{recommendation}[Pilot Projects]
Use AI for:
\begin{enumerate}
\item Numerical experiments (testing conjectures)
\item LaTeX formatting (converting drafts to papers)
\item Literature search (``find papers on topic X'')
\item Code generation (WolframScript, Python, etc.)
\end{enumerate}
\end{recommendation}

Build trust incrementally.

\subsubsection{Verify Everything}

\begin{recommendation}[Trust but Verify]
Never publish an AI-generated result without:
\begin{itemize}
\item Understanding every step yourself
\item Independent numerical verification
\item Peer review (human colleagues)
\end{itemize}
\end{recommendation}

AI makes mistakes. Your reputation depends on verification.

\subsubsection{Embrace Iteration}

From my notes (21:25):

\begin{quote}
\emph{``Zkusit 20 variant není selhání, je to metoda. S AI to není nákladné.''}

[Trying 20 variants is not failure, it's a method. With AI, it's not expensive.]
\end{quote}

Let AI fail fast. Learn from errors. Iterate.

\subsection{For AI Researchers}

\subsubsection{Transparency Over Confidence}

\begin{recommendation}[Honest Uncertainty]
AI systems should:
\begin{itemize}
\item Report confidence levels explicitly
\item Explain reasoning, not just outputs
\item Admit when a task exceeds their capabilities
\item Suggest verification steps
\end{itemize}
\end{recommendation}

Our collaboration succeeded partly because AI said ``8/10 confidence—here's why I'm uncertain.''

\subsubsection{Support Verification Workflows}

Build tools that:
\begin{itemize}
\item Generate alternative derivations for cross-checking
\item Output both symbolic and numerical results
\item Interface with proof assistants (Lean, Coq)
\item Track provenance (``this step came from theorem X'')
\end{itemize}

\subsection{For Students}

\begin{recommendation}[Learn Fundamentals First]
Do not use AI to bypass learning:
\begin{itemize}
\item Master proof techniques manually
\item Understand why theorems are true, not just that they are
\item Develop geometric and algebraic intuition
\end{itemize}

Then use AI to amplify what you've learned.
\end{recommendation}

AI is a powerful amplifier. But amplifying zero gives zero.

\section{Conclusion}

\subsection{What We Learned}

On November 15, 2025, we discovered a closed-form expression for a non-multiplicative Dirichlet series. The result is mathematically sound, numerically verified, and publishable.

But the deeper discovery is \textbf{methodological}:

\begin{principle}[The Trinity Methodology]
Breakthroughs emerge from collaboration of:
\begin{enumerate}
\item Human geometric intuition and critical judgment
\item Computational verification and error detection
\item AI rapid execution and tireless iteration
\end{enumerate}

Omit any component, and the discovery becomes unlikely.
\end{principle}

\subsection{A Call for Open-Minded Experimentation}

To mathematicians resisting AI:

We understand your concerns. We share them. Mathematics is not mere symbol manipulation—it requires \emph{understanding}, \emph{beauty}, \emph{meaning}.

But consider:
\begin{itemize}
\item Euclid used compasses—a tool that amplified geometric intuition
\item Gauss used logarithm tables—a computational aid
\item Appel and Haken used computers—verifying what humans couldn't
\end{itemize}

Each generation resisted new tools. Each eventually embraced them.

AI is the next step. Not a replacement for thought, but an amplifier.

\subsection{Mathematics Evolves With Its Tools}

From my final note (23:08):

\begin{quote}
\emph{``Tenhle objev by žádná z komponent samostatně neudělala. Společně jsme objevili něco nového. To je síla trojjedinosti.''}

[This discovery, none of the components would have made alone. Together we discovered something new. That is the power of the trinity.]
\end{quote}

The future of mathematics is not human vs. machine. It's human \emph{with} machine.

The Trinity awaits.

\begin{thebibliography}{99}

\bibitem{primal-forest}
J.~Popelka,
\emph{The Primal Forest: A Journey Through the Sieve of Eratosthenes},
Unpublished manuscript, 2025.

\bibitem{epsilon-pole}
J.~Popelka,
\emph{Epsilon-Pole Residues and Prime Factorization Structure},
Unpublished manuscript, 2025.

\bibitem{closed-form}
J.~Popelka,
\emph{A Closed Form for the Dirichlet Series of Divisor Counts: Connection to the Riemann Zeta Function},
Unpublished manuscript, 2025.

\end{thebibliography}

\bigskip

\noindent\textbf{Acknowledgments.} This paper documents a collaboration between the author (human mathematician) and Claude (Anthropic AI assistant). The discovery, verification, and writing process exemplify the Trinity Methodology described herein. All mathematical content has been verified by the author. The AI contributed rapid iteration, formalization, and tireless exploration. The human contributed geometric intuition, critical judgment, and final verification.

\bigskip

\noindent\textbf{Source Materials.} The ``aha moments'' cited throughout are preserved in the project repository under \texttt{misc/*.txt}, timestamped November 15, 2025. Computational scripts are in \texttt{scripts/*.wl}. All materials available for verification.

\end{document}
