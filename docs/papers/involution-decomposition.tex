\documentclass[11pt,a4paper]{article}

\usepackage{amsmath,amssymb,amsthm}
\usepackage{hyperref}
\usepackage{booktabs}
\usepackage{enumitem}

\theoremstyle{plain}
\newtheorem{theorem}{Theorem}
\newtheorem{lemma}[theorem]{Lemma}
\newtheorem{corollary}[theorem]{Corollary}
\newtheorem{proposition}[theorem]{Proposition}

\theoremstyle{definition}
\newtheorem{definition}[theorem]{Definition}
\newtheorem{example}[theorem]{Example}

\theoremstyle{remark}
\newtheorem{remark}[theorem]{Remark}
\newtheorem{conjecture}[theorem]{Conjecture}

\title{Atomic Involution Decomposition of Calkin--Wilf Generators}
\author{Jan Popelka}
\date{December 2025 \\ \small Draft v0.1}

\begin{document}

\maketitle

\begin{abstract}
We show that the Calkin--Wilf tree generators $L(x) = x/(1+x)$ and $R(x) = x+1$
can be decomposed into compositions of three elementary M\"obius involutions:
$\sigma(x) = (1-x)/(1+x)$, $\kappa(x) = 1-x$, and $\iota(x) = 1/x$.
Specifically, $L = \kappa\iota\kappa\sigma\iota\sigma$ and $R = \kappa\sigma\iota\sigma$.
This decomposition reveals that the transitive action of $\langle L, R \rangle$ on $\mathbb{Q}^+$
can be generated by involutions with coefficients in $\{-1, 0, 1\}$ only.
We also identify the orbit structure of the subgroup $\langle \sigma, \kappa \rangle$
restricted to $(0,1) \cap \mathbb{Q}$, which is governed by the invariant
$I(p/q) = \mathrm{odd}(p(q-p))$.
\end{abstract}

\section{Introduction}

The Calkin--Wilf tree \cite{CalkinWilf2000} provides an elegant enumeration of positive
rational numbers. Starting from the root $1$, each vertex $a/b$ (in lowest terms) has
two children: $a/(a+b)$ (left) and $(a+b)/b$ (right). Every positive rational appears
exactly once in the tree.

The tree structure is governed by two generators:
\begin{align}
L(x) &= \frac{x}{1+x}, \qquad \text{(left child)} \\
R(x) &= 1 + x. \qquad \text{(right child)}
\end{align}

The Calkin--Wilf theorem states that $\langle L, R \rangle$ acts transitively on $\mathbb{Q}^+$.

In this note, we show that $L$ and $R$ are not ``atomic'' --- they decompose into
simpler M\"obius involutions, expressing the Calkin--Wilf generators in terms of
elementary building blocks.

\section{Elementary M\"obius Involutions}

\begin{definition}
A \emph{M\"obius transformation} is a function of the form $f(x) = (ax+b)/(cx+d)$
where $ad - bc \neq 0$. It is an \emph{involution} if $f \circ f = \mathrm{id}$.
\end{definition}

We consider three elementary involutions with coefficients in $\{-1, 0, 1\}$:

\begin{center}
\begin{tabular}{lccc}
\toprule
Name & Formula & Matrix & Fixed points \\
\midrule
$\sigma$ (silver) & $(1-x)/(1+x)$ & $\begin{pmatrix} -1 & 1 \\ 1 & 1 \end{pmatrix}$ & $\sqrt{2}-1$ \\[2mm]
$\kappa$ (copper) & $1-x$ & $\begin{pmatrix} -1 & 1 \\ 0 & 1 \end{pmatrix}$ & $1/2$ \\[2mm]
$\iota$ (inv) & $1/x$ & $\begin{pmatrix} 0 & 1 \\ 1 & 0 \end{pmatrix}$ & $1$ \\
\bottomrule
\end{tabular}
\end{center}

\begin{remark}
These are arguably the simplest non-trivial M\"obius involutions, using only
coefficients from $\{-1, 0, 1\}$. The name ``silver'' reflects the connection
to the silver ratio $\delta_S = 1 + \sqrt{2}$, as $\sigma$ has fixed point
$\sqrt{2} - 1 = 1/\delta_S$.
Compositions are written as juxtaposition (right-to-left): $\sigma\iota\sigma(x) = \sigma(\iota(\sigma(x)))$.
\end{remark}

\section{Main Result}

\begin{theorem}[Involution Decomposition]\label{thm:main}
The Calkin--Wilf generators decompose as:
\begin{align}
L &= \kappa\iota\kappa\sigma\iota\sigma, \\
R &= \kappa\sigma\iota\sigma.
\end{align}
Consequently, $\langle \sigma, \kappa, \iota \rangle$ acts transitively on $\mathbb{Q}^+$.
\end{theorem}

\begin{proof}
We verify the composition for $L$ algebraically. Let $x \in \mathbb{Q}^+$.

\textbf{Step 1:} Compute $\sigma\iota\sigma$:
\begin{align*}
\sigma(x) &= \frac{1-x}{1+x}, \\
\iota\sigma(x) &= \frac{1+x}{1-x}, \\
\sigma\iota\sigma(x) &= \frac{1 - \frac{1+x}{1-x}}{1 + \frac{1+x}{1-x}}
= \frac{(1-x)-(1+x)}{(1-x)+(1+x)} = \frac{-2x}{2} = -x.
\end{align*}

Thus $\sigma\iota\sigma = -\mathrm{id}$ (negation).

\textbf{Step 2:} Complete the composition:
\begin{align*}
\kappa(-x) &= 1 - (-x) = 1 + x, \\
\iota\kappa(-x) &= \frac{1}{1+x}, \\
\kappa\iota\kappa(-x) &= 1 - \frac{1}{1+x} = \frac{x}{1+x} = L(x).
\end{align*}

Thus $L = \kappa\iota\kappa \cdot \sigma\iota\sigma = \kappa\iota\kappa\sigma\iota\sigma$.

For $R$: We have $\sigma\iota\sigma(x) = -x$, so
$\kappa\sigma\iota\sigma(x) = \kappa(-x) = 1-(-x) = 1+x = R(x)$.

The transitivity follows from the Calkin--Wilf theorem \cite{CalkinWilf2000}:
since $L, R \in \langle \sigma, \kappa, \iota \rangle$ and $\langle L, R \rangle$
is transitive on $\mathbb{Q}^+$, so is $\langle \sigma, \kappa, \iota \rangle$.
\end{proof}

\begin{corollary}
The identity $\sigma\iota\sigma = -\mathrm{id}$ provides a ``negation gate''
constructed from three involutions.
\end{corollary}

\section{Orbit Structure of $\langle \sigma, \kappa \rangle$}

While the full group $\langle \sigma, \kappa, \iota \rangle$ acts transitively
on $\mathbb{Q}^+$, the subgroup $\langle \sigma, \kappa \rangle$ has a richer
orbit structure when restricted to $(0,1) \cap \mathbb{Q}$.

\begin{proposition}[Orbit Invariant]\label{prop:invariant}
For $p/q \in (0,1) \cap \mathbb{Q}$ in lowest terms, define
\[
I(p/q) = \mathrm{odd}(p(q-p)),
\]
where $\mathrm{odd}(n) = n / 2^{\nu_2(n)}$ is the odd part of $n$.
Then $I$ is invariant under $\sigma$ and $\kappa$.
\end{proposition}

\begin{proof}
For $\kappa$: $\kappa(p/q) = (q-p)/q$, so
$I(\kappa(p/q)) = \mathrm{odd}((q-p) \cdot p) = I(p/q)$.

For $\sigma$: $\sigma(p/q) = (q-p)/(q+p)$, so
$I(\sigma(p/q)) = \mathrm{odd}((q-p) \cdot 2p) = \mathrm{odd}(2p(q-p)) = \mathrm{odd}(p(q-p)) = I(p/q)$.
\end{proof}

\begin{conjecture}[Completeness]
$I$ completely characterizes orbits: two fractions in $(0,1)$ are in the same
$\langle \sigma, \kappa \rangle$-orbit if and only if they have the same $I$ value.
\end{conjecture}

\noindent\textit{Evidence:} Verified computationally for all $p/q$ with $q \leq 100$.

\begin{example}
Fractions with $I = 1$: $\{1/2, 1/3, 2/3, 1/5, 2/5, 3/5, 4/5, 1/9, \ldots\}$.
These form a single orbit under $\langle \sigma, \kappa \rangle$.

The canonical representative for orbit $I = k$ is $1/(k+1)$ when $k+1$ is a valid denominator.
\end{example}

\section{Path Length Analysis}

For fractions within the same $I$-orbit, the involution path can be significantly
shorter than the continued fraction representation.

\begin{proposition}
For $k \geq 1$, the distance from $1/2$ to $1/(2^k+1)$ under $\langle \sigma, \kappa \rangle$
is exactly $2k-1$, achieved by the path $\sigma(\kappa\sigma)^{k-1}$.
\end{proposition}

\begin{proof}
By induction. Base case ($k=1$): $\sigma(1/2) = (1-1/2)/(1+1/2) = 1/3 = 1/(2^1+1)$.

Inductive step: Assume $(\sigma\kappa)^{k-1}\sigma(1/2) = 1/(2^k+1)$.
Then $\kappa(1/(2^k+1)) = 2^k/(2^k+1)$, and
\[
\sigma\left(\frac{2^k}{2^k+1}\right) = \frac{1 - \frac{2^k}{2^k+1}}{1 + \frac{2^k}{2^k+1}}
= \frac{1}{2^{k+1}+1}.
\]
Hence $\sigma(\kappa\sigma)^k(1/2) = 1/(2^{k+1}+1)$, with path length $2(k+1)-1 = 2k+1$.
\end{proof}

\begin{center}
\begin{tabular}{ccccc}
\toprule
$k$ & Target & Path & Length & $|$CF$|$ \\
\midrule
1 & $1/3$ & $\sigma$ & 1 & 3 \\
2 & $1/5$ & $\sigma\kappa\sigma$ & 3 & 5 \\
3 & $1/9$ & $\sigma\kappa\sigma\kappa\sigma$ & 5 & 9 \\
4 & $1/17$ & $\sigma(\kappa\sigma)^3$ & 7 & 17 \\
\bottomrule
\end{tabular}
\end{center}

For this family, path length is $\Theta(\log q)$ while CF sum equals $q$.
This demonstrates that involutions can provide shortcuts for fractions with
large continued fraction coefficients. For typical fractions (where CF sum
is also $O(\log q)$ by Khinchin's theorem \cite{Khinchin1964}), no such speedup is guaranteed.

\section{Continued Fractions and the Euclidean Algorithm}

The well-known correspondence between continued fractions and the Euclidean
algorithm acquires a new interpretation through involutions: each GCD step
corresponds to an application of~$\iota$.

\begin{proposition}[CF-Involution Correspondence]
For $p/q \in (0,1) \cap \mathbb{Q}$ with continued fraction $[0; a_1, a_2, \ldots, a_n]$,
the involution representation is:
\[
\frac{p}{q} = \iota R^{a_1} \iota R^{a_2} \cdots \iota R^{a_n}(0),
\]
where $R(x) = 1+x$ and $\iota(x) = 1/x$.
\end{proposition}

\begin{proof}
The continued fraction $[0; a_1, \ldots, a_n]$ satisfies the recurrence
\[
\frac{p}{q} = \frac{1}{a_1 + \frac{1}{a_2 + \cdots}}.
\]
Starting from $a_n$ and working backwards: $R^{a_n}(0) = a_n$,
then $\iota R^{a_n}(0) = 1/a_n$, then $R^{a_{n-1}} \iota R^{a_n}(0) = a_{n-1} + 1/a_n$,
and so on. The final $\iota$ converts the result to a proper fraction.
\end{proof}

\begin{example}
For $7/11 = [0; 1, 1, 1, 3]$, the GCD, CF, and involution perspectives align:

\begin{center}
\begin{tabular}{clcl}
\toprule
Step & Euclidean Algorithm & $a_i$ & Involution Build-up \\
\midrule
0 & $\gcd(11, 7)$ start & --- & $R^3(0) = 3$ \\
1 & $11 = 1 \cdot 7 + 4$ & 1 & $\iota R^3(0) = 1/3$ \\
2 & $7 = 1 \cdot 4 + 3$ & 1 & $R\iota R^3 = 4/3 \xrightarrow{\iota} 3/4$ \\
3 & $4 = 1 \cdot 3 + 1$ & 1 & $R\iota R\iota R^3 = 7/4 \xrightarrow{\iota} 4/7$ \\
4 & $3 = 3 \cdot 1 + 0$ & 3 & $R\iota R\iota R\iota R^3 = 11/7 \xrightarrow{\iota} 7/11$ \\
\bottomrule
\end{tabular}
\end{center}

Each GCD quotient $q_i$ equals CF coefficient $a_i$, and each $\iota$ corresponds
to one GCD step (swapping dividend and remainder).
\end{example}

Since $R = \kappa\sigma\iota\sigma$ decomposes into 4 atomic involutions,
the total atomic involution count for $p/q$ is $4 \sum a_i + n$,
where $n$ is the CF length.

\section{Connection to Stern--Brocot Tree}

The Calkin--Wilf tree is closely related to the Stern--Brocot tree \cite{GrahamKnuthPatashnik1994}.
Both enumerate $\mathbb{Q}^+$ using binary tree structures. Wildberger \cite{Wildberger2010}
studied $L$-$R$ factorizations in the context of Pell equations and the Stern--Brocot tree.

Our contribution is the observation that $L$ and $R$ themselves factor into
elementary involutions, providing a finer decomposition than the $L$-$R$ level.

\section{Discussion}

The decomposition of Calkin--Wilf generators into elementary involutions
provides a finer-grained view of rational number arithmetic, with
$\{\sigma, \kappa, \iota\}$ serving as atomic building blocks.

\textbf{Open questions:}
\begin{enumerate}[label=(\arabic*)]
\item Is there a direct ``GCD-like'' algorithm for finding the shortest involution
path between two rationals, without going through Calkin--Wilf?
\item What is the structure of $\langle \sigma, \kappa, \iota \rangle$ as an
abstract group? Is it isomorphic to a known group?
\item Can the invariant $I$ be generalized to characterize orbits of other
involution subgroups?
\end{enumerate}

\section*{Acknowledgments}

This work was developed in collaboration with Claude (Anthropic).

\begin{thebibliography}{9}

\bibitem{CalkinWilf2000}
N.~Calkin and H.~S.~Wilf,
``Recounting the Rationals,''
\emph{American Mathematical Monthly}, vol.~107, no.~4, pp.~360--363, 2000.

\bibitem{GrahamKnuthPatashnik1994}
R.~L.~Graham, D.~E.~Knuth, and O.~Patashnik,
\emph{Concrete Mathematics}, 2nd ed.
Addison-Wesley, 1994.
(Stern--Brocot tree, Section 4.5)

\bibitem{Khinchin1964}
A.~Ya.~Khinchin,
\emph{Continued Fractions}.
University of Chicago Press, 1964.

\bibitem{Wildberger2010}
N.~J.~Wildberger,
``Solving the Pell Equation,''
\emph{Journal of Integer Sequences}, vol.~13, Article 10.4.3, 2010.

\end{thebibliography}

\end{document}
