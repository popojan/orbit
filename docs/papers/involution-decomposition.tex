\documentclass[11pt]{article}
\usepackage[utf8]{inputenc}
\usepackage{amsmath,amssymb,amsthm}
\usepackage{geometry}
\geometry{margin=2.5cm}
\usepackage[hidelinks]{hyperref}
\usepackage{booktabs}
\usepackage{enumitem}

\theoremstyle{plain}
\newtheorem{theorem}{Theorem}
\newtheorem{lemma}[theorem]{Lemma}
\newtheorem{corollary}[theorem]{Corollary}
\newtheorem{proposition}[theorem]{Proposition}

\theoremstyle{definition}
\newtheorem{definition}[theorem]{Definition}
\newtheorem{example}[theorem]{Example}

\theoremstyle{remark}
\newtheorem{remark}[theorem]{Remark}
\newtheorem{conjecture}[theorem]{Conjecture}

\title{Atomic Involution Decomposition of Calkin--Wilf Generators}
\author{Jan Popelka}
\date{December 2025 \\ \small Draft v0.1}

\begin{document}

\maketitle

\begin{abstract}
We show that the Calkin--Wilf tree generators $L(x) = x/(1+x)$ and $R(x) = x+1$
can be decomposed into compositions of three elementary M\"obius involutions:
$\sigma(x) = (1-x)/(1+x)$, $\kappa(x) = 1-x$, and $\iota(x) = 1/x$.
Specifically, $L = \kappa\iota\kappa\sigma\iota\sigma$ and $R = \kappa\sigma\iota\sigma$.
This decomposition reveals that the transitive action of $\langle L, R \rangle$ on $\mathbb{Q}^+$
can be generated by involutions with coefficients in $\{-1, 0, 1\}$ only.
We also identify the orbit structure of the subgroup $\langle \sigma, \kappa \rangle$
restricted to $(0,1) \cap \mathbb{Q}$, which is governed by the invariant
$I(p/q) = \mathrm{odd}(p(q-p))$, where $\mathrm{odd}(n) = n/2^{v_2(n)}$ denotes the odd part of~$n$.
\end{abstract}

\section{Introduction}

The Calkin--Wilf tree \cite{CalkinWilf2000} provides an elegant enumeration of positive
rational numbers. Starting from the root $1$, each vertex $a/b$ (in lowest terms) has
two children: $a/(a+b)$ (left) and $(a+b)/b$ (right). Every positive rational appears
exactly once in the tree.

The tree structure is governed by two generators:
\begin{align}
L(x) &= \frac{x}{1+x}, \qquad \text{(left child)} \\
R(x) &= 1 + x. \qquad \text{(right child)}
\end{align}

The Calkin--Wilf theorem states that $\langle L, R \rangle$ acts transitively on $\mathbb{Q}^+$.

In this note, we show that $L$ and $R$ are not ``atomic'' --- they decompose into
simpler M\"obius involutions, expressing the Calkin--Wilf generators in terms of
elementary building blocks.

\section{Elementary M\"obius Involutions}

\begin{definition}
A \emph{M\"obius transformation} is a function of the form $f(x) = (ax+b)/(cx+d)$
where $ad - bc \neq 0$. It is an \emph{involution} if $f \circ f = \mathrm{id}$.
\end{definition}

We consider three elementary involutions with coefficients in $\{-1, 0, 1\}$:

\begin{center}
\begin{tabular}{lccc}
\toprule
Name & Formula & Matrix & Fixed points \\
\midrule
$\sigma$ (silver) & $(1-x)/(1+x)$ & $\begin{pmatrix} -1 & 1 \\ 1 & 1 \end{pmatrix}$ & $\sqrt{2}-1$ \\[2mm]
$\kappa$ (copper) & $1-x$ & $\begin{pmatrix} -1 & 1 \\ 0 & 1 \end{pmatrix}$ & $1/2$ \\[2mm]
$\iota$ (inv) & $1/x$ & $\begin{pmatrix} 0 & 1 \\ 1 & 0 \end{pmatrix}$ & $1$ \\
\bottomrule
\end{tabular}
\end{center}

\begin{remark}
These are arguably the simplest non-trivial M\"obius involutions, using only
coefficients from $\{-1, 0, 1\}$. The name ``silver'' reflects the connection
to the silver ratio $\delta_S = 1 + \sqrt{2}$, as $\sigma$ has fixed point
$\sqrt{2} - 1 = 1/\delta_S$.
Compositions are written as juxtaposition (right-to-left): $\sigma\iota\sigma(x) = \sigma(\iota(\sigma(x)))$.
\end{remark}

\section{Main Result}

\begin{theorem}[Involution Decomposition]\label{thm:main}
The Calkin--Wilf generators decompose as:
\begin{align}
L &= \kappa\iota\kappa\sigma\iota\sigma, \\
R &= \kappa\sigma\iota\sigma.
\end{align}
Consequently, $\langle \sigma, \kappa, \iota \rangle$ acts transitively on $\mathbb{Q}^+$.
\end{theorem}

\begin{proof}
We verify the composition for $L$ algebraically. Let $x \in \mathbb{Q}^+$.

\textbf{Step 1:} Compute $\sigma\iota\sigma$:
\begin{align*}
\sigma(x) &= \frac{1-x}{1+x}, \\
\iota\sigma(x) &= \frac{1+x}{1-x}, \\
\sigma\iota\sigma(x) &= \frac{1 - \frac{1+x}{1-x}}{1 + \frac{1+x}{1-x}}
= \frac{(1-x)-(1+x)}{(1-x)+(1+x)} = \frac{-2x}{2} = -x.
\end{align*}

Thus $\sigma\iota\sigma = -\mathrm{id}$ (negation).

\textbf{Step 2:} Complete the composition:
\begin{align*}
\kappa(-x) &= 1 - (-x) = 1 + x, \\
\iota\kappa(-x) &= \frac{1}{1+x}, \\
\kappa\iota\kappa(-x) &= 1 - \frac{1}{1+x} = \frac{x}{1+x} = L(x).
\end{align*}

Thus $L = \kappa\iota\kappa \cdot \sigma\iota\sigma = \kappa\iota\kappa\sigma\iota\sigma$.

For $R$: We have $\sigma\iota\sigma(x) = -x$, so
$\kappa\sigma\iota\sigma(x) = \kappa(-x) = 1-(-x) = 1+x = R(x)$.

The transitivity follows from the Calkin--Wilf theorem \cite{CalkinWilf2000}:
since $L, R \in \langle \sigma, \kappa, \iota \rangle$ and $\langle L, R \rangle$
is transitive on $\mathbb{Q}^+$, so is $\langle \sigma, \kappa, \iota \rangle$.
\end{proof}

\begin{corollary}
The identity $\sigma\iota\sigma = -\mathrm{id}$ provides a ``negation gate''
constructed from three involutions.
\end{corollary}

\section{Orbit Structure of $\langle \sigma, \kappa \rangle$}

While the full group $\langle \sigma, \kappa, \iota \rangle$ acts transitively
on $\mathbb{Q}^+$, the subgroup $\langle \sigma, \kappa \rangle$ has a richer
orbit structure when restricted to $(0,1) \cap \mathbb{Q}$.

\begin{proposition}[Orbit Invariant]\label{prop:invariant}
For $p/q \in (0,1) \cap \mathbb{Q}$ in lowest terms, define
\[
I(p/q) = \mathrm{odd}(p(q-p)),
\]
where $\mathrm{odd}(n) = n / 2^{\nu_2(n)}$ is the odd part of $n$.
Then $I$ is invariant under $\sigma$ and $\kappa$.
\end{proposition}

\begin{proof}
For $\kappa$: $\kappa(p/q) = (q-p)/q$, so
$I(\kappa(p/q)) = \mathrm{odd}((q-p) \cdot p) = I(p/q)$.

For $\sigma$: $\sigma(p/q) = (q-p)/(q+p)$, so
$I(\sigma(p/q)) = \mathrm{odd}((q-p) \cdot 2p) = \mathrm{odd}(2p(q-p)) = \mathrm{odd}(p(q-p)) = I(p/q)$.
\end{proof}

\begin{theorem}[Orbit Completeness]\label{thm:completeness}
$I$ completely characterizes orbits: two fractions in $(0,1)$ are in the same
$\langle \sigma, \kappa \rangle$-orbit if and only if they have the same $I$ value.
\end{theorem}

\begin{proof}
We have already shown (Proposition~\ref{prop:invariant}) that $I$ is preserved
by $\sigma$ and $\kappa$. It remains to show that fractions with the same $I$
value are connected.

Consider the \emph{logit coordinate} $y = x/(1-x)$ for $x \in (0,1)$.
In these coordinates:
\[
\sigma\kappa\colon y \mapsto y/2, \qquad
\kappa\sigma\colon y \mapsto 2y, \qquad
\kappa\colon y \mapsto 1/y.
\]
For $p/q$ in lowest terms, we have $y = p/(q-p)$, and $I(p/q) = \mathrm{odd}(p(q-p))$.

\textbf{Key observation:} For any $p/q$ with $I = k$ (odd), there exists a unique
\emph{canonical representative} of the form $m/(k+m)$ where $\gcd(m, k+m) = 1$
and $m$ is a power of~$2$. Specifically, apply $\sigma\kappa$ (which halves $y$)
or $\kappa\sigma$ (which doubles $y$) until the numerator of $y = p/(q-p)$
becomes a power of~$2$.

\textbf{Algorithm to connect $p_1/q_1$ and $p_2/q_2$ with $I(p_1/q_1) = I(p_2/q_2) = k$:}
\begin{enumerate}
\item Reduce $p_1/q_1$ to canonical form $2^{a_1}/(k + 2^{a_1})$ via
      $(\sigma\kappa)^{a_1}$ or $(\kappa\sigma)^{a_1}$.
\item Reduce $p_2/q_2$ to canonical form $2^{a_2}/(k + 2^{a_2})$ via
      $(\sigma\kappa)^{a_2}$ or $(\kappa\sigma)^{a_2}$.
\item Connect the canonical forms: if $a_1 < a_2$, apply $(\kappa\sigma)^{a_2 - a_1}$.
\end{enumerate}

Since every fraction with invariant $k$ reduces to a canonical form, and all
canonical forms for the same $k$ are connected by powers of $\sigma\kappa$,
the orbit is exactly the set of fractions with invariant~$k$.
\end{proof}

\begin{example}
Fractions with $I = 1$: $\{1/2, 1/3, 2/3, 1/5, 2/5, 3/5, 4/5, 1/9, \ldots\}$.
These form a single orbit under $\langle \sigma, \kappa \rangle$.

The canonical representative for orbit $I = k$ is $1/(k+1)$ when $k+1$ is a valid denominator.
\end{example}

\section{Path Length Analysis}

For fractions within the same $I$-orbit, the involution path can be significantly
shorter than the continued fraction representation.

\begin{proposition}
For $k \geq 1$, the distance from $1/2$ to $1/(2^k+1)$ under $\langle \sigma, \kappa \rangle$
is exactly $2k-1$, achieved by the path $\sigma(\kappa\sigma)^{k-1}$.
\end{proposition}

\begin{proof}
By induction. Base case ($k=1$): $\sigma(1/2) = (1-1/2)/(1+1/2) = 1/3 = 1/(2^1+1)$.

Inductive step: Assume $(\sigma\kappa)^{k-1}\sigma(1/2) = 1/(2^k+1)$.
Then $\kappa(1/(2^k+1)) = 2^k/(2^k+1)$, and
\[
\sigma\left(\frac{2^k}{2^k+1}\right) = \frac{1 - \frac{2^k}{2^k+1}}{1 + \frac{2^k}{2^k+1}}
= \frac{1}{2^{k+1}+1}.
\]
Hence $\sigma(\kappa\sigma)^k(1/2) = 1/(2^{k+1}+1)$, with path length $2(k+1)-1 = 2k+1$.
\end{proof}

\begin{center}
\begin{tabular}{ccccc}
\toprule
$k$ & Target & Path & Length & $|$CF$|$ \\
\midrule
1 & $1/3$ & $\sigma$ & 1 & 3 \\
2 & $1/5$ & $\sigma\kappa\sigma$ & 3 & 5 \\
3 & $1/9$ & $\sigma\kappa\sigma\kappa\sigma$ & 5 & 9 \\
4 & $1/17$ & $\sigma(\kappa\sigma)^3$ & 7 & 17 \\
\bottomrule
\end{tabular}
\end{center}

For this family, path length is $\Theta(\log q)$ while CF sum equals $q$.
This demonstrates that involutions can provide shortcuts for fractions with
large continued fraction coefficients.

\begin{remark}[Generic Path Length]
For \emph{typical} rationals, no systematic speedup is guaranteed. By Khinchin's
theorem \cite{Khinchin1964}, the geometric mean of CF coefficients converges to
$K_0 \approx 2.685$ for almost all reals, so a random $p/q$ has CF sum $\approx c \log q$
(with $c$ depending on the distribution). In this generic regime, the involution path
length is also $O(\log q)$---neither representation is asymptotically superior.

The $1/(2^k+1)$ family achieves exponential speedup precisely because its CF representation
$[0; 1, 2^k]$ has a single large coefficient. Similarly, fractions near quadratic surds
(with eventually periodic CFs) may benefit from involution shortcuts. The involution
approach thus offers advantages for \emph{structured} fractions, not random ones.
\end{remark}

\section{Continued Fractions and the Euclidean Algorithm}

The well-known correspondence between continued fractions and the Euclidean
algorithm acquires a new interpretation through involutions: each GCD step
corresponds to an application of~$\iota$.

\begin{proposition}[CF-Involution Correspondence]
For $p/q \in (0,1) \cap \mathbb{Q}$ with continued fraction $[0; a_1, a_2, \ldots, a_n]$,
the involution representation is:
\[
\frac{p}{q} = \iota R^{a_1} \iota R^{a_2} \cdots \iota R^{a_n}(0),
\]
where $R(x) = 1+x$ and $\iota(x) = 1/x$.
\end{proposition}

\begin{proof}
The continued fraction $[0; a_1, \ldots, a_n]$ satisfies the recurrence
\[
\frac{p}{q} = \frac{1}{a_1 + \frac{1}{a_2 + \cdots}}.
\]
Starting from $a_n$ and working backwards: $R^{a_n}(0) = a_n$,
then $\iota R^{a_n}(0) = 1/a_n$, then $R^{a_{n-1}} \iota R^{a_n}(0) = a_{n-1} + 1/a_n$,
and so on. The final $\iota$ converts the result to a proper fraction.
\end{proof}

\begin{example}
For $7/11 = [0; 1, 1, 1, 3]$, the Euclidean algorithm and involution build-up
run in opposite directions:

\begin{center}
\small
\begin{tabular}{clclc}
\toprule
& GCD (top-down) & $a_i$ & Build-up (bottom-up) & Result \\
\midrule
& & & $\iota R \iota R \iota R \iota R^3(0)$ & $7/11$ \\
\midrule
1 & $11 = 1 \cdot 7 + 4$ & 1 & $R \iota R \iota R \iota R^3(0)$ & $11/7$ \\
2 & $7 = 1 \cdot 4 + 3$ & 1 & $R \iota R \iota R^3(0)$ & $7/4$ \\
3 & $4 = 1 \cdot 3 + 1$ & 1 & $R \iota R^3(0)$ & $4/3$ \\
4 & $3 = 3 \cdot 1 + 0$ & 3 & $R^3(0)$ & $3$ \\
\bottomrule
\end{tabular}
\end{center}

The GCD quotients read top-to-bottom give $[1,1,1,3]$; the build-up reads
them bottom-to-top as $[3,1,1,1]$, with each row's result being the ratio
from the corresponding GCD step. The final $\iota$ inverts row~1 to produce $7/11 < 1$.

Note: The CF build-up uses only $R$ and $\iota$, regardless of whether the
target is in $(0,1)$ or greater than~$1$. The generator $L$ appears when
traversing the Calkin--Wilf tree to a specific rational (Theorem~\ref{thm:main}).
\end{example}

Since $R = \kappa\sigma\iota\sigma$ decomposes into 4 atomic involutions,
the total atomic involution count for $p/q$ is $4 \sum a_i + n$,
where $n$ is the CF length.

\section{Connection to Stern--Brocot Tree}

The Calkin--Wilf tree is closely related to the Stern--Brocot tree \cite{GrahamKnuthPatashnik1994}.
Both enumerate $\mathbb{Q}^+$ using binary tree structures. Wildberger \cite{Wildberger2010}
studied $L$-$R$ factorizations in the context of Pell equations and the Stern--Brocot tree.

Our contribution is the observation that $L$ and $R$ themselves factor into
elementary involutions, providing a finer decomposition than the $L$-$R$ level.

\begin{remark}[Convention Summary]
Several $L$-$R$ conventions appear in the literature. For clarity:
\begin{center}
\begin{tabular}{llll}
\toprule
System & $L(x)$ & $R(x)$ & Matrix $L$ \\
\midrule
Calkin--Wilf (this paper) & $x/(1+x)$ & $1+x$ & $\begin{pmatrix} 1 & 0 \\ 1 & 1 \end{pmatrix}$ \\[2mm]
Stern--Brocot & $x/(1+x)$ & $(1+x)/x$ & $\begin{pmatrix} 1 & 0 \\ 1 & 1 \end{pmatrix}$ \\[2mm]
Wildberger (Pell) & swapped & swapped & $\begin{pmatrix} 1 & 1 \\ 0 & 1 \end{pmatrix}$ \\
\bottomrule
\end{tabular}
\end{center}
Wildberger's $L'_{\mathrm{W}}$ corresponds to Stern--Brocot $R_{\mathrm{SB}}$
(and vice versa) when acting on quadratic forms.
All systems share the same underlying group $\mathrm{SL}(2,\mathbb{Z})$;
differences are purely notational.
\end{remark}

\begin{remark}[Algebraic Structure]
A key algebraic relation connects the generators:
\begin{equation}\label{eq:LR-relation}
L = \kappa\iota \cdot R,
\end{equation}
where $\kappa\iota(x) = 1 - 1/x = (x-1)/x$. This follows from
$L = \kappa\iota\kappa\sigma\iota\sigma = \kappa\iota \cdot (\kappa\sigma\iota\sigma) = \kappa\iota \cdot R$.
Thus $L$ is $R$ ``wrapped'' in the involution pair $\kappa\iota$.
\end{remark}

\begin{remark}[Connection to Quadratic Forms]
The CW generators correspond to $\mathrm{SL}(2,\mathbb{Z})$ matrices
(up to projective equivalence):
\[
L \sim \begin{pmatrix} 1 & 0 \\ 1 & 1 \end{pmatrix}, \quad
R \sim \begin{pmatrix} 1 & 1 \\ 0 & 1 \end{pmatrix}.
\]
Wildberger's algorithm for the Brahmagupta--Bh\=askara equation $x^2 - dy^2 = 1$
uses $L'$-$R'$ operations on binary quadratic forms $(a,b,c)$.
The induced $3 \times 3$ matrices satisfy
$L'_{\text{W}} = \rho(R_{\text{SB}})$ and $R'_{\text{W}} = \rho(L_{\text{SB}})$,
where $\rho\colon \mathrm{SL}(2,\mathbb{Z}) \to \mathrm{GL}(3,\mathbb{Z})$
is the natural representation on quadratic forms.
The $L \leftrightarrow R$ swap arises from differing conventions.

By Lagrange's theorem, the continued fraction of $\sqrt{d}$ has palindromic period,
which manifests as palindromic Wildberger paths (e.g., $LRRL$ for $\sqrt{2}$).
Using \eqref{eq:LR-relation}, each such path becomes an alternating sequence
of $R$ and $\kappa\iota R$ operations, exposing the involutive structure
underlying quadratic irrational approximation.
\end{remark}

\begin{remark}[Binary Encoding and Powers of Two]
The position $n$ in level-order traversal of the Calkin--Wilf tree encodes
the path from root: dropping the leading 1 from $n$'s binary representation,
each 0 corresponds to $L$ and each 1 to $R$. In particular:
\begin{itemize}
\item Position $2^k$ (binary $10\cdots0$) gives $L^k(1) = 1/(k+1)$
\item Position $2^k-1$ (binary $11\cdots1$) gives $R^{k-1}(1) = k$
\end{itemize}
Thus the ``left spine'' of the tree produces unit fractions $1/2, 1/3, 1/4, \ldots$
via $L^k = (\kappa\iota\kappa\sigma\iota\sigma)^k$, requiring $6k$ atomic involutions.
The ``right spine'' produces positive integers $2, 3, 4, \ldots$ via
$R^k = (\kappa\sigma\iota\sigma)^k$, requiring $4k$ atomic involutions.
\end{remark}

\section{Group Structure and Word Problem}

The group $\langle \sigma, \kappa, \iota \rangle$ generated by our three involutions
has a natural group-theoretic characterization.

\begin{proposition}[Coxeter Structure]\label{prop:coxeter}
The group $\langle \sigma, \kappa, \iota \rangle$ is isomorphic to the universal
Coxeter group on three generators:
\[
W_\infty^{(3)} = \langle \sigma, \kappa, \iota \mid \sigma^2 = \kappa^2 = \iota^2 = 1 \rangle,
\]
i.e., the free product $(\mathbb{Z}/2) \ast (\mathbb{Z}/2) \ast (\mathbb{Z}/2)$.
All products of distinct generators have infinite order.
\end{proposition}

\begin{proof}
That $\sigma^2 = \kappa^2 = \iota^2 = 1$ follows from direct computation.
For infinite order, consider the matrix representation in $\mathrm{PGL}(2, \mathbb{Q})$:
\[
\sigma\kappa \sim \begin{pmatrix} 1 & 0 \\ -1 & 2 \end{pmatrix}, \quad
(\sigma\kappa)^n \sim \begin{pmatrix} 1 & 0 \\ 1-2^n & 2^n \end{pmatrix}.
\]
Since $\sigma\kappa$ has eigenvalues $\{1, 2\}$ (not roots of unity), it has infinite order.
Similarly, $\kappa\iota$ and $\sigma\iota$ have infinite order (their matrices have
eigenvalues including $\pm\sqrt{2}$ and the golden ratio, respectively).

\textbf{Completeness of presentation.} To verify that no additional relations exist,
we observe that the representation $\rho\colon \langle \sigma, \kappa, \iota \rangle \to \mathrm{PGL}(2, \mathbb{Q})$
is faithful. The matrices
\[
M_\sigma = \begin{pmatrix} -1 & 1 \\ 1 & 1 \end{pmatrix}, \quad
M_\kappa = \begin{pmatrix} -1 & 1 \\ 0 & 1 \end{pmatrix}, \quad
M_\iota = \begin{pmatrix} 0 & 1 \\ 1 & 0 \end{pmatrix}
\]
generate a subgroup of $\mathrm{GL}(2, \mathbb{Z})$. The only relations among these
are the involution relations: any additional relation $w = 1$ in the abstract group
would imply $\rho(w) = I$, but the infinite order of all non-trivial products
precludes such relations. Hence the presentation is complete.
\end{proof}

\begin{remark}[Coxeter--Dynkin Type]
In Coxeter notation, $W_\infty^{(3)}$ corresponds to the diagram with three nodes
and all edges labeled $\infty$ (i.e., all $m_{ij} = \infty$ for $i \neq j$).
This is \emph{not} the affine type $\tilde{A}_2$ (which would have $m_{ij} = 3$),
but rather the fully infinite case where no finite-order products exist.
\end{remark}

\begin{corollary}[Word Problem]
The word problem for $\langle \sigma, \kappa, \iota \rangle$ is solvable in
linear time $O(n)$, where $n$ is the word length.
\end{corollary}

\begin{proof}
Since the only relations are $\sigma^2 = \kappa^2 = \iota^2 = 1$, the normal
form of any word is obtained by removing adjacent identical letters. This
can be done in a single left-to-right pass: maintain a stack, and for each
new letter, either push it (if different from top) or pop (if same as top).
The resulting word has no adjacent identical letters.
\end{proof}

\begin{remark}[Chomsky Classification]
The reduction rules $\sigma\sigma \to \varepsilon$, $\kappa\kappa \to \varepsilon$,
$\iota\iota \to \varepsilon$ form a context-free grammar. Unlike typical CFGs
(which require $O(n^3)$ parsing via CYK), our specific grammar admits linear-time
parsing because the rules are purely local (adjacent pairs only).
\end{remark}

The Calkin--Wilf generators have fixed-length representations:
\begin{itemize}
\item $R = \kappa\sigma\iota\sigma$ (4 symbols)
\item $L = \kappa\iota\kappa\sigma\iota\sigma$ (6 symbols)
\end{itemize}
Both are already in normal form (alternating letters). Hence the atomic
representation of any CW path has length $4 n_R + 6 n_L$, where $n_R$ and $n_L$
are the numbers of $R$ and $L$ steps respectively.

\section{Discussion}

The decomposition of Calkin--Wilf generators into elementary involutions
provides a finer-grained view of rational number arithmetic, with
$\{\sigma, \kappa, \iota\}$ serving as atomic building blocks.

\subsection*{Significance of the Decomposition}

Why decompose $L$ and $R$ into involutions? Several perspectives illuminate
the value of this viewpoint:

\begin{enumerate}
\item \textbf{Geometric interpretation.} The three involutions have natural
geometric meanings: $\iota$ is inversion about~$1$ (the point swapping
$(0,\infty)$), $\kappa$ is reflection about~$1/2$ (the midpoint of $(0,1)$),
and $\sigma$ is the real Cayley transform, conjugating hyperbolic/trigonometric
functions to exponentials (cf.\ Open Question~6). The Calkin--Wilf generators
inherit these geometric properties as compositions.

\item \textbf{Coxeter structure.} The group $\langle \sigma, \kappa, \iota \rangle$
is an infinite Coxeter group (Proposition~\ref{prop:coxeter}), generated by reflections with all
products of distinct generators having infinite order. This places the rational
enumeration problem in the broader context of Coxeter-theoretic combinatorics,
where word-length and normal forms have rich theory.

\item \textbf{Minimal coefficients.} The involutions $\sigma$, $\kappa$, $\iota$
use only coefficients from $\{-1, 0, 1\}$. This is computationally significant:
multiplication and division in the group action reduce to additions and sign changes,
enabling efficient symbolic and numerical implementation.

\item \textbf{Orbit invariants.} The subgroup $\langle \sigma, \kappa \rangle$
admits a non-trivial orbit invariant $I(p/q) = \mathrm{odd}(p(q-p))$, partitioning
$(0,1) \cap \mathbb{Q}$ into infinitely many orbits. The full group's transitivity
arises precisely because $\iota$ connects these orbits.
\end{enumerate}

\textbf{Open questions:}
\begin{enumerate}[label=(\arabic*)]
\item The group $\langle \sigma, \kappa, \iota \rangle$ is an infinite
Coxeter group. What is its relationship to other well-known Coxeter groups?
\item Can the invariant $I$ be generalized to characterize orbits of other
involution subgroups?
\item For fractions within the same $I$-orbit, can the $\langle \sigma, \kappa \rangle$
path be computed directly, without enumerating the orbit?
\item The involution $\sigma$ appears to simplify continued fraction representations.
In a companion paper~\cite{PopelkaEgypt2025}, we establish a bijection between
CF coefficients and Egyptian fraction decompositions. Does the action of
$\langle \sigma, \kappa \rangle$ on rationals have a natural interpretation
in terms of this CF--Egypt correspondence?
\item The involution $\sigma$ extends naturally to irrational numbers.
Preliminary computations suggest remarkable behavior on transcendental constants:
\begin{itemize}
\item $\sigma$ fixes $\sqrt{2}-1$ (whose CF is $[0; \overline{2}]$).
\item $\sigma$ maps $1/\varphi = (\sqrt{5}-1)/2$ (CF $[0; \overline{1}]$)
      to a number with CF $[0; \overline{4}]$.
\item $\sigma(e-2) = (3-e)/(e-1) = \coth(1/2) - 2$ has CF $[0; 6, 10, 14, \ldots]$.
\end{itemize}
The last identity follows from $\coth(1/2) = (e+1)/(e-1) = [2; 6, 10, 14, \ldots]$
\cite{OEIS-A016825}. Since $e$ is transcendental \cite{Hermite1873},
so is $\sigma(e-2)$. The arithmetic CF arises because
$\sigma$ transforms the irregular CF of $e-2$ into the fractional part of $\coth(1/2)$.
This provides a natural interpretation: $\sigma$ ``regularizes'' the Euler pattern
$[0; 1, 2, 1, 1, 4, \ldots]$ via M\"obius conjugation to hyperbolic functions.
\item The identities $\sigma(\tanh t) = e^{-2t}$ and $\sigma(i\tan\theta) = e^{-2i\theta}$
suggest a deep connection to Fourier/Hartley analysis. Since $\sigma(\tan\theta) = \tan(\pi/4 - \theta)$,
the involution acts as \emph{angle reflection} around $\pi/8$, where its fixed point
$\sqrt{2}-1 = \tan(\pi/8)$ lies. The Hartley transform uses the kernel
$\mathrm{cas}(\theta) = \cos\theta + \sin\theta = \sqrt{2}\cos(\theta - \pi/4)$,
for which $\sigma$-reflection is natural. Can this connection be exploited algorithmically?
\end{enumerate}

\section*{Acknowledgments}

This work was developed in collaboration with Claude (Anthropic).

\begin{thebibliography}{9}

\bibitem{CalkinWilf2000}
N.~Calkin and H.~S.~Wilf,
``Recounting the Rationals,''
\emph{American Mathematical Monthly}, vol.~107, no.~4, pp.~360--363, 2000.

\bibitem{GrahamKnuthPatashnik1994}
R.~L.~Graham, D.~E.~Knuth, and O.~Patashnik,
\emph{Concrete Mathematics}, 2nd ed.
Addison-Wesley, 1994.
(Stern--Brocot tree, Section 4.5)

\bibitem{Khinchin1964}
A.~Ya.~Khinchin,
\emph{Continued Fractions}.
University of Chicago Press, 1964.

\bibitem{Wildberger2010}
N.~J.~Wildberger,
``Solving the Pell Equation,''
\emph{Journal of Integer Sequences}, vol.~13, Article 10.4.3, 2010.

\bibitem{PopelkaEgypt2025}
J.~Popelka,
``The Continued Fractions--Egyptian Fractions Bijection: Symbolic Telescoping Representation,''
preprint, 2025.

\bibitem{Hermite1873}
C.~Hermite,
``Sur la fonction exponentielle,''
\emph{Comptes Rendus de l'Acad\'emie des Sciences}, vol.~77, pp.~18--24, 74--79, 226--233, 285--293, 1873.

\bibitem{OEIS-A016825}
OEIS Foundation Inc.,
``Sequence A016825: Positive integers congruent to 2 mod 4,''
\emph{The On-Line Encyclopedia of Integer Sequences}, 2024.
\url{https://oeis.org/A016825}

\end{thebibliography}

\end{document}
