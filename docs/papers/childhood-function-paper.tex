% The M(n) Childhood Function and Its Dirichlet Series:
% From Primal Forest Geometry to Analytic Number Theory
%
% Authors: Jan Popelka & Claude (Anthropic)
% Date: November 2025
% Status: DRAFT - Work in Progress

\documentclass[11pt,a4paper]{article}

\usepackage{amsmath,amssymb,amsthm}
\usepackage{mathtools}
\usepackage{hyperref}
\usepackage{graphicx}
\usepackage{enumitem}

% Theorem environments
\newtheorem{theorem}{Theorem}[section]
\newtheorem{lemma}[theorem]{Lemma}
\newtheorem{proposition}[theorem]{Proposition}
\newtheorem{corollary}[theorem]{Corollary}

\theoremstyle{definition}
\newtheorem{definition}[theorem]{Definition}
\newtheorem{example}[theorem]{Example}
\newtheorem{remark}[theorem]{Remark}

\theoremstyle{remark}
\newtheorem*{note}{Note}

% Custom commands
\newcommand{\N}{\mathbb{N}}
\newcommand{\Z}{\mathbb{Z}}
\newcommand{\Q}{\mathbb{Q}}
\newcommand{\R}{\mathbb{R}}
\newcommand{\C}{\mathbb{C}}

\newcommand{\eps}{\varepsilon}
\newcommand{\Om}{\Omega}

% Paper-specific notation
\newcommand{\Mn}{M(n)}
\newcommand{\taun}{\tau(n)}
\newcommand{\LM}{L_M}
\newcommand{\Fn}{F_n}

\title{The $M(n)$ Childhood Function and Its Dirichlet Series:\\
From Primal Forest Geometry to Analytic Number Theory}

\author{
Jan Popelka\thanks{Independent researcher. Email: [to be added]}
\and
Claude (Anthropic)\thanks{AI research assistant. This work represents a human-AI collaboration under the Trinity framework.}
}

\date{November 2025\\[1em]
\textit{Draft Version --- Comments Welcome}}

\begin{document}

\maketitle

\begin{abstract}
We introduce $M(n)$, the \emph{childhood function}, which counts divisors of $n$ strictly between $2$ and $\sqrt{n}$. Starting from a geometric visualization called the \emph{primal forest}, we develop an epsilon-pole regularization framework that connects local factorization structure to global arithmetic distribution. We establish that $M(n) = \lfloor (\tau(n)-1)/2 \rfloor$ and derive a closed-form expression for its Dirichlet series $\LM(s) = \sum_{n=1}^\infty M(n)/n^s$. The function $\LM(s)$ exhibits a double pole at $s=1$ with residue $2\gamma-1$, where $\gamma$ is the Euler-Mascheroni constant. The $\sqrt{n}$ boundary emerges as a fundamental scale across geometry, asymptotics, and complex analysis. We conclude with open questions, including a ``Mellin puzzle'' regarding the discrepancy between summatory function constants and Laurent expansion residues.

\medskip
\noindent\textbf{Keywords:} divisor function, Dirichlet series, analytic number theory, regularization, childhood function

\medskip
\noindent\textbf{MSC 2020:} 11N37 (Asymptotic results on arithmetic functions), 11M41 (Other Dirichlet series and zeta functions)

\medskip
\noindent\textbf{Epistemic Status:} Many results are \emph{numerically verified} but not rigorously proven. We clearly mark the status of each claim.
\end{abstract}

\tableofcontents

\section{Introduction}

\subsection{Motivation: The Primal Forest}

The genesis of this work lies in a geometric visualization we call the \emph{primal forest}. For each positive integer $n$, consider the lattice of points $(d, k) \in \N^2$ satisfying $kd + d^2 \leq n$. Each such point represents a potential factorization structure: if $n = kd + d^2$, then $n$ is composite with divisor $d$. Points failing this equation lie at a ``distance'' from exact factorization.

This geometric perspective naturally partitions divisors by the $\sqrt{n}$ boundary:
\begin{itemize}[nosep]
\item Divisors $d \leq \sqrt{n}$ appear explicitly in the lattice
\item Divisors $d > \sqrt{n}$ appear implicitly (via complementary factors $n/d < \sqrt{n}$)
\end{itemize}

The asymmetry around $\sqrt{n}$ proves fundamental to our analysis.

\subsection{The Childhood Function}

\begin{definition}[Childhood Function]
\label{def:childhood}
For $n \in \N$, define the \emph{childhood function} $M(n)$ by
\[
M(n) := \#\{d \in \N : d \mid n,\, 2 \leq d \leq \sqrt{n}\}.
\]
We interpret $M(n)$ as counting the ``childhood divisors'' --- those in the range where factorization structure is most directly visible.
\end{definition}

\begin{example}
\label{ex:basic}
\begin{align*}
M(1) &= 0 \quad \text{(no divisors in range)} \\
M(12) &= 2 \quad \text{(divisors: } 2, 3 \text{)} \\
M(60) &= 5 \quad \text{(divisors: } 2, 3, 4, 5, 6 \text{)} \\
M(p) &= 0 \quad \text{for all primes } p
\end{align*}
\end{example}

\subsection{Connection to Classical Invariants}

The function $M(n)$ is intimately related to the classical divisor function $\tau(n) = \#\{d : d \mid n\}$.

\begin{proposition}[Relation to $\tau(n)$]
\label{prop:tau-relation}
For all $n \geq 1$,
\[
M(n) = \left\lfloor \frac{\tau(n) - 1}{2} \right\rfloor.
\]
\end{proposition}

\begin{proof}
% TODO: Fill in rigorous proof
(Proof to be completed in next revision.)

Sketch: The divisors of $n$ partition into three disjoint sets:
\begin{enumerate}
\item $d = 1$ (always)
\item $2 \leq d \leq \sqrt{n}$ (counted by $M(n)$)
\item $\sqrt{n} < d \leq n$ (complementary divisors $n/d$ from set 2)
\end{enumerate}
When $n$ is a perfect square, $d = \sqrt{n}$ is counted once. Otherwise, sets 2 and 3 have equal cardinality. The floor function accounts for this parity.
\end{proof}

\begin{corollary}
\label{cor:non-multiplicative}
The function $M(n)$ is \emph{not multiplicative}. That is, $M(mn) \neq M(m)M(n)$ in general for coprime $m, n$.
\end{corollary}

\begin{proof}
Immediate from the floor function in Proposition \ref{prop:tau-relation}, since rounding destroys multiplicativity even though $\tau(n)$ is multiplicative.
\end{proof}

\subsection{Historical Context}

The study of arithmetic functions and their Dirichlet series is classical. The divisor function $\tau(n)$ and its summatory behavior
\[
\sum_{n \leq x} \tau(n) = x \log x + (2\gamma - 1)x + O(\sqrt{x})
\]
is the famous \emph{Dirichlet divisor problem}. Our function $M(n)$, as roughly half of $\tau(n)$, inherits some of this structure but with critical modifications due to the $\sqrt{n}$ cutoff.

\subsection{Outline}

The paper is organized as follows:
\begin{itemize}[nosep]
\item Section \ref{sec:properties}: Basic properties and distribution of $M(n)$
\item Section \ref{sec:epsilon-pole}: Epsilon-pole regularization framework
\item Section \ref{sec:dirichlet}: Dirichlet series $\LM(s)$ and closed form
\item Section \ref{sec:analytic}: Analytic properties and Laurent expansion
\item Section \ref{sec:asymptotics}: Asymptotic analysis and summatory function
\item Section \ref{sec:sqrt-n}: The $\sqrt{n}$ universality phenomenon
\item Section \ref{sec:bridge}: Bridging local and global perspectives
\item Section \ref{sec:open}: Open questions and conjectures
\end{itemize}

Throughout, we clearly distinguish between \emph{proven} results and those that are \emph{numerically verified} but lack rigorous proof.

\section{Basic Properties and Distribution}
\label{sec:properties}

\subsection{Elementary Bounds}

\begin{proposition}[Bounds on $M(n)$]
\label{prop:bounds}
For all $n \geq 1$,
\[
0 \leq M(n) \leq \frac{\tau(n) - 1}{2}.
\]
Furthermore:
\begin{enumerate}[label=(\alph*),nosep]
\item $M(n) = 0$ if and only if $n \in \{1\} \cup \{\text{primes}\}$
\item $M(n) \geq 1$ if and only if $n$ is composite
\item $M(n^2) \geq M(n)$ for all $n$ (squares increase childhood)
\end{enumerate}
\end{proposition}

\begin{proof}
(a) and (b) are immediate from Definition \ref{def:childhood}. For (c), observe that if $d \mid n$ with $2 \leq d \leq \sqrt{n}$, then $d^2 \mid n^2$ and $d^2 \leq n < \sqrt{n^2}$, so divisors of $n$ contribute to divisors of $n^2$.
\end{proof}

\subsection{Distribution Statistics}

% TODO: Add theorem on distribution characteristics
% Based on computational data (n <= 10000):
% - Most common: M(n) = 1 (26.33%)
% - M(n) = 0 for primes (12.30%)
% - Highly skewed toward small values

\begin{remark}[Numerical Observations]
Computational experiments for $n \leq 10{,}000$ reveal:
\begin{itemize}[nosep]
\item $M(n) = 1$ is most frequent (26.33\% of values)
\item $M(n) = 0$ for 12.30\% of values (primes and $n=1$)
\item Maximum observed: $M(7560) = 31$
\item Mean: $\langle M(n) \rangle \approx 3.69$
\item Correlation with $\tau(n)$: Pearson $r = 0.9999$
\end{itemize}
These observations guide our asymptotic analysis in Section \ref{sec:asymptotics}.
\end{remark}

\subsection{Average Order}

The following is central to our asymptotic understanding:

\begin{theorem}[Average Order --- Numerical]
\label{thm:average-order}
For large $n$,
\[
M(n) \sim \frac{\log n}{2}.
\]
More precisely, the arithmetic mean over $[1, N]$ satisfies
\[
\frac{1}{N} \sum_{n=1}^N M(n) \sim \frac{\log N}{2}.
\]
\end{theorem}

\begin{proof}[Proof (sketch)]
From Proposition \ref{prop:tau-relation}, $M(n) \approx \tau(n)/2$ for large $n$. The average order of $\tau(n)$ is $\log n$, hence $M(n) \sim \log(n)/2$ on average.

\textbf{Status:} Numerically verified for $n \leq 10^4$. Rigorous proof requires careful analysis of the floor function and error terms.
\end{proof}

\section{Epsilon-Pole Regularization Framework}
\label{sec:epsilon-pole}

% TODO: This section needs significant expansion
% Core idea: F_n(alpha, epsilon) with power-law regularization
% Residue theorem: lim_{eps->0} eps^alpha * F_n = M(n)

\subsection{The Primal Forest Function}

Motivated by the geometric visualization, we introduce a regularized function encoding factorization structure.

\begin{definition}[Primal Forest Function]
\label{def:primal-forest}
For $n \in \N$, $\alpha > 0$, and $\eps > 0$, define
\[
F_n(\alpha, \eps) := \sum_{\substack{d \geq 2 \\ k \geq 0}} \left[(n - kd - d^2)^2 + \eps\right]^{-\alpha}.
\]
The sum ranges over all $(d, k)$ pairs in the primal forest lattice.
\end{definition}

\begin{remark}
When $n = kd + d^2$ exactly (exact factorization), the term $(n - kd - d^2)^2 = 0$, creating a pole as $\eps \to 0$. The residue at this pole measures compositeness.
\end{remark}

\subsection{Dominant Term Approximation}

% TODO: Derive this rigorously

For computational efficiency, we use:
\[
F_n^{\text{dom}}(\alpha, \eps) := \sum_{d=2}^{\lfloor\sqrt{n}\rfloor} \left[r_d^2 + \eps\right]^{-\alpha}
+ \sum_{d > \sqrt{n}}^{d_{\max}} \left[(d^2 - n)^2 + \eps\right]^{-\alpha},
\]
where $r_d = (n - d^2) \bmod d$ and $d_{\max}$ is chosen for convergence.

\subsection{Residue Theorem}

The central result connecting regularization to $M(n)$:

\begin{theorem}[Residue Theorem --- Numerical]
\label{thm:residue}
For all $n \geq 1$ and $\alpha > 0$,
\[
\lim_{\eps \to 0^+} \eps^\alpha \cdot F_n(\alpha, \eps) = M(n).
\]
\end{theorem}

\begin{proof}[Proof (sketch)]
Each exact factorization $n = kd + d^2$ with $2 \leq d \leq \sqrt{n}$ contributes a pole $\eps^{-\alpha}$ to $F_n$. Multiplying by $\eps^\alpha$ extracts the residue (count = 1). Summing over all such divisors yields $M(n)$.

\textbf{Status:} Numerically verified to $< 0.2\%$ error for individual $n$ up to $n = 1000$.
\end{proof}

\subsection{Non-Uniform Convergence}

% TODO: Explain the n^{-1/(2alpha)} scaling

\begin{remark}[Convergence Rate]
The limit in Theorem \ref{thm:residue} is \emph{non-uniform} in $n$. Numerical experiments suggest that accurate residue extraction requires
\[
\eps \ll n^{-1/(2\alpha)}.
\]
For $\alpha = 3$, this gives $\eps \ll n^{-1/6} \approx 1/\sqrt[6]{n}$, again reflecting the $\sqrt{n}$ boundary.
\end{remark}

\section{The Dirichlet Series $\LM(s)$}
\label{sec:dirichlet}

\subsection{Definition and Convergence}

\begin{definition}[Dirichlet Series]
\label{def:LM}
The Dirichlet series associated to $M(n)$ is
\[
\LM(s) := \sum_{n=1}^\infty \frac{M(n)}{n^s}.
\]
\end{definition}

\begin{proposition}[Convergence Region]
\label{prop:convergence}
The series $\LM(s)$ converges absolutely for $\Re(s) > 1$.
\end{proposition}

\begin{proof}
Since $M(n) = O(\tau(n)) = O(n^\eps)$ for any $\eps > 0$, the series behaves like $\sum n^{-s + \eps}$, which converges for $\Re(s) > 1$.
\end{proof}

\subsection{Closed-Form Expression}

The following is our main analytic result:

\begin{theorem}[Closed Form for $\LM(s)$ --- Numerical]
\label{thm:closed-form}
For $\Re(s) > 1$,
\[
\LM(s) = \zeta(s)[\zeta(s) - 1] - \sum_{j=2}^\infty \frac{H_{j-1}(s)}{j^s},
\]
where $\zeta(s)$ is the Riemann zeta function and
\[
H_k(s) := \sum_{\ell=1}^k \frac{1}{\ell^s}
\]
is the $k$-th partial sum of the zeta series.
\end{theorem}

\begin{proof}[Proof (sketch)]
% TODO: Rigorous derivation from M(n) = floor((tau(n)-1)/2)

Starting from $M(n) = \lfloor(\tau(n) - 1)/2\rfloor$ and using the Dirichlet convolution formula for $\tau(n) = \sum_{d \mid n} 1$, we expand:
\[
\sum_{n=1}^\infty \frac{\tau(n)}{n^s} = \zeta(s)^2.
\]
The correction term arising from the floor function and the $-1$ offset requires careful analysis of partial sums, yielding the $H_{j-1}(s)/j^s$ correction series.

\textbf{Status:} Numerically verified to high precision for multiple values of $s$ with $\Re(s) > 1$. Rigorous proof in progress.
\end{proof}

\begin{remark}
The closed form allows efficient computation of $\LM(s)$ without summing over all $n$. The correction series $\sum_{j=2}^\infty H_{j-1}(s)/j^s$ converges rapidly for $\Re(s) > 1$.
\end{remark}

\section{Analytic Properties}
\label{sec:analytic}

\subsection{Laurent Expansion Near $s=1$}

% TODO: Derive Laurent expansion rigorously

\begin{theorem}[Double Pole at $s=1$ --- Numerical]
\label{thm:laurent}
As $s \to 1^+$,
\[
\LM(s) \sim \frac{A}{(s-1)^2} + \frac{B}{s-1} + C + O(s-1),
\]
where:
\begin{itemize}[nosep]
\item $A = 1$ (double pole coefficient)
\item $B = 2\gamma - 1$ (residue, $\gamma$ = Euler-Mascheroni constant)
\item $C$ = (next-order constant, to be determined)
\end{itemize}
\end{theorem}

\begin{proof}[Proof (sketch)]
From Theorem \ref{thm:closed-form}, near $s=1$:
\[
\LM(s) = \zeta(s)[\zeta(s) - 1] - \text{(correction)}.
\]
Using $\zeta(s) \sim 1/(s-1) + \gamma + O(s-1)$:
\begin{align*}
\zeta(s)^2 &\sim \frac{1}{(s-1)^2} + \frac{2\gamma}{s-1} + O(1), \\
\zeta(s) &\sim \frac{1}{s-1} + \gamma + O(s-1).
\end{align*}
Subtracting and including the correction term yields the stated expansion.

\textbf{Status:} Numerical evidence strong. The residue $B = 2\gamma - 1 \approx 0.1544$ is particularly significant (see Section \ref{sec:sqrt-n}).
\end{proof}

\subsection{Schwarz Symmetry}

\begin{theorem}[Schwarz Symmetry --- Numerical]
\label{thm:schwarz}
For $\Re(s) > 1$,
\[
\LM(\overline{s}) = \overline{\LM(s)}.
\]
\end{theorem}

\begin{proof}
Since $M(n) \in \Z$ (real coefficients), the Dirichlet series satisfies Schwarz reflection principle.

\textbf{Status:} Numerically verified to machine precision ($< 10^{-10}$ error) for grid of points in $\{1.1 \leq \Re(s) \leq 3, |\Im(s)| \leq 30\}$.
\end{proof}

\subsection{Lack of Euler Product}

\begin{proposition}[No Euler Product]
\label{prop:no-euler}
The function $\LM(s)$ does \emph{not} admit an Euler product representation.
\end{proposition}

\begin{proof}
Immediate from Corollary \ref{cor:non-multiplicative}: $M(n)$ is not multiplicative, hence $\LM(s)$ cannot be written as a product over primes.
\end{proof}

This distinguishes $\LM(s)$ from classical L-functions and makes it more complex analytically.

\section{Asymptotic Analysis}
\label{sec:asymptotics}

\subsection{Summatory Function}

Define the summatory function:
\[
S(x) := \sum_{n \leq x} M(n).
\]

\begin{theorem}[Summatory Asymptotics --- Numerical]
\label{thm:summatory}
As $x \to \infty$,
\[
S(x) = \frac{x \log x}{2} + (\gamma - 1) x + O(\sqrt{x}).
\]
\end{theorem}

\begin{proof}[Proof (sketch)]
From $M(n) = \lfloor(\tau(n) - 1)/2\rfloor$:
\begin{align*}
\sum_{n \leq x} M(n) &\approx \frac{1}{2} \left[\sum_{n \leq x} \tau(n) - x\right] \\
&\sim \frac{1}{2} \left[x \log x + (2\gamma - 1)x - x\right] + O(\sqrt{x}) \\
&= \frac{x \log x}{2} + (\gamma - 1) x + O(\sqrt{x}).
\end{align*}

\textbf{Status:} Numerically consistent for $x \leq 10^4$. Error decreases from 22\% at $x=100$ to 12\% at $x=10{,}000$, suggesting $O(\sqrt{x})$ term is significant.
\end{proof}

\subsection{The Mellin Puzzle}

\begin{remark}[Mellin Puzzle]
\label{rem:mellin-puzzle}
Theorem \ref{thm:summatory} gives a constant $(\gamma - 1)$ in the summatory function, while Theorem \ref{thm:laurent} gives a residue $(2\gamma - 1)$ in the Laurent expansion. These \emph{differ by a factor of 2}!

Naively, Mellin inversion of $\LM(s)$ should relate:
\[
\sum_{n \leq x} M(n) = \frac{1}{2\pi i} \int_{(c)} \LM(s) \frac{x^s}{s} \, ds,
\]
where the residue at $s=1$ contributes. However, the floor function in $M(n) = \lfloor(\tau(n)-1)/2\rfloor$ introduces subtle corrections.

\textbf{Open Question:} Rigorously derive the relationship between $(2\gamma - 1)$ and $(\gamma - 1)$ via Mellin inversion, accounting for the floor function.
\end{remark}

\subsection{Highly Composite Numbers}

% TODO: Theorem on max M(n) at highly composite numbers

\begin{remark}[Highly Composite Pattern]
Computational experiments show that $\max_{n \leq x} M(n)$ is attained at classical highly composite numbers (e.g., $60, 360, 840, 2520, 7560, \ldots$). This suggests $M(n)$ inherits extremal properties from $\tau(n)$, but the $\sqrt{n}$ cutoff modifies the structure.
\end{remark}

\section{The $\sqrt{n}$ Universality Phenomenon}
\label{sec:sqrt-n}

A striking feature of this work is the ubiquity of the $\sqrt{n}$ scale across multiple contexts:

\begin{enumerate}[label=(\roman*),nosep]
\item \textbf{Definition:} $M(n)$ counts divisors $d$ with $2 \leq d \leq \sqrt{n}$
\item \textbf{Regularization:} Optimal $\eps \sim n^{-1/(2\alpha)} \approx 1/\sqrt[6]{n}$ (for $\alpha=3$)
\item \textbf{Asymptotics:} $M(n) \sim \log(\sqrt{n}) = \frac{1}{2} \log n$
\item \textbf{Residue:} The constant $2\gamma - 1$ encodes divisor asymmetry around $\sqrt{n}$
\item \textbf{Geometry:} The primal forest function $F_n$ splits naturally at $d = \sqrt{n}$
\end{enumerate}

This is \emph{not coincidence} --- the $\sqrt{n}$ boundary is fundamentally encoded in multiplicative structure.

\subsection{Connection to the Divisor Problem}

The Dirichlet divisor problem studies
\[
\Delta(x) := \sum_{n \leq x} \tau(n) - x \log x - (2\gamma - 1)x.
\]
The appearance of $2\gamma - 1$ relates to the $\sqrt{n}$ boundary in the hyperbola method for counting divisor pairs $(d, n/d)$.

Our function $M(n)$ makes this asymmetry \emph{explicit} by restricting to $d \leq \sqrt{n}$, and the residue $2\gamma - 1$ in Theorem \ref{thm:laurent} directly reflects this geometric structure.

\section{Bridging Local and Global: $G(s, \alpha, \eps)$}
\label{sec:bridge}

% TODO: Define G(s,alpha,eps) and prove limit

\subsection{The Unified Function}

Combining the epsilon-pole framework with the Dirichlet series:

\begin{definition}[Global Regularized Function]
For $\Re(s) > 1$, $\alpha > 0$, $\eps > 0$,
\[
G(s, \alpha, \eps) := \sum_{n=1}^\infty \frac{F_n(\alpha, \eps)}{n^s}.
\]
\end{definition}

\begin{theorem}[Bridge Theorem --- Numerical]
\label{thm:bridge}
For $\Re(s) > 1$ and $\alpha > 0$,
\[
\lim_{\eps \to 0^+} \eps^\alpha \cdot G(s, \alpha, \eps) = \LM(s).
\]
\end{theorem}

\begin{proof}[Proof (sketch)]
Interchange limit and sum (justified by dominated convergence for $\Re(s) > 1$):
\[
\lim_{\eps \to 0} \eps^\alpha G(s, \alpha, \eps)
= \sum_{n=1}^\infty \frac{1}{n^s} \lim_{\eps \to 0} \eps^\alpha F_n(\alpha, \eps)
= \sum_{n=1}^\infty \frac{M(n)}{n^s} = \LM(s),
\]
using Theorem \ref{thm:residue}.

\textbf{Status:} Numerically verified. The "systematic shortfall" initially observed was identified as truncation error ($\sum_{n > n_{\max}} M(n)/n^s$), not a violation of the limit.
\end{proof}

\subsection{Complementarity of Regularizations}

The function $G(s, \alpha, \eps)$ unifies two regularization schemes:
\begin{itemize}[nosep]
\item \textbf{Power law} (via $\eps$): Local, detects exact factorizations, has poles
\item \textbf{Exponential} (via $s$): Global, smooth distribution, analytic
\end{itemize}

These are \emph{complementary}, not equivalent. The parameter $\alpha$ controls pole strength, $\eps$ is an IR cutoff, and $s$ is a UV cutoff. All three play independent roles in the three-parameter regularization.

\section{Open Questions and Conjectures}
\label{sec:open}

\subsection{Rigorous Proofs}

Many results in this paper are numerically verified but lack rigorous proof:
\begin{itemize}[nosep]
\item Laurent expansion (Theorem \ref{thm:laurent})
\item Closed form derivation (Theorem \ref{thm:closed-form})
\item Residue theorem (Theorem \ref{thm:residue})
\item Bridge theorem (Theorem \ref{thm:bridge})
\end{itemize}

\subsection{Functional Equation}

\begin{question}
Does $\LM(s)$ satisfy a functional equation relating $\LM(s)$ to $\LM(k - s)$ for some $k$? If so, what is the gamma factor?
\end{question}

Classical attempts with $\gamma(s) = \pi^{-s/2} \Gamma(s/2)$ have failed (analytic continuation to $\Re(s) \leq 1$ fails). A non-classical gamma factor may be required, or no functional equation may exist due to non-multiplicativity.

\subsection{Zeros of $\LM(s)$}

\begin{question}
Does $\LM(s)$ have zeros in $\Re(s) > 1$? If so, where are they located?
\end{question}

No zeros were observed numerically in the region $\{1.1 \leq \Re(s) \leq 3.5, |\Im(s)| \leq 30\}$.

\begin{question}
Is there a connection to the Riemann zeta zeros? Specifically, does $\LM(s_0) = 0$ when $\zeta(s_0) = 0$?
\end{question}

Testing this requires analytic continuation to $\Re(s) = 1/2$, which currently fails.

\subsection{The Mellin Puzzle}

Resolve Remark \ref{rem:mellin-puzzle}: Why does the summatory function have constant $(\gamma - 1)$ while the Laurent residue is $(2\gamma - 1)$? Perform rigorous Mellin inversion accounting for the floor function.

\subsection{Variance Asymptotics}

\begin{question}
How does $\text{Var}(M(n))$ grow as $n \to \infty$?
\end{question}

Numerical data suggests $\text{Var}(M(n)) \approx 15.7$ for $n \leq 10^4$. Does it follow a quarter-rule $\text{Var}(M(n)) \sim \text{Var}(\tau(n))/4$?

\subsection{Connections to Other Theories}

\begin{itemize}[nosep]
\item \textbf{Pell equations:} Both involve $\sqrt{D}$ or $\sqrt{n}$ boundaries. Is there a regulator-type quantity related to $M(n)$?
\item \textbf{Modular forms:} Can $M(n)$ be interpreted as Fourier coefficients of a weight-2 modular form (despite non-multiplicativity)?
\item \textbf{Random matrix theory:} Does $\LM(1/2 + it)$ (if AC works) follow GUE/GOE statistics?
\end{itemize}

\section{Conclusion}

We have introduced the childhood function $M(n)$, developed its epsilon-pole regularization framework, derived a closed-form Dirichlet series $\LM(s)$, and analyzed its analytic properties. The $\sqrt{n}$ boundary emerges as a universal scale connecting geometry, asymptotics, and complex analysis.

Key discoveries include:
\begin{itemize}[nosep]
\item The residue $2\gamma - 1$ at the double pole $s=1$
\item Schwarz symmetry (real coefficients)
\item Non-uniform convergence in epsilon-pole regularization
\item The "Mellin puzzle" relating summatory and residue constants
\end{itemize}

Many results remain numerically verified pending rigorous proof. The function $\LM(s)$ presents a rich structure worthy of further investigation, particularly regarding its potential functional equation, zeros, and connections to classical analytic number theory.

\subsection*{Acknowledgments}

This work represents a collaboration between human mathematical intuition (J.P.) and AI-assisted computation and analysis (Claude). We acknowledge the Trinity framework for human-AI co-authorship in mathematical research.

\begin{thebibliography}{99}

% TODO: Add references
% - Apostol: Introduction to Analytic Number Theory
% - Ivic: The Riemann Zeta Function
% - Hardy & Wright: An Introduction to the Theory of Numbers
% - Tenenbaum: Introduction to Analytic and Probabilistic Number Theory

\bibitem{apostol}
T. M. Apostol, \emph{Introduction to Analytic Number Theory}, Springer, 1976.

\bibitem{ivic}
A. Ivi\'c, \emph{The Riemann Zeta-Function: Theory and Applications}, Dover, 2003.

\bibitem{hw}
G. H. Hardy and E. M. Wright, \emph{An Introduction to the Theory of Numbers}, 6th ed., Oxford University Press, 2008.

\bibitem{tenenbaum}
G. Tenenbaum, \emph{Introduction to Analytic and Probabilistic Number Theory}, 3rd ed., American Mathematical Society, 2015.

\end{thebibliography}

\appendix

\section{Computational Methods}

% TODO: Brief overview of Python scripts (without detailed code)

All numerical computations were performed using Python 3 with NumPy, SciPy, and Matplotlib. Key scripts include:
\begin{itemize}[nosep]
\item \texttt{analyze\_M\_asymptotics.py}: Distribution and summatory function analysis
\item \texttt{visualize\_L\_M\_complex.py}: Complex plane visualizations
\item \texttt{domain\_coloring\_L\_M.py}: Domain coloring plots
\item \texttt{analyze\_systematic\_shortfall.py}: Verification of residue theorem
\end{itemize}

Source code available in the repository: \texttt{https://github.com/[to-be-added]}.

\section{Visualizations}

% TODO: Include key plots
% - M(n) distribution histogram
% - Summatory function comparison
% - Domain coloring of L_M(s)
% - Phase portrait

[Figures to be inserted in final version]

\end{document}
