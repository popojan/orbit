\documentclass[11pt]{article}
\usepackage[T1]{fontenc}
\usepackage[utf8]{inputenc}
\usepackage[czech]{babel}
\usepackage{amsmath,amssymb,amsthm}
\usepackage{graphicx}
\usepackage{xcolor}
\usepackage{tikz}
\usepackage[margin=1in]{geometry}
\usepackage{listings}
\usepackage{hyperref}
\hypersetup{
  pdfpagemode=UseOutlines,
  bookmarksnumbered=true
}

\title{Prvoles:\\
Cesta skrz Eratosthenovo síto}

\author{Jan Popelka}

\date{\today}

\begin{document}

\maketitle

\begin{abstract}
Představujeme vzdělávací vizualizaci, která transformuje klasické Eratosthenovo síto z jednorozměrného seznamu do dvourozměrného ``prvolesí.'' Mapováním složených čísel $n = p(p+k)$ do souřadnic $(kp+p^2, kp+1)$ vytváříme pohlcující pohled, kdy stojíte na jižním okraji lesa a díváte se na sever, přičemž složená čísla se objevují jako stromy v různých hloubkách. Prvočísla se stanou viditelnými jako průhledy---pozice, kde žádné složené číslo neblokuje váš výhled skrz les. Tato geometrická perspektiva činí strukturu síta vizuálně intuitivní a poskytuje poutavý úvod do multiplikativní struktury v teorii čísel.
\end{abstract}

\section{Úvod: Proč je tak těžké najít prvočísla?}

Prvočísla jsou atomy aritmetiky---každé celé číslo se rozpadá na jedinečný součin prvočísel. Přesto se, navzdory své fundamentální důležitosti, prvočísla zdají objevovat na číselné ose nepředvídatelně:

\[
2, 3, 5, 7, 11, 13, 17, 19, 23, 29, 31, 37, 41, 43, 47, \ldots
\]

Mezery mezi po sobě jdoucími prvočísly se liší: někdy jen 2 (dvojčata jako 11 a 13), někdy mnohem více (mezi 113 a 127 je mezera 14). Je v tomto chaosu skrytý vzorec?

Staří Řekové vyvinuli chytrý algoritmus---\emph{Eratosthenovo síto}---který systematicky nachází všechna prvočísla odstraněním složených čísel. Ale procházení sítem na lineárním seznamu může působit mechanicky. \textbf{Co kdybychom mohli tu strukturu \emph{vidět}?}

Představujeme \textbf{Prvoles}: geometrickou transformaci, která mění síto na vizuální krajinu, kde složená čísla tvoří pravidelné vzorce a prvočísla se objevují jako průhledy. Představte si, že stojíte na jižním okraji rozsáhlého lesa a díváte se na sever mezi stromy. Můžete chodit na západ nebo východ podél tohoto okraje (osa x), a na každé pozici se podívat přímo na sever. Stromy jsou složená čísla, rozptýlená v různých hloubkách lesa. Na určitých pozicích podél jižního okraje máte dokonale čistý průhled---žádné stromy vám neblokují výhled v žádné hloubce. Tyto čisté koridory odhalují prvočísla.

Tento geometrický pohled skrývá hlubší tajemství: když pečlivě měříme vzdálenosti k těmto stromům, objevíme, že prvočíselnost existuje na spojitém spektru, s prvočísly na hranici a složenými čísly stratifikovanými podle jejich faktorizační složitosti.

\section{Klasické síto (lineární pohled)}

Připomeňme si stručně, jak Eratosthenovo síto funguje v tradičním provedení. Vypíšeme všechna celá čísla počínaje od 2:

\[
2, 3, 4, 5, 6, 7, 8, 9, 10, 11, 12, 13, 14, 15, 16, 17, 18, 19, 20, 21, 22, 23, 24, \ldots
\]

Pak opakovaně:
\begin{enumerate}
\item Označíme první neoznačené číslo jako prvočíslo
\item Vyškrtneme všechny jeho násobky (složená čísla)
\item Opakujeme
\end{enumerate}

Například:
\begin{itemize}
\item Označíme 2 jako prvočíslo, vyškrtneme 4, 6, 8, 10, 12, 14, 16, 18, 20, 22, 24, \ldots
\item Označíme 3 jako prvočíslo, vyškrtneme 6, 9, 12, 15, 18, 21, 24, \ldots
\item Označíme 5 jako prvočíslo, vyškrtneme 10, 15, 20, 25, \ldots
\item A tak dále\ldots
\end{itemize}

Čísla, která zůstanou neoznačená, jsou prvočísla. To funguje krásně, ale je to jednorozměrný pohled. Každé složené číslo zmizí v přeškrtnutém symbolu. Můžeme vizualizovat \emph{strukturu} toho, proč jsou složená čísla odstraňována?

\section{Geometrická transformace}

Zde je klíčový poznatek: každé složené číslo lze zapsat jako $n = p(p+k)$ pro nějaká kladná celá čísla $p$ a $k \geq 0$. Například:
\begin{align*}
4 &= 2 \times 2 = 2(2+0) \quad \text{(tedy } p=2, k=0\text{)} \\
6 &= 2 \times 3 = 2(2+1) \quad \text{(tedy } p=2, k=1\text{)} \\
8 &= 2 \times 4 = 2(2+2) \quad \text{(tedy } p=2, k=2\text{)} \\
9 &= 3 \times 3 = 3(3+0) \quad \text{(tedy } p=3, k=0\text{)}
\end{align*}

Nyní, místo umístění těchto složených čísel na čáru, je umístíme do 2D roviny pomocí tohoto mapování:

\[
\boxed{n = p(p+k) \quad \mapsto \quad \text{bod v } (kp + p^2,\, kp + 1)}
\]

Pojďme to rozebrat:
\begin{itemize}
\item \textbf{x-souřadnice}: $kp + p^2 = p(k+p)$ je přesně složené číslo $n$ samo o sobě
\item \textbf{y-souřadnice}: $kp + 1$ je zvolena tak, aby body rozmístila vertikálně v pravidelném vzorci
\end{itemize}

\subsection{Proč tato volba souřadnic?}

Souřadnice y jako $kp + 1$ není svévolná---vytváří \textbf{pravidelný les}. Když vykreslíme body pro různé hodnoty $k$ a $p$, tvoří symetrický mřížkový vzorec. Tato symetrie odhaluje hlubokou pravidelnost v tom, jak jsou faktorizace rozděleny.

\textbf{Poznámka}: Mohli bychom vykreslit i případ $p=1$ (který dává posloupnost $k+1$ podél hlavní diagonály), ale jeho vynechání činí vzorec jasnějším: \emph{každá tečka je faktorizace a prvočísla jsou absence}. To vyžaduje aktivní uvažování spíše než pasivní rozpoznávání.

\section{Les se objevuje}

Obrázek~\ref{fig:forest} ukazuje prvolesí pro rozsah $n \leq 31$. Vykreslujeme pouze faktorizace s $p \geq 2$---čistou faktorizační mřížku.

\begin{figure}[h]
\centering
\includegraphics[width=0.8\textwidth]{../visualizations/primal-forest-31.pdf}
\caption{Prvoles: Stojíte na jižním okraji (spodek diagramu, y=0) a můžete chodit západo-východně (vlevo-vpravo, osa x). Na každé pozici se díváte přímo na sever (nahoru v diagramu, osa y). Každá tečka představuje faktorizaci $n = p(p+k)$ s $p \geq 2$, vykreslenou v souřadnicích $(kp+p^2, kp+1)$. Prvočísla jsou x-souřadnice s \emph{čistými průhledy}---žádné složené stromy neblokují váš výhled v žádné hloubce na sever.}
\label{fig:forest}
\end{figure}

\subsection{Co vidíme?}

\begin{itemize}
\item \textbf{Stromy (tečky)}: Každá tečka představuje způsob, jak zapsat n jako součin dvou činitelů, kde menší činitel je alespoň 2. Konkrétně strom na pozici $(kp+p^2, kp+1)$ ukazuje faktorizaci $n = p \times (p+k)$, kde $p \geq 2$ je menší činitel. Y-souřadnice $(kp+1)$ představuje, jak hluboko v lese tato konkrétní faktorizace stojí. Počet stromů na x-souřadnici n (napříč všemi hloubkami) se rovná počtu dělitelů n v rozsahu $[2, \sqrt{n}]$. Tyto stromy tvoří ``les'', který může blokovat váš průhled.

\item \textbf{Vaše pozice}: Stojíte na pozici $(x, 0)$ na jižním okraji---doslova na y=0. Můžete chodit západo-východně a zvolit si, kterou x-souřadnici chcete testovat, pak se podívat přímo na sever do lesa. Většina pozic má alespoň jeden strom někde na sever od vás. Ale některé pozice \emph{nemají žádné stromy v žádné hloubce na sever}. To jsou \textbf{prvočísla}! Jsou to čisté průseky---čísla, která nelze zapsat jako $p(p+k)$ pro žádné $p \geq 2$.

\item \textbf{Test prvočíselnosti}: Postavte se na libovolnou x-souřadnici podél jižního okraje a podívejte se přímo na sever. Pokud vidíte stromy blokující váš výhled v nějaké hloubce, toto číslo je složené. Pokud máte dokonale čistý průhled celou cestou skrz les---nic neblokuje váš výhled v žádné vzdálenosti na sever---toto číslo je prvočíslo.
\end{itemize}

\subsection{Klíčový poznatek}

\begin{quote}
\textbf{Počet stromů na pozici n (napříč všemi hloubkami) se rovná počtu dělitelů n v rozsahu $[2, \sqrt{n}]$. Prvočísla mají nula takových dělitelů, takže nemají žádné stromy blokující průhled---jsou to čisté průseky skrz faktorový les.}
\end{quote}

Například:
\begin{itemize}
\item $n = 18 = 2 \times 9 = 3 \times 6$: dělitelé v $[2, \sqrt{18}] = [2, 4{,}24]$ jsou $\{2, 3\}$, takže \textbf{2 stromy} blokují váš výhled
\item $n = 16 = 2 \times 8 = 4 \times 4$: dělitelé v $[2, \sqrt{16}] = [2, 4]$ jsou $\{2, 4\}$, takže \textbf{2 stromy} blokují váš výhled
\item $n = 13$ (prvočíslo): žádné dělitele v $[2, \sqrt{13}] = [2, 3{,}6]$, takže \textbf{čistý průhled}
\end{itemize}

Les neukazuje pouze která čísla jsou složená---geometricky počítá jejich malé dělitele.

\section{Průzkum lesa}

\subsection{Proč složená čísla tvoří tento vzorec?}

Každý řádek $k = \text{konst}$ představuje všechna složená čísla tvaru $p(p+k)$:
\begin{itemize}
\item $k=0$: Druhé mocniny $p^2$ na pozicích $(p^2, 1)$
\item $k=1$: Součiny $p(p+1)$ na pozicích $(p^2+p, p+1)$
\item $k=2$: Součiny $p(p+2)$ na pozicích $(p^2+2p, 2p+1)$
\item A tak dále\ldots
\end{itemize}

Každý sloupec $p = \text{konst}$ představuje všechna složená čísla s $p$ jako menším činitelem.

\subsection{Čísla s více stromy}

Čím více způsobů se číslo dá faktorizovat, tím více stromů blokuje váš výhled. Mocniny prvočísel jako $9 = 3^2$ nebo $25 = 5^2$ mají přesně jeden strom. Polprvočísla jako $15 = 3 \times 5$ mají dva. Ale vysoce složená čísla mají více stromů v řadě: $24 = 2 \times 12 = 3 \times 8 = 4 \times 6$ má tři stromy v různých hloubkách, zatímco $36$ má čtyři. Počet stromů se rovná počtu dělitelů v $[2, \sqrt{n}]$, což činí vysoce složená čísla hustými překážkami. Prvočísla, bez takových dělitelů, zůstávají jako průseky---dokonale čisté průhledy.

\subsection{Diagonální řady a rostoucí mezery}

Všimněte si diagonálních řad stromů stoupajících na severovýchod? Každé $p$ vytváří svou vlastní diagonálu začínající v bodě $(p^2, 1)$ s rozestupy po $p$. Čím větší čísla, tím více diagonál se překrývá a tím hlouběji se les rozšiřuje na sever. To odhaluje hlubokou pravdu o rozložení prvočísel:

\begin{quote}
\textbf{Čím vyšší je x-souřadnice (čím větší číslo), tím hlouběji se les rozšiřuje na sever. Více stromů v různých hloubkách znamená vyšší pravděpodobnost, že nějaký strom zablokuje váš průhled.}
\end{quote}

Jak kráčíme na východ podél jižního okraje (zvyšující se x), les se zahušťuje a rozprostírá se dále na sever. Více činitelů znamená více stromů, rozptýlených napříč větším rozsahem hloubek. Šance najít dokonale čistý průsek---prvočíslo---se snižuje. Tento geometrický pohled činí Větu o prvočíslech intuitivní: \emph{prvočísla řídnou}, protože ``les činitelů'' se zahušťuje a prohlubuje.

\section{Průzkum a cvičení}

\begin{figure}[h]
\centering
% TODO: Vložit primal-forest-100.pdf (nebo jiný větší rozsah)
% \includegraphics[width=\textwidth]{../visualizations/primal-forest-100.pdf}
\fbox{\parbox{0.9\textwidth}{\centering\vspace{2cm}[Zde bude vizualizace prvolesí pro větší rozsah, např. $n \leq 100$]\vspace{2cm}}}
\caption{Prvoles pro větší rozsah: Diagonální řady se překrývají, les se zahušťuje směrem na východ a sever. Průseky (prvočísla) se stávají vzácnějšími.}
\label{fig:forest-large}
\end{figure}

\subsection{Cvičení}

\begin{enumerate}
\item \textbf{Hledání prvočíselných dvojčat}: Postavte se na pozici 11 (prvočíslo s čistým průhledem). Jděte pomalu na východ. Další čistý koridor je na 13---jen 2 kroky daleko (prvočíselná dvojčata!). Pokračujte v chůzi a najděte další dvojčata v rozsahu do 100. Kolik jich je? Studiem struktury lesa zkuste odhadnout, proč jsou některé oblasti ``bohatší'' na dvojčata než jiné. (Poznámka: Předpovědět přesný vzorec rozložení dvojčat je otevřený výzkumný problém!)\footnote{Pro pokročilé: Viz \emph{Domněnka o prvočíselných dvojčatech} na \href{https://cs.wikipedia.org/wiki/Prvo\%C4\%8D\%C3\%ADseln\%C3\%A1_dvoj\%C4\%8Data}{cs.wikipedia.org}}

\item \textbf{Goldbachův průzkum}: Vyberte si pozici sudého čísla, řekněme 20. Můžete najít dvě prvočíselné pozice (čisté koridory), která v součtu dají 20? Například pozice 3 a 17 obě mají čisté průhledy a $3+17=20$. Zkuste jiná sudá čísla do 50. Vždy najdete alespoň jeden takový pár? Co vám les říká o aditivní struktuře? (Poznámka: Otázka ``funguje to pro všechna sudá čísla?'' je slavná Goldbachova domněnka, dosud nedokázaná!)\footnote{Pro pokročilé: Více o \emph{Goldbachově domněnce} na \href{https://cs.wikipedia.org/wiki/Goldbachova_dom\%C4\%9B\%C5\%88ka}{cs.wikipedia.org}}

\item \textbf{Modifikace lesa}: Co kdybyste zahrnuli stromy s $p=1$ do vizualizace? Nebo co kdybyste vykreslili pouze lichá složená čísla? Vytvořte tyto varianty a popište, co se změní. Činí odstranění určitých stromů vzorce zřetelnějšími nebo méně zřetelnými? Co vám to říká o roli malých prvočísel?

\item \textbf{Vizuální test prvočíselnosti}: Prostudujte vizualizaci lesa pro $n \leq 30$. Zkuste odhadnout, zda budou pozice 91 a 97 prvočísla (čisté průhledy). Vysvětlete své uvažování pouze z toho, co vidíte v diagramu---diagonální řady, hustota stromů---bez faktorizací. Pak ověřte výpočtem.

\item \textbf{Pozorování hustoty}: Spočítejte, kolik stromů se objevuje v oblastech 1-30, 31-60 a 61-90. Roste nebo klesá počet stromů? Zkuste odhadnout, proč se hustota lesa mění tímto způsobem.\footnote{Pro pokročilé: Tento trend souvisí s \emph{Větou o prvočíslech}, která popisuje, jak prvočísla řídnou. Viz \href{https://cs.wikipedia.org/wiki/V\%C4\%9Bta_o_prvo\%C4\%8D\%C3\%ADslech}{cs.wikipedia.org}}
\end{enumerate}

\section{Spojení s jinými vizualizacemi}

Existují další geometrické vizualizace prvočísel, z nichž každá zdůrazňuje jiné aspekty:

\begin{itemize}
\item \textbf{Ulamova spirála}\footnote{Viz \href{https://cs.wikipedia.org/wiki/Ulamova_spir\%C3\%A1la}{cs.wikipedia.org}} (1963): Uspořádání celých čísel do spirály; prvočísla se shlukují podél diagonálních čar
\item \textbf{Sacksova spirála}\footnote{Viz \url{https://en.wikipedia.org/wiki/Ulam_spiral\#Variations} (anglicky). Podobná Ulamově spirále, ale používá Archimedovu spirálu, kde vzdálenost od středu roste lineárně s úhlem.}: Uspořádání celých čísel na Archimedově spirále; prvočísla tvoří zakřivené vzorce
\item \textbf{Prvočíselné mřížky}: Různá 2D uspořádání odhalující strukturu
\end{itemize}

Metafora lesa je zvláště intuitivní: \textbf{prvočísla nejsou náhodně rozptýlená---jsou to průseky, které zůstanou po systematickém vysazení stromů.}

\section{Vzdělávací hodnota}

Tato vizualizace pomáhá odpovědět na časté studentské otázky:

\paragraph{Proč jsou prvočísla ``speciální''?}
A: Podívejte se na les---prvočísla jsou jediná čísla s \emph{dokonale čistými průhledy}. Žádné stromy neblokují váš výhled na sever v žádné hloubce. Jsou zásadně odlišná od složených čísel.

\paragraph{Proč mezery mezi prvočísly rostou?}
A: Vidíte, jak se les rozprostírá hlouběji na sever a zahušťuje se, jak čísla rostou? Více složených čísel v více hloubkách přeplňuje větší x-oblasti, činíc čisté průhledy vzácnějšími. Mezery mezi prvočísly se přirozeně rozšiřují.

\paragraph{Co faktorizace znamená geometricky?}
A: Každá tečka je složené číslo, umístěné podle svých činitelů $p$ a $k+p$. Prvočísla nemohou být takto umístěna---nemají faktorizaci.

\paragraph{Je nějaký vzorec v prvočíslech?}
A: Ano! Prvočísla jsou přesně mezery v pravidelné faktorové mřížce. Vzorec je viditelný, jakmile zmapujeme faktorizace do 2D.

\paragraph{Proč se pravděpodobnost nalezení prvočísla snižuje?}
A: Jak čísla rostou, každá pozice na ose x (jižní okraj) má více potenciálních činitelů. Les se rozprostírá hlouběji na sever a zahušťuje se, protože existuje jednoduše více způsobů, jak faktorizovat větší čísla. Geometricky více stromů ve více hloubkách blokuje vaše průhledy, takže dokonale čisté průseky (prvočísla) se stávají vzácnějšími.

\section{Paradox pravidelnosti}

\begin{center}
\fbox{\begin{minipage}{0.9\textwidth}
\vspace{0.5em}
\textbf{Hluboké tajemství poodhaleno}

Prvoles odhaluje ohromující paradox v srdci rozložení prvočísel:

\begin{itemize}
\item \textbf{Vstup}: Dokonale pravidelné diagonální vzorce
  \begin{itemize}
  \item Násobky 2: rovnoměrně rozložené v intervalech 2
  \item Násobky 3: rovnoměrně rozložené v intervalech 3
  \item Násobky 5: rovnoměrně rozložené v intervalech 5
  \item Každé prvočíslo generuje svou vlastní dokonale uniformní mřížku
  \end{itemize}

\item \textbf{Transformace}: Jednoduchý kvadratický posun $p^2 + kp$
  \begin{itemize}
  \item Zcela deterministické
  \item Žádná náhodnost, žádný chaos v samotném pravidle
  \item Pouze posunuje každý diagonální řádek na pozici určenou druhou mocninou
  \end{itemize}

\item \textbf{Výstup}: Tajemné rozložení prvočísel
  \begin{itemize}
  \item Mezery mezi prvočísly: 1, 2, 2, 4, 2, 4, 2, 4, 6, 2, 6, 4, 2, 4, 6, 6, \ldots---chaoticky kolísají a celkově rostou\footnote{Pro pokročilé: Maximální velikost mezer mezi po sobě jdoucími prvočísly souvisí s \emph{Cramérovou domněnkou}. Viz anglická Wikipedie: \href{https://en.wikipedia.org/wiki/Cram\%C3\%A9r\%27s_conjecture}{Cramér's conjecture}}
  \item Prvočíselná dvojčata se objevující nepředvídatelně
  \item Předmět Riemannovy hypotézy\footnote{\emph{Riemannova hypotéza}, formulovaná v roce 1859, zůstává nevyřešena již přes 165 let a je jedním z Millennium Prize Problems s odměnou \$1 milion. Viz česká Wikipedie: \href{https://cs.wikipedia.org/wiki/Riemannova_hypot\%C3\%A9za}{Riemannova hypotéza}}
  \item Jeden z nejhlubších nevyřešených problémů matematiky
  \end{itemize}
\end{itemize}
\vspace{0.5em}
\end{minipage}}
\end{center}

\textbf{Otázka}: Jak může sjednocení nekonečně mnoha dokonale pravidelných vzorů, systematicky posunutých, vytvořit něco tak složitého a tajemného, jako je rozložení prvočísel?

Struktura je dokonale pravidelná, předvídatelná---a přesto se průseky vymykají předpovědi. Vizualizace odhaluje podstatu tajemství: \textbf{jednoduchá geometrická pravidla generují nepopsatelnou složitost.}

Proto geometrický pohled, navzdory své jasnosti, nenabízí výpočetní zjednodušení. Průseky jsou globální jev---prvočíslo se musí vyhnout \emph{všem} pravidelným vzorům najednou. Test prvočíselnosti zůstává stejně obtížný, jen vidíme jeho geometrickou podstatu.

\section{Poznámka pro pedagogy}

\subsection{Proč se to neučí ve školách?}

Matematické vzdělávání tradičně upřednostňuje algoritmické postupy před vizuálním porozuměním. Eratosthenovo síto je typickým příkladem: přestože je základem výuky prvočísel, je téměř univerzálně prezentováno jako \emph{lineární, sekvenční algoritmus}: vypište čísla 2, 3, 4, 5, \ldots na řádek, pak systematicky škrtejte násobky. Tento přístup zapojuje \textbf{sekvenční zpracování levou hemisférou}\footnote{Poznámka: Rozdělení na "levou analytickou" a "pravou kreativní" hemisféru je zjednodušený populárně-naučný model. V realitě obě hemisféry intenzivně spolupracují na většině kognitivních úkolů. Jde zde o pedagogickou metaforu. Viz \href{https://cs.wikipedia.org/wiki/Lateralizace_mozku}{cs.wikipedia.org}}---krok 1, krok 2, následujte postup, dojděte k odpovědi.

Prvoles nabízí něco zásadně odlišného: \textbf{prostorové vnímání pravou hemisférou}, kde je celá struktura viditelná najednou. Místo ``vyškrtněte další složené číslo'' studenti vidí ``složená čísla tvoří pravidelnou mřížku; prvočísla jsou mezery.''

Proč geometrické vizualizace prvočísel zůstávají raritou ve výuce? S moderními výpočetními nástroji---GeoGebra, Python, Wolfram, interaktivní notebooky---je jejich vytváření a zkoumání triviální. Ulamova spirála existuje od roku 1963, přesto se neučí. Proč vzdělávání setrvává u lineárního algoritmu?

\paragraph{Historická setrvačnost}
Matematické vzdělávání se vyvinulo z ústních a písemných tradic, kde jste mohli ukázat pouze jedno číslo najednou. Lineární posloupnost 2, 3, 4, 5, 6, \ldots je to, co se vejde na papyrus, řádek tabule nebo tištěnou stránku. Dvourozměrné vizualizace vyžadují nástroje (milimetrový papír, vykreslování, barvu), které nebyly snadno dostupné až do nedávna.

\paragraph{Výpočetní zaujatost}
Síto se tradičně učí jako \emph{algoritmus}---metoda pro efektivní výpočet prvočísel. Školy zdůrazňují ``jak najít prvočísla'' (procedurální dovednost) nad ``proč se prvočísla chovají tímto způsobem'' (konceptuální porozumění). Lineární metoda je skutečně snazší implementovat ručně.

\paragraph{Kompatibilita s hodnocením}
Lineární algoritmy je snadné testovat: ``Vyškrtal student správná čísla? Identifikoval všechna prvočísla do 100?'' Prostorové porozumění je těžší hodnotit: ``Rozpoznává student, proč se tvoří mezery? Dokáže vysvětlit vzorec geometricky?'' Tradiční testování upřednostňuje procedurální úkoly.

\paragraph{Příprava učitelů}
Většina učitelů matematiky se sama naučila lineární síto a možná se nikdy nesetkala s geometrickými alternativami. Vyučování toho, co znáte, udržuje algoritmickou tradici. Zavedení nových vizualizací vyžaduje profesní rozvoj, aktualizované materiály a pohodlí s výpočetními nástroji.

\subsection{Vzdělávací příležitost}

Lineární síto odpovídá na otázky \emph{jak}, zatímco geometrický pohled odpovídá na otázky \emph{proč}. Lineární síto je užitečný nástroj. Škoda je, když zůstane jediným setkáním žáků s prvočísly, protože pro zvídavé žáky algoritmus sám nevybízí k hlubšímu zkoumání a kladení otázek \emph{proč}:

\begin{center}
\begin{tabular}{p{6cm}|p{6cm}}
\textbf{Lineární síto (Jak)} & \textbf{Prvoles (Proč)} \\ \hline
Vyškrtněte násobky 2 & Násobky tvoří diagonální řady s pravidelnými rozestupy \\
Vyškrtněte násobky 3 & Každé prvočíslo vytváří svoji diagonální řadu \\
Prvočísla jsou to, co zbylo & Prvočísla se nedají rozložit na menší činitele \\
Mezery mezi prvočísly rostou & Hustota složených čísel roste s velikostí \\
Pokračujte ve škrtání\ldots & Přehlédnete celou strukturu jedním pohledem \\
\end{tabular}
\end{center}

To jsou přesně otázky, které zvídaví studenti kladou:
\begin{itemize}
\item ``Proč prvočísla řídnou?'' $\rightarrow$ Les se zahušťuje (více činitelů blokuje váš výhled)
\item ``Proč jsou prvočísla `speciální'?'' $\rightarrow$ Jsou to mezery ve faktorové mřížce (neexistují faktorizace)
\item ``Je nějaký vzorec v prvočíslech?'' $\rightarrow$ Ano i ne---prvočísla jsou to, co zůstane, když odstraníme pravidelnost
\end{itemize}

Lineární síto na tyto otázky neodpovídá---pouze říká: provádějte algoritmus až do konce.

\subsection{Doplňující materiál, ne náhrada}

Netvrdíme, že by algoritmické Eratosthenovo síto nemělo být vyučováno---ba právě naopak. Klíčový poznatek: \textbf{různí studenti myslí různě}. Někteří pochopí prvočísla z algoritmu, jiní z vizualizace, někteří potřebují obojí. Proč se snažit všechny směstnat do jediné šablony?

Navrhujeme kombinaci obou přístupů. Pořadí může být flexibilní:

\begin{enumerate}
\item \textbf{Tradiční cesta (algoritmus → vizualizace)}: Začnete klasickým sítem---škrtáním násobků. To je konkrétní, snadno proveditelné tužkou a papírem. Pak nabídnete překvapivý obrat: ``Co kdybychom tu samou strukturu viděli geometricky?'' Vizualizace odhaluje \emph{proč} algoritmus funguje. Výhoda: jednoduchost na začátku. Riziko: struktura se může ztratit v mechanickém škrtání.

\item \textbf{Alternativní cesta (vizualizace → algoritmus)}: Začnete geometrickou strukturou---les složených čísel, průseky jsou prvočísla. Studenti vidí \emph{proč} prvočísla existují dřív, než se naučí \emph{jak} je najít. Pak algoritmus jako praktický nástroj. Výhoda: struktura hned viditelná. Riziko: abstraktnější vstup, 2D souřadnice mohou být náročnější.

\item \textbf{Diskuse a reflexe}: Ať už zvolíte jakékoli pořadí, zahrňte diskusi: Proč obě metody fungují? Co každá odhaluje? Který přístup vám připadá přirozenější? Cíl není vybrat jeden správný přístup, ale poskytnout různé pohledy pro různé myšlení.
\end{enumerate}

\subsection{Nástroje to umožňují}

S moderními výpočetními nástroji---Wolfram Language, Python s matplotlib, GeoGebra, Desmos, interaktivními notebooky---je vytváření a zkoumání těchto vizualizací nyní \emph{jednoduché}. Studenti mohou:
\begin{itemize}
\item Vykreslovat vizualizace pro různé rozsahy čísel
\item Zaměřit se na konkrétní oblasti
\item Experimentovat s úpravami souřadnic
\item Animovat vznikající vzor
\item Objevovat další vzory sami
\end{itemize}

Bariéra už není technologie. Bariérou je nedostatek \emph{povědomí}, že takové vizualizace existují, a \emph{odvahy} odchýlit se od tradičního učebního plánu.

\subsection{Výzva k akci}

Pokud jste pedagog a čtete toto, zvažte:
\begin{itemize}
\item Zkuste ukázat Prvoles jako doplněk algoritmického síta
\item Zeptejte se žáků a studentů, který pohled jim dává větší smysl
\item Použijte vizualizace k odpovědi na otázky ``proč'' o rozložení prvočísel
\item Povzbuďte žáky a studenty ke zkoušení variací a k vlastním objevům
\item Sdílejte s kolegy funkční přístupy a podílejte se na tvorbě učebních plánů
\end{itemize}

Cílem není přidat další téma do již přeplněného učebního plánu. Cílem je prohloubit porozumění důležitému tématu v souvislostech, zapojením jak algoritmického tak prostorového myšlení.

\section{Závěr}

Navržená vizualizace Prvoles transformuje linární algoritmus mechanického vyškrtávání násobků na konkrétní vizuální zkušenost.

Tento přístup je pedagogicky hodnotný na různých stupních:
\begin{itemize}
\item \textbf{Základní škola}: Vizuální intuice důležitější než přesné algoritmy
\item \textbf{Střední škola}: Propojení faktorizace čísel s geometrickým chápáním
\item \textbf{Univerzita}: Transformace souřadnic odhalují strukturu
\item \textbf{Výzkum}: Alternativní perspektivy snad mohou inspirovat nové přístupy
\end{itemize}

Příště, až budete přemýšlet o prvočíslech, představte si, že stojíte na jižním okraji rozsáhlého lesa a rozhodujete se, kudy do něj vstoupit. Na většině míst stromy blokují váš výhled na sever, ale občas narazíte na dokonale čistý průsek, kterým vidíte až do nekonečna. To je prvočíslo. Les je pravidelný, ale průhledy jsou tajemné.

Přesto má les mnohá další tajemství k odhalení. Jeden z možných směrů pro další nezvyklé bádání nabízíme v příloze A.

\clearpage
\appendix

\section{Spojité skóre prvočíselnosti}

Klasická charakterizace prvočísel je binární: číslo $n \in \mathbb{N}$, $n \geq 2$, je buď prvočíslo, nebo složené. Geometrická konstrukce z hlavního textu umisťuje složená čísla $kp + p^2$ (kde $p$ je prvočíslo, $k \in \mathbb{N}_0$) do bodové mřížky v rovině; číslo $n$ je prvočíslo právě tehdy, když přímka $x = n$ neprotíná žádný bod této mřížky.

Tuto diskrétní charakterizaci lze zobecnit na spojitou míru definováním vzdálenosti přímky $x = n$ k nejbližším bodům mřížky v každé hloubce $p$. Součin těchto vzdáleností poskytuje \emph{spojité skóre prvočíselnosti}.

\subsection{Definice}

\begin{definition}[Vzdálenost v hloubce]
\label{def:depth-distance}
Pro celé číslo $n \geq 2$ a prvočíslo $p \leq n$ definujeme \textbf{vzdálenost v hloubce $p$} jako
\begin{equation}
d_p(n) = \min_{0 \leq k \leq \lfloor n/p \rfloor} |n - (kp + p^2)|.
\end{equation}
Geometricky je to vzdálenost (v ose $x$) přímky $x = n$ od nejbližšího bodu tvaru $(kp + p^2, p)$ v mřížce složených čísel.
\end{definition}

Platí: $d_p(n) = 0$ právě tehdy, když existuje $k$ takové, že $n = kp + p^2$, tj. $p \mid (n - p^2)$. Pro prvočíslo $n$ je $d_p(n) > 0$ pro všechna $p < n$.

\begin{definition}[Soft-minimum aproximace]
\label{def:soft-min}
Pro $\beta > 0$ definujeme \textbf{soft-minimum aproximaci} vzdálenosti $d_p(n)$ předpisem
\begin{equation}
d_p^{\text{soft}}(n, \beta) = -\beta \log \sum_{k=0}^{\lfloor n/p \rfloor} \exp\left(-\frac{|n - (kp + p^2)|}{\beta}\right).
\end{equation}
\end{definition}

Jde o hladkou aproximaci funkce $\min$, známou jako \emph{log-sum-exp} trik.

\begin{definition}[Spojité skóre prvočíselnosti]
\label{def:primality-score}
Pro celé číslo $n \geq 2$ a parametr $\beta > 0$ definujeme \textbf{spojité skóre prvočíselnosti} jako
\begin{equation}
S(n, \beta) = \prod_{p \text{ prvočíslo}, \, 2 \leq p \leq n} d_p^{\text{soft}}(n, \beta).
\end{equation}
\end{definition}

Pro účely vizualizace se často používá logaritmická komprese $\log(1 + S(n, \beta))$.

\begin{observation}[Limitní chování]
\label{obs:limit}
Pro pevné $n$ platí $\lim_{\beta \to 0^+} d_p^{\text{soft}}(n, \beta) = d_p(n)$, a tedy
\begin{equation}
\lim_{\beta \to 0^+} S(n, \beta) = \prod_{p \leq n} d_p(n).
\end{equation}
Pro prvočíslo $n$ je tento součin kladný; pro složené $n$ existuje $p$ s $d_p(n) = 0$, takže limitní součin je nulový.
\end{observation}

\subsection{Empirická pozorování}

\begin{observation}[Obálková struktura]
\label{obs:envelope}
Pro pevné $\beta > 0$ (např. $\beta = 1/7$) funkce $n \mapsto \log(1 + S(n, \beta))$ vykazuje následující empirickou strukturu (viz Obrázek~\ref{fig:envelope}):
\begin{enumerate}
\item Prvočísla tvoří přibližně hladkou horní obálku.
\item Složená čísla leží pod touto obálkou, stratifikovaná podle faktorizační struktury.
\end{enumerate}
\end{observation}

\begin{figure}[h]
\centering
\includegraphics[width=0.7\textwidth]{../visualizations/soft-distance-envelope-127.pdf}
\caption{Spektrum prvočíselnosti: prvočísla (oranžová) tvoří hladkou obálku, složená čísla (modrá) se rozptylují níže. Vzdálenost každého složeného čísla od obálky kóduje jeho faktorizační strukturu.}
\label{fig:envelope}
\end{figure}

\begin{observation}[Stratifikace podle $\Omega(n)$]
\label{obs:stratification}
Empiricky se složená čísla stratifikují podle počtu prvočíselných činitelů s násobností, $\Omega(n) = \sum_{p^k \| n} k$ (viz Obrázek~\ref{fig:stratification}):
\begin{enumerate}
\item Mocniny prvočísel ($p^2, p^3, \ldots$) s $\Omega(n) = 2, 3, \ldots$ leží nejblíže obálce.
\item Polprvočísla ($pq$ s $p \neq q$) s $\Omega(n) = 2$ leží ve střední hloubce.
\item Čísla s $\Omega(n) \geq 3$ leží hlouběji pod obálkou.
\end{enumerate}
Obecně: hodnota $S(n, \beta)$ klesá s rostoucím $\Omega(n)$.
\end{observation}

\begin{figure}[h]
\centering
\includegraphics[width=0.7\textwidth]{../visualizations/soft-distance-composite-types.pdf}
\caption{Stratifikace složených čísel podle $\Omega(n)$: mocniny prvočísel (zelená) nejblíže k obálce, polprvočísla (modrá) níže, čísla s $\Omega(n) \geq 3$ (červená) na dně.}
\label{fig:stratification}
\end{figure}

\begin{observation}[Závislost na geometrické formulaci]
\label{obs:geometric}
Empiricky: definování analogického součinu měkkých vzdáleností pomocí klasické modulární aritmetiky (vzdálenost k násobkům $p$) neprodukuje obálkovou strukturu. Geometrická formulace $(kp + p^2, p)$ je podstatná pro pozorovanou stratifikaci.
\end{observation}

\subsection{Výpočetní složitost}

Výpočet $S(n, \beta)$ vyžaduje vyhodnocení $d_p^{\text{soft}}(n, \beta)$ pro všechna prvočísla $p \leq n$. Pro každé $p$ je třeba sečíst $\lfloor n/p \rfloor$ členů, celkově tedy $\Theta(n \log \log n)$ operací.

Tato složitost je srovnatelná s Eratosthenovým sítem, ale poskytuje spojitou míru místo binární klasifikace. Metoda nenabízí algoritmické zrychlení pro rozhodování prvočíselnosti.

\subsection{Otevřené problémy}

Následující otázky zůstávají otevřené (není však jasné, zda mají netriviální odpovědi):

\begin{enumerate}
\item Lze asymptotické chování obálky $\max_{p \leq n, \, p \text{ prvočíslo}} S(p, \beta)$ vyjádřit explicitně?
\item Je stratifikace podle $\Omega(n)$ ostrá, nebo se vrstvy překrývají?
\item Existuje spojitost s klasickými funkcemi rozložení prvočísel (Riemannova zeta, $\pi(x)$, atd.)?
\end{enumerate}

\section*{Poděkování}

Tato vizualizace vznikla z rekreačních zkoumání ve výpočetní teorii čísel. Děkuji komunitě Wolfram Language za nástroje, které dělají takové experimenty potěšením.

\paragraph{Kód a reprodukovatelnost:} Wolfram Language kód pro generování vizualizací je k dispozici na vyžádání. Kontaktujte autora pro zdrojové soubory.\footnote{Kód a další materiály budou dostupné v repozitáři: \texttt{https://github.com/popojan/orbit}}

\end{document}
