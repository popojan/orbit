\documentclass[11pt]{article}
\usepackage[T1]{fontenc}
\usepackage[utf8]{inputenc}
\usepackage[czech]{babel}
\usepackage{amsmath,amssymb,amsthm}
\usepackage{graphicx}
\usepackage{xcolor}
\usepackage{tikz}
\usepackage[margin=1in]{geometry}
\usepackage{listings}
\usepackage{hyperref}
\hypersetup{
  pdfpagemode=UseOutlines,
  bookmarksnumbered=true
}

% Definice theorem prostředí
\newtheorem{definition}{Definice}
\newtheorem{observation}{Pozorování}
\newtheorem{theorem}{Věta}
\newtheorem{lemma}{Lemma}

\title{Prvoles -- Cesta skrz Eratosthenovo síto}

\author{Jan Popelka}

\date{\today}

\begin{document}

\maketitle

\begin{abstract}
Představujeme geometrickou vizualizaci, která transformuje klasické Eratosthenovo síto z jednorozměrného seznamu do dvourozměrného ``prvolesa.'' Mapováním dělitelů čísel podle vzorce $n = p(p+k)$ do souřadnic $(kp+p^2, kp+1)$ vytváříme pohlcující pohled, kdy stojíte na jižním okraji lesa a díváte se na sever, přičemž dělitelé složených čísel se objevují jako stromy v různých hloubkách. Prvočísla se stanou viditelnými jako průhledy---pozice, kde žádný strom neblokuje váš výhled skrz les. Tato perspektiva nabízí nový způsob, jak vidět strukturu prvočísel.
\end{abstract}

\section{Úvod: Proč je tak těžké najít prvočísla?}

Prvočísla jsou atomy aritmetiky---každé celé číslo se rozpadá na jedinečný součin prvočísel. Přesto se, navzdory své fundamentální důležitosti, prvočísla zdají objevovat na číselné ose nepředvídatelně:

\[
2, 3, 5, 7, 11, 13, 17, 19, 23, 29, 31, 37, 41, 43, 47, \ldots
\]

Mezery mezi po sobě jdoucími prvočísly se liší: někdy jen 2 (dvojčata jako 11 a 13), někdy mnohem více (mezi 113 a 127 je mezera 14). Je v tomto chaosu skrytý vzorec?

Staří Řekové vyvinuli chytrý algoritmus---\emph{Eratosthenovo síto}---který systematicky nachází všechna prvočísla odstraněním složených čísel. Ale procházení sítem na lineárním seznamu může působit mechanicky. \textbf{Co kdybychom mohli tu strukturu \emph{vidět}?}

Představujeme \textbf{Prvoles}: geometrickou transformaci, která mění síto na vizuální krajinu, kde dělitelé tvoří pravidelné vzorce a prvočísla se objevují jako průhledy. Představte si, že stojíte na jižním okraji rozsáhlého lesa a díváte se na sever mezi stromy. Můžete chodit na západ nebo východ podél tohoto okraje (osa x), a na každé pozici se podívat přímo na sever. Stromy představují dělitele čísel na příslušných x-souřadnicích, rozptýlené v různých hloubkách lesa. Na určitých pozicích podél jižního okraje máte dokonale čistý průhled---to jsou prvočísla.

Tento geometrický pohled skrývá hlubší tajemství: když pečlivě měříme vzdálenosti k těmto stromům, objevíme, že prvočíselnost existuje na spojitém spektru, s prvočísly na hranici a složenými čísly stratifikovanými podle jejich faktorizační složitosti.

\section{Klasické síto (lineární pohled)}

Eratosthenovo síto najde prvočísla vyškrtáváním násobků: označíme 2 a vyškrtneme 4, 6, 8, \ldots; označíme 3 a vyškrtneme 6, 9, 12, \ldots; označíme 5 a vyškrtneme 10, 15, 20, \ldots; a tak dále. Co zbyde, jsou prvočísla.

To funguje, ale je to jednorozměrný pohled. Můžeme vidět \emph{strukturu}, proč jsou složená čísla odstraňována?

\section{Geometrická transformace}

Zde je klíčový poznatek: každé složené číslo lze zapsat jako $n = p(p+k)$ pro nějaká kladná celá čísla $p$ a $k \geq 0$. Například:
\begin{align*}
4 &= 2 \times 2 = 2(2+0) \quad \text{(tedy } p=2, k=0\text{)} \\
6 &= 2 \times 3 = 2(2+1) \quad \text{(tedy } p=2, k=1\text{)} \\
8 &= 2 \times 4 = 2(2+2) \quad \text{(tedy } p=2, k=2\text{)} \\
9 &= 3 \times 3 = 3(3+0) \quad \text{(tedy } p=3, k=0\text{)}
\end{align*}

Nyní, místo umístění těchto složených čísel na čáru, je umístíme do 2D roviny pomocí tohoto mapování:

\[
\boxed{n = p(p+k) \quad \mapsto \quad \text{strom na } (kp + p^2,\, kp + 1)}
\]

Pojďme to rozebrat:
\begin{itemize}
\item \textbf{x-souřadnice}: $kp + p^2 = p(k+p)$ je přesně složené číslo $n$ samo o sobě
\item \textbf{y-souřadnice}: $kp + 1$ je zvolena tak, aby body rozmístila vertikálně v pravidelném vzorci
\end{itemize}

\subsection{Proč tato volba souřadnic?}

Souřadnice y jako $kp + 1$ není svévolná---vytváří \textbf{pravidelný les}. Když vysázíme stromy pro různé hodnoty $k$ a $p$, tvoří symetrický mřížkový vzorec. Tato symetrie odhaluje hlubokou pravidelnost v tom, jak jsou dělitelé rozděleny.

\textbf{Poznámka}: Mohli bychom vykreslit i případ $p=1$ (který dává posloupnost $k+1$ podél hlavní diagonály), ale jeho vynechání činí vzorec jasnějším: \emph{každý strom odpovídá děliteli a prvočísla jsou absence}.

\section{Les se objevuje}

Obrázek~\ref{fig:forest} ukazuje prvoles pro rozsah $n \leq 31$. Vysazujeme pouze stromy pro $p \geq 2$---čistou mřížku dělitelů. Šedá křivka zvýrazňuje vrstvu $k=1$ (body tvaru $n = p + p^2$).

\begin{figure}[!ht]
\centering
\includegraphics[width=\textwidth]{visualizations/primal-forest-100-parabola.pdf}
\caption{Prvoles: Stojíte na jižním okraji (spodek diagramu, y=0) a můžete chodit západo-východně (vlevo-vpravo, osa x). Na každé pozici se díváte přímo na sever (nahoru v diagramu, osa y). Každý strom odpovídá děliteli $p$ čísla $n = p(p+k)$ s $p \geq 2$, vysazenému v souřadnicích $(kp+p^2, kp+1)$. Prvočísla jsou x-souřadnice s \emph{čistými průhledy} (čárkované šipky). Šedá křivka ukazuje vrstvu $k=1$.}
\label{fig:forest}
\end{figure}

\subsection{Co vidíme?}

\begin{itemize}
\item \textbf{Stromy}: Každý strom odpovídá jednomu děliteli $p$ čísla $n$, kde $2 \leq p \leq \sqrt{n}$. Konkrétně pro $n = p \times (p+k)$ je strom vysazen na pozici $(kp+p^2, kp+1)$. Y-souřadnice $(kp+1)$ představuje, jak hluboko v lese tento dělitel stojí. Počet stromů na x-souřadnici $n$ se rovná počtu dělitelů $n$ v rozsahu $[2, \sqrt{n}]$. Tyto stromy tvoří les, který může blokovat váš průhled.

\item \textbf{Vaše pozice}: Stojíte na pozici $(x, 0)$ na jižním okraji---doslova na y=0. Můžete chodit západo-východně a zvolit si, kterou x-souřadnici chcete testovat, pak se podívat přímo na sever do lesa. Většina pozic má alespoň jeden strom někde na sever od vás. Ale některé pozice \emph{nemají žádné stromy v žádné hloubce na sever}. To jsou \textbf{prvočísla}---čísla s čistým průhledem skrz les.

\item \textbf{Test prvočíselnosti}: Postavte se na libovolnou x-souřadnici podél jižního okraje a podívejte se přímo na sever. Pokud vidíte stromy blokující váš výhled v nějaké hloubce, toto číslo je složené. Pokud máte dokonale čistý průhled skrz les, toto číslo je prvočíslo.
\end{itemize}

\subsection{Klíčový poznatek}

\begin{quote}
\textbf{Počet stromů na pozici n (napříč všemi hloubkami) se rovná počtu dělitelů n v rozsahu $[2, \sqrt{n}]$. Prvočísla mají nula takových dělitelů, takže nabízejí čisté průhledy skrz les.}
\end{quote}

Například:
\begin{itemize}
\item $n = 18 = 2 \times 9 = 3 \times 6$: dělitelé v $[2, \sqrt{18}] = [2, 4{,}24]$ jsou $\{2, 3\}$, takže \textbf{2 stromy} blokují váš výhled
\item $n = 16 = 2 \times 8 = 4 \times 4$: dělitelé v $[2, \sqrt{16}] = [2, 4]$ jsou $\{2, 4\}$, takže \textbf{2 stromy} blokují váš výhled
\item $n = 13$ (prvočíslo): žádné dělitele v $[2, \sqrt{13}] = [2, 3{,}6]$, takže \textbf{čistý průhled}
\end{itemize}

Les neukazuje pouze která čísla jsou složená---geometricky počítá jejich malé dělitele.

\section{Průzkum lesa}

\subsection{Geometrie lesa: Pravidelné diagonály ze zakřivených vrstev}

Klíčový nápad této vizualizace spočívá v překvapivém geometrickém paradoxu. Zatímco klasické Eratosthenovo síto postupně škrtá násobky prvočísel, Prvoles zasazuje \textbf{každé číslo $p \geq 2$} do pravidelné diagonální řady:

\begin{quote}
Každé $p$ (prvočíslo i složené) generuje diagonálu dělitelů tvaru $p(p+k)$ pro $k = 0, 1, 2, \ldots$\\
Body: $(p^2, 1)$, $(p^2+p, p+1)$, $(p^2+2p, 2p+1)$, $(p^2+3p, 3p+1)$, \ldots\\
Rozestupy: konstantní $(p, p)$ v obou směrech $\Rightarrow$ \textbf{sklon přesně 1} (úhel 45°)
\end{quote}

Například:
\begin{itemize}
\item $p=2$: diagonála $(4,1)$, $(6,3)$, $(8,5)$, $(10,7)$, \ldots s rozestupy $(2,2)$
\item $p=3$: diagonála $(9,1)$, $(12,4)$, $(15,7)$, $(18,10)$, \ldots s rozestupy $(3,3)$
\item $p=5$: diagonála $(25,1)$, $(30,6)$, $(35,11)$, $(40,16)$, \ldots s rozestupy $(5,5)$
\end{itemize}

\textbf{Paradox}: Horizontální ``vrstvy'' (body s pevným $k$) tvoří zakřivené řady. Například vrstva $k=1$ (zvýrazněná růžově) sleduje body $(x,y)$ splňující $x = y^2 - y$, tedy diskrétní odmocninnou funkci. Přesto se z těchto zakřivených vrstev poskládá dokonale \emph{pravidelná diagonální struktura} se sklonem 1!

Složené číslo s více děliteli se objeví vícekrát. Například $24 = 2 \times 12 = 3 \times 8 = 4 \times 6$ leží na třech různých diagonálách ($p=2,3,4$) jako tři stromy v různých hloubkách.

\subsection{Čísla s více stromy}

Čím více dělitelů má číslo, tím více stromů blokuje váš výhled. Mocniny prvočísel ($9, 25$) mají jeden strom, polprvočísla ($15$) dva, vysoce složená čísla ($36$) čtyři. Prvočísla nabízejí čisté průhledy.

\subsection{Diagonální řady a rostoucí mezery}

Všimněte si diagonálních řad stromů stoupajících na severovýchod? Každé $p$ vytváří svou vlastní diagonálu začínající v bodě $(p^2, 1)$ s rozestupy po $p$. Čím větší čísla, tím více diagonál se překrývá a tím hlouběji se les rozšiřuje na sever. To odhaluje hlubokou pravdu o rozložení prvočísel:

\begin{quote}
\textbf{Čím vyšší je x-souřadnice (čím větší číslo), tím hlouběji se les rozšiřuje na sever. Více stromů v různých hloubkách znamená vyšší pravděpodobnost, že nějaký strom zablokuje váš průhled.}
\end{quote}

Jak kráčíme na východ podél jižního okraje, les se zahušťuje a rozprostírá se dále na sever. Více stromů v různých hloubkách snižuje šanci najít čistý průhled. Tento geometrický pohled činí Větu o prvočíslech intuitivní: \emph{prvočísla řídnou}, protože les se zahušťuje a prohlubuje.

\section{Průzkum a cvičení}

\subsection{Cvičení}

\begin{enumerate}
\item \textbf{Hledání prvočíselných dvojčat}: Postavte se na pozici 11 (prvočíslo s čistým průhledem). Jděte pomalu na východ. Další čistý průhled je na 13---jen 2 kroky daleko (prvočíselná dvojčata!). Pokračujte v chůzi a najděte další dvojčata v rozsahu do 100. Kolik jich je? Studiem struktury lesa zkuste odhadnout, proč jsou některé oblasti ``bohatší'' na dvojčata než jiné. (Poznámka: Předpovědět přesný vzorec rozložení dvojčat je otevřený výzkumný problém!)\footnote{Pro zájemce: Viz \emph{Domněnka o prvočíselných dvojčatech} na \href{https://cs.wikipedia.org/wiki/Prvo\%C4\%8D\%C3\%ADseln\%C3\%A1_dvoj\%C4\%8Data}{cs.wikipedia.org}}

\item \textbf{Goldbachův průzkum}: Vyberte si pozici sudého čísla, řekněme 20. Můžete najít dvě prvočíselné pozice, která v součtu dají 20? Například $3+17=20$. Zkuste jiná sudá čísla do 50. Vždy najdete alespoň jeden takový pár? (Poznámka: Otázka ``funguje to pro všechna sudá čísla?'' je slavná Goldbachova domněnka, dosud nedokázaná!)\footnote{Pro zájemce: Více o \emph{Goldbachově domněnce} na \href{https://cs.wikipedia.org/wiki/Goldbachova_dom\%C4\%9B\%C5\%88ka}{cs.wikipedia.org}}

\item \textbf{Modifikace lesa}: Co kdybyste zahrnuli stromy s $p=1$ do vizualizace? Nebo co kdybyste vykreslili pouze lichá složená čísla? Vytvořte tyto varianty a popište, co se změní. Činí odstranění určitých stromů vzorce zřetelnějšími nebo méně zřetelnými? Co vám to říká o roli malých prvočísel?

\item \textbf{Vizuální test prvočíselnosti}: Prostudujte vizualizaci lesa pro $n \leq 30$. Zkuste odhadnout, zda budou pozice 91 a 97 prvočísla (čisté průhledy). Vysvětlete své uvažování pouze z toho, co vidíte v diagramu---diagonální řady, hustota stromů. Pak ověřte výpočtem.

\item \textbf{Pozorování hustoty}: Spočítejte, kolik stromů se objevuje v oblastech 1-30, 31-60 a 61-90. Roste nebo klesá počet stromů? Zkuste odhadnout, proč se hustota lesa mění tímto způsobem.\footnote{Pro zájemce: Tento trend souvisí s \emph{Větou o prvočíslech}, která popisuje, jak prvočísla řídnou. Viz \href{https://cs.wikipedia.org/wiki/V\%C4\%9Bta_o_prvo\%C4\%8D\%C3\%ADslech}{cs.wikipedia.org}}
\end{enumerate}

\section{Spojení s jinými vizualizacemi}

Existují další geometrické vizualizace prvočísel, z nichž každá zdůrazňuje jiné aspekty:

\begin{itemize}
\item \textbf{Ulamova spirála}\footnote{Viz \href{https://cs.wikipedia.org/wiki/Ulamova_spir\%C3\%A1la}{cs.wikipedia.org}} (1963): Uspořádání celých čísel do spirály; prvočísla se shlukují podél diagonálních čar
\item \textbf{Sacksova spirála}\footnote{Viz \url{https://en.wikipedia.org/wiki/Ulam_spiral\#Variations} (anglicky). Podobná Ulamově spirále, ale používá Archimedovu spirálu, kde vzdálenost od středu roste lineárně s úhlem.}: Uspořádání celých čísel na Archimedově spirále; prvočísla tvoří zakřivené vzorce
\item \textbf{Prvočíselné mřížky}: Různá 2D uspořádání odhalující strukturu
\end{itemize}

Metafora lesa je zvláště intuitivní: \textbf{prvočísla nejsou náhodně rozptýlená---jsou to průhledy, které zůstanou po systematickém vysazení stromů.}

\section{Časté otázky}

Vizualizace nabízí odpovědi na otázky, které se často objevují:

\paragraph{Proč jsou prvočísla ``speciální''?}
Prvočísla jsou jediná čísla s čistými průhledy. Jsou zásadně odlišná od složených čísel.

\paragraph{Proč mezery mezi prvočísly rostou?}
Les se rozprostírá hlouběji a zahušťuje se s rostoucími čísly. Průhledy se stávají vzácnějšími, mezery mezi prvočísly se přirozeně rozšiřují.

\paragraph{Je nějaký vzorec v prvočíslech?}
Ano! Prvočísla jsou přesně mezery v pravidelné mřížce dělitelů. Vzorec je viditelný, jakmile vysázíme stromy do 2D.

\paragraph{Proč se pravděpodobnost nalezení prvočísla snižuje?}
Jak čísla rostou, mají více dělitelů. Les se rozprostírá hlouběji a zahušťuje se. Více stromů ve více hloubkách blokuje průhledy, takže prvočísla se stávají vzácnějšími.

\section{Paradox pravidelnosti}

\textbf{Hluboké tajemství:} Prvoles odhaluje ohromující paradox v srdci rozložení prvočísel.

\textbf{Vstup} jsou dokonale pravidelné diagonální vzorce---násobky 2 v intervalech 2, násobky 3 v intervalech 3, násobky 5 v intervalech 5. Každé prvočíslo generuje svou vlastní dokonale uniformní mřížku.

\textbf{Transformace} je jednoduchý kvadratický posun $p^2 + kp$, zcela deterministický, bez náhodnosti. Pouze posunuje každý diagonální řádek na pozici určenou druhou mocninou.

\textbf{Výstup} je tajemné rozložení prvočísel: mezery mezi nimi chaoticky kolísají (1, 2, 2, 4, 2, 4, 2, 4, 6, 2, 6, 4, \ldots), prvočíselná dvojčata se objevují nepředvídatelně, a celá struktura je předmětem Riemannovy hypotézy---jednoho z nejhlubších nevyřešených problémů matematiky.\footnote{Pro zájemce: Maximální velikost mezer souvisí s \emph{Cramérovou domněnkou} (\href{https://en.wikipedia.org/wiki/Cram\%C3\%A9r\%27s_conjecture}{en.wikipedia.org}), a \emph{Riemannova hypotéza}, formulovaná 1859, zůstává nevyřešena přes 165 let (\href{https://cs.wikipedia.org/wiki/Riemannova_hypot\%C3\%A9za}{cs.wikipedia.org}).}

\textbf{Otázka}: Jak může sjednocení nekonečně mnoha dokonale pravidelných vzorů, systematicky posunutých, vytvořit něco tak složitého a tajemného, jako je rozložení prvočísel?

Struktura je dokonale pravidelná, předvídatelná---a přesto se průhledy vymykají předpovědi. Vizualizace odhaluje podstatu tajemství: \textbf{jednoduchá geometrická pravidla generují nepopsatelnou složitost.}

Proto geometrický pohled, navzdory své jasnosti, nenabízí výpočetní zjednodušení. Průhledy jsou globální jev---prvočíslo se musí vyhnout \emph{všem} pravidelným vzorům najednou. Test prvočíselnosti zůstává stejně obtížný, jen vidíme jeho geometrickou podstatu.

\section{Závěr}

Vizualizace Prvoles transformuje lineární algoritmus Eratosthenova síta do geometrického pohledu, kde prvočísla nejsou to, co zbyde po škrtání, ale průhledy skrz pravidelnou mřížku dělitelů.

Je zajímavé, jak stejná struktura může oslovit různé typy myšlení---někdo vidí algoritmus, někdo geometrický vzorec, někdo vizuální krajinu. A to je právě krása matematiky.

\end{document}
