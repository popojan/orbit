\appendix

\section{Spojité skóre prvočíselnosti}

Klasická charakterizace prvočísel je binární: číslo $n \in \mathbb{N}$, $n \geq 2$, je buď prvočíslo, nebo složené. Geometrická konstrukce z hlavního textu umisťuje složená čísla $kp + p^2$ (kde $p \geq 2$, $k \in \mathbb{N}_0$) do bodové mřížky v rovině; číslo $n$ je prvočíslo právě tehdy, když přímka $x = n$ neprotíná žádný bod této mřížky.

Tuto diskrétní charakterizaciju lze zobecnit na \emph{spojitou míru} měřením, jak blízko je $n$ ke stromům v každé hloubce. Místo tvrdého minima použijeme \textbf{p-normový soft-minimum}---algebraickou aproximaci, která zachycuje ``typickou vzdálenost'' k nejbližším bodům.

\subsection{Definice P-normového skóre}

\begin{definition}[Vzdálenosti k blízkým bodům]
\label{def:distances}
Pro celé číslo $n \geq 2$ a celé $d \geq 2$ definujeme množinu \textbf{čtverců vzdáleností} ke všem bodům ve ``vrstvě'' $d$:
\begin{equation}
D_d(n) = \left\{ (n - (kd + d^2))^2 + \varepsilon \;\bigg|\; k = 0, 1, \ldots, \lfloor n/d \rfloor \right\},
\end{equation}
kde $\varepsilon > 0$ je malá regularizační konstanta (typicky $\varepsilon = 1$) zabraňující dělení nulou.
\end{definition}

Geometricky: každý bod $(kd+d^2, kd+1)$ v síti stromů odpovídá složenému číslu $kd+d^2$. Měříme \emph{čtverce} vzdáleností v ose $x$ od testované hodnoty $n$ ke všem takovým bodům pro pevné $d$.

\begin{definition}[P-normový soft-minimum]
\label{def:pnorm}
Pro pevné $d$ a hodnotu $p > 0$ (typicky $p = 3$) definujeme \textbf{p-normový soft-minimum} jako zobecněný průměr:
\begin{equation}
\mu_d(n, p, \varepsilon) = \left( \frac{1}{|D_d(n)|} \sum_{r \in D_d(n)} r^{-p} \right)^{-1/p}.
\end{equation}
\end{definition}

Toto je \emph{power mean} (také Hölderův průměr) s exponentem $-p$. Pro $p \to \infty$ konverguje k minimu, pro $p = 1$ je to harmonický průměr. Volba $p = 3$ empiricky poskytuje dobrou rovnováhu mezi hladkostí a citlivostí.

\begin{definition}[Spojité skóre prvočíselnosti]
\label{def:primality-score}
Pro celé číslo $n \geq 2$ definujeme \textbf{spojité skóre prvočíselnosti} agregací přes všechny hloubky:
\begin{equation}
F_n(s) = \sum_{d=2}^{n} \mu_d(n, 3, 1)^{-s},
\end{equation}
kde $s \in \mathbb{R}$ je parametr určující váhu jednotlivých vrstev (typicky $s = 1$).
\end{definition}

Pro účely vizualizace se často používá logaritmická komprese $\log(1 + F_n(s))$, která zviditelňuje strukturu i pro velká $n$.

\begin{observation}[Limitní chování]
\label{obs:limit}
Pro složené $n = kd + d^2$ existuje hloubka $d$, kde $\mu_d(n, 3, \varepsilon) \approx \sqrt{\varepsilon}$ (minimum dosaženo), takže $F_n(s)$ roste pomalu. Pro prvočíslo $n$ všechny $\mu_d(n, 3, \varepsilon)$ zůstávají výrazně nad $1$, takže $F_n(s)$ je malé. To vytváří rozdíl mezi prvočísly a složenými čísly.
\end{observation}

\subsection{Empirická pozorování}

\begin{observation}[Obálková struktura]
\label{obs:envelope}
Pro pevné $s = 1$ funkce $n \mapsto \log(1 + F_n(1))$ vykazuje následující empirickou strukturu (viz Obrázek~\ref{fig:envelope}):
\begin{enumerate}
\item Prvočísla tvoří přibližně hladkou \textbf{dolní obálku} (malé hodnoty $F_n$).
\item Složená čísla leží \textbf{nad} touto obálkou (větší hodnoty $F_n$), stratifikovaná podle faktorizační složitosti.
\end{enumerate}
Poznámka: Na rozdíl od původní exp/log formulace, kde prvočísla tvořila \emph{horní} obálku, p-normová formulace invertuje pořadí---prvočísla mají \emph{nejmenší} $F_n$ hodnoty.
\end{observation}

\begin{figure}[h]
\centering
\includegraphics[width=0.7\textwidth]{visualizations/pnorm-envelope-paper.pdf}
\caption{P-normové spektrum prvočíselnosti ($p=3$, $\varepsilon=1/100$): prvočísla (oranžová) tvoří hladkou \emph{dolní} obálku (malé $F_n$), složená čísla (modrá) se rozptylují výše. Vzdálenost každého složeného čísla od obálky kóduje jeho faktorizační strukturu.}
\label{fig:envelope}
\end{figure}

\begin{observation}[Stratifikace podle $\Omega(n)$]
\label{obs:stratification}
Empiricky se složená čísla stratifikují podle počtu prvočíselných činitelů s násobností, $\Omega(n) = \sum_{p^k \| n} k$ (viz Obrázek~\ref{fig:stratification}):
\begin{enumerate}
\item Mocniny prvočísel ($p^2, p^3, \ldots$) s $\Omega(n) = 2, 3, \ldots$ leží nejblíže k prvočíselné obálce.
\item Polprvočísla ($pq$ s $p \neq q$) s $\Omega(n) = 2$ leží ve střední výšce.
\item Čísla s $\Omega(n) \geq 3$ leží nejvýše nad obálkou.
\end{enumerate}
Obecně: hodnota $F_n(1)$ \emph{roste} s rostoucím $\Omega(n)$. Čím složitější faktorizace, tím větší $F_n$.
\end{observation}

\begin{figure}[h]
\centering
\includegraphics[width=0.7\textwidth]{visualizations/pnorm-stratification-paper.pdf}
\caption{Stratifikace složených čísel podle $\Omega(n)$ ($p=3$, $\varepsilon=1/100$): mocniny prvočísel (zelená) nejblíže k prvočíselné obálce, polprvočísla (modrá) výše, čísla s $\Omega(n) \geq 3$ (červená) nejvýše.}
\label{fig:stratification}
\end{figure}

\begin{observation}[Inverzní zesílení prvočísel]
\label{obs:inverse}
Nejpřekvapivější objev: definujeme-li \textbf{inverzní agregaci}
\begin{equation}
G_{\text{inv}}(s, \sigma) = \sum_{n=2}^{N} \frac{1}{F_n(s) \cdot n^\sigma},
\end{equation}
pak prvočísla (s malými $F_n$) spontánně \textbf{dominují} součet. Pro $s=1$, $\sigma=1.5$, $N=50$:
\begin{itemize}
\item Přímá agregace $G(s,\sigma) = \sum F_n/n^\sigma$: prvočísla přispívají \textbf{37\,\%}
\item Inverzní agregace $G_{\text{inv}}$: prvočísla přispívají \textbf{84\,\%}!
\end{itemize}
Žádná umělá váha, žádný trik---geometrická struktura přirozeně zesílí prvočíselný signál inverzí.
\end{observation}

\begin{figure}[h]
\centering
\includegraphics[width=0.6\textwidth]{visualizations/inverse-prime-dominance.pdf}
\caption{Ilustrace inverzního zesílení prvočísel (pro $s=1$, $\sigma=1.5$, $N=50$): přímá agregace $G$ má 37\,\% příspěvek prvočísel, inverzní agregace $G_{\text{inv}}$ spontánně zesílí prvočísla na 84\,\%. Geometrická struktura přirozeně zvýrazňuje prvočísla bez umělých vah.}
\label{fig:inverse}
\end{figure}

\begin{observation}[Závislost na geometrické formulaci]
\label{obs:geometric}
Empiricky: definování analogického p-normu pomocí klasické modulární aritmetiky (vzdálenost k násobkům $p$) neprodukuje obálkovou strukturu. Geometrická formulace $(kp + p^2, kp+1)$ je podstatná pro pozorovanou stratifikaci a inverzní zesílení.
\end{observation}

\subsection{Výpočetní složitost}

Výpočet $F_n(s)$ vyžaduje vyhodnocení $\mu_d(n, p, \varepsilon)$ pro všechna $d \in \{2, 3, \ldots, n\}$. Pro každé $d$ je třeba sečíst $\lfloor n/d \rfloor$ členů, celkově tedy $\Theta(n \log n)$ operací.

Optimalizace: lze limitovat na $d \leq \sqrt{n}$, protože větší $d$ přispívají zanedbatelně. To snižuje složitost na $\Theta(n^{3/2})$.

Tato složitost je srovnatelná s Eratosthenovým sítem, ale poskytuje \emph{spojitou míru} místo binární klasifikace. Metoda nenabízí algoritmické zrychlení pro rozhodování prvočíselnosti, ale odhaluje gradient faktorizační složitosti.

\subsection{Otevřené problémy}

Následující otázky zůstávají otevřené:

\begin{enumerate}
\item \textbf{Asymptotické chování}: Lze chování prvočíselné obálky $\min_{p \leq n, \, p \text{ prvočíslo}} F_p(s)$ vyjádřit explicitně? Roste jako $\log n$, $\log \log n$, nebo jinak?

\item \textbf{Ostrost stratifikace}: Je stratifikace podle $\Omega(n)$ ostrá (disjunktní vrstvy), nebo se vrstvy překrývají? Existují složená čísla s různými $\Omega$ ale stejným $F_n$?

\item \textbf{Spojení s $\zeta(s)$}: Inverzní agregace $G_{\text{inv}}(s, \sigma)$ spontánně zesílí prvočísla. Souvisí tato funkce s Riemannovou zeta funkcí $\zeta(s)$, s funkcí počtu prvočísel $\pi(x)$, nebo s jinými klasickými funkcemi rozložení prvočísel?

\item \textbf{Funkcionální rovnice}: Má $G_{\text{inv}}(s, \sigma)$ funkcionální rovnici analogickou k $\zeta(s)$? Existuje symetrie v komplexní rovině?

\item \textbf{Nuly v komplexní rovině}: Empiricky se zdá, že $F_n(s)$ nemá nuly pro $\Re(s) > 0$. Je to pravda? Pokud ano, má $G_{\text{inv}}(s, \sigma)$ nuly analogické k Riemannovým nulám?

\item \textbf{Teoretické zdůvodnění inverzního zesílení}: Proč přesně inverzní agregace zesílí prvočísla z 37\,\% na 84\,\%? Lze tento efekt teoreticky předpovědět, nebo jde o čistě empirický objev?
\end{enumerate}

\section*{Poděkování}

Tato vizualizace vznikla z rekreačních zkoumání ve výpočetní teorii čísel. Děkuji komunitě Wolfram Language za nástroje, které dělají takové experimenty potěšením.

\paragraph{Kód a reprodukovatelnost:} Wolfram Language kód pro generování vizualizací je k dispozici na vyžádání. Kontaktujte autora pro zdrojové soubory.\footnote{Kód a další materiály budou dostupné v repozitáři: \texttt{https://github.com/popojan/orbit}}

