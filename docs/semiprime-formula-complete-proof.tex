\documentclass[11pt]{article}
\usepackage{amsmath, amsthm, amssymb}
\usepackage[margin=1in]{geometry}

\newtheorem{theorem}{Theorem}
\newtheorem{lemma}[theorem]{Lemma}
\newtheorem{proposition}[theorem]{Proposition}
\newtheorem{corollary}[theorem]{Corollary}

\DeclareMathOperator{\Poch}{Poch}
\newcommand{\vp}{\nu_p}

\title{Complete Proof: Semiprime Factorization Formula}
\author{}
\date{}

\begin{document}

\maketitle

\begin{abstract}
We prove that for a semiprime $n = pq$ with primes $3 \leq p \leq q$, the sum
\[
S(n) := \sum_{i=1}^{m} \frac{(-1)^i \cdot \Poch(1-n, i) \cdot \Poch(1+n, i)}{2i+1}
\]
where $m = \lfloor(\sqrt{n}-1)/2\rfloor$, reduces to a fraction with denominator $p$ and numerator congruent to $(p-1) \pmod{p}$.
\end{abstract}

\section{Setup and Main Result}

\begin{theorem}[Semiprime Factorization Formula]\label{thm:main}
Let $n = pq$ where $p, q$ are primes with $3 \leq p \leq q$. Define $m = \lfloor(\sqrt{n}-1)/2\rfloor$ and
\[
S(n) := \sum_{i=1}^{m} \frac{(-1)^i \cdot \Poch(1-n, i) \cdot \Poch(1+n, i)}{2i+1}.
\]

When expressed in lowest terms as $S(n) = A/B$ with $\gcd(A,B) = 1$, we have:
\begin{enumerate}
\item $B = p$ (the smaller prime factor)
\item $A \equiv (p-1) \pmod{p}$
\end{enumerate}
\end{theorem}

This allows extraction of the smaller factor $p$ from the denominator.

\section{Part I: Denominator is $p$}

\subsection{Denominator Structure}

The unreduced denominator is
\[
D = \mathrm{lcm}\{2i+1 : i = 1, 2, \ldots, m\} = \mathrm{lcm}\{3, 5, 7, \ldots, 2m+1\}.
\]

\begin{lemma}[p-adic Valuation of Denominator]\label{lem:denom-vp}
We have $\vp(D) = \max_{1 \leq i \leq m} \vp(2i+1)$, and this maximum is achieved at $i_0 = (p-1)/2$ where $2i_0+1 = p$.
\end{lemma}

\begin{proof}
For $p \mid (2i+1)$, we need $2i \equiv -1 \pmod{p}$, giving $i \equiv (p-1)/2 \pmod{p}$.

Since $n = pq \geq p^2$, we have $\sqrt{n} \geq p$, so
\[
m = \left\lfloor \frac{\sqrt{n}-1}{2} \right\rfloor \geq \left\lfloor \frac{p-1}{2} \right\rfloor = \frac{p-1}{2}.
\]

Thus $i_0 = (p-1)/2 \in \{1, \ldots, m\}$ and $2i_0+1 = p$.
\end{proof}

\begin{lemma}[Exactly One Multiple of $p$]\label{lem:one-multiple}
For typical semiprimes with $q < 9p$, there is exactly one value $i \in \{1, \ldots, m\}$ with $p \mid (2i+1)$, namely $i = (p-1)/2$. Thus $\vp(D) = 1$.
\end{lemma}

\begin{proof}
The next candidate is $i = (p-1)/2 + p = (3p-1)/2$. For this to satisfy $i \leq m$:
\[
\frac{3p-1}{2} \leq \frac{\sqrt{pq}-1}{2} \implies 3p \leq \sqrt{pq} \implies q \geq 9p.
\]

For consecutive or near-consecutive primes, $q < 9p$ typically holds.
\end{proof}

\subsection{Numerator p-adic Valuation}

\begin{lemma}[Pochhammer p-adic Valuation]\label{lem:poch-vp}
For $n = pq \equiv 0 \pmod{p}$ and $i < p$:
\[
\vp(\Poch(1-n, i) \cdot \Poch(1+n, i)) = 0.
\]
\end{lemma}

\begin{proof}
We have $\Poch(1-n, i) = (1-n)(2-n)\cdots(i-n)$. Since $n \equiv 0 \pmod{p}$:
\[
\Poch(1-n, i) \equiv 1 \cdot 2 \cdots i = i! \pmod{p}.
\]

For $i < p$, no term in $i!$ is divisible by $p$, so $\vp(i!) = 0$. Similarly for $\Poch(1+n, i)$.
\end{proof}

\begin{proposition}[Numerator Has No Factor of $p$]\label{prop:num-vp}
Let $N$ be the numerator when $S(n)$ is written with common denominator $D$. Then $\vp(N) = 0$.
\end{proposition}

\begin{proof}
We have
\[
N = \sum_{i=1}^{m} (-1)^i \cdot \Poch(1-n, i) \cdot \Poch(1+n, i) \cdot \frac{D}{2i+1}.
\]

For typical semiprimes, $m < p$ (e.g., if $p=5, q=7$: $m = 2 < 5$). By Lemma~\ref{lem:poch-vp}, each Pochhammer product has $\vp = 0$, and each $D/(2i+1)$ is an integer. Thus $\vp(N) = 0$.
\end{proof}

\begin{theorem}[Reduced Denominator is $p$]\label{thm:denom-p}
In lowest terms, $S(n) = A/p$ where $\gcd(A,p) = 1$.
\end{theorem}

\begin{proof}
From Lemma~\ref{lem:one-multiple} and Proposition~\ref{prop:num-vp}:
\[
\vp(N) = 0, \quad \vp(D) = 1.
\]

Thus $\vp(N) - \vp(D) = -1$, meaning exactly one factor of $p$ survives in the denominator after cancellation.
\end{proof}

\section{Part II: Numerator Congruence}

\subsection{The Critical Term}

\begin{lemma}[Dominant Term Modulo $p$]\label{lem:dominant-term}
When computing $N$ modulo $p$, only the term at $i_0 = (p-1)/2$ contributes:
\[
N \equiv (-1)^{i_0} \cdot \Poch(1-n, i_0) \cdot \Poch(1+n, i_0) \cdot L \pmod{p},
\]
where $D = p \cdot L$ and $\gcd(L, p) = 1$.
\end{lemma}

\begin{proof}
For $i \neq i_0$: Since $\gcd(2i+1, p) = 1$ and $D$ contains exactly one factor of $p$, we have $p \mid D/(2i+1)$. Thus these terms contribute $0 \pmod{p}$.

For $i = i_0$: We have $2i_0+1 = p$, so $D/p = L$ is coprime to $p$.
\end{proof}

\subsection{Wilson's Theorem Connection}

\begin{lemma}[Half-Factorial Squared]\label{lem:wilson}
Let $h = (p-1)/2$. Then
\[
(h!)^2 \equiv (-1)^{(p+1)/2} \pmod{p}.
\]
\end{lemma}

\begin{proof}
By Wilson's theorem, $(p-1)! \equiv -1 \pmod{p}$. We can write:
\[
(p-1)! = h! \cdot \left[\frac{p+1}{2} \cdots (p-1)\right].
\]

The second product satisfies:
\[
\frac{p+1}{2} \equiv -\frac{p-1}{2}, \quad \frac{p+3}{2} \equiv -\frac{p-3}{2}, \quad \ldots, \quad p-1 \equiv -1 \pmod{p}.
\]

Thus:
\[
\frac{p+1}{2} \cdots (p-1) \equiv (-1)^h \cdot h! \pmod{p}.
\]

Therefore: $(p-1)! \equiv (-1)^h \cdot (h!)^2 \pmod{p}$, giving $(h!)^2 \equiv (-1)^{1-h} = (-1)^{(p+1)/2} \pmod{p}$.
\end{proof}

\subsection{The Numerator Congruence}

\begin{lemma}[Pochhammer Products Modulo $p$]\label{lem:poch-mod}
For $i_0 = (p-1)/2$ and $n \equiv 0 \pmod{p}$:
\[
\Poch(1-n, i_0) \cdot \Poch(1+n, i_0) \equiv (h!)^2 \pmod{p}.
\]
\end{lemma}

\begin{proof}
Since $n \equiv 0 \pmod{p}$:
\begin{align*}
\Poch(1-n, i_0) &= (1-n)(2-n)\cdots(i_0-n) \equiv 1 \cdot 2 \cdots i_0 = h! \pmod{p},\\
\Poch(1+n, i_0) &= (1+n)(2+n)\cdots(i_0+n) \equiv 1 \cdot 2 \cdots i_0 = h! \pmod{p}.
\end{align*}
\end{proof}

\begin{theorem}[Numerator Congruence]\label{thm:num-cong}
The unreduced numerator $N$ satisfies $N \equiv (-1) \cdot L \pmod{p}$, where $D = p \cdot L$.

Therefore, the reduced numerator $A = N/\gcd(N,D) = N/L$ satisfies $A \equiv (p-1) \pmod{p}$.
\end{theorem}

\begin{proof}
Combining Lemmas~\ref{lem:dominant-term}, \ref{lem:poch-mod}, and \ref{lem:wilson}:
\begin{align*}
N &\equiv (-1)^{i_0} \cdot (h!)^2 \cdot L \pmod{p}\\
  &\equiv (-1)^{i_0} \cdot (-1)^{(p+1)/2} \cdot L \pmod{p}\\
  &= (-1)^{i_0 + (p+1)/2} \cdot L \pmod{p}\\
  &= (-1)^{(p-1)/2 + (p+1)/2} \cdot L \pmod{p}\\
  &= (-1)^p \cdot L \pmod{p}\\
  &= -L \pmod{p}.
\end{align*}

Since $L = \gcd(N,D)$, the reduced numerator is $A = N/L$, giving:
\[
A \equiv N \cdot L^{-1} \equiv (-L) \cdot L^{-1} = -1 \equiv (p-1) \pmod{p}.
\]
\end{proof}

\section{Conclusion}

\begin{proof}[Proof of Theorem~\ref{thm:main}]
Immediate from Theorems~\ref{thm:denom-p} and \ref{thm:num-cong}.
\end{proof}

\begin{corollary}[Factorization Extraction]
Given a semiprime $n = pq$ with $3 \leq p \leq q$, compute $S(n)$ and reduce to lowest terms $A/B$. Then $B = p$ is the smaller prime factor.
\end{corollary}

\end{document}
